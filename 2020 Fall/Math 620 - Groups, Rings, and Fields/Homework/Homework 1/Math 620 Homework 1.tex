\documentclass{article}
\usepackage{amsmath}
\usepackage{amssymb}
\usepackage{bm}
\usepackage{amsthm}
\usepackage{enumerate}
\usepackage{graphicx}
\usepackage{psfrag}
\usepackage{color}
\usepackage{url}
\usepackage{listings}
\usepackage{xcolor}
\usepackage{tikz}
\usetikzlibrary{positioning}
\tikzset{main node/.style={circle,fill=gray!20,draw,minimum size=.5cm,inner sep=0pt},}

\definecolor{codegreen}{rgb}{0,0.5,0}
\definecolor{codewhite}{rgb}{1,1,1}
\definecolor{codegray}{rgb}{0.5,0.5,0.5}
\definecolor{codepurple}{rgb}{0.58,0,0.82}
\definecolor{codeblack}{rgb}{0,0,0}
\definecolor{codeorange}{rgb}{0.8,0.4,0}

\lstdefinestyle{mystyle}{
    backgroundcolor=\color{codewhite},   
    commentstyle=\color{codegray},
    keywordstyle=\color{codegreen},
    numberstyle=\color{codegray},
    stringstyle=\color{codeorange},
    basicstyle=\ttfamily ,
    breakatwhitespace=false,         
    breaklines=true,                 
    captionpos=b,                    
    keepspaces=true,                 
    numbers=left,                    
    numbersep=5pt,                  
    showspaces=false,                
    showstringspaces=false,
    showtabs=false,                  
    tabsize=4
}
\lstset{style=mystyle}


\setlength{\hoffset}{-1in}
\addtolength{\textwidth}{1.5in}
\setlength{\voffset}{-1in}
\addtolength{\textheight}{1.5in}

\newcommand{\be}{\begin{enumerate}}
\newcommand{\ee}{\end{enumerate}}
\newcommand{\BigO}[1]{\ensuremath\mathcal{O}\left(#1\right)}
\newcommand{\il}[1]{\lstinline!#1!}
\newcommand{\gnorm}[1]{\left|\left|#1\right|\right|}
\newcommand{\abs}[1]{\left|#1\right|}
\newcommand{\parens}[1]{\left(#1\right)}
\newcommand{\bracks}[1]{\left\{#1\right\}}
\newcommand{\sqbracks}[1]{\left[#1\right]}
\newcommand{\vep}{\varepsilon}
\newcommand{\ceiling}[1]{\left\lceil#1\right\rceil}
\newcommand{\R}{\mathbb{R}}
\newcommand{\N}{\mathbb{N}}
\newcommand{\Z}{\mathbb{Z}}
\newcommand{\distrib}[2]{\text{#1}\left(#2\right)}
\newcommand{\dd}[1]{\frac{d}{d#1}}
\newcommand{\abracks}[1]{\left< #1\right>}

\DeclareMathOperator\lcm{lcm} % lcm

\begin{document}
	\begin{center}
		\textbf{Fall 2020, Math 620:\ Homework 1} \\
		\textbf{Due:\ Wednesday, September 2nd, 2020} \\
		\textbf{Joseph Diaz: 819947915}
	\end{center}
	\bigskip
\noindent
\textbf{Homework problems.}
You must submit \emph{all} homework problems in order to receive full credit.
\noindent\makebox[\linewidth]{\rule{\paperwidth}{0.4pt}}
\begin{enumerate}[(H1)]
\item 
Determine whether each of the following sets $G$ form a group under the given operation $*$.  Prove your assertions.  

\begin{enumerate}[(a)]
\item 
$G = \{1, 3, 5, 7, 9\} \subset \Z_{10}$; $a * b = ab$ (i.e.\ standard multiplication in $\Z_{10}$).  

\begin{proof}[Answer]
$\parens{G, *}$ does not form a group; because while it does have an identity
element $e = 1$, 
not every element of $G$ has an
inverse under the given operation. For example, $1,\ 3,\ 7,\ 9$ each have an inverse:
$$\sqbracks{1\cdot1}_{10} = 1,\ \sqbracks{3\cdot7}_{10} = 
\sqbracks{21}_{10} = 1,\ \sqbracks{9\cdot9}_{10} = \sqbracks{81}_{10} 
= 1$$
But $5$ does not, because:
$$\forall a \in G:\ [5 \cdot a]_{10} = [a \cdot 5]_{10} = 5$$
So $\parens{G, *}$ does not form a group.
\end{proof}


\item 
$G = \R$; $a * b = a + b + 3$.  

\begin{proof}[Answer]
$\parens{G, *}$ does form a group.
\\\\
Closure under $*$:\\
$G$ is the real numbers and $*$ is nothing more than the addition of
3 real numbers, so $G$ is closed under $*$.
\\
Associativity:\\
Let $a,b,c \in G$, then:
\begin{align*}
(a*b)*c &= (a + b + 3) + c + 3\\
&= a + b + 3 + c + 3 \\
&= a + b + c + 3 + 3 \\
&= a + (b + c + 3) + 3 \\
&= a*(b*c)
\end{align*}
Addition is associative among the reals, and so is $*$ among the 
elements of $G$.
\\
Identity:\\
Let $a \in G$, then the identity of $G$ is $e = -3$:
\begin{align*}
a * e = a + e + 3 = a + (-3) + 3 = a\\
e * a = e + a + 3 = -3 + a + 3 = a
\end{align*}
So, $\exists e\in G: a*e = e*a = a, \forall a \in G$, and $G$ has an identity.
\\
Inverses:\\
Let $a \in G$, we want to find $a^{-1} \in G$ such that $e = a*a^{-1}
= a^{-1}*a$:
\begin{align*}
e = a*a^{-1} \implies -3 = a + a^{-1} + 3 \implies 0 = a + a^{-1} \implies a^{-1} = -a \\
e = a^{-1}*a \implies -3 = a^{-1} + a + 3 \implies 0 = a^{-1} + a \implies a^{-1} = -a
\end{align*}
So we have $\forall a \in G, a^{-1} = -a:\ a^{-1}*a = a*a^{-1} = e$ and $G$ has inverses.
\\
\end{proof}


\item 
$G$ is the set of nonzero real numbers; $a * b = |a| \cdot b$.  

\begin{proof}[Answer]
$\parens{G, *}$ does not form a group, because $G$ does not have an identity.\\
Since $G = \R\backslash\{0\}$, the expected candidate for an identity
in $G$ is $e = 1$ but the operation $*$ means that it doesn't 
satisfy the identity property; for example, let $a \in G$, then:
\begin{align*}
1*a &= \abs{1}\cdot a = a \\
a*1 &= \abs{a}\cdot 1 = \abs{a}
\end{align*}
$\abs{a} = a$ only for $a > 0$, but $\abs{a} \neq a$ for $a < 0$; 
i.e. $a*1 \neq 1*a$, so $1$ is clearly not the identity. There are 
no other candidates for an inverse in $G$, so it is not a group.
\end{proof}

\end{enumerate}


\item 
Suppose $G$ is a group and $a, b, c \in G$.  Prove each of the 
following statements 
Using \textbf{only group axioms}, prove that if $ab = ac$, then $b = 
c$.  Be especially careful with associativity!  In particular, any 
triple products $xyz$ should be written as either $(xy)z$ or $x(yz)$.  

\begin{proof}
We have that $a,b,c \in G$ is a group and that $ab = ac$. 
$a \in G$ so there exists a unique $a^{-1} \in G$ such that $a^{-1}a 
= aa^{-1} = e$ where $e \in G$ is the identity. Then:
\begin{align*}
ab &= ac \\
a^{-1}\parens{ab} &= a^{-1}\parens{ac} \\
\parens{a^{-1}a}b &= \parens{a^{-1}a}c \\
eb &= ec \\
b &= c
\end{align*}
As desired.
\end{proof}


\item 
Determine whether each of the following statements is true or false.  
Prove your assertions.  

\begin{enumerate}
\item 
If $(G, \cdot)$ is a group and $a, b \in G$, then $(ab)^{-1} = b^{-1}a^{-1}$.  

\begin{proof}[Answer]
We will prove this equality using the properties of groups. Let $a, b 
\in G$, then $ab \in G$ and also $\parens{ab}^{-1} \in G$ such that
$(ab)^{-1}(ab) = (ab)(ab)^{-1} = e $ where $e \in G$ is the identity.
Take $(ab)(ab)^{-1} = e$, then:
\begin{align*}
(ab)(ab)^{-1} &= e \\
a(b(ab)^{-1}) &= e \\
(a^{-1}a)(b(ab)^{-1}) &= a^{-1}e \\
e(b(ab)^{-1}) &= a^{-1} \\
b(ab)^{-1} &= a^{-1} \\
(b^{-1}b)(ab)^{-1} &= b^{-1}a^{-1} \\
e(ab)^{-1} &= b^{-1}a^{-1} \\
(ab)^{-1} &= b^{-1}a^{-1}
\end{align*}
As desired.

\end{proof}

\pagebreak

\item 
If $(G, \cdot)$ is a group and $a, b, \in G$ with $|a| = n$ and $|b| = m$, then $\abs{ab} \leq \lcm(n,m)$.  

\begin{proof}[Answer]
We will show that this is not true in general by providing a 
counter-example. Consider the permutations possible on the 
elements of set 
$L = \bracks{1, 2, 3}$, which is isomorphic to $D_3$. 
These permutations are defined as
\begin{itemize}
\item $i: \bracks{1,2,3} \mapsto \bracks{1,2,3}$ (identity)
\item $c: \bracks{1,2,3} \mapsto \bracks{2,3,1}$ (cycle left once)
\item $c^2: \bracks{1,2,3} \mapsto \bracks{3,1,2}$ (cycle left twice)
\item $f_1: \bracks{1,2,3} \mapsto \bracks{1,3,2}$ (fix 1st item, 
swap other 2)
\item $f_2: \bracks{1,2,3} \mapsto \bracks{3,2,1}$ (fix 2nd item, 
swap other 2)
\item $f_3: \bracks{1,2,3} \mapsto \bracks{2,1,3}$ (fix 3rd item, 
swap other 2)
\end{itemize}
with orders:
$$\abs{i} = 1,\ \abs{c} = \abs{c^2} = 3,\ \abs{f_1} =
\abs{f_2} = \abs{f_3} =  2$$
So $f_3 f_2: \bracks{1,2,3} \mapsto \bracks{2,3,1}$, which means 
that $f_3 f_2 = c$ and $\abs{f_3 f_2} = \abs{c} = 3$. Going back to 
what we wanted to show:
$$\lcm\parens{\abs{f_3},\abs{f_2}} = \lcm(2,2) = 2$$
Clearly $3 \nleq 2$, so in general the inequality doesn't hold.
\end{proof}

\end{enumerate}

\end{enumerate}



\bigskip
\noindent
\textbf{Challenge problems.}
Challenge problems are not required for submission, but bonus points 
will be awarded for submitting a partial attempt or a complete 
solution.  

\begin{enumerate}[(C1)]
\item 
Suppose $(G, *)$ is a group, where $G = \{0, 1, 2, 3, 4, 5, 6, 7\}$ 
and $*$ is an operation satisfying
\begin{enumerate}[(i)]
\item 
$a * b \le a + b$ for every $a, b \in G$, and

\item 
$a * a = 0$ for every $a \in G$.  

\end{enumerate}

Write out the operation table for $G$, and \textbf{briefly} justify 
why this is the only possibility.  

\begin{proof}

\end{proof}

\end{enumerate}

\noindent\makebox[\linewidth]{\rule{\paperwidth}{0.4pt}}
	
\end{document}
