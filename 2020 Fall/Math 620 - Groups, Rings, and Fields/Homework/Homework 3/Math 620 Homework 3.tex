\documentclass{article}
\usepackage{amsmath}
\usepackage{amssymb}
\usepackage{bm}
\usepackage{amsthm}
\usepackage{enumerate}
\usepackage{graphicx}
\usepackage{psfrag}
\usepackage{color}
\usepackage{url}
\usepackage{listings}
\usepackage{xcolor}
\usepackage{tikz}
\usepackage{soul}
\usetikzlibrary{positioning}
\tikzset{main node/.style={circle,fill=gray!20,draw,minimum size=.5cm,inner sep=0pt},}

% In line code stuff%
\definecolor{codegreen}{rgb}{0,0.5,0}
\definecolor{codewhite}{rgb}{1,1,1}
\definecolor{codegray}{rgb}{0.5,0.5,0.5}
\definecolor{codepurple}{rgb}{0.58,0,0.82}
\definecolor{codeblack}{rgb}{0,0,0}
\definecolor{codeorange}{rgb}{0.8,0.4,0}

\lstdefinestyle{mystyle}{
    backgroundcolor=\color{codewhite},   
    commentstyle=\color{codegray},
    keywordstyle=\color{codegreen},
    numberstyle=\color{codegray},
    stringstyle=\color{codeorange},
    basicstyle=\ttfamily ,
    breakatwhitespace=false,         
    breaklines=true,                 
    captionpos=b,                    
    keepspaces=true,                 
    numbers=left,                    
    numbersep=5pt,                  
    showspaces=false,                
    showstringspaces=false,
    showtabs=false,                  
    tabsize=4
}
\lstset{style=mystyle}
\setlength{\hoffset}{-1in}
\addtolength{\textwidth}{1.5in}
\setlength{\voffset}{-1in}
\addtolength{\textheight}{1.5in}

% Custom commands%
\newcommand{\be}{\begin{enumerate}}
\newcommand{\ee}{\end{enumerate}}
\newcommand{\BigO}[1]{\ensuremath\mathcal{O}\left(#1\right)}
\newcommand{\il}[1]{\lstinline!#1!}
\newcommand{\norm}[1]{\left|\left|#1\right|\right|}
\newcommand{\abs}[1]{\left|#1\right|}
\newcommand{\parens}[1]{\left(#1\right)}
\newcommand{\bracks}[1]{\left\{#1\right\}}
\newcommand{\sqbracks}[1]{\left[#1\right]}
\newcommand{\vep}{\varepsilon}
\newcommand{\ceiling}[1]{\left\lceil#1\right\rceil}
\newcommand{\R}{\mathbb{R}}
\newcommand{\N}{\mathbb{N}}
\newcommand{\Z}{\mathbb{Z}}
\newcommand{\F}{\mathbb{F}}
\newcommand{\Q}{\mathbb{Q}}
\newcommand{\A}{\mathcal{A}}
\newcommand{\distrib}[2]{\text{#1}\left(#2\right)}
\newcommand{\dd}[1]{\frac{d}{d#1}}
\newcommand{\abracks}[1]{\left< #1\right>}

\begin{document}
	\begin{center}
		\textbf{Fall 2020, Math 620:\ Homework 3} \\
		\textbf{Due:\ Wednesday, September 16th, 2020} \\
		\textbf{Joseph Diaz: 819947915}\\
		\st{Collaborators} \st{\textbf{Conspirators}}
		\underline{\textbf{Partners in Crime}}:\\
		Heidi Allen, Jake Samuelson, Richard Castro,\\ 
		Anthony Milazzo, Mario Alvarez


\end{center}
\bigskip
\noindent
\textbf{Homework problems.}
You must submit \emph{all} homework problems in order to receive full 
credit.  
\noindent\makebox[\linewidth]{\rule{\paperwidth}{0.4pt}}
\begin{enumerate}[(H1)]
\item 
Prove $S_n$ is isomorphic to a subgroup of $A_{n+2}$.  
\begin{proof}
To show that $S_n$ is isomorphic to a subgroup of $A_{n+2}$, let
$\varphi: S_n \to Q \subset A_{n+2}$ be defined as
$$\forall \tau \in S_n,\ 
\varphi\parens{\tau} = \left\{\begin{array}{cc}
\tau & \text{if }\tau \in A_n \\
\sigma\tau & \text{if }\tau \in B_n
\end{array}\right.,\ \sigma = \parens{n+1\ n+2}$$
This way, every even permutation in $S_n$ is mapped to itself 
(which means that it's in $A_{n+2}$, because $A_n \subset A_{n+2}$)
and every odd permutation is mapped to an even permutation
in $A_{n+2}$ which will be distinct from the permutations mapped
directly from $A_n$ because they'll have the transposition $\sigma$.
So we have that $\abs{A_n} = \abs{Q}$ from this mapping. $\varphi$
also has the homomorphism property; to show this, let 
$\tau_1,\tau_2 \in S_n$ and we'll consider some cases.\\\\
$\tau_1\tau_2 \in A_n$:\\
This case must also be broken up, because of the behavior of
composition of even and odd permutations:\\
Both are even:\\
As both are even, this is trivial.
$$\varphi\parens{\tau_1\tau_2} = \tau_1\tau_2 =
\varphi\parens{\tau_1}\varphi\parens{\tau_2}$$ 
Both are odd:\\
This case requires us to use the 
disjointedness of $\sigma$ with $\tau_1, \tau_2$.
\begin{align*}
\varphi\parens{\tau_1\tau_2} &= \tau_1\tau_2 \\
&= \sigma\sigma\tau_1\tau_2 \\ 
&= \sigma\tau_1\sigma\tau_2 \\
&= \varphi\parens{\tau_1}\varphi\parens{\tau_2}
\end{align*}        
$\tau_1\tau_2 \in B_n$:\\
Suppose that $\tau_1$ is even and $\tau_2$ is odd, then
\begin{align*}
\varphi\parens{\tau_1\tau_2} &= \sigma\tau_1\tau_2 \\ 
&= \tau_1\sigma\tau_2 \\ 
&= \varphi\parens{\tau_1}\varphi\parens{\tau_2}
\end{align*}
By symmetry, we may conclude that the property is respected when
swapping the parities of $\tau_1$ and $\tau_2$, as well.
So, $\varphi$ is a bijective map with the homomorphism property and 
$S_n \cong Q \subset A_{n+2}$.
\end{proof}


\item 
Locate a generating set for $S_n$ consisting of only 2 generators.  
\begin{proof}
$S_n$ can be generated by $(1\ 2)$ and $(1\ 2\ \cdots\ n-1\ n)$, 
because they can generate the other 2-cycles in $S_n$. First let $b = 
(1\ 2\ \cdots\ n-1\ n)$, then the following procedure can be used to
generate adjacent transpositions.
\begin{align*}
b\parens{1\ 2}b^{-1} &= (2\ 3) \\
b\parens{2\ 3}b^{-1} &= (3\ 4) \\
b\parens{3\ 4}b^{-1} &= (4\ 5) \\
&\vdots \\
b\parens{n-2\ n-1}b^{-1} &= (n-1\ n)
\end{align*}
So $(i\ i+1) \in \abracks{(1\ 2),b}, \forall i \in \bracks{1, \cdots,
n-1}$. With those, yet more transpositions can be generated.
\begin{align*}
(2\ 3)(1\ 2)(2\ 3) &= (1\ 3) \\
(3\ 4)(1\ 3)(3\ 4) &= (1\ 4) \\
(4\ 5)(1\ 4)(4\ 5) &= (1\ 5) \\
&\vdots \\
(n-1\ n)(1\ n-1)(n-1\ n) &= (1\ n) \\
\end{align*}
So $(1\ i) \in \abracks{(1\ 2),b}, \forall i \in \bracks{2, \cdots,
n}$; and with these 2-cycles we can generate arbitrary 2-cycles in 
$S_n$ like so:
$$(i\ j) = (1\ i)(1\ j)(1\ i),\ \forall i,j \in \bracks{2,\cdots,n}$$
Since we can generate all 2-cycles in $S_n$, this implies that 
$\abracks{(1\ 2), (1\ 2\ \cdots\ n)} = S_n$. 

\end{proof}


\item 
Determine whether each of the following statements is true or false.  
Prove your assertions.  

\begin{enumerate}[(a)]
\item 
For each $n \ge 3$, every permutation in $S_n$ can be written as a 
product of 3-cycles.  

\begin{proof}[Answer]
This is false, because no 2-cycle in $S_n$ for $n \geq 2$ can be 
expressed as a product of 3-cycles. All 3-cycles are the product of a
pair of transpositions, like so:
$$(a\ b)(a\ c) = (a\ c\ b),\ \forall a,b,c \in \bracks{1,\cdots, n}, 
a \neq b \neq c$$ 
As such, this means that any 3-cycle is an even permutation and 
so is \emph{any} product of multiple 3-cycles. All 2-cycles are odd, 
so this means that some permutations cannot be written as 3-cycle 
products.
\end{proof}


\item 
For each $n \ge 2$, every permutation in $S_n$ is a product of at 
most $n-1$ transpositions.  

\begin{proof}[Answer]
This is false, because a counter example exists in $S_2$. 
The elements of $S_2$ are $\bracks{e, (1\ 2)}$. While $(1\ 2)$ is a 
product of a single transposition; the identity $e$ is not, because 
it's an even permutation and cannot be written as a 2-cycle.
\end{proof}


\item 
For each $n \ge 3$, every permutation in $S_n$ is a product of 
adjacent transpositions.  

\begin{proof}[Answer]
This is true. From (H2), we know that $S_n = \abracks{(1\ 2), 
(1\ 2\ \cdots\ n)}$. So to show that the every permutation in 
$S_n$ is 
a product of adjacent transpositions, we'll show
$$(1\ 2)(2\ 3)\cdots(n-2\ n-1)(n-1\ n) = (1\ 2\ \cdots\ n)$$
for all $n \in \N$ by using induction.\\\\
\textbf{Base case}:\\
Let $n = 3$, this gives
$$(1\ 2)(2\ 3) = \parens{
\begin{matrix}
1 & 2 & 3 \\
2 & 3 & 1
\end{matrix}} = (1\ 2\ 3)
$$
which establishes our base case.\\
\textbf{Inductive Hypothesis}:\\
Suppose that 
$$(1\ 2)(2\ 3)\cdots(k-2\ k-1)(k-1\ k) = (1\ 2\ \cdots\ k)$$
holds for some $n = k$.\\
\textbf{Induction Step}:\\
We now seek to show that the property hold for $n = k+1$. From the left side of the equation, we have
$$(1\ 2)(2\ 3) \cdots (k-2\ k-1)(k-1\ k)(k\ k+1) = (1\ 2\ 
\cdots\ k-1\ k)(k\ k+1)$$
by our inductive hypothesis and associativity of permutation 
composition. Now, composing these 2 permutations gives:
\begin{align*}
(1\ 2\ \cdots\ k-1\ k)(k\ k+1) &= \parens{
\begin{matrix}
1 & 2 & 3 & \cdots & k & k+1 \\
2 & 3 & 4 & \cdots  & k+1& 1
\end{matrix}} \\
&= (1\ 2\ 3\ \cdots\ k\ k+1)
\end{align*} 
which is what we wanted to show. Again, invoking the result from 
(H2), this implies that $\abracks{(1\ 2), (1\ 2\ \cdots\ n)} = 
\abracks{(1\ 2),(2\ 3),\cdots,(n-2\ n-1),(n-1\ n)} = S_n$.
\end{proof}

\end{enumerate}

\end{enumerate}


\bigskip
\noindent
\textbf{Challenge problems.}
Challenge problems are not required for submission, but bonus points will be awarded for submitting a partial attempt or a complete solution.  

\begin{enumerate}[(C1)]
\item 
Determine what familiar group is isomorphic to $\text{Aut}(S_3)$.  

\end{enumerate}
\noindent\makebox[\linewidth]{\rule{\paperwidth}{0.4pt}}
	
\end{document}
