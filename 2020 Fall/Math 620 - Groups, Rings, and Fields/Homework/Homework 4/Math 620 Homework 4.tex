\documentclass{article}
\usepackage{amsmath}
\usepackage{amssymb}
\usepackage{bm}
\usepackage{amsthm}
\usepackage{enumerate}
\usepackage{graphicx}
\usepackage{psfrag}
\usepackage{color}
\usepackage{url}
\usepackage{listings}
\usepackage{xcolor}
\usepackage{tikz}
\usepackage{soul}
\usetikzlibrary{positioning}
\tikzset{main node/.style={circle,fill=gray!20,draw,minimum size=.5cm,inner sep=0pt},}

% In line code stuff%
\definecolor{codegreen}{rgb}{0,0.5,0}
\definecolor{codewhite}{rgb}{1,1,1}
\definecolor{codegray}{rgb}{0.5,0.5,0.5}
\definecolor{codepurple}{rgb}{0.58,0,0.82}
\definecolor{codeblack}{rgb}{0,0,0}
\definecolor{codeorange}{rgb}{0.8,0.4,0}

\lstdefinestyle{mystyle}{
    backgroundcolor=\color{codewhite},   
    commentstyle=\color{codegray},
    keywordstyle=\color{codegreen},
    numberstyle=\color{codegray},
    stringstyle=\color{codeorange},
    basicstyle=\ttfamily ,
    breakatwhitespace=false,         
    breaklines=true,                 
    captionpos=b,                    
    keepspaces=true,                 
    numbers=left,                    
    numbersep=5pt,                  
    showspaces=false,                
    showstringspaces=false,
    showtabs=false,                  
    tabsize=4
}
\lstset{style=mystyle}
\setlength{\hoffset}{-1in}
\addtolength{\textwidth}{1.5in}
\setlength{\voffset}{-1in}
\addtolength{\textheight}{1.5in}

% Custom commands%
\newcommand{\be}{\begin{enumerate}}
\newcommand{\ee}{\end{enumerate}}
\newcommand{\BigO}[1]{\ensuremath\mathcal{O}\left(#1\right)}
\newcommand{\il}[1]{\lstinline!#1!}
\newcommand{\norm}[1]{\left|\left|#1\right|\right|}
\newcommand{\abs}[1]{\left|#1\right|}
\newcommand{\parens}[1]{\left(#1\right)}
\newcommand{\bracks}[1]{\left\{#1\right\}}
\newcommand{\sqbracks}[1]{\left[#1\right]}
\newcommand{\vep}{\varepsilon}
\newcommand{\ceiling}[1]{\left\lceil#1\right\rceil}
\newcommand{\R}{\mathbb{R}}
\newcommand{\N}{\mathbb{N}}
\newcommand{\Z}{\mathbb{Z}}
\newcommand{\F}{\mathbb{F}}
\newcommand{\Q}{\mathbb{Q}}
\newcommand{\A}{\mathcal{A}}
\newcommand{\distrib}[2]{\text{#1}\left(#2\right)}
\newcommand{\dd}[1]{\frac{d}{d#1}}
\newcommand{\abracks}[1]{\left< #1\right>}

\begin{document}
	\begin{center}
		\textbf{Fall 2020, Math 620:\ Homework 4} \\
		\textbf{Due:\ Thursday, September 24th, 2020} \\
		\textbf{Joseph Diaz: 819947915}\\
		\underline{\textbf{Partners in Crime}}:\\
		


\end{center}
\bigskip
\noindent
\textbf{Homework problems.}
You must submit \emph{all} homework problems in order to receive full 
credit.  
\noindent\makebox[\linewidth]{\rule{\paperwidth}{0.4pt}}
\begin{enumerate}[(H1)]
\item 
Suppose $G$ is a group and $H \subset G$ is a subgroup.  Prove 
\textbf{without using any theorems} that $b \in aH$ if and only if 
$bH = aH$.  

\begin{proof}
To prove this, we'll show that the equivalence holds from the
left and from the right.
\be
    \item[$\Longrightarrow$:] We have that $b \in aH$ and $bH 
    \subseteq 
    aH$. So there exists $h \in H$ such that $b = ah$. 
    There must also exist $h' \in H$ such that $hh' = h'h = e$, where
    $e$ is the identity element of $H$. This means
    $$b = ah \iff bh' = ahh' = a$$
    which itself implies that $aH \subseteq bH$. $aH$ and $bH$ are 
    subsets of one another, so $aH = bH$.
    
    \item[$\Longleftarrow$:] We have that $aH = bH$, so there exists
    $h_1, h_2 \in H$ such that $ah_1 = bh_2$. As $H$ is a subgroup,
    there must also be an element $h_2' \in H$, such that $h_2h_2'
    = h_2'h_2 = e$, where $e$ is the identity element of $H$. This
    means that 
    $$bh_2 = ah_1 \iff bh_2h_2' = b = ah_1h_2'$$
    and this implies that $b \in aH$, because $h_1h_2' \in H$.
    
\ee 
\end{proof}


\item 
Prove $A_n$ is a normal subgroup of $S_n$.  

\begin{proof}
To prove that $A_n$ is a normal subgroup, we'll use the test for
normality. The test for normality is 
$$A_n \text{ is normal if and only if }\tau A_n\tau^{-1} \subseteq 
A_n,\ \forall \tau \in S_n$$
To show that $\tau A_n\tau^{-1} \subseteq A_n$, we need only show 
that all permutations in $\tau A_n\tau^{-1}$ are even.
We know that 
for all $\tau \in S_n$ there exists $\tau^{-1} \in S_n$ such that
$$\tau\tau^{-1} = \tau^{-1}\tau = e$$
where $e$ is the identity element of $S_n$. The identity is an even
permutation in $S_n$ so this implies that the parities of $\tau$ and 
$\tau^{-1}$ are identical, because otherwise their composition would
result in an odd permutation. By it's own definition 
$$\tau A_n\tau^{-1} = \bracks{\tau a\tau^{-1}\ \big|\ a\in A_n}$$
and as all permutations in $A_n$ are even, we need only consider the
parity of $\tau$.
If $\tau$ is even, 
$\tau a\tau^{-1}$ must also be even because every element would be
the composition of three even permutations. 
If $\tau$ is odd, we have the
same result because $\parens{\tau a}\tau^{-1} = \tau\parens{a
\tau^{-1}}$ so regardless of which composition is done first, we'll 
have an odd permutation composed with an odd permutation, which will 
be even.\\\\As every permutation in $\tau A_n\tau^{-1}$ is even, we 
have that $\tau A_n\tau^{-1} \subseteq A_n$ and so $A_n$ is a normal
subgroup. 
\end{proof}


\item 
Fix a group $G$.  Define the \emph{center} of $G$ as the set
$$C = \{c \in G : ca = ac \text{ for all } a \in G\}$$
of elements that commute with every element of $G$.  

\begin{enumerate}[(a)]
\item 
Prove $C$ is a normal subgroup of $G$.  

\begin{proof}
To prove that $C$ is normal, we'll show that $aC = Ca$ for all 
$a \in G$. We have that 
$$aC = \bracks{ac\ \big|\ c \in C}$$
but by it's own definition, the elements of $C$ commute with every 
other element of $G$, so $ac = ca, a\in G, c \in C$. This property 
also implies that every element of $aC$ is also in $Ca$, so 
$aC \subseteq Ca$. We could make a similar argument for $Ca 
\subseteq aC$ again by the commutativity of the elements of $C$ with
those of $G$. So we have that $aC \subseteq Ca$ and $Ca 
\subseteq aC$, this implies that $aC = Ca$ which means that $C$ is 
a normal subgroup of $G$.
\end{proof}


\item 
Prove or disprove:\ $G/C$ has trivial center.  

\begin{proof}[Disproof]
This is not true in general. Consider the dihedral group $D_4$;
The center of $D_4$, which is we'll denote $C$, is $\bracks{e,\ r^2}
$. The quotient of $D_4$ with its center is
$$D_4/C = \bracks{[e]_C,\ [r]_C,\ [f]_C,\ [rf]_C}$$
This quotient group is isomorphic to $D_2$, and to show that we'll 
use the FIT.\\\\
Let $\varphi: D_4 \to D_2$ be a homomorphism defined as
\begin{gather*}
r \mapsto r\\
f \mapsto f
\end{gather*}
$\varphi$ is surjective because every element of $D_2$ can obtained
from an element of $D_4$ under $\varphi$, and we have that 
$\ker(\varphi) = \bracks{e, r^2} = C$. So by the FIT, 
$D_4/C \cong D_2$.\\\\ 
As $D_2$ is abelian, $D_4/C$ must be abelian,
as well. This means that the center of $D_4/C$ cannot be trivial,
because every element of $D_4/C$ commutes with every other element.

\end{proof}

\end{enumerate}

\item 
Fix a group $(G, \cdot)$ and a normal subgroup $H \subset G$, and 
consider the map 
$$\begin{array}{r@{}c@{}l}
\varphi:G &{}\longrightarrow{}& G/H \\
a &{}\longmapsto{}& [a]_H.
\end{array}$$
The goal of this problem is to prove the correspondence theorem from 
Problem~(D2).  

\begin{enumerate}[(a)]
\item 
Prove that if $K \subseteq G/H$ is a subgroup, then $\varphi^{-1}(K)$ 
is a subgroup of $G$ containing $H$.

\begin{proof}
To prove this, we must show that $\varphi^{-1}(K)$ is a subgroup of 
$G$ and that $H \subseteq \varphi^{-1}(K)$.\\\\
We have that 
$K \subseteq G/H$ is a subgroup, as such it's elements are cosets of
$H$; particularly, one of those cosets must be $eH = H$, the coset 
corresponding to the identity of $G$.  This implies that $H 
\subseteq \varphi^{-1}(K)$, because $H$ is normal subgroup of $G$ and
all the elements of $H$, including $e$ would map to $eH$ under $
\varphi$. So $H \subseteq \varphi^{-1}(K)$ and $e \in \varphi^{-1}(K)
$.\\\\
Now, it remains to show that $\varphi^{-1}(K)$ is closed and 
that each element has inverses. $\varphi$ is defined as
$a \mapsto [a]_H$; which means that, as a homomorphism, $\varphi$
preserves the group operation of $G$ into $G/H$. As $K$ is a subgroup
of $G/H$, it contains inverses for each of its elements and is 
closed; this would be impossible if $\varphi^{-1}(K)$ didn't satisfy
these properties as well, because of the preservation of the group
structure under the homomorphism.\\\\
So $\varphi^{-1}(K)$ must be a subgroup of $G$ containing $H$.
\end{proof}


\item 
Prove that if $K \subseteq G$ is a subgroup with $H \subseteq K$, 
then $\varphi(K)$ is a subgroup of $G/H$.  

\begin{proof}
We have that $H \subseteq K$, so there exists a mapping from $K$ into
$K/H$ whose elements would be cosets of $H$ and form a 
subgroup. Further, as $K \subseteq G$, all cosets of $H$ in $K/H$
would also be in $G/H$ and the image of $K$ under the mapping from
$K$ into $K/H$ would be identical to the image of $K$ under 
$\varphi$. So $\varphi(K)$ is a subgroup of $G/H$.
\end{proof}

\end{enumerate}

\item 
The following questions pertain to the correspondence theorem from 
Problem~(D2).  

\begin{enumerate}[(a)]
\item 
If a subgroup of $G/H$ is normal, is its corresponding subgroup of $G
$ normal?  What about the converse of this statement?  

\begin{proof} We'll consider these implications separately, under 
the homomorphism $\varphi: G \to G/H$.\\\\
\textbf{Original Statement}:\\

\textbf{Converse}:\\
If $K \lhd G$, then is $\varphi(K) \lhd G/H$? If $K$ is a normal
subgroup of $G$, then for all $k \in K$ and $g \in G,\ gkg^{-1} \in K
$. From this, the image of $K$ under $\varphi$ is 
$$\varphi(K) = \bracks{\varphi(k)\ \big|\ k \in K}$$
and, by the properties of homomorphisms, for all $k \in K$ and 
$g \in G,\ \varphi(gkg^{-1}) = \varphi(g)\varphi(k)\varphi(g)^{-1}$. 
Since every element of the form $gkg^{-1} \in K$, it stands to reason
that $\varphi(g)\varphi(k)\varphi(g)^{-1} \in \varphi(K)$. 
So $\varphi(K)$ is a normal subgroup of $G/H$.
\end{proof}


\item 
If a subgroup of $G/H$ is cyclic, is its corresponding subgroup of $G
$ cyclic?  What about the converse of this statement?  

\begin{proof}

\end{proof}

\end{enumerate}

\end{enumerate}


\bigskip
\noindent
\textbf{Challenge problems.}
Challenge problems are not required for submission, but bonus points 
will be awarded for submitting a partial attempt or a complete 
solution.  

\begin{enumerate}[(C1)]
\item 
Suppose $p$ is prime.  Find all groups $G$ (up to isomorphism) with 
$|G| = p^2$.  

\begin{proof}[Answer]
Let $G_1 = \Z/\abracks{p^2}$, with addition, for some prime $p$; 
clearly, this group has order $p^2$. It is also isomorphic to any
cyclic group of the same order.\\\\
Let $G_2 = \Z/\abracks{p} \times \Z/\abracks{p}$, with 
component-wise addition,
for some prime $p$. This group has order $p^2$ because that's the
number of ways to choose 2 elements from a list of $p$ elements 
with replacement.
\end{proof}

\end{enumerate}
\noindent\makebox[\linewidth]{\rule{\paperwidth}{0.4pt}}
	
\end{document}
