\documentclass{article}
\usepackage{amsmath}
\usepackage{amssymb}
\usepackage{bm}
\usepackage{amsthm}
\usepackage{enumerate}
\usepackage{graphicx}
\usepackage{psfrag}
\usepackage{color}
\usepackage{url}
\usepackage{listings}
\usepackage{xcolor}
\usepackage{tikz}
\usepackage{soul}
\usetikzlibrary{positioning}
\tikzset{main node/.style={circle,fill=gray!20,draw,minimum size=.5cm,inner sep=0pt},}

% In line code stuff%
\definecolor{codegreen}{rgb}{0,0.5,0}
\definecolor{codewhite}{rgb}{1,1,1}
\definecolor{codegray}{rgb}{0.5,0.5,0.5}
\definecolor{codepurple}{rgb}{0.58,0,0.82}
\definecolor{codeblack}{rgb}{0,0,0}
\definecolor{codeorange}{rgb}{0.8,0.4,0}

\lstdefinestyle{mystyle}{
    backgroundcolor=\color{codewhite},   
    commentstyle=\color{codegray},
    keywordstyle=\color{codegreen},
    numberstyle=\color{codegray},
    stringstyle=\color{codeorange},
    basicstyle=\ttfamily ,
    breakatwhitespace=false,         
    breaklines=true,                 
    captionpos=b,                    
    keepspaces=true,                 
    numbers=left,                    
    numbersep=5pt,                  
    showspaces=false,                
    showstringspaces=false,
    showtabs=false,                  
    tabsize=4
}
\lstset{style=mystyle}
\setlength{\hoffset}{-1in}
\addtolength{\textwidth}{1.5in}
\setlength{\voffset}{-1in}
\addtolength{\textheight}{1.5in}

% Custom commands%
\newcommand{\be}{\begin{enumerate}}
\newcommand{\ee}{\end{enumerate}}
\newcommand{\BigO}[1]{\ensuremath\mathcal{O}\left(#1\right)}
\newcommand{\il}[1]{\lstinline!#1!}
\newcommand{\norm}[1]{\left|\left|#1\right|\right|}
\newcommand{\abs}[1]{\left|#1\right|}
\newcommand{\parens}[1]{\left(#1\right)}
\newcommand{\bracks}[1]{\left\{#1\right\}}
\newcommand{\sqbracks}[1]{\left[#1\right]}
\newcommand{\vep}{\varepsilon}
\newcommand{\ceiling}[1]{\left\lceil#1\right\rceil}
\newcommand{\R}{\mathbb{R}}
\newcommand{\N}{\mathbb{N}}
\newcommand{\Z}{\mathbb{Z}}
\newcommand{\F}{\mathbb{F}}
\newcommand{\Q}{\mathbb{Q}}
\newcommand{\A}{\mathcal{A}}
\newcommand{\distrib}[2]{\text{#1}\left(#2\right)}
\newcommand{\dd}[1]{\frac{d}{d#1}}
\newcommand{\abracks}[1]{\left< #1\right>}

\begin{document}
	\begin{center}
		\textbf{Fall 2020, Math 620:\ Homework 5} \\
		\textbf{Due:\ Thursday, October 1st, 2020} \\
		\textbf{Joseph Diaz: 819947915}\\
		\underline{\textbf{Partners in Crime}}:\\
		


\end{center}
\bigskip
\noindent
\textbf{Homework problems.}
You must submit \emph{all} homework problems in order to receive full 
credit.  
\noindent\makebox[\linewidth]{\rule{\paperwidth}{0.4pt}}
\begin{enumerate}[(H1)]
\item 
Prove that if $H \subseteq G$ is a subgroup and $[G:H] = 2$, then $H$ 
is normal.  

\begin{proof}

\end{proof}


\item 
Suppose $G$ is a group.  Given $a \in G$, define $f_a:G \to G$ by 
$f_a(x) = axa^{-1}$.  
    \begin{enumerate}[(a)]
    \item 
    Prove $f_a$ is an automorphism (these are known as \emph{inner     
    automorphisms}).  

    \begin{proof}
    To prove this, we'll show that $f_a$ is a bijective 
    homomorphism from $G$ to itself.\\\\
    \textbf{Homomorphism property}:\\
    First, let $x,y \in G$, then 
    \begin{align*}
    f_a(xy) &= a(xy)a^{-1} \\
    &= (ax)(ya^{-1}) \\
    &= (ax)(a^{-1}a)(ya^{-1}) \\
    &= (axa^{-1})(aya^{-1}) \\
    &= f_a(x)f_a(y)
    \end{align*}
    which means that $f_a$ has the homomorphism property.\\\\
    \textbf{Injectivity}:\\
    Let $x,y \in G$, such that $f_a(x) = f_a(y)$, now 
    \begin{align*}
    f_a(x) &= f_a(y) \\
    axa^{-1} &= aya^{-1} \\
    a(xa^{-1}) &= a(ya^{-1}) \\
    (a^{-1}a)(xa^{-1}) &= (a^{-1}a)(ya^{-1}) \\
    (xa^{-1}) &= (ya^{-1}) \\
    (xa^{-1})a &= (ya^{-1})a \\
    x(a^{-1}a) &= y(a^{-1}a) \\
    x &= y
    \end{align*}
    So, $f_a$ is injective.\\\\
    \textbf{Surjectivity}:\\
    Let $y \in G$, such that for some $x \in G,\ f_a(x) = y$, then
    \begin{align*}
    f_a(x) &= y \\
    axa^{-1} &= y \\
    \Longrightarrow x &= a^{-1}ya 
    \end{align*}
    $a^{-1}, a, y \in G$, so $x \in G$ and $f_a$ is surjective.\\\\
    $f_a$ satisfies the definition of an isomorphism from $G$ to 
    itself; so $f_a$ is an automorpism.
    \end{proof}


    \item 
    Let $G' = \{f_a : a \in G\} \subset \text{Aut}(G)$.  Prove $G'$ 
    is a 
    normal subgroup of $\text{Aut}(G)$.  

    \begin{proof}

    \end{proof}


    \item 
    Let $\varphi:G \to G'$ denote the map $a \mapsto f_a$.      
    Characterize the elements of $G$ in $\ker(\varphi)$.  

    \begin{proof}

    \end{proof}


    \item 
    Characterize which groups $G$ have a unique inner automorphism.  

    \begin{proof}

    \end{proof}
    \end{enumerate}


\item 
Determine whether each of the following statements is true or false.  
Prove your assertions.  
    \begin{enumerate}[(a)]

    \item 
    If $G$ is a group and $H, K \lhd G$ with $K \subset H$, then $G/H     
    \times H/K \cong G/K$.  

    \begin{proof}

    \end{proof}


    \item 
    If $G, G'$ are groups and $H \lhd G$, $H' \lhd G'$, then $(G 
    \times G') / (H \times H') \cong (G/H) \times (G'/H')$.  

    \begin{proof}

    \end{proof}




    \end{enumerate}

\end{enumerate}

\bigskip
\noindent
\textbf{Challenge problems.}
Challenge problems are not required for submission, but bonus points 
will be awarded for submitting a partial attempt or a complete 
solution.  

\begin{enumerate}[(C1)]
\item 
Suppose $G$ is a group and $H, K \lhd G$ with $HK = G$.  Determine under what condition(s) involving $H$ and $K$ we have $G \cong G/H \times G/K$.  
\end{enumerate}
\noindent\makebox[\linewidth]{\rule{\paperwidth}{0.4pt}}
	
\end{document}
