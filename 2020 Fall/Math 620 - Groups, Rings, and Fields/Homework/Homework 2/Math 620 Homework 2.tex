\documentclass{article}
\usepackage{amsmath}
\usepackage{amssymb}
\usepackage{bm}
\usepackage{amsthm}
\usepackage{enumerate}
\usepackage{graphicx}
\usepackage{psfrag}
\usepackage{color}
\usepackage{url}
\usepackage{listings}
\usepackage{xcolor}
\usepackage{tikz}
\usepackage{soul}
\usetikzlibrary{positioning}
\tikzset{main node/.style={circle,fill=gray!20,draw,minimum size=.5cm,inner sep=0pt},}

% In line code stuff%
\definecolor{codegreen}{rgb}{0,0.5,0}
\definecolor{codewhite}{rgb}{1,1,1}
\definecolor{codegray}{rgb}{0.5,0.5,0.5}
\definecolor{codepurple}{rgb}{0.58,0,0.82}
\definecolor{codeblack}{rgb}{0,0,0}
\definecolor{codeorange}{rgb}{0.8,0.4,0}

\lstdefinestyle{mystyle}{
    backgroundcolor=\color{codewhite},   
    commentstyle=\color{codegray},
    keywordstyle=\color{codegreen},
    numberstyle=\color{codegray},
    stringstyle=\color{codeorange},
    basicstyle=\ttfamily ,
    breakatwhitespace=false,         
    breaklines=true,                 
    captionpos=b,                    
    keepspaces=true,                 
    numbers=left,                    
    numbersep=5pt,                  
    showspaces=false,                
    showstringspaces=false,
    showtabs=false,                  
    tabsize=4
}
\lstset{style=mystyle}
\setlength{\hoffset}{-1in}
\addtolength{\textwidth}{1.5in}
\setlength{\voffset}{-1in}
\addtolength{\textheight}{1.5in}

% Custom commands%
\newcommand{\be}{\begin{enumerate}}
\newcommand{\ee}{\end{enumerate}}
\newcommand{\BigO}[1]{\ensuremath\mathcal{O}\left(#1\right)}
\newcommand{\il}[1]{\lstinline!#1!}
\newcommand{\norm}[1]{\left|\left|#1\right|\right|}
\newcommand{\abs}[1]{\left|#1\right|}
\newcommand{\parens}[1]{\left(#1\right)}
\newcommand{\bracks}[1]{\left\{#1\right\}}
\newcommand{\sqbracks}[1]{\left[#1\right]}
\newcommand{\vep}{\varepsilon}
\newcommand{\ceiling}[1]{\left\lceil#1\right\rceil}
\newcommand{\R}{\mathbb{R}}
\newcommand{\N}{\mathbb{N}}
\newcommand{\Z}{\mathbb{Z}}
\newcommand{\F}{\mathbb{F}}
\newcommand{\Q}{\mathbb{Q}}
\newcommand{\A}{\mathcal{A}}
\newcommand{\distrib}[2]{\text{#1}\left(#2\right)}
\newcommand{\dd}[1]{\frac{d}{d#1}}
\newcommand{\abracks}[1]{\left< #1\right>}

\begin{document}
	\begin{center}
		\textbf{Fall 2020, Math 620:\ Homework 2} \\
		\textbf{Due:\ Wednesday, September 9th, 2020} \\
		\textbf{Joseph Diaz: 819947915}\\
		\st{Collaborators} \textbf{Conspirators}: 
		Almira Decena, Anthony 
		Milazzo,\\ Jake
		Samuelson, Richard Castro, Heidi Allen

	\end{center}
\bigskip
\noindent
\textbf{Homework problems.}
You must submit \emph{all} homework problems in order to receive full 
credit.  
\noindent\makebox[\linewidth]{\rule{\paperwidth}{0.4pt}}


\begin{enumerate}[(H1)]
\item 
Identify a subgroup of $GL_2(\R)$ isomorphic to $D_4$.  Identify a 
subgroup isomorphic to $\Z_6$.  

\begin{proof}
The following subgroups of $GL_2(\R)$ have the desired properties.
\be 
    \item[$Q \cong D_4$:] Let $a = \displaystyle
    \sqbracks{\begin{matrix}
    \cos\parens{\frac{\pi}{2}} & -\sin\parens{\frac{\pi}{2}} \\
    \sin\parens{\frac{\pi}{2}} & \cos\parens{\frac{\pi}{2}}
    \end{matrix}} = \sqbracks{\begin{matrix}
    0 & -1 \\ 1 & 0 
    \end{matrix}}$
    and $b = 
    \sqbracks{\begin{matrix}
    0 & 1 \\ 1 & 0 
    \end{matrix}}$, $r$ and $f$ are both elements of $GL_2(\R)$, 
    because $\det a \neq 0$ and $\det b \neq 0$ (they're invertible).
    Indeed, under matrix multiplication:
    $$a\cdot a^3 = I_2,\qquad b^2 = I_2$$ 
    which implies that $a^2\cdot a^2 = I_2$, since matrix 
    multiplication is associative. Let $A \in GL_2(\R)$, then $a\cdot
    A$ is a $\frac{\pi}{2}$ radians counterclockwise rotation of
    the column vectors of $A$ and $b\cdot A$ is the swapping of the
    column vectors of $A$. Finally, let
    $$Q = \bracks{I_2, a, a^2, a^3, b, ab, a^2b, 
    a^3b}$$
    So let $\varphi: Q \to D_4$ be defined like so:
    \begin{align*}
    I_2 \mapsto e &\qquad b \mapsto f \\
    a \mapsto r &\qquad ab \mapsto rf \\
    a^2 \mapsto r^2 &\qquad a^2b \mapsto r^2f \\
    a^3 \mapsto r^3 &\qquad a^3b \mapsto r^3f 
    \end{align*}         
    So $Q \subset GL_2(\R)$ is isomorphic to $D_4$.
    
    
    \item[$P \cong \Z_6$:] Let $a = \sqbracks{\begin{matrix}
    \cos\parens{\frac{\pi}{3}} & -\sin\parens{\frac{\pi}{3}} \\
    \sin\parens{\frac{\pi}{3}} & \cos\parens{\frac{\pi}{3}}
    \end{matrix}} = \sqbracks{\begin{matrix}
    \frac{1}{2} & -\frac{\sqrt{3}}{2} \\
    \frac{\sqrt{3}}{2} & \frac{1}{2}
    \end{matrix}
    } \in GL_2(\R)$. Now let $G$ be a group that is generated by $a$
    under matrix multiplication,
    i.e.
    $$G = \abracks{a} = \bracks{a^k\big| k \in \Z}$$
    This creates a cyclic group, because $a^6 = I_2$, and as every 
    element of $G$ is simply a $\frac{\pi}{3}$ radians 
    counterclockwise rotation
    of the column vectors of another matrix, their products are 
    commutative. So let $\varphi: G \to \Z_6$, where
    $$a_k \mapsto \sqbracks{k}_6$$
    So $G \subset GL_2(\R)$ is isomorphic to $\Z_6$.
\ee
\end{proof}


\item 
It turns out $GL_2(\Z_2)$ is isomorphic to either $S_3$ or $\Z_6$.  Determine which.  

\begin{proof}
The elements of $GL_2(\Z_2)$ are
$$\bracks{\sqbracks{\begin{matrix}
1 & 0 \\
0 & 1
\end{matrix}},
\sqbracks{\begin{matrix}
0 & 1 \\
1 & 0
\end{matrix}},
\sqbracks{\begin{matrix}
1 & 0 \\
1 & 1
\end{matrix}},
\sqbracks{\begin{matrix}
0 & 1 \\
1 & 1
\end{matrix}},
\sqbracks{\begin{matrix}
1 & 1 \\
0 & 1
\end{matrix}},
\sqbracks{\begin{matrix}
1 & 1 \\
1 & 0
\end{matrix}}
}$$
With normal matrix multiplication, $GL_2(\Z_2)$ is isomorphic to $S_3$. Let $f: GL_2(\Z_2) \to S_3$, with:
\begin{align*}
\sqbracks{\begin{matrix}
1 & 0 \\
0 & 1
\end{matrix}} &\mapsto (1)(2)(3) &
\sqbracks{\begin{matrix}
0 & 1 \\
1 & 0
\end{matrix}} &\mapsto (1\ 2) \\
\sqbracks{\begin{matrix}
1 & 0 \\
1 & 1
\end{matrix}} &\mapsto (2\ 3) &
\sqbracks{\begin{matrix}
0 & 1 \\
1 & 1
\end{matrix}} &\mapsto (1\ 2\ 3) \\
\sqbracks{\begin{matrix}
1 & 1 \\
0 & 1
\end{matrix}} &\mapsto (1\ 3) &
\sqbracks{\begin{matrix}
1 & 1 \\
1 & 0
\end{matrix}} &\mapsto (1\ 2\ 3)^2
\end{align*}
$S_3$ elements are expressed using cycle notation. The order of each 
element is preserved by the mapping, and so is composition/
multiplication between corresponding elements. So $GL_2(\Z_2)$ is
isomorphic to $S_3$.
\end{proof}


\item 
Identify an element of $S_9$ of order $20$.
\begin{proof}
The elements $S_9$ are permutations on $\bracks{1,2,3,4,5,6,7,8,9}$;
so take the composition of 2 permutations in cycle notation:
$$(1\ 3\ 5\ 7\ 9)(2\ 4\ 6\ 8) = 
\parens{\begin{matrix}
 1 & 2 & 3 & 4 & 5 & 6 & 7 & 8 & 9 \\
 9 & 8 & 1 & 2 & 3 & 4 & 5 & 6 & 7
\end{matrix}}$$
This permutation has order 20, because it's the least common 
multiple of the order of each sub-cycle; which are 5 and 4, respectively.
\end{proof}


\item 
Let $B_n \subset S_n$ denote the set of odd permutations in $S_n$.  Prove that $|B_n| = |A_n| = n!/2$.  Hint:\ can you find a bijection $f:A_n \to B_n$?  

\begin{proof}
Let $\tau$ be an even permutation in $S_n$ (which also means that 
$\tau \in A_n$), then let 
$$\parens{i\ j},\ 1 \leq i, j \leq n, i\neq 
j$$ 
be
a 2-cycle permutation in $S_n$ (it's also odd, so $\parens{i\ j} \in 
B_n$). Now let 
$$\tau' = \parens{i\ j}\tau$$
The composition of the an
even and odd permutation is an odd permutation, so $\tau' \in B_n$.
This means that for any even permutation, we have a composition that 
can map it into the set of odd permutations. The same is true the 
other 
way around; indeed if we had defined $\tau$ to instead be an odd 
permutation, then $\tau' = \parens{i\ j}\tau$ would be an even 
permutation because the composition of 2 odd permutations is even.
Any
2-cycle is it's own inverse, so this effectively creates a bijection
between $A_n$ and $B_n$; because for any 2-cycle we can map 
elements of one to elements of 
the other and back again, which ultimately implies $\abs{A_n} = 
\abs{B_n}$.\\\\
As permutations can be even or odd but not both, this means that 
$A_n$ and $B_n$ are disjoint sets in $S_n$ and that they are jointly
exhaustive, i.e. $S_n = A_n \cup B_n$. The number of permutations on 
a set of $n$ distinct elements is $n!$, so $\abs{S_n} = n!$. Taking
this all together, we have:
\begin{align*}
\abs{S_n} &= \abs{A_n} + \abs{B_n} \\
n! &= \abs{A_n} + \abs{A_n} \\
n! &= 2\abs{A_n} \\
\implies \abs{A_n} &= \frac{n!}{2}
\end{align*}
So, $\abs{A_n} = \abs{B_n} = \frac{n!}{2}$, as desired.
\end{proof}


\item 
Determine whether each of the following statements is true or false.  Prove your assertions.  

\begin{enumerate}
\item 
The group $(\Q, +)$ is cyclic.  

\begin{proof}[Answer]
No, $\parens{\Q, +}$ is not a cyclic group, because there is no 
generator for which all $q \in \Q$ could generated. For example, take
$\abracks{\frac{a}{b}},\ a,b \in \Z, b \neq 0$. 
Under the given group operation
$\frac{a}{2b}$ could not be generated, and using it as the generator 
instead doesn't solve the problem either, as $\frac{a}{4b}$ could 
also not be generated. Indeed, $\Q$ cannot be finitely generated 
because for any countable list of elements, this problem would 
persist. So $\parens{\Q, +}$ cannot be cyclic.
\end{proof}


\item 
If $G$ is a group and every proper subgroup of $G$ is cyclic, then 
$G$ is cyclic.  

\begin{proof}[Answer]
This is false, in general. For example, consider 
$$G = \Z_2\times\Z_2 = \bracks{\parens{0,0},\parens{0,1},
\parens{1,0},\parens{1,1}}$$
with component-wise $+_2$. The proper subgroups of $G$ are:
$$\bracks{\parens{0,0},\parens{0,1}},\ \bracks{\parens{0,0},
\parens{1,0}},\ \bracks{\parens{0,0}, \parens{1,1}},
\bracks{\parens{0,0}}$$
which are all cyclic groups (the first 3 are isomorphic to $\Z_2$ and 
the last is isomorphic to the trivial cyclic group). $G$, 
however, is not cyclic. 
\end{proof}


\item 
For any groups $G_1$ and $G_2$, we have $\text{Aut}(G_1 \times G_2) 
\cong \text{Aut}(G_1) \times \text{Aut}(G_2)$.  

\begin{proof}[Answer]
This, in general, is not true. Consider the case that
$G_1 = G_2 = \Z_2$; then $\text{Aut}(\Z_2) =
\bracks{e}$, where $e$ represents the identity automorphism; then
$\text{Aut}(\Z_2) \times \text{Aut}(\Z_2) = \bracks{\parens{e,e}}$.
Now consider 
$$\Z_2 \times \Z_2 = \bracks{\parens{0,0},\parens{0,1},\parens{1,0},
\parens{1,1}}$$ 
but $\text{Aut}(\Z_2\times\Z_2) = \bracks{e, \varphi}$, where $
\varphi$ is defined as:
\begin{align*}
\varphi: \Z_2\times\Z_2 &\to \Z_2\times\Z_2 \\
\parens{0,0} \mapsto \parens{0,0}\quad &\quad 
\parens{1,1} \mapsto \parens{1,1} \\
\parens{1,0} \mapsto \parens{0,1}\quad &\quad 
\parens{0,1} \mapsto \parens{1,0} \\
\end{align*}
Notice that $\varphi^2 = e$, and also $\abs{\text{Aut}(\Z_2) \times \text{Aut}(\Z_2)} = 1$ and 
$\abs{\text{Aut}(\Z_2\times\Z_2)} = 2$, which clearly means 
$\text{Aut}(G_1) \times \text{Aut}(G_2)$ and 
$\text{Aut}(G_1\times G_2)$ are not isomorphic in general.
\end{proof}


\end{enumerate}

\end{enumerate}



\bigskip
\noindent
\textbf{Challenge problems.}
Challenge problems are not required for submission, but bonus points will be awarded for submitting a partial attempt or a complete solution.  

\begin{enumerate}[(C1)]
\item 
A \emph{graph} $H$ is a collection $E$ of 2-element subsets of $\{1, \ldots, n\}$ (called \emph{edges}).  An \emph{automorphism} of a graph is a permutation $\sigma$ of the integers $1, \ldots, n$ such that $\{a,b\} \in E$ if and only if $\{\sigma(a), \sigma(b)\} \in E$.  The set $\text{Aut}(H)$ of automorphisms of a graph $H$ is a group under composition (you are not required to prove this).  For example, if $H$ is the 4-cycle graph, with edges $\{1,2\}$, $\{2,3\}$, $\{3,4\}$, and $\{4,1\}$, then $\text{Aut}(H) \cong D_4$.  

Identify a graph $H$ whose automorphism group $\text{Aut}(H)$ is isomorphic to $(\Z_5, +)$.  

\end{enumerate}
\noindent\makebox[\linewidth]{\rule{\paperwidth}{0.4pt}}
	
\end{document}
