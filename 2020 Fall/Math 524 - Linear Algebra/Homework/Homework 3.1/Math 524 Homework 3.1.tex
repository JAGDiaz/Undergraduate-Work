\documentclass{article}
\usepackage{amsmath}
\usepackage{amssymb}
\usepackage{bm}
\usepackage{amsthm}
\usepackage{enumerate}
\usepackage{graphicx}
\usepackage{psfrag}
\usepackage{color}
\usepackage{url}
\usepackage{listings}
\usepackage{xcolor}
\usepackage{tikz}
\usetikzlibrary{positioning}
\tikzset{main node/.style={circle,fill=gray!20,draw,minimum size=.5cm,inner sep=0pt},}

\definecolor{codegreen}{rgb}{0,0.5,0}
\definecolor{codewhite}{rgb}{1,1,1}
\definecolor{codegray}{rgb}{0.5,0.5,0.5}
\definecolor{codepurple}{rgb}{0.58,0,0.82}
\definecolor{codeblack}{rgb}{0,0,0}
\definecolor{codeorange}{rgb}{0.8,0.4,0}

\lstdefinestyle{mystyle}{
    backgroundcolor=\color{codewhite},   
    commentstyle=\color{codegray},
    keywordstyle=\color{codegreen},
    numberstyle=\color{codegray},
    stringstyle=\color{codeorange},
    basicstyle=\ttfamily ,
    breakatwhitespace=false,         
    breaklines=true,                 
    captionpos=b,                    
    keepspaces=true,                 
    numbers=left,                    
    numbersep=5pt,                  
    showspaces=false,                
    showstringspaces=false,
    showtabs=false,                  
    tabsize=4
}
\lstset{style=mystyle}


\setlength{\hoffset}{-1in}
\addtolength{\textwidth}{1.5in}
\setlength{\voffset}{-1in}
\addtolength{\textheight}{1.5in}
\newcommand{\be}{\begin{enumerate}}
\newcommand{\ee}{\end{enumerate}}
\newcommand{\BigO}[1]{\ensuremath\mathcal{O}\left(#1\right)}
\newcommand{\il}[1]{\lstinline!#1!}
\newcommand{\gnorm}[1]{\left|\left|#1\right|\right|}
\newcommand{\abs}[1]{\left|#1\right|}
\newcommand{\parens}[1]{\left(#1\right)}
\newcommand{\bracks}[1]{\left\{#1\right\}}
\newcommand{\sqbracks}[1]{\left[#1\right]}
\newcommand{\vep}{\varepsilon}
\newcommand{\ceiling}[1]{\left\lceil#1\right\rceil}
\newcommand{\R}{\mathbb{R}}
\newcommand{\N}{\mathbb{N}}
\newcommand{\Z}{\mathbb{Z}}
\newcommand{\F}{\mathbb{F}}
\newcommand{\LT}{\mathcal{L}}
\newcommand{\poly}{\mathcal{P}}
\newcommand{\distrib}[2]{\text{#1}\left(#2\right)}
\newcommand{\dd}[1]{\frac{d}{d#1}}
\newcommand{\abracks}[1]{\left< #1\right>}

\begin{document}
	\begin{center}
		\textbf{Fall 2020, Math 524:\ Homework 3.1} \\
		\textbf{Due:\ Monday, September 28th, 2020} \\
		\textbf{Joseph Diaz: 819947915}
	\end{center}
\noindent\makebox[\linewidth]{\rule{\paperwidth}{0.4pt}}

\subsection*{3A}
\be
    \item[4.] Suppose $T \in \LT\parens{V,\ W}$ and $v_1, 
    \cdots, v_m$ is a list of vectors in $V$ such that $Tv_1,\cdots,
    Tv_m$ is a linearly independent list of vectors in $W$. Show that
    $v_1, \cdots, v_m$ is a linearly independent list of vectors.
    \begin{proof}
    Suppose there exists $a_i \in \F,\ 1 \leq i \leq m$ such that
    $$a_1v_1 + a_2v_2 + \cdots + a_mv_m = 0$$
    Now we apply the linear map $T$ to the equation:
    \begin{align*}
    T\parens{a_1v_1 + a_2v_2 + \cdots + a_mv_m} &= T(0) \\
    T\parens{a_1v_1} + \cdots + T\parens{a_mv_m} &= 0 \\
    a_1Tv_1 + \cdots + a_mTv_m &= 0
    \end{align*}
    We know that $Tv_1,\cdots, Tv_m$ is a linearly independent 
    list of vectors, so this implies that $a_1 = a_2 = \cdots
    = a_m = 0$ is the only solution to the equation. Which means
    that $v_1,\cdots, v_m$ is also a 
    linearly independent list of vectors.
    \end{proof}
    
    \item[14.] Suppose $V$ is finite-dimensional with 
    $\dim V \geq 2$. Show that there exists $S,T \in \LT\parens{V,V}$
    such that $ST \neq TS$.
    \begin{proof}
    Let $V = \R^3$, and define $S,T \in \LT\parens{\R^3, \R^3}$ such
    that
    $$\forall \parens{x,y,z} \in \R^3, T\parens{x,y,z} = 
    \parens{y,z,x},\ S\parens{x,y,z} = \parens{x,z,y}$$
    i.e. $T$ cycles the coordinates once, and $S$ swaps the last
    2 coordinates. Then we have
    \begin{gather*}
    ST(x,y,z) = S(y,z,x) = (y,x,z) \\
    TS(x,y,z) = T(x,z,y) = (z,y,x) \\\\
    (y,x,z) \neq (z,y,x) \implies ST\neq TS    
    \end{gather*}
    which is what we wanted to show. (p.s. I was inspired to 
    formulate these transformations by the
    permutation groups from my class on Group theory. Cyclic 
    permutations do not generally commute, and that led me to what 
    is above.)
    \end{proof}
    
\ee

\subsection*{3B}
\be
    \item[5.] Give an example of a linear map $T: \R^4 \to \R^4$ such 
    that
    $$\text{range }T= \text{null }T$$
    \begin{proof}[Answer]
    Let $T \in \LT\parens{\R^4, \R^4}$ be a linear map defined by:
    \begin{align*}
    T(1,0,0,0) &= (0,0,1,0)\\
    T(0,1,0,0) &= (0,0,0,1)\\
    T(0,0,1,0) &= (0,0,0,0)\\
    T(0,0,0,1) &= (0,0,0,0)
    \end{align*}
    The equivalent matrix of the transformation is
    $$\mathcal{M}(T) = \parens{\begin{array}{cccc}
    0 & 0 & 0 & 0\\
    0 & 0 & 0 & 0\\
    1 & 0 & 0 & 0\\
    0 & 1 & 0 & 0
    \end{array}}$$
    This gives us $\text{range }T = \text{span}\parens{
    (0,0,1,0),(0,0,0,1)} = \text{null }T$, as desired.
    \end{proof}
    
    \item[6.] Prove that there does not exist a linear map 
    $T: \R^5 \to \R^5$ such that
    $$\text{range }T= \text{null }T$$
    \begin{proof}[Proof by Contradiction]
    Suppose that there exists a linear map $T \in \LT\parens{\R^5,
    \R^5}$ that satisfies
    $$\text{range }T= \text{null }T$$
    By the Fundamental Theorem of Linear Maps, we must have:
    $$\dim\R^5 = \dim\text{null }T + \dim\text{range }T$$
    So by our assumption:
    \begin{align*}
    \dim\R^5 &= \dim\text{null }T + \dim\text{range }T \\
    5 &= \dim\text{range }T + \dim\text{range }T \\
    5 &= 2\dim\text{range }T\\
    \frac{5}{2} &= \dim\text{range }T
    \end{align*}
    Clearly, this cannot be the case because
    the dimension of vector spaces must be a natural number. So
    our assumption that such a linear map exists must be wrong,
    and there exists no map in $\LT\parens{\R^5,
    \R^5}$ that can satisfy the condition above. 
    \end{proof}
    
    \item[9.] Suppose $T \in \LT\parens{V,\ W}$ is injective and 
    $v_1, \cdots, v_n$ is linearly independent in $V$. Show that
    $Tv_1,\cdots, Tv_n$ is linearly independent in 
    $W$.
    \begin{proof}
    Suppose there exists $a_i \in \F,\ 1 \leq i \leq m$ such that
    $$a_1Tv_1 + a_2Tv_2 + \cdots + a_nTv_n = 0$$
    $T$ is a linear map, so the above is equivalent to
    $$T\parens{a_1v_1 + a_2v_2 + \cdots + a_nv_n} = 0$$
    and as $T$ is injective, this implies
    $$a_1v_1 + a_2v_2 + \cdots + a_nv_n = 0$$
    By assumption, $v_1, \cdots, v_n$ is a list of linearly
    independent vectors, so the only solution to the above equation
    is $a_1 = a_2 = \cdots = a_n = 0$. The same must be true of 
    the equation at the top, and so $Tv_1,\cdots, Tv_n$ is a 
    linearly independent list of vectors in $W$. 
    \end{proof}
\ee

\subsection*{3C}
\be
    \item[2.] Suppose $D \in \LT\parens{\poly_3(\R), \poly_2(\R)}$ is 
    the differentiation map defined by $Dp = p'$. Find a basis for
    $\poly_3(\R)$ and a basis for $\poly_2(\R)$ such that the matrix
    of $D$ with respect these bases is
    $$\parens{\begin{array}{cccc}
    1 & 0 & 0 & 0\\
    0 & 1 & 0 & 0\\
    0 & 0 & 1 & 0
    \end{array}}$$
    \begin{proof}[Answer]
    For $\poly_3\parens{\R}$ the standard basis $\bracks{1, z, 
    z^2, z^3}$ will suffice, to find the basis for 
    $\poly_2\parens{\R}$ that satisfies the specified matrix, 
    we'll consider $D$
    operating on $p(z)=a + bz + cz^2 + dz^3 \in \poly_3\parens{\R}$:
    $$Dp = \parens{\begin{array}{cccc}
    1 & 0 & 0 & 0\\
    0 & 1 & 0 & 0\\
    0 & 0 & 1 & 0\\
    \end{array}}\cdot\parens{\begin{array}{c}
    d\\
    c\\
    b\\
    a
    \end{array}} = 
    \parens{\begin{array}{c}
    d\\
    c\\
    b
    \end{array}} = p'$$
    We want a basis that respects 
    $p'(z) = b + 2cz + 3dz^2$, so we choose the basis 
    $\bracks{1, 2z, 3z^2}$ for $\poly_2\parens{\R}$. 
    \end{proof}
    
    \item[3.] Suppose $V$ and $W$ are finite-dimensional and
    $T \in \LT\parens{V,W}$. Prove that there exists a basis of $V$
    and a basis of $W$ such that with respect to these bases, all 
    entries of $\mathcal{M}(T)$ are $0$ except that the entries in 
    row $j$, column $j$, equal 1 for $1 \leq j \leq \dim\text{range }
    T$.
    \begin{proof}
    Let $v_1,\cdots,\ v_n$ be a basis of range $T$, then extend this 
    to a basis of $V$ by appending $u_1,\ \cdots,\ u_m$ (which must
    be a basis for the null space of $T$ by the Fundamental 
    Theorem of Linear Maps) to 
    that list. We have now have a basis for $V$ that consists of
    $$\bracks{v_1,\cdots,\ v_n,\ u_1,\ \cdots,\ u_m}$$
    So we have that $\dim\text{range }T = n,\ \dim\text{null }T = m$,
    and $\dim V = n + m$. Now let $v \in V$, then applying $T$ to $v$
    gives
    \begin{align*}
    Tv &= T\parens{a_1v_1 + \cdots + a_nv_n + b_1u_1 + \cdots + 
    b_mu_m} \\
    &= a_1Tv_1 + \cdots + a_nTv_n
    \end{align*}
    for $a_i, b_j \in \F$. Now we extend $Tv_1,\ \cdots,\ Tv_n$ to
    a basis in $W$ by appending $\gamma_1,\ \cdots,\ \gamma_k$. These
    bases for $V$ and $W$ guarantee that $\mathcal{M}(T)$ will only
    have $1$'s in the specified positions, because only vectors in
    the span of the range of $T$ will be mapped into $W$ (by 
    definition). 
    \end{proof}
    
    
\ee

\noindent\makebox[\linewidth]{\rule{\paperwidth}{0.4pt}}
	
\end{document}
