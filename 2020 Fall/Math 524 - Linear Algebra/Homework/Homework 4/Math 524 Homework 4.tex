\documentclass{article}
\usepackage{amsmath}
\usepackage{amssymb}
\usepackage{bm}
\usepackage{amsthm}
\usepackage{enumerate}
\usepackage{graphicx}
\usepackage{psfrag}
\usepackage{color}
\usepackage{url}
\usepackage{listings}
\usepackage{xcolor}
\usepackage{tikz}
\usetikzlibrary{positioning}
\tikzset{main node/.style={circle,fill=gray!20,draw,minimum size=.5cm,inner sep=0pt},}

\definecolor{codegreen}{rgb}{0,0.5,0}
\definecolor{codewhite}{rgb}{1,1,1}
\definecolor{codegray}{rgb}{0.5,0.5,0.5}
\definecolor{codepurple}{rgb}{0.58,0,0.82}
\definecolor{codeblack}{rgb}{0,0,0}
\definecolor{codeorange}{rgb}{0.8,0.4,0}

\lstdefinestyle{mystyle}{
    backgroundcolor=\color{codewhite},   
    commentstyle=\color{codegray},
    keywordstyle=\color{codegreen},
    numberstyle=\color{codegray},
    stringstyle=\color{codeorange},
    basicstyle=\ttfamily ,
    breakatwhitespace=false,         
    breaklines=true,                 
    captionpos=b,                    
    keepspaces=true,                 
    numbers=left,                    
    numbersep=5pt,                  
    showspaces=false,                
    showstringspaces=false,
    showtabs=false,                  
    tabsize=4
}
\lstset{style=mystyle}


\setlength{\hoffset}{-1in}
\addtolength{\textwidth}{1.5in}
\setlength{\voffset}{-1in}
\addtolength{\textheight}{1.5in}
\newcommand{\be}{\begin{enumerate}}
\newcommand{\ee}{\end{enumerate}}
\newcommand{\BigO}[1]{\ensuremath\mathcal{O}\left(#1\right)}
\newcommand{\il}[1]{\lstinline!#1!}
\newcommand{\gnorm}[1]{\left|\left|#1\right|\right|}
\newcommand{\abs}[1]{\left|#1\right|}
\newcommand{\parens}[1]{\left(#1\right)}
\newcommand{\bracks}[1]{\left\{#1\right\}}
\newcommand{\sqbracks}[1]{\left[#1\right]}
\newcommand{\vep}{\varepsilon}
\newcommand{\ceiling}[1]{\left\lceil#1\right\rceil}
\newcommand{\R}{\mathbb{R}}
\newcommand{\N}{\mathbb{N}}
\newcommand{\Z}{\mathbb{Z}}
\newcommand{\F}{\mathbb{F}}
\newcommand{\LT}{\mathcal{L}}
\newcommand{\poly}{\mathcal{P}}
\newcommand{\distrib}[2]{\text{#1}\left(#2\right)}
\newcommand{\dd}[1]{\frac{d}{d#1}}
\newcommand{\abracks}[1]{\left< #1\right>}

\begin{document}
	\begin{center}
		\textbf{Fall 2020, Math 524:\ Homework 4} \\
		\textbf{Due:\ Monday, October 12th, 2020} \\
		\textbf{Joseph Diaz: 819947915}
	\end{center}
\noindent\makebox[\linewidth]{\rule{\paperwidth}{0.4pt}}

\subsection*{4}
\be
    \item[2.] Suppose $m \in \N$. Is the set
    $$\bracks{0}\cup \bracks{p \in \mathcal{P}\parens{\F}\ \big|\ 
    \deg p = m}$$
    a subspace of $\mathcal{P}\parens{\F}$?
    \begin{proof}
    No, this is not a subspace of $\mathcal{P}\parens{\F}$. While 
    this set of polynomials contains a 0 element and does respect
    scalar multiplication with elements of $\F$, it isn't closed 
    under addition. Let $p_1(x)$ and $p_2(x)$ be elements of this
    set such that
    \begin{gather*}
    p_1(x) = a_0 + a_1x + \cdots + a_m x^m, a_i \in \F, a_m \neq 0 \\
    p_2(x) = b_0 + b_1x + \cdots + b_m x^m, b_i \in \F, b_m = -a_m
    \end{gather*}
    The sum of these 2 polynomials, which I'll denote $q(x)$, is 
    given by
    \begin{align*}
    q(x) &= p_1(x) + p_2(x) \\
    &= (a_0 + b_0) + (a_1+b_1)x + \cdots + (a_{m-1} + b_{m-1})x^{m-1}
    + (a_m + b_m)x^m \\
    &= (a_0 + b_0) + (a_1+b_1)x + \cdots + (a_{m-1} + b_{m-1})x^{m-1}
    + (a_m - a_m)x^m \\    
    &= (a_0 + b_0) + (a_1+b_1)x + \cdots + (a_{m-1} + b_{m-1})x^{m-1}
    \end{align*}
    $q$ has only degree $m-1$, so clearly $q$ is not in the set of 
    polynomials. Therefore, it's not closed under addition and 
    it isn't a subspace of $\mathcal{P}\parens{\F}$.
    \end{proof}
    
    \item[3.] Is the set
    $$\bracks{0}\cup \bracks{p \in \mathcal{P}\parens{\F}\ \big|\ 
    \deg p \equiv 0 \mod 2}$$
    a subspace of $\mathcal{P}\parens{\F}$?
    \begin{proof}
    Like problem 2, this set is not a subspace of $\mathcal{P}
    \parens{\F}$ either. Let $p_1(x)$ and $p_2(x)$ be defined 
    identically to those in problem 2, with the caveat that $m \equiv
    0 \mod 2$. The sum of these 2 polynomials, which I'll again
    denote $q(x)$, is 
    \begin{align*}
    q(x) &= p_1(x) + p_2(x) \\
    &= (a_0 + b_0) + (a_1+b_1)x + \cdots + (a_{m-1} + b_{m-1})x^{m-1}
    + (a_m + b_m)x^m \\
    &= (a_0 + b_0) + (a_1+b_1)x + \cdots + (a_{m-1} + b_{m-1})x^{m-1}
    + (a_m - a_m)x^m \\    
    &= (a_0 + b_0) + (a_1+b_1)x + \cdots + (a_{m-1} + b_{m-1})x^{m-1}
    \end{align*}
    $q$ has only degree $m-1$. Since $m$ was even, $m-1$ must be 
    odd; so $q$ is not in the set of polynomials because it has odd 
    degree. Therefore, the set is not closed under addition and 
    it isn't a subspace of $\mathcal{P}\parens{\F}$; just like the 
    set in problem 2.
    \end{proof}
        
\ee


\noindent\makebox[\linewidth]{\rule{\paperwidth}{0.4pt}}
	
\end{document}
