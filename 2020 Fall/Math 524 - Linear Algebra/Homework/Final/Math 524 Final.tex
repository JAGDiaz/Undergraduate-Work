\documentclass{article}
\usepackage{amsmath}
\usepackage{amssymb}
\usepackage{bm}
\usepackage{amsthm}
\usepackage{enumerate}
\usepackage{graphicx}
\usepackage{psfrag}
\usepackage{color}
\usepackage{url}
\usepackage{listings}
\usepackage{xcolor}
\usepackage{tikz}
\usetikzlibrary{positioning}
\tikzset{main node/.style={circle,fill=gray!20,draw,minimum size=.5cm,inner sep=0pt},}

% In line code stuff%
\definecolor{codegreen}{rgb}{0,0.5,0}
\definecolor{codewhite}{rgb}{1,1,1}
\definecolor{codegray}{rgb}{0.5,0.5,0.5}
\definecolor{codepurple}{rgb}{0.58,0,0.82}
\definecolor{codeblack}{rgb}{0,0,0}
\definecolor{codeorange}{rgb}{0.8,0.4,0}

\lstdefinestyle{mystyle}{
    backgroundcolor=\color{codewhite},   
    commentstyle=\color{codegray},
    keywordstyle=\color{codegreen},
    numberstyle=\color{codegray},
    stringstyle=\color{codeorange},
    basicstyle=\ttfamily ,
    breakatwhitespace=false,         
    breaklines=true,                 
    captionpos=b,                    
    keepspaces=true,                 
    numbers=left,                    
    numbersep=5pt,                  
    showspaces=false,                
    showstringspaces=false,
    showtabs=false,                  
    tabsize=4
}
\lstset{style=mystyle}
\setlength{\hoffset}{-1in}
\addtolength{\textwidth}{1.5in}
\setlength{\voffset}{-1in}
\addtolength{\textheight}{1.5in}

% Custom commands%
\newcommand{\be}{\begin{enumerate}}
\newcommand{\ee}{\end{enumerate}}
\newcommand{\BigO}[1]{\ensuremath\mathcal{O}\left(#1\right)}
\newcommand{\il}[1]{\lstinline!#1!}
\newcommand{\norm}[1]{\left|\left|#1\right|\right|}
\newcommand{\abs}[1]{\left|#1\right|}
\newcommand{\parens}[1]{\left(#1\right)}
\newcommand{\bracks}[1]{\left\{#1\right\}}
\newcommand{\sqbracks}[1]{\left[#1\right]}
\newcommand{\vep}{\varepsilon}
\newcommand{\ceiling}[1]{\left\lceil#1\right\rceil}
\newcommand{\R}{\mathbb{R}}
\newcommand{\N}{\mathbb{N}}
\newcommand{\Z}{\mathbb{Z}}
\newcommand{\F}{\mathbb{F}}
\newcommand{\C}{\mathbb{C}}
\newcommand{\A}{\mathcal{A}}
\newcommand{\distrib}[2]{\text{#1}\left(#2\right)}
\newcommand{\dd}[2]{\frac{d#1}{d#2}}
\newcommand{\abracks}[1]{\left< #1\right>}
\newcommand{\nullspace}[1]{\text{null }#1}
\newcommand{\nullp}[1]{\text{null}\parens{#1}}
\newcommand{\LT}[1]{\mathcal{L}\parens{#1}}
\newcommand{\poly}[2]{\mathcal{P}_{#1}\parens{#2}}
\newcommand{\cont}[3]{\mathcal{C}_{#1}^{#2}\parens{#3}}
\newcommand{\range}[1]{\text{range }#1}
\newcommand{\one}{\mathbbm{1}}

\begin{document}
\noindent\makebox[\linewidth]{\rule{\paperwidth}{0.4pt}}
\be[1.] 
    \item In [Notes\#6] we discussed different types of inner products for polynomials; and
    used the inner product 
    $\displaystyle\abracks{f, g} = \int_{-1}^1 f(x)g(x)\ dx$ 
    to produce orthogonal polynomials - the ``Legendre Polynomials.'' The polynomials orthogonal 
    with respect tothe inner product 
    $\displaystyle\abracks{f, g} = \int_{-1}^1 \frac{f(x)g(x)}{\sqrt{1-x^2}}\
    dx$ are known as the ``Chebyshev Polynomials,'' and are an extremely useful basis for high
    degree polynomial approximations. Deriving them invariably turns into a mess of 
    trigonometric ``magic''. 
    \textbf{Instead}, this question deals with the ``Laguerre Polynomials,'' which are orthogonal
    with respect to the inner product 
    $\displaystyle\abracks{f, g} = \int_0^\infty f(x)g(x) e^{-x}\ dx$. 
    Use the fact that $\forall p, q \in \N \cup \{0\},\ \abracks{x^p, x^q} = (p+q)!$ 
    (``p plus q factorial''), to derive the first 4 (order 0, 1, 2, 3) orthonormal 
    Laguerre Polynomials starting from elements of the standard
    polynomial basis $\{1, x, x^2, x^3\}$.
    \begin{proof}[Answer]
    Let 
    \begin{align*}
    L_0(x) &= a_0 \\
    L_1(x) &= a_1 + b_1x \\
    L_2(x) &= a_2 + b_2x + c_2x^2 \\
    L_3(x) &= a_3 + b_3x + c_3x^2 + d_3x^3
    \end{align*}
    with $a_i, b_i, c_i, d_i \in \R$ be the first 4 orthonormal Laguerre Polynomials; to find the coefficients for each, we'll 
    use the fact that 
    $$\forall n,m \in \N \cup \{0\},\ \abracks{L_n, L_m} = \left\{\begin{array}{lcr}
    0 & & n \neq m \\
    1 & & n = m
    \end{array}\right.$$ and the properties of the inner product. So for $L_0$, we have
    $$\abracks{L_0, L_0} = \abracks{a_0, a_0} = a_0a_0\abracks{1,1} = a_0^2 \abracks{x^0, x^0}
    = a_0^2 \cdot 1 = 1$$
    $a_0$ could be either $1$ or $-1$; we'll take $a_0 = 1$, so $L_0(x) = 1$. Now for $L_1$:
    \begin{align*}
    \abracks{L_1, L_1} &= \abracks{a_1 + b_1x, a_1 + b_1x} = \abracks{a_1,a_1} + 
    2\abracks{a_1, b_1x} + \abracks{b_1x, b_1x} \\
    &= a_1^2\abracks{1,1} + 2a_1b_1\abracks{1, x} + b_1^2\abracks{x, x} = 
    a_1^2 + 2a_1b_1 + 2b_1^2 = 1 \\\\
    \abracks{L_0, L_1} &= \abracks{1, a_1 + b_1x} = a_1\abracks{1,1} + b_1\abracks{1,1} = 
    a_1 + b_1 = 0
    \end{align*}
    So we have that $b_1 = -a_1$, with this:
    $$a_1^2 + 2a_1b_1 + 2b_1^2 = a_1^2 - 2a_1^2 + 2(-a_1)^2 = a_1^2 = 1$$
    $a_1$ could be either $1$ or $-1$; we'll take $a_1 = 1$, so $L_1(x) = 1 - x$, by the other
    $L_1$ equation. Now for $L_2$:
    \begin{align*}
    \abracks{L_2, L_2} &= \abracks{a_2 + b_2x + c_2x^2, a_2 + b_2x + c_2x^2} \\ 
    &= \abracks{a_2, a_2} + 2\abracks{a_2, b_2x} + 2\abracks{a_2, c_2x^2} + 
    \abracks{b_2x, b_2x} + 2\abracks{b_2x, c_2x^2} + \abracks{c_2x^2, c_2x^2} \\
    &= a_2^2\abracks{1, 1} + 2a_2b_2\abracks{1, x} + 2a_2c_2\abracks{1, x^2} + 
    b_2^2\abracks{x, x} + 2b_2c_2\abracks{x, x^2} + c_2^2\abracks{x^2, x^2} \\
    &= a_2^2 + 2a_2b_2 + 4a_2c_2 + 2b_2^2 + 12b_2c_2 + 24c_2^2 = 1 \\
    \end{align*}
    \begin{align*}
    \abracks{L_1, L_2} &= \abracks{1 - x, a_2 + b_2x + c_2x^2} = 
    \abracks{1, a_2} + \abracks{1, b_2x} + \abracks{1, c_2x^2} - \abracks{x, a_2} - 
    \abracks{x, b_2x} - \abracks{x, c_2x^2} \\
    &= a_2\abracks{1, 1} + b_2\abracks{1, x} + c_2\abracks{1, x^2} - a_2\abracks{x, 1} - 
    b_2\abracks{x, x} - c_2\abracks{x, x^2} \\
    &= a_2 + b_2 + 2c_2 - a_2 - 
    2b_2 - 6c_2  \\
    &= -b_2 - 4c_2 = 0 \\\\
    \abracks{L_0, L_2} &= \abracks{1, a_2 + b_2x + c_2x^2} = \abracks{1, a_2} + 
    \abracks{1, b_2x} + \abracks{1, c_2x^2} \\
    &= a_2\abracks{1, 1} + b_2\abracks{1, x} + c_2\abracks{1, x^2} \\
    &= a_2 + b_2 + 2c_2 = 0 
    \end{align*}
    Using these last 2 equations, we have that $a_2 = 2c_2,\ b_2 = -4c_2$; plugging these 
    into the larger equation gives:
    \begin{align*}
    a_2^2 + 2a_2b_2 + 4a_2c_2 + 2b_2^2 + 12b_2c_2 + 24c_2^2 &= 
    (2c_2)^2 + 2(2c_2)(-4c_2) + 4(2c_2)c_2 \\
    &\qquad + 2(-4c_2)^2 + 12(-4c_2)c_2 + 24c_2^2 \\
    &= 4c_2^2 - 16c_2^2 + 8c_2^2 + 32c_2^2 - 48c_2^2 + 24c_2^2 \\
    &= 4c_2^2 = 1 \\\\
    \implies c_2^2 &= \frac{1}{4}
    \end{align*}

    $c_2$ could be either $1/2$ or $-1/2$; we'll take $c_2 = 1/2$, this yields $a_2 = 1,\ b_2 =
    -2$; so $L_2(x) = 1 - 2x + \frac{1}{2}x^2$. Finally, for $L_3$:
    \begin{align*}
    \abracks{L_3, L_3} &= \abracks{a_3 + b_3x + c_3x^2 + d_3x^3, a_3 + b_3x + c_3x^2 + d_3x^3}\\
    &= \abracks{a_3, a_3} + \abracks{a_3, b_3x} + \abracks{a_3, c_3x^2} + 
    \abracks{a_3, d_3x^3} \\
    &\quad + \abracks{b_3x, a_3} + \abracks{b_3x, b_3x} + \abracks{b_3x, c_3x^2} 
    + \abracks{b_3x, d_3x^3} \\
    &\quad + \abracks{c_3x^2, a_3} + \abracks{c_3x^2, b_3x} + \abracks{c_3x^2, c_3x^2} 
    + \abracks{c_3x^2, d_3x^3} \\
    &\quad + \abracks{d_3x^3, a_3} + \abracks{d_3x^3, b_3x} + \abracks{d_3x^3, c_3x^2} 
    + \abracks{d_3x^3, d_3x^3} \\
    &= a_3^2\abracks{1,1} + 2a_3b_3\abracks{1,x} + 2a_3c_3\abracks{1,x^2} + 2a_3d_3
    \abracks{1,x^3} \\
    &\quad + 2b_3c_3\abracks{x,x^2} + 2b_3d_3\abracks{x, x^3} + 2c_3d_3\abracks{x^2, x^3} \\
    &\quad + b_3^2\abracks{x,x} + c_3^2\abracks{x^2,x^2} + d_3^2\abracks{x^3,x^3} \\
    &= a_3^2 + 2a_3b_3 + 4a_3c_3 + 12a_3d_3 + 12b_3c_3 + 48b_3d_3 \\
    &\quad + 240c_3d_3 + 2b_3^2 + 24c_3^2 + 720d_3^2 = 1 \\
    \end{align*}
    \begin{align*}
    \abracks{L_2, L_3} &= \abracks{1 - 2x + \frac{1}{2}x^2, a_3 + b_3x + c_3x^2 + d_3x^3} \\
    &= \abracks{1, a_3} + \abracks{1, b_3x} + \abracks{1, c_3x^2} + \abracks{1, d_3x^3} \\
    &\quad - \abracks{2x, a_3} - \abracks{2x, b_3x} - \abracks{2x, c_3x^2} - 
    \abracks{2x, d_3x^3} \\
    &\quad + \abracks{\frac{1}{2}x^2, a_3} + \abracks{\frac{1}{2}x^2, b_3x} 
    + \abracks{\frac{1}{2}x^2, c_3x^2} + \abracks{\frac{1}{2}x^2, d_3x^3} \\
    &= a_3\abracks{1, 1} + b_3\abracks{1, x} + c_3\abracks{1, x^2} + d_3\abracks{1, x^3} \\
    &\quad - 2a_3\abracks{x, 1} - 2b_3\abracks{x, x} - 2c_3\abracks{x, x^2} - 
    2d_3\abracks{x, x^3} \\
    &\quad + \frac{a_3}{2}\abracks{x^2, 1} + \frac{b_3}{2}\abracks{x^2, x} 
    + \frac{c_3}{2}\abracks{x^2, x^2} + \frac{d_3}{2}\abracks{x^2, x^3} \\
    &= a_3 + b_3 + 2c_3 + 6d_3  - 2a_3 - 4b_3 - 12c_3 - 
    48d_3 \\
    &\quad + a_3 + 3b_3 + 12c_3 + 60d_3 \\
    &= 2c_3 + 18d_3 = 0 
    \end{align*}
    \begin{align*}
    \abracks{L_1, L_3} &= \abracks{1-x, a_3 + b_3x + c_3x^2 + d_3x^3} \\
    &= \abracks{1, a_3} + \abracks{1, b_3x} + \abracks{1, c_3x^2} + \abracks{1, d_3x^3} \\
    &\quad - \abracks{x, a_3} - \abracks{x, b_3x} - \abracks{x, c_3x^2} - \abracks{x, d_3x^3} \\
    &= a_3\abracks{1, 1} + b_3\abracks{1, x} + c_3\abracks{1, x^2} + d_3\abracks{1, x^3} \\
    &\quad - a_3\abracks{x, 1} - b_3\abracks{x, x} - c_3\abracks{x, x^2} - d_3\abracks{x, x^3} \\
    &= a_3 + b_3 + 2c_3 + 6d_3 - a_3 - 2b_3 - 6c_3 - 24d_3 \\
    &= -b_3 - 4c_3 - 18d_3 = 0 \\\\
    \abracks{L_0, L_3} &= \abracks{1, a_3 + b_3x + c_3x^2 + d_3x^3} \\
    &= \abracks{1, a_3} + \abracks{1, b_3x} + \abracks{1, c_3x^2} + \abracks{1, d_3x^3} \\
    &= a_3\abracks{1, 1} + b_3\abracks{1, x} + c_3\abracks{1, x^2} + d_3\abracks{1, x^3} \\
    &= a_3 + b_3 + 2c_3 + 6d_3 = 0 \\
    \end{align*}
    With these lst equations, we have that $a_3 = -6d_3,\ b_3 = 18d_3,\ c_3 = -9d_3$, plugging
    these into the larger equation gives:
    \begin{align*}
    &a_3^2 + 2a_3b_3 + 4a_3c_3 + 12a_3d_3 + 12b_3c_3 + 48b_3d_3 + 240c_3d_3 + 2b_3^2 + 
    24c_3^2 + 720d_3^2 & \\
    &= (-6d_3)^2 + 2(-6d_3)(18d_3) + 4(-6d_3)(-9d_3) + 12(-6d_3)d_3 + 
    12(18d_3)(-9d_3) + 48(18d_3)d_3 & \\
    &+ 240(-9d_3)d_3 + 2(18d_3)^2 + 24(-9d_3)^2 + 720d_3^2& \\
    &= 36d_3^2 = 1& \\
    &\implies d_3^2 = \frac{1}{36}&
    \end{align*}
    $d_3$ could be either $1/6$ or $-1/6$; we'll take $d_3 = 1/6$, this yields $a_3 = -1,\ 
    b_3 = 3,\ c_3 = -3/2$; so $L_3(x) = -1 + 3x - \frac{3}{2}x^2 + \frac{1}{6}x^3$. 
    So our 4 orthonormal Laguerre Polynomials are:
    \begin{align*}
    L_0(x) &= 1 \\
    L_1(x) &= 1 - x \\
    L_2(x) &= 1 - 2x + \frac{1}{2}x^2 \\
    L_3(x) &= -1 + 3x - \frac{3}{2}x^2 + \frac{1}{6}x^3
    \end{align*}
    \end{proof}

    \item Let $T \in \LT{\C^7}$ be defined by 
    $$T(z_1, z_2, z_3, z_4, z_5, z_6, z_7) = (4z_1 + z_2 + z_3 + z_4, 4z_2 + z_3 + z_4,
    4z_3 + z_4, 4z_4, 3z_5 + z_6 + z_7, 3z_6 + z_7, 3z_7)$$
    Let $\mathfrak{B}\parens{\C^7} = \bracks{e_1, e_2, e_3, e_4, e_5, e_6, e_7}$ be the standard 
    basis of $\C^7$.
    \be[(a)]
        \item Find $\mathcal{M}\parens{T, \mathfrak{B}\parens{\C^7}}$.
        \begin{proof}[Answer]
        The matrix of the transformation for $T$ with respect to the standard basis is:
        $$\mathcal{M}\parens{T, \mathfrak{B}\parens{\C^7}} = 
        \sqbracks{\begin{matrix}
            4 & 1 & 1 & 1 & 0 & 0 & 0 \\
            0 & 4 & 1 & 1 & 0 & 0 & 0 \\
            0 & 0 & 4 & 1 & 0 & 0 & 0 \\
            0 & 0 & 0 & 4 & 0 & 0 & 0 \\
            0 & 0 & 0 & 0 & 3 & 1 & 1 \\
            0 & 0 & 0 & 0 & 0 & 3 & 1 \\
            0 & 0 & 0 & 0 & 0 & 0 & 3 
        \end{matrix}}$$
        \end{proof}

        \item Find the eigenvalues $\bracks{\lambda_k}_{k = 1,2,\cdots,?}$.
        \begin{proof}[Answer]
        $\mathcal{M}\parens{T, \mathfrak{B}\parens{\C^7}}$ is an upper-triangular matrix,
        so we can just get the eigenvalues right off the main diagonal; they are
        $\lambda_1 = 4$ with multiplicity 4 and $\lambda_2 = 3$ with multiplicity 3. 
        \end{proof}

        \item For each eigenvalue, $\lambda_k$:
        \be[i.]
            \item Find the eigenspace $E\parens{\lambda_k,T}$.
            \begin{proof}[Answer]
            There are only 2 distinct eigenvalues, so we need only find
            $E\parens{4, T} = \nullp{T-4I}$ and $E\parens{3, T} = \nullp{T-3I}$. 
            Starting with $E\parens{4,T}$, for $z \in \C^7$:
            \begin{gather*}
                (T-4I)z = 0 \\
                \implies (z_2 + z_3 + z_4, z_3 + z_4, z_4, 0, -z_5 + z_6 + z_7, -z_6 + z_7, -z_7)
                = (0,0,0,0,0,0,0)
            \end{gather*}
            With back substition, we have that $z_1$ is free, and $z_2 = z_3 = z_4 =z_5 = 
            z_6 = z_7 = 0$, so 
            $$E(4,T) = \nullp{T-4I} = \text{span}(e_1)$$
            Now, with $E\parens{3,T}$:
            \begin{gather*}
                (T-3I)z = 0 \\
                \implies (z_1 + z_2 + z_3 + z_4, z_2 + z_3 + z_4, z_3 + z_4, z_4, 
                z_6 + z_7, z_7, 0)
                = (0,0,0,0,0,0,0)
            \end{gather*}
            again, with back substition, we have that $z_5$ is free, and $z_1 = z_2 = z_3 = z_4= 
            z_6 = z_7 = 0$, so 
            $$E(3,T) = \nullp{T-3I} = \text{span}(e_5)$$
            \end{proof}

            \item Find the generalized eigenspace $G\parens{\lambda_k,T}$.
            \begin{proof}[Answer]
            To find the generalzed eigenspaces for $\lambda_1 = 4,\ \lambda_2 = 3$, we'll find 
            $\parens{T-4I}^7$ and $\parens{T-3I}^7$ by taking powers of 
            $\mathcal{M}\parens{T, \mathfrak{B}\parens{\C^7}} - 4I$ and 
            $\mathcal{M}\parens{T, \mathfrak{B}\parens{\C^7}} - 3I$:
            $$\parens{\mathcal{M}\parens{T, \mathfrak{B}\parens{\C^7}} - 4I}^7 = 
            \sqbracks{\begin{matrix}
                0 & 0 & 0 & 0 & 0 & 0 & 0 \\
                0 & 0 & 0 & 0 & 0 & 0 & 0 \\
                0 & 0 & 0 & 0 & 0 & 0 & 0 \\
                0 & 0 & 0 & 0 & 0 & 0 & 0 \\
                0 & 0 & 0 & 0 & -1 & 7 & -14 \\
                0 & 0 & 0 & 0 & 0 & -1 & 7 \\
                0 & 0 & 0 & 0 & 0 & 0 & -1 
            \end{matrix}}$$
            $$\parens{\mathcal{M}\parens{T, \mathfrak{B}\parens{\C^7}} - 3I}^7 = 
            \sqbracks{\begin{matrix}
                1 & 7 & 28 & 84 & 0 & 0 & 0 \\
                0 & 1 & 7 & 28 & 0 & 0 & 0 \\
                0 & 0 & 1 & 7 & 0 & 0 & 0 \\
                0 & 0 & 0 & 1 & 0 & 0 & 0 \\
                0 & 0 & 0 & 0 & 0 & 0 & 0 \\
                0 & 0 & 0 & 0 & 0 & 0 & 0 \\
                0 & 0 & 0 & 0 & 0 & 0 & 0 
            \end{matrix}}$$
            Hence
            \begin{align*}
            (T-4I)^7z &= (0,0,0,0,-z_5 + 7z_6 - 14z_7, -z_6 + 7 z_7, -z_7) \\
            (T-3I)^7z &= (z_1 + 7z_2 + 28z_3 + 84z_4, z_2 +7z_3 + 28z_4, z_3 + 7z_4, z_4,0,0,0)
            \end{align*}
            Solving $(T-4I)^7z=0$, we see that $z_1,\ z_2,\ z_3$, and $z_4$ are free while
            $z_5= z_6 = z_7 = 0$, and solving $(T-3I)^7z =0$, yields $z_1 = z_2 = z_3 = z_4 = 0$
            while $z_5,\ z_6$, and $z_7$ are free. So the generalized eigenspaces we're 
            interested in are
            $$G\parens{4,T} = \text{span}\parens{e_1,e_2,e_3,e_4},\quad G\parens{3,T} = 
            \text{span}\parens{e_5,e_6,e_7}$$

            \end{proof}
        \ee

        \item What does the Jordan Form of $\mathcal{M}\parens{T, \mathfrak{B}\parens{\C^7}}$
        look like?
        \begin{proof}[Answer]
        Due to the multiplicity of the eigenvalues of $T$, the Jordan Form of 
        $\mathcal{M}\parens{T, \mathfrak{B}\parens{\C^7}}$ would be
        $$\mathcal{M}\parens{\mathcal{M}\parens{T, \mathfrak{B}\parens{\C^7}}, \mathfrak{B}_J}
        = \sqbracks{\begin{matrix}
            4 & 1 & 0 & 0 & 0 & 0 & 0 \\
            0 & 4 & 1 & 0 & 0 & 0 & 0 \\
            0 & 0 & 4 & 1 & 0 & 0 & 0 \\
            0 & 0 & 0 & 4 & 0 & 0 & 0 \\
            0 & 0 & 0 & 0 & 3 & 1 & 0 \\
            0 & 0 & 0 & 0 & 0 & 3 & 1 \\
            0 & 0 & 0 & 0 & 0 & 0 & 3 
        \end{matrix}}$$
        \end{proof}

        \item Identify the Jordan Basis, \textit{i.e.} the basis with respect to which 
        $\mathcal{M}\parens{\mathcal{M}\parens{T, \mathfrak{B}\parens{\C^7}}, \mathfrak{B}_J}$
        is in Jordan Form:
        \be[i.]
            \item Identify the Jordan Chains.
            \begin{proof}[Answer]
            Applying $T-4I$ to each of the basis vectors in $G\parens{4, T}$ gives
            $$\bracks{\sqbracks{\begin{matrix}
            0 \\
            0 \\
            0 \\
            0 \\
            0 \\
            0 \\
            0 
            \end{matrix}}, \sqbracks{\begin{matrix}
            1 \\
            0 \\
            0 \\
            0 \\
            0 \\
            0 \\
            0 
            \end{matrix}}},             
            \bracks{\sqbracks{\begin{matrix}
            1 \\
            0 \\
            0 \\
            0 \\
            0 \\
            0 \\
            0 
            \end{matrix}}, \sqbracks{\begin{matrix}
            0 \\
            1 \\
            0 \\
            0 \\
            0 \\
            0 \\
            0 
            \end{matrix}}},
            \bracks{\sqbracks{\begin{matrix}
            1 \\
            1 \\
            0 \\
            0 \\
            0 \\
            0 \\
            0 
            \end{matrix}}, \sqbracks{\begin{matrix}
            0 \\
            0 \\
            1 \\
            0 \\
            0 \\
            0 \\
            0 
            \end{matrix}}}, 
            \bracks{\sqbracks{\begin{matrix}
            1 \\
            1 \\
            1 \\
            0 \\
            0 \\
            0 \\
            0 
            \end{matrix}}, \sqbracks{\begin{matrix}
            0 \\
            0 \\
            0 \\
            1 \\
            0 \\
            0 \\
            0 
            \end{matrix}}}$$
            and applying $T - 3I$ to the basis vectors of $G(3,T)$ gives
            $$\bracks{\sqbracks{\begin{matrix}
            0 \\
            0 \\
            0 \\
            0 \\
            0 \\
            0 \\
            0 
            \end{matrix}}, \sqbracks{\begin{matrix}
            0 \\
            0 \\
            0 \\
            0 \\
            1 \\
            0 \\
            0 
            \end{matrix}}},             
            \bracks{\sqbracks{\begin{matrix}
            0 \\
            0 \\
            0 \\
            0 \\
            1 \\
            0 \\
            0 
            \end{matrix}}, \sqbracks{\begin{matrix}
            0 \\
            0 \\
            0 \\
            0 \\
            0 \\
            1 \\
            0 
            \end{matrix}}},
            \bracks{\sqbracks{\begin{matrix}
            0 \\
            0 \\
            0 \\
            0 \\
            1 \\
            0 \\
            0 
            \end{matrix}}, \sqbracks{\begin{matrix}
            0 \\
            0 \\
            0 \\
            0 \\
            1 \\
            1 \\
            0 
            \end{matrix}},
            \sqbracks{\begin{matrix}
            0 \\
            0 \\
            0 \\
            0 \\
            0 \\
            0 \\
            1 
            \end{matrix}}}$$ 

            \end{proof}

            \item Collect the Jordan Basis, $\mathfrak{B}_J$.
            \begin{proof}[Answer]
            By taking the 2nd and 4th set of vectors from the $T-4I$ case and
            the last set from the $T-3I$ case, we have the Jordan Basis:
            $$\mathfrak{B}_J = 
            \bracks{\sqbracks{\begin{matrix}
            1 \\
            0 \\
            0 \\
            0 \\
            0 \\
            0 \\
            0 
            \end{matrix}}, \sqbracks{\begin{matrix}
            0 \\
            1 \\
            0 \\
            0 \\
            0 \\
            0 \\
            0 
            \end{matrix}}, \sqbracks{\begin{matrix}
            1 \\
            1 \\
            1 \\
            0 \\
            0 \\
            0 \\
            0 
            \end{matrix}}, \sqbracks{\begin{matrix}
            0 \\
            0 \\
            0 \\
            1 \\
            0 \\
            0 \\
            0 
            \end{matrix}},\sqbracks{\begin{matrix}
            0 \\
            0 \\
            0 \\
            0 \\
            1 \\
            0 \\
            0 
            \end{matrix}}, \sqbracks{\begin{matrix}
            0 \\
            0 \\
            0 \\
            0 \\
            1 \\
            1 \\
            0 
            \end{matrix}},\sqbracks{\begin{matrix}
            0 \\
            0 \\
            0 \\
            0 \\
            0 \\
            0 \\
            1 
            \end{matrix}}}$$            

            \end{proof}
        \ee
    \ee

    \item Let $T \in \LT{V}$, and $\mathfrak{B}$ be an orthonormal basis, so that
    $$\mathcal{M}\parens{T, \mathfrak{B}} = \sqbracks{
    \begin{matrix}
    5 & 1 & & & \\
    & 5 & 1 & & \\
    & & 5 & 1 & \\
    & & & 5 & 1 \\
    & & & & 5 
    \end{matrix}}$$
    \be[(a)]
        \item Is $T$ self-adjoint? Why/Why not?
        \begin{proof}[Answer]
        $T$ is not self adjoint, because if it were then 
        $\mathcal{M}(T, \mathfrak{B}) = \mathcal{M}(T^*, \mathfrak{B})$; as in,
        the given matrix of the transformation would equal it's own conjugate transpose. 
        This matrix is upper-triangular, so obviously it's not equal to its own conjugate
        transpose.
        \end{proof}

        \item Is $T$ normal? Why/Why not?
        \begin{proof}[Answer]
        $T$ is normal. $\mathcal{M}\parens{T^*, \mathfrak{B}}$ is the tranpose of the 
        matrix above, so 
        $$\mathcal{M}\parens{T^*, \mathfrak{B}} = \sqbracks{
        \begin{matrix}
        5 & & & & \\
        1 & 5 & & & \\
        & 1 & 5 & & \\
        & & 1 & 5 &  \\
        & & & 1 & 5
        \end{matrix}}$$
        Then:
        $$\mathcal{M}\parens{T, \mathfrak{B}}\cdot \mathcal{M}\parens{T^*, \mathfrak{B}} = 
        \sqbracks{\begin{matrix}
        5 & 1 & & & \\
        & 5 & 1 & & \\
        & & 5 & 1 & \\
        & & & 5 & 1 \\
        & & & & 5 
        \end{matrix}}\cdot \sqbracks{
        \begin{matrix}
        5 & & & & \\
        1 & 5 & & & \\
        & 1 & 5 & & \\
        & & 1 & 5 &  \\
        & & & 1 & 5
        \end{matrix}} = \sqbracks{
        \begin{matrix}
        26 & 5 & & & \\
        5 & 26 & 5 & & \\
        & 5 & 26 & 5 & \\
        & & 5 & 26 & 5 \\
        & & & 5 & 26
        \end{matrix}}$$
        and 
        $$\mathcal{M}\parens{T^*, \mathfrak{B}}\cdot\mathcal{M}\parens{T, \mathfrak{B}} = 
        \sqbracks{
        \begin{matrix}
        5 & & & & \\
        1 & 5 & & & \\
        & 1 & 5 & & \\
        & & 1 & 5 &  \\
        & & & 1 & 5
        \end{matrix}} \cdot \sqbracks{\begin{matrix}
        5 & 1 & & & \\
        & 5 & 1 & & \\
        & & 5 & 1 & \\
        & & & 5 & 1 \\
        & & & & 5 
        \end{matrix}} = \sqbracks{
        \begin{matrix}
        26 & 5 & & & \\
        5 & 26 & 5 & & \\
        & 5 & 26 & 5 & \\
        & & 5 & 26 & 5 \\
        & & & 5 & 26
        \end{matrix}}$$
        So, $\mathcal{M}\parens{T, \mathfrak{B}}\cdot \mathcal{M}\parens{T^*, \mathfrak{B}}=
        \mathcal{M}\parens{T^*, \mathfrak{B}}\cdot \mathcal{M}\parens{T, \mathfrak{B}}$ and 
        $T$ is a normal operator.
        \end{proof}

    \ee
    
\ee
\noindent\makebox[\linewidth]{\rule{\paperwidth}{0.4pt}}
	
\end{document}
