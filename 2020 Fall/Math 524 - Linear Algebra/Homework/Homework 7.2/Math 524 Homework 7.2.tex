\documentclass{article}
\usepackage{amsmath}
\usepackage{amssymb}
\usepackage{bm}
\usepackage{amsthm}
\usepackage{enumerate}
\usepackage{graphicx}
\usepackage{psfrag}
\usepackage{color}
\usepackage{url}
\usepackage{listings}
\usepackage{xcolor}
\usepackage{tikz}
\usetikzlibrary{positioning}
\tikzset{main node/.style={circle,fill=gray!20,draw,minimum size=.5cm,inner sep=0pt},}

\definecolor{codegreen}{rgb}{0,0.5,0}
\definecolor{codewhite}{rgb}{1,1,1}
\definecolor{codegray}{rgb}{0.5,0.5,0.5}
\definecolor{codepurple}{rgb}{0.58,0,0.82}
\definecolor{codeblack}{rgb}{0,0,0}
\definecolor{codeorange}{rgb}{0.8,0.4,0}

\lstdefinestyle{mystyle}{
    backgroundcolor=\color{codewhite},   
    commentstyle=\color{codegray},
    keywordstyle=\color{codegreen},
    numberstyle=\color{codegray},
    stringstyle=\color{codeorange},
    basicstyle=\ttfamily ,
    breakatwhitespace=false,         
    breaklines=true,                 
    captionpos=b,                    
    keepspaces=true,                 
    numbers=left,                    
    numbersep=5pt,                  
    showspaces=false,                
    showstringspaces=false,
    showtabs=false,                  
    tabsize=4
}
\lstset{style=mystyle}


\setlength{\hoffset}{-1in}
\addtolength{\textwidth}{1.5in}
\setlength{\voffset}{-1in}
\addtolength{\textheight}{1.5in}
\newcommand{\be}{\begin{enumerate}}
\newcommand{\ee}{\end{enumerate}}
\newcommand{\BigO}[1]{\ensuremath\mathcal{O}\left(#1\right)}
\newcommand{\il}[1]{\lstinline!#1!}
\newcommand{\norm}[1]{\left|\left|#1\right|\right|}
\newcommand{\abs}[1]{\left|#1\right|}
\newcommand{\parens}[1]{\left(#1\right)}
\newcommand{\bracks}[1]{\left\{#1\right\}}
\newcommand{\sqbracks}[1]{\left[#1\right]}
\newcommand{\vep}{\varepsilon}
\newcommand{\ceiling}[1]{\left\lceil#1\right\rceil}
\newcommand{\R}{\mathbb{R}}
\newcommand{\N}{\mathbb{N}}
\newcommand{\Z}{\mathbb{Z}}
\newcommand{\F}{\mathbb{F}}
\newcommand{\C}{\mathbb{C}}
\newcommand{\LT}[1]{\mathcal{L}\parens{#1}}
\newcommand{\poly}[2]{\mathcal{P}_{#1}\parens{#2}}
\newcommand{\range}[1]{\text{range }#1}
\renewcommand{\null}[1]{\text{null }#1}
\newcommand{\distrib}[2]{\text{#1}\left(#2\right)}
\newcommand{\dd}[1]{\frac{d}{d#1}}
\newcommand{\abracks}[1]{\left< #1\right>}

\begin{document}
	\begin{center}
		\textbf{Fall 2020, Math 524:\ Homework 7.2} \\
		\textbf{Due:\ Monday, November 30th, 2020} \\
		\textbf{Joseph Diaz: 819947915}
	\end{center}
\noindent\makebox[\linewidth]{\rule{\paperwidth}{0.4pt}}

\subsection*{7C}
\be
    \item[2.] Suppose $T$ is a positive operator on $V$. Suppose $v,w \in V$ 
    are such that $Tv = w$ and $Tw = v$. Prove that $v = w$.
    \begin{proof}
    By assumption, we have that $Tv = w$ and $Tw = v$, so 
    $$Tv - Tw = T(v-w) = w - v$$
    Since $T$ is a positive operator, we also have that
    \begin{align*}
    \abracks{T(v-w), v-w} &\geq 0 \\
    \abracks{w-v, v-w} &\geq \\
    \abracks{-(v-w), v-w} &\geq \\
    -\abracks{v-w, v-w} &\geq \\
    \abracks{v-w,v-w} & \leq 0 \\
    \norm{v-w}^2 &\leq 0
    \end{align*}
    As norms are non-negative, it must be the case that $\norm{v-w} = 0$; 
    which implies that $v-w = 0$ and $v = w$.
    \end{proof}

    \item[4.] Suppose that $T \in \LT{V, W}$. Prove that $T^*T$ is a positive 
    operator on $V$ and $TT^*$ is a positive operator on $W$.
    \begin{proof}
    First, et $T^* \in \LT{W,V}$ be the adjoint of $T$. Next, we establish that 
    \begin{gather*}
    (T^*T)^*=T^*(T^*)^* = T^*T \\
    (TT^*)^* = (T^*)^*T^* = TT^*
    \end{gather*}
    which means that both $T^*T$ and $TT^*$ are self-adjoint and $T^*T \in \LT{V},\
    TT^* \in \LT{W}$. Now, let $v \in V$, then
    $$\abracks{T^*Tv, v} = \abracks{Tv, (T^*)^*v} = \abracks{Tv,Tv} \geq 0$$
    so $T^*T$ is a positive operator on $V$. Lastly, let $w \in W$, then 
    $$\abracks{TT^*w, w} = \abracks{T^*w, T^*w} \geq 0$$
    so $TT^*$ is a positive operator on $W$.
    \end{proof}

    \item[7.] Suppose $T$ is a positive operator on $V$. Prove that $T$ is 
    invertible if and only if
    $$\abracks{Tv,v} > 0$$
    for every non-zero $v \in V$.
    \begin{proof}
    Let $T \in \LT{V}$ be a positive operator. Then, we'll prove this from the 
    left and from the right.
    \be
        \item[$\Longrightarrow$:]
        Suppose that $T$ is invertible, this means that for every non-zero $v \in V,\ 
        Tv \neq 0$. So $\abracks{Tv, v} = 0$ only when $v =0$; as $T$ is also a 
        positive operator this implies that
        $$\abracks{Tv, v} > 0$$
        for non-zero $v$.

        \item[$\Longleftarrow$:] 
        Suppose that for every non-zero $v \in V$, it is the case that 
        $$\abracks{Tv, v} > 0$$
        Now suppose, by contradiction, that $T$ is not invertible. This means 
        that there exists at least one non-zero $u \in V$ such that $Tu = 0$, and 
        so 
        $$\abracks{Tu, u} = \abracks{0, u} = 0$$
        But this contradicts the condition that $\abracks{Tv, v} >0$ for all non-zero 
        $v \in V$. So $T$ must be invertible.
    \ee
    \end{proof}

\ee

\subsection*{7D}
\be
    \item[1.] Fix $u,x \in V$ with $u \neq 0$. Define $T \in \LT{V}$ by
    $$Tv = \abracks{v,u}x$$
    for every $v \in V$. Prove that 
    $$\sqrt{T^*T}v = \frac{\norm{x}}{\norm{u}}\abracks{v,u}u$$
    for every $v \in V$.
    \begin{proof}
    Using the usual process, we find that $T^* \in \LT{V}$ is given by 
    $$T^*v = \abracks{x,v}u$$
    for all $v \in V$. So, composing $T$ with this gives
    \begin{align*}
    T^*Tv &= \abracks{x, Tv}u \\
    &= \abracks{x, \abracks{v,u}x}u \\
    &= \abracks{v,u}\abracks{x, x}u \\
    &= \norm{x}^2\abracks{v,u}u \\
    \end{align*}
    Now, let $R \in \LT{V}$ be defined as 
    $$\forall v\in V,\ Rv = \frac{\norm{x}}{\norm{u}}\abracks{v,u}u$$
    We now wish to show that $R$ is self-adjoint and a square root of $T^*T$.
    Let $w \in V$, then 
    \begin{align*}
    \abracks{Rv, w} &= \abracks{\frac{\norm{x}}{\norm{u}}\abracks{v,u}u, w} \\
    &= \frac{\norm{x}}{\norm{u}}\abracks{v,u}\abracks{u, w} \\
    &= \frac{\norm{x}}{\norm{u}}\abracks{w,u}\abracks{v, u} \\
    &= \abracks{v, \frac{\norm{x}}{\norm{u}}\abracks{w,u}u} \\
    &= \abracks{v, Rw}
    \end{align*}
    So $R$ is self-adjoint. Further:
    \begin{align*}
    R^2v &= \frac{\norm{x}}{\norm{u}}\abracks{\frac{\norm{x}}{\norm{u}}\abracks{v,u}u,u}u \\
    &= \frac{\norm{x}^2}{\norm{u}^2}\abracks{v,u}\abracks{u,u}u \\ 
    &= \frac{\norm{x}^2}{\norm{u}^2}\abracks{v,u}\norm{u}^2u \\
    &= \norm{x}^2\abracks{v,u}u \\
    &= T^*Tv
    \end{align*}
    So, $R$ is a square root of $T^*T$ and 
    $$Rv = \sqrt{T^*T}v = \frac{\norm{x}}{\norm{u}}\abracks{v,u}u$$
    \end{proof} 

    \item[2.] Give an example of $T \in \LT{\C^2}$ such that 0 is the only 
    eigenvalue of $T$ and the singular values of $T$ are 5 and 0.
    \begin{proof}[Answer]
    Let $T \in \LT{\C^2}$ be defined as $T(x,y) = (5y, 0)$, then
    $$\mathcal{M}(T) = \parens{\begin{matrix}
    0 & 5 \\ 0 & 0    
    \end{matrix}}$$
    and the eigenvalues of $T$ are both 0. We can find the 
    singular values using $\mathcal{M}(T)$ and $\mathcal{M}(T^*)$,
    which in this case is just the transpose of $\mathcal{M}(T)$, like so 
    $$\mathcal{M}\parens{T^*T} = \mathcal{M}(T^*)\mathcal{M}(T) =
    \parens{\begin{matrix}
    0 & 0 \\ 5 & 0
    \end{matrix}}\parens{\begin{matrix}
    0 & 5 \\ 0 & 0
    \end{matrix}} = \parens{\begin{matrix}
    0 & 0 \\ 0 & 25
    \end{matrix}}$$
    So $T^*T(x,y) = (0, 25y)$, and the singular values of $T$ are 0 and 5.
    \end{proof}

    \item[5.] Suppose $T \in \LT{\C^2}$ is defined as $T(x,y) = (-4y, x)$.
    Find the singular values of $T$.
    \begin{proof}[Answer]
    The singular values of $T$ are just the square-roots of the eigenvalues of 
    $T^*T$. We can find $T^*$ by taking the transpose of $\mathcal{M}(T)$:
    $$\mathcal{M}(T) = \parens{\begin{matrix}
    0 & -4 \\ 1 & 0
    \end{matrix}} \implies \mathcal{M}(T^*) = \parens{\begin{matrix}
    0 & 1 \\ -4 & 0    
    \end{matrix}}$$
    then 
    $$\mathcal{M}\parens{T^*T} = \mathcal{M}(T^*)\mathcal{M}(T) = 
    \parens{\begin{matrix}
    0 & 1 \\ -4 & 0
    \end{matrix}}\parens{\begin{matrix}
    0 & -4 \\ 1 & 0
    \end{matrix}} = \parens{\begin{matrix}
    1 & 0 \\ 0 & 16
    \end{matrix}}$$
    This means that $T^*T(x,y) = (x, 16y)$, so the singular values of $T$ are 
    1 and 4.
    \end{proof} 

\ee

\noindent\makebox[\linewidth]{\rule{\paperwidth}{0.4pt}}
	
\end{document}
 