\documentclass{article}
\usepackage{amsmath}
\usepackage{amssymb}
\usepackage{bm}
\usepackage{amsthm}
\usepackage{enumerate}
\usepackage{graphicx}
\usepackage{psfrag}
\usepackage{color}
\usepackage{url}
\usepackage{listings}
\usepackage{xcolor}
\usepackage{tikz}
\usetikzlibrary{positioning}
\tikzset{main node/.style={circle,fill=gray!20,draw,minimum size=.5cm,inner sep=0pt},}

\definecolor{codegreen}{rgb}{0,0.5,0}
\definecolor{codewhite}{rgb}{1,1,1}
\definecolor{codegray}{rgb}{0.5,0.5,0.5}
\definecolor{codepurple}{rgb}{0.58,0,0.82}
\definecolor{codeblack}{rgb}{0,0,0}
\definecolor{codeorange}{rgb}{0.8,0.4,0}

\lstdefinestyle{mystyle}{
    backgroundcolor=\color{codewhite},   
    commentstyle=\color{codegray},
    keywordstyle=\color{codegreen},
    numberstyle=\color{codegray},
    stringstyle=\color{codeorange},
    basicstyle=\ttfamily ,
    breakatwhitespace=false,         
    breaklines=true,                 
    captionpos=b,                    
    keepspaces=true,                 
    numbers=left,                    
    numbersep=5pt,                  
    showspaces=false,                
    showstringspaces=false,
    showtabs=false,                  
    tabsize=4
}
\lstset{style=mystyle}


\setlength{\hoffset}{-1in}
\addtolength{\textwidth}{1.5in}
\setlength{\voffset}{-1in}
\addtolength{\textheight}{1.5in}
\newcommand{\be}{\begin{enumerate}}
\newcommand{\ee}{\end{enumerate}}
\newcommand{\BigO}[1]{\ensuremath\mathcal{O}\left(#1\right)}
\newcommand{\il}[1]{\lstinline!#1!}
\newcommand{\gnorm}[1]{\left|\left|#1\right|\right|}
\newcommand{\abs}[1]{\left|#1\right|}
\newcommand{\parens}[1]{\left(#1\right)}
\newcommand{\bracks}[1]{\left\{#1\right\}}
\newcommand{\sqbracks}[1]{\left[#1\right]}
\newcommand{\vep}{\varepsilon}
\newcommand{\ceiling}[1]{\left\lceil#1\right\rceil}
\newcommand{\R}{\mathbb{R}}
\newcommand{\N}{\mathbb{N}}
\newcommand{\Z}{\mathbb{Z}}
\newcommand{\F}{\mathbb{F}}
\newcommand{\distrib}[2]{\text{#1}\left(#2\right)}
\newcommand{\dd}[1]{\frac{d}{d#1}}
\newcommand{\abracks}[1]{\left< #1\right>}
%\newcommand{\span}[1]{\text{span}\parens{#1}}

\begin{document}
	\begin{center}
		\textbf{Fall 2020, Math 524:\ Homework 2} \\
		\textbf{Due:\ Monday, September 21st, 2020} \\
		\textbf{Joseph Diaz: 819947915}
	\end{center}
\noindent\makebox[\linewidth]{\rule{\paperwidth}{0.4pt}}

\subsection*{2A}
\be
    \item[8.] Prove or give a counterexample: If $v_1, v_2, \cdots, 
    v_m$ is a linearly independent list of vectors in $V$ and $
    \lambda \in \F$ with $\lambda \neq 0$, then $\lambda v_1, 
    \lambda v_2, \cdots, \lambda v_m$ is linearly independent.
    \begin{proof}
    We have that $v_1, v_2, \cdots, v_m$ is a linearly independent
    list of vectors in $V$. This implies that
    $$a_1v_1 + a_2v_2 + \cdots + a_mv_m = 0, a_i \in \F, 1 \leq i 
    \leq m$$
    has only a single solution: $a_1 = a_2 = \cdots = a_m = 0$.
    Now let $\lambda \in \F,\ \lambda \neq 0$, then:
    \begin{align*}
    a_1v_1 + a_2v_2 + \cdots + a_mv_m &= 0 \\
    \lambda\parens{a_1v_1 + a_2v_2 + \cdots + a_mv_m} 
    &= \lambda\parens{0} \\
    \lambda\parens{a_1v_1} + \lambda\parens{a_2v_2} + \cdots + 
    \lambda\parens{a_mv_m} &= \lambda\parens{0} \\
    a_1\parens{\lambda v_1} + a_2\parens{\lambda v_2} + \cdots + 
    a_m\parens{\lambda v_m} &= 0
    \end{align*}
    As $\lambda \neq 0$ and all the $v_i$ vectors have been uniformly
    scaled by $\lambda$, we may conclude that the only solution to 
    the equation is $a_1 = a_2 = \cdots = a_m = 0$; which implies 
    that the list of vectors $\lambda v_1, \lambda v_2, \cdots, 
    \lambda v_m$ is linearly independent.
    \end{proof}
    
    \item[9.] Prove or give a counterexample: If $v_1, v_2, \cdots, 
    v_m$ and $w_1, w_2, \cdots, w_m$ are linearly independent lists
    of vectors in $V$, then $v_1 + w_1, v_2 + w_1, \cdots, v_m +w_m$
    is linearly independent.
    \begin{proof}
    This is not true in general. For example, take the possibility 
    that $\forall i \in \bracks{1, \cdots, m},\ v_i = -w_i$; this 
    does not violate the assumption that $v_1, v_2, \cdots, 
    v_m$ and $w_1, w_2, \cdots, w_m$ are linearly independent lists,
    but we have that the equation 
    $$a_1(v_1 + w_1) + a_2(v_2 + w_2) + \cdots + a_m(v_m + w_m) = 0, 
    a_i \in \F, 1\leq i \leq m$$
    has solutions other $a_1 = a_2 = \cdots a_m = 0$, because $v_i 
    + w_i = 0$. In this case, $v_1 + w_1, v_2 + w_1, \cdots, v_m 
    +w_m$ is not a linearly independent list.
    
    \end{proof}
    
    \item[11.] Suppose $v_1, \cdots, v_m$ is linearly independent in 
    $V$ and $w \in V$. Show that $v_1, \cdots, v_m, w$ is 
    linearly independent list in $V$ if and only if
    $$w \notin \text{span}\parens{v_1, \cdots, v_m}$$
    \begin{proof}
    We will prove this from the left and from the right:
    \be 
    \item[$\implies$:]
    We have that $v_1, v_2, \cdots v_m, w$ is a linearly independent 
    list of vectors in $V$, which we denote $L$. This implies that
    $$\forall v \in L: v \notin \text{span}\parens{L\backslash v}$$
    as none of the vectors in $L$ can be written as a linear 
    combination of the other vectors in $L$. $w$ is not special is 
    this regard and we may conclude that
    $$w \notin \text{span}\parens{v_1, \cdots, v_m}$$
    \item[$\Longleftarrow$:]
    We have that 
    $$w \notin \text{span}\parens{v_1, \cdots, v_m}$$
    This means that $w$ cannot be written as a linear combination of 
    the $v_i$'s, which implies that $w$ is linearly independent from
    the list of vectors $v_1, v_2, \cdots v_m$. So $v_1, v_2, \cdots 
    v_m, w$ is a linearly independent list in $V$. 
    
    \ee
    \end{proof}
\ee

\subsection*{2B}
\be
    \item[3.]
    \be[(a)]
        \item Let $U$ be a subspace of $\R^5$ defined by
        $$U = \bracks{\parens{x_1, x_2, x_3, x_4, x_5} \in \R^5 
        \big| x_1 = 3x_2,\ x_3 = 7x_4}$$
        Find a basis in $U$.
        \begin{proof}[Answer]
        The set
        $$L = \bracks{\parens{3,1,0,0,0},\ \parens{0,0,7,1,0},\
        \parens{0,0,0,0,1}}$$
        spans $U$.
        \end{proof}
        
        \item Extend the basis in part ~(a) to a basis in $\R^5$.
        \begin{proof}[Answer]
        To extend the basis for $U$ to $\R^5$, we need only add 2 
        more vectors from $\R^5$ that are linearly independent 
        from each other and also the vectors of $L$. So
        we choose $\parens{1,0,0,0,0}$ and $\parens{0,0,1,0,0}$, then
        we have that  
        $$L' = \bracks{\parens{3,1,0,0,0},\ \parens{0,0,7,1,0},\
        \parens{0,0,0,0,1},\ \parens{1,0,0,0,0},\ 
        \parens{0,0,1,0,0}}$$
        spans $\R^5$.
        \end{proof}
        \item Find a subspace $W$ of $\R^5$ such that $\R^5 = U 
        \oplus W$.
        \begin{proof}[Answer]
        From part (b), we found what we needed to add to the basis
        of $U$ to get a basis for $\R^5$. So let 
        $$W = \text{span}\parens{\parens{1,0,0,0,0},\ 
        \parens{0,0,1,0,0}}$$
        be a subspace of $\R^5$. The only element in common between
        $U$ and $W$ is $0$, so we may conclude that $\R^5 = U \oplus 
        W$. 
        \end{proof}
        
    
    \ee
    
    \item[5.] Prove or disprove: there exists a basis $p_0, p_1, p_2, 
    p_3$ of $\mathcal{P}_3\parens{\F}$ such that none of the 
    polynomials $p_0, p_1, p_2, p_3$ has degree 2.
    \begin{proof}
    The following list of polynomials is a basis for $\mathcal{P}_3
    \parens{\F}$ that satisfies the desired property:
    \begin{align*}
    p_0 &= 1 \\
    p_1 &= z \\
    p_2 &= z^2 + z^3 \\
    p_3 &= z^3
    \end{align*}
    These polynomials are linearly independent, i.e. none of them can
    be rewritten as a linear combination of the others, they span the
    space $\mathcal{P}_3\parens{\F}$ since there are 4 of them, and
    none have degree 2. 
    \end{proof}
    
\ee

\subsection*{2C}
\be
    \item[5.] 
    \be[(a)]
        \item Let $$U = \bracks{p \in \mathcal{P}_4\parens{\R}\big|
        p''(6) = 0}$$
        Find a basis for $U$.
        \begin{proof}[Answer]
        Let 
        $$p(z) = a_0 + a_1z + a_2z^2 + a_3z^3 + a_4z^4 \in U$$
        Then $p$ satisfies
        \begin{align*}
        p(z) &= a_0 + a_1z + a_2z^2 + a_3z^3 + a_4z^4 \\
        p'(z) &= a_1 + 2a_2z + 3a_3z^2 + 4a_4z^3 \\
        p''(z) &= 2a_2 + 6a_3z + 12a_4z^2 \\
        \implies p''(6) &= 2a_2 + 36a_3 + 432a_4  = 0\\
        \end{align*}
        So we have that $a_0,a_1$ are free variables, but we must 
        parameterize one of $a_2, a_3, a_4$. We'll solve the above 
        equation for $a_2$:
        $$2a_2 + 36a_3 + 432a_4  = 0 \implies a_2 = -18a_3 - 216a_4$$         
        Plugging this into our original $p(z)$ and regrouping terms
        by coefficients gives
        \begin{align*}
        p(z) &= a_0 + a_1z + a_2z^2 + a_3z^3 + a_4z^4 \\
        &= a_0 + a_1z + \parens{-18a_3 - 216a_4}z^2 + a_3z^3 + a_4z^4
        \\
        &= a_0 + a_1z + a_3\parens{z^3 - 18z^2} + 
        a_4\parens{z^4 -216z^2}
        \end{align*}
        With this, we have that 
        $$P = \bracks{1,\ z,\ z^3 - 18z^2,\ z^4 -216z^2}$$ 
        forms a basis for $U$.
        \end{proof}         
        
        \item Extend the basis in part ~(a) to a basis of 
        $\mathcal{P}_4\parens{\R}$.
        \begin{proof}[Answer]
        We need only add a $z^2$ vector to our basis for $U$ to 
        extend it to $\mathcal{P}_4\parens{\R}$, so
        $$P' = \bracks{1,\ z,\ z^2,\ z^3 - 18z^2,\ z^4 -216z^2}$$
        forms a basis for $\mathcal{P}_4\parens{\R}$.
        \end{proof}
        
        \item Find a subspace $W$ of $\mathcal{P}_4\parens{\R}$
        such that $\mathcal{P}_4\parens{\R} = U \oplus W$.
        \begin{proof}[Answer]
        From the answer for part (b), we know that 
        $$W = \bracks{cz^2 \big|\ c \in \R}$$
        will satisfy $\mathcal{P}_4\parens{\R} = U \oplus W$.
        \end{proof}
    \ee
    
    \item[9.] Suppose $v_1, \cdots, v_m$ is linearly independent in 
    $V$ and $w \in V$. Prove that
    $$\dim\text{span}\parens{v_1 + w, \cdots, v_m + w} \geq m - 1$$
    \begin{proof}
    We'll prove this by considering some cases.\\\\
    \textbf{Suppose $w = 0$}:\\
    In this case it is trivial to show that the inequality holds,
    because 
    $$\dim\text{span}\parens{v_1 + w, \cdots, v_m + w} = 
    \dim\text{span}\parens{v_1, \cdots, v_m} = m \geq m-1$$
    
    \textbf{Suppose $w \neq 0, w \in 
    \text{span}\parens{v_1, \cdots, v_m}$}:\\
    This means that $w$ can be expressed as a linear combination 
    of the $v$'s, so there exists a set of coefficients, 
    $c_i \in \F$, such that 
    $$w = c_1v_1 + \cdots + c_mv_m$$
    where not \emph{all} of the $c$'s are 0. This means 
    $$v_i + w \in \text{span}\parens{v_1,\cdots, v_m}$$
    So clearly, $v_1 + w, \cdots, v_m + w$ is not \emph{necessarily}
    a linearly independent list, because we could have that $w = -v_                
    \ell, 1 \leq \ell \leq m$, which would leave us with $m-1$ 
    linearly independent vectors, and our inequality would be 
    satisfied. However, this sort of vector nullification can only
    happen one at a time, so the smallest the dimension that 
    span$\parens{v_1 + w, \cdots, v_m + w}$ could be is $m-1$. So
    $$\dim\text{span}\parens{v_1 + w, \cdots, v_m + w} = 
    m \geq m-1$$
    
        
    \textbf{Suppose $w \neq 0, w \notin 
    \text{span}\parens{v_1, \cdots, v_m}$}:\\
    This means that $w$ is linearly independent of the $v$'s, 
    so the list of vectors $v_1 + w, \cdots, v_m + w$ must also be 
    linearly independent because
    \begin{align*}
    a_1(v_1 + w) + \cdots + a_m(v_m + w) &= 0, a_i \in \F\\
    a_1v_1 + a_1w + \cdots + a_mv_m + a_mw &= 0\\
    a_1v_1 + \cdots a_mv_m + a_1w + \cdots + a_mw &= 0\\
    a_1v_1 + \cdots a_mv_m + \parens{a_1 + \cdots + a_m}w &= 0\\
    \end{align*}
    which only has the solution $a_1 = \cdots = a_m = 0$.
    So 
    $$\dim\text{span}\parens{v_1 + w, \cdots, v_m + w} = 
    m \geq m-1$$
    Regardless of case, we see that the inequality is respected.
    \end{proof}
    
\ee

\noindent\makebox[\linewidth]{\rule{\paperwidth}{0.4pt}}
	
\end{document}
