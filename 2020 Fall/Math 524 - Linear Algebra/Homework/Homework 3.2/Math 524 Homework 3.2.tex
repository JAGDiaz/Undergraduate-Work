\documentclass{article}
\usepackage{amsmath}
\usepackage{amssymb}
\usepackage{bm}
\usepackage{amsthm}
\usepackage{enumerate}
\usepackage{graphicx}
\usepackage{psfrag}
\usepackage{color}
\usepackage{url}
\usepackage{listings}
\usepackage{xcolor}
\usepackage{tikz}
\usetikzlibrary{positioning}
\tikzset{main node/.style={circle,fill=gray!20,draw,minimum size=.5cm,inner sep=0pt},}

\definecolor{codegreen}{rgb}{0,0.5,0}
\definecolor{codewhite}{rgb}{1,1,1}
\definecolor{codegray}{rgb}{0.5,0.5,0.5}
\definecolor{codepurple}{rgb}{0.58,0,0.82}
\definecolor{codeblack}{rgb}{0,0,0}
\definecolor{codeorange}{rgb}{0.8,0.4,0}

\lstdefinestyle{mystyle}{
    backgroundcolor=\color{codewhite},   
    commentstyle=\color{codegray},
    keywordstyle=\color{codegreen},
    numberstyle=\color{codegray},
    stringstyle=\color{codeorange},
    basicstyle=\ttfamily ,
    breakatwhitespace=false,         
    breaklines=true,                 
    captionpos=b,                    
    keepspaces=true,                 
    numbers=left,                    
    numbersep=5pt,                  
    showspaces=false,                
    showstringspaces=false,
    showtabs=false,                  
    tabsize=4
}
\lstset{style=mystyle}


\setlength{\hoffset}{-1in}
\addtolength{\textwidth}{1.5in}
\setlength{\voffset}{-1in}
\addtolength{\textheight}{1.5in}
\newcommand{\be}{\begin{enumerate}}
\newcommand{\ee}{\end{enumerate}}
\newcommand{\BigO}[1]{\ensuremath\mathcal{O}\left(#1\right)}
\newcommand{\il}[1]{\lstinline!#1!}
\newcommand{\gnorm}[1]{\left|\left|#1\right|\right|}
\newcommand{\abs}[1]{\left|#1\right|}
\newcommand{\parens}[1]{\left(#1\right)}
\newcommand{\bracks}[1]{\left\{#1\right\}}
\newcommand{\sqbracks}[1]{\left[#1\right]}
\newcommand{\vep}{\varepsilon}
\newcommand{\ceiling}[1]{\left\lceil#1\right\rceil}
\newcommand{\R}{\mathbb{R}}
\newcommand{\N}{\mathbb{N}}
\newcommand{\Z}{\mathbb{Z}}
\newcommand{\F}{\mathbb{F}}
\newcommand{\LT}{\mathcal{L}}
\newcommand{\poly}{\mathcal{P}}
\newcommand{\distrib}[2]{\text{#1}\left(#2\right)}
\newcommand{\dd}[1]{\frac{d}{d#1}}
\newcommand{\abracks}[1]{\left< #1\right>}

\begin{document}
	\begin{center}
		\textbf{Fall 2020, Math 524:\ Homework 3.2} \\
		\textbf{Due:\ Monday, October 5th, 2020} \\
		\textbf{Joseph Diaz: 819947915}
	\end{center}
\noindent\makebox[\linewidth]{\rule{\paperwidth}{0.4pt}}

\subsection*{3D}
\be
    \item[2.] Suppose $V$ is finite-dimensional and $\dim V > 1$.
    Prove that the set of noninvertible operators on $V$ is not a 
    subspace of $\LT\parens{V}$.
    \begin{proof}
    Let $V$ be a finite-dimensional vector space with $\dim V = n 
    > 1$ and basis vectors $(v_1,\ \cdots,\ v_n)$; we make this
    requirement that $n > 1$ because if $\dim V = 1$ then the set of
    noninvertible operators in $\LT(V)$ is $\bracks{0}$, which is a
    subspace of $\LT(V)$. Now we define 
    $S,T \in \LT(V)$ such that
    \begin{gather*}
    S(a_1v_1 + \cdots + a_nv_n) = \sum_{k \text{ odd}}^n a_kv_k \\
    T(a_1v_1 + \cdots + a_nv_n) = \sum_{k \text{ even}}^n a_kv_k \\
    \end{gather*}
    where $a_i \in \F$. So clearly neither $S$ nor $T$ are injective,
    because the null space of $S$ is all of the linear combinations
    of the ``even vectors'' of $V$ and the null space of $T$ is all 
    of the linear combinations of the ``odd vectors'' of $V$; and by 
    theorem 3.69 in the text, they aren't invertible either. However,
    their sum is the identity:
    \begin{align*}
    (S + T)(a_1v_1 + \cdots + a_nv_n) &= S(a_1v_1 + \cdots + a_nv_n) 
    + T(a_1v_1 + \cdots + a_nv_n) \\
    &= \sum_{k \text{ odd}}^n a_kv_k + \sum_{k \text{ even}}^n a_kv_k
    \\
    &= \sum_{k =1}^n a_kv_k \\
    &= a_1v_1 + \cdots + a_nv_n \\
    &= I(a_1v_1 + \cdots + a_nv_n)
    \end{align*}
    The identity is invertible, so this implies that the set of 
    noninvertible operators is not closed under addition; therefore,
    they do not form a subspace of $\LT(V)$.
    \end{proof}
    
    \item[3.] Suppose $V$ is finite-dimensional, $U$ is a subspace of 
    $V$, and $S \in \LT\parens{U,V}$. Prove that there exists an
    invertible operator $T \in \LT\parens{V}$ such that $Tu = Su$ for
    every $u \in U$ if and only if $S$ is injective.
    \begin{proof}
    To show this, we'll prove the equivalence from the left and from 
    the right.
    \be
        \item[$\Longrightarrow$:]
        We have that there exists an invertible $T \in \LT(V)$ such
        that $Tu = Su$ for all $u \in U$. By theorem 3.69 in the 
        text, $T$ must also be injective; this implies that $S$ must 
        be injective too, since $\text{range }S \subset 
        \text{range }T$ and a vector in $\text{range }S$ that isn't 
        mapped to under $S$ would violate the equality $Tu = Su$.
        
        \item[$\Longleftarrow$:] Let
        $u_1,\ \cdots,\ u_n$ be a basis for $U$, which we may extend 
        to a basis of $V$ as $u_1,\ \cdots,\ u_n, u_{n+1},\ \cdots,\
        u_m$. $S$ is injective, so $Su_1,\ \cdots,\ Su_n$ is 
        linearly independent in $V$. Now we extend that as a basis of 
        $V$ as $Su_1,\ \cdots,\ Su_n,\ v_{n+1},\ \cdots,\ v_m$, and 
        define $T \in \LT(V)$ such that
        $$Tu_i = Su_i, 1 \leq i \leq n,\ Tu_j = v_j, n+1 \leq j
        \leq m$$
        This definition also guarantees that $T$ is surjective and 
        therefore invertible as an operator.
        Finally, for all $u \in U, u = a_1u_1 + \cdots + a_nu_n$, we 
        have that
        \begin{align*}
        Tu &= T(a_1u_1 + \cdots + a_nu_n) \\
        &= T(a_1u_1) + \cdots + T(a_nu_n) \\
        &= a_1Tu_1 + \cdots + a_nTu_n \\
        &= a_1Su_1 + \cdots + a_nSu_n \\
        &= S(a_1u_1) + \cdots + S(a_nu_n) \\
        &= S(a_1u_1 + \cdots + a_nu_n) \\
        &= Su
        \end{align*}
        
    \ee
    \end{proof}
\ee

\subsection*{3E}
\be
    \item[2.] Suppose $V_1,\cdots, V_m$ are vector spaces such that
    $V_1 \times \cdots \times V_m$ is finite-dimensional. Prove that
    $V_j$ is finite-dimensional for each $j=1,\ \cdots,\ m$. 
    \begin{proof}
    Let $\dim V_1 \times \cdots \times V_m = n$, then by theorem 
    3.76 from the text, we have
    $$n = \dim V_1 + \cdots +\dim V_m$$
    As each of the $V$'s are themselves vector spaces, their 
    dimensions must be natural numbers; the sum of these dimensions
    is $n$, so this implies that 
    $$0 \leq \dim V_j \leq n,\ 1 \leq j \leq m$$
    and each vector space $V_j$ is finite-dimensional.
    \end{proof}
    
    \item[4.] Suppose $V_1,\cdots, V_m$ are vector spaces. Prove that
    $\LT\parens{V_1 \times \cdots \times V_m, W}$ and \\
    $\LT\parens{V_1,W}\times\cdots\times\LT\parens{V_m,W}$ are
    isomorphic vector spaces. 
    \begin{proof}
    To show that $\LT\parens{V_1 \times \cdots \times V_m, W} = U$ 
    and $\LT\parens{V_1,W}\times\cdots\times\LT\parens{V_m,W} = U'$ 
    are isomorphic, we'll define an isomorphism $\varphi$ between the 
    vector spaces.\\\\
    Let $\varphi: U \to U'$ be defined as 
    $$\varphi\parens{T(v_1,\ \cdots,\ v_m)} = 
    \parens{T_1,\ T_2,\ \cdots,\ T_m},\ \forall
    T \in U$$
    where $T: V_1 \times \cdots \times V_m \to W$ and $T_i: V_i \to
    W$ is defined as 
    $$T_iv_i = T(0,\ \cdots,\ v_i,\ \cdots,\ 0),\ v_i \in V_i$$
    Now, we define the $\varphi^{-1}: U' \to U$ as
    $$\varphi^{-1}(T_1,\ \cdots,\ T_m) =
    T(v_1,\ \cdots,\ v_m),\ v_i \in V_i, \forall T_i \in U'$$
    Both mappings $\varphi$ and $\varphi^{-1}$ are linear, for     
    example:
    \begin{align*}
    \varphi\parens{(T + S)(v_1,\ \cdots,\ v_m)} 
    &= \parens{T_1 + S_1,\ \cdots,\ T_m + S_m} 
    \\
    &= \parens{T_1,\ \cdots,\ T_m} + \parens{S_1,\ \cdots,\ S_m} \\
    &= \varphi\parens{T(v_1,\ \cdots,\ v_m)} + 
    \varphi\parens{S(v_1,\ \cdots,\ v_m)}
    \end{align*}
    and
    \begin{align*}
    \varphi^{-1}\parens{T_1 + S_1,\ \cdots,\ T_m + S_m}
    &= (T + S)(v_1,\ \cdots,\ v_m) \\
    &= T(v_1,\ \cdots,\ v_m) + S(v_1,\ \cdots,\ v_m)
    \end{align*}
    and they serve as inverse of one another:
    \begin{gather*}
    \varphi\circ\varphi^{-1}\parens{T_1,\ \cdots,\ T_m} = 
    \parens{T_1,\ \cdots,\ T_m} \\
    \varphi^{-1}\circ\varphi\parens{T(v_1,\ \cdots,\ v_m)} = 
    T(v_1,\ \cdots,\ v_m)
    \end{gather*}
    $\varphi$ is an isomorphism between $\LT\parens{V_1 \times \cdots 
    \times V_m, W}$ and $\LT\parens{V_1,W}\times\cdots\times\LT
    \parens{V_m,W}$, and those 2 spaces are isomorphic.
    \end{proof}        
\ee



\noindent\makebox[\linewidth]{\rule{\paperwidth}{0.4pt}}
	
\end{document}
