\documentclass{article}
\usepackage{amsmath}
\usepackage{amssymb}
\usepackage{bm}
\usepackage{amsthm}
\usepackage{enumerate}
\usepackage{graphicx}
\usepackage{psfrag}
\usepackage{color}
\usepackage{url}
\usepackage{listings}
\usepackage{xcolor}
\usepackage{tikz}
\usetikzlibrary{positioning}
\tikzset{main node/.style={circle,fill=gray!20,draw,minimum size=.5cm,inner sep=0pt},}

\definecolor{codegreen}{rgb}{0,0.5,0}
\definecolor{codewhite}{rgb}{1,1,1}
\definecolor{codegray}{rgb}{0.5,0.5,0.5}
\definecolor{codepurple}{rgb}{0.58,0,0.82}
\definecolor{codeblack}{rgb}{0,0,0}
\definecolor{codeorange}{rgb}{0.8,0.4,0}

\lstdefinestyle{mystyle}{
    backgroundcolor=\color{codewhite},   
    commentstyle=\color{codegray},
    keywordstyle=\color{codegreen},
    numberstyle=\color{codegray},
    stringstyle=\color{codeorange},
    basicstyle=\ttfamily ,
    breakatwhitespace=false,         
    breaklines=true,                 
    captionpos=b,                    
    keepspaces=true,                 
    numbers=left,                    
    numbersep=5pt,                  
    showspaces=false,                
    showstringspaces=false,
    showtabs=false,                  
    tabsize=4
}
\lstset{style=mystyle}


\setlength{\hoffset}{-1in}
\addtolength{\textwidth}{1.5in}
\setlength{\voffset}{-1in}
\addtolength{\textheight}{1.5in}
\newcommand{\be}{\begin{enumerate}}
\newcommand{\ee}{\end{enumerate}}
\newcommand{\BigO}[1]{\ensuremath\mathcal{O}\left(#1\right)}
\newcommand{\il}[1]{\lstinline!#1!}
\newcommand{\gnorm}[1]{\left|\left|#1\right|\right|}
\newcommand{\abs}[1]{\left|#1\right|}
\newcommand{\parens}[1]{\left(#1\right)}
\newcommand{\bracks}[1]{\left\{#1\right\}}
\newcommand{\sqbracks}[1]{\left[#1\right]}
\newcommand{\vep}{\varepsilon}
\newcommand{\ceiling}[1]{\left\lceil#1\right\rceil}
\newcommand{\R}{\mathbb{R}}
\newcommand{\N}{\mathbb{N}}
\newcommand{\Z}{\mathbb{Z}}
\newcommand{\C}{\mathbb{C}}
\newcommand{\F}{\mathbb{F}}
\newcommand{\distrib}[2]{\text{#1}\left(#2\right)}
\newcommand{\dd}[1]{\frac{d}{d#1}}
\newcommand{\abracks}[1]{\left< #1\right>}

\begin{document}
	\begin{center}
		\textbf{Fall 2020, Math 524:\ Homework 1} \\
		\textbf{Due:\ Wednesday, September 9th, 2020} \\
		\textbf{Joseph Diaz: 819947915}
	\end{center}
\noindent\makebox[\linewidth]{\rule{\paperwidth}{0.4pt}}
\be
	\item[1A.]
		\be
			\item[5.]
			Show that $\parens{\alpha + \beta} + \lambda = \alpha + \parens{\beta + \lambda},\ \forall \alpha,\beta,\lambda \in \C$.
			\begin{proof}
			Let $\alpha,\beta,\lambda \in \C$. Now as these are 
			complex numbers, they can all be rewritten as:
			$$\alpha = a+bi,\ \beta = c+di,\ \lambda = e+fi,\ a,b,c,d,e,f \in \R,\ i := \sqrt{-1}$$
			So we have that:
			\begin{align*}
			\parens{\alpha+\beta}+\lambda &= \parens{a + bi + c +di} 
			+ e + fi\\
			&= a + bi + c + di + e + fi\\
			&= a + bi + \parens{c + e} + \parens{d + f}i \\
			&= a + bi + \parens{\parens{c + e} + \parens{d + f}i} \\
			\parens{\alpha+\beta} + \lambda &= \alpha + \parens{\beta+\lambda}
			\end{align*}
			As desired.
			\end{proof}
			
			\item[6.] 
			Show that $\parens{\alpha\beta}\lambda = \alpha\parens{\beta\lambda},\ \forall \alpha,\beta,\lambda \in \C$.
			\begin{proof}
			Let $\alpha,\beta,\lambda \in \C$. Now as these are 
			complex numbers, they can all be rewritten as:
			$$\alpha = a+bi,\ \beta = c+di,\ \lambda = e+fi,\ a,b,c,d,e,f \in \R,\ i := \sqrt{-1}$$
			So we have that:
			\begin{align*}
			\parens{\alpha\beta}\lambda &= \parens{(a+bi)(c+di)}
			(e+fi)\\
			&= \parens{ac+adi+bci - bd}(e+fi) \\
			&= \parens{(ac-bd) + (ad+bc)i}(e+fi) \\
			&= (ac-bd)e + (ac-bd)fi + (ad+bc)ei - (ad+bc)f \\
			&= ace-bde - adf - bcf + acfi - bdfi + adei + bcei \\
			&= a(ce-df + cfi + dei) + bi(ce - df + cfi + dei)\\
			&= (a+bi)\parens{ce + cfi + dei - df} \\
			&= (a+bi)\parens{ce + cfi + dei + i^2df} \\
			&= (a+bi)\parens{(c+di)(e+fi)} \\
			\parens{\alpha\beta}\lambda &= \alpha\parens{\beta
			\lambda}
			\end{align*}
			As desired.
			\end{proof}
		\ee
	\item[1B.]
		\be
			\item[1.]
			Prove that $-(-v) = v,\ \forall v \in V$.
			\begin{proof}
			Let $V$ be vector space. As $V$ is a vector 
			space, we know that 
			$$\forall v \in V, \exists!v' \in V:\ v+v'=0$$
			We denote this additive inverse of $v$ as $v' = -v$, 
			so clearly
			$-v \in V$. This also implies that 
			$$\exists! w \in V:\ -v+w=0$$
			By the same convention, we denote this additive inverse
			as $w = -(-v)$. As we already knew that $v + (-v) = 0$,
			and that additive inverses are unique, we may conclude 
			that $w = -(-v) = v$. 
			\\\\So $v = -(-v),\ \forall v \in V$, as desired. 
			\end{proof}
			
			\item[3.]
			Suppose $v,\ w \in V$. Explain why 
			$\exists! x \in V$ such that $v + 3x = w$.
			\begin{proof}[Explanation]
			This explanation will consist of two parts. Let $V$ be 
			a vector space, and $v, w \in V$.\\
	        \textbf{How we know $x$ exists}:\\
	        As $V$ is a vector space, we know that $V$ is closed
	        under addition, subtraction, and scalar multiplication.
	        So:
	        $$v + 3x = w \implies x = \frac{1}{3}w - \frac{1}{3}v$$
	        Which means that, as an expression of $v$ and $w$, 
	        $x \in V$. \\
	        \textbf{How we know $x$ is unique}:\\
	        Again, given the equation that we have at the top, $x$ is 
	        uniquely defined by it's relationship with
	        $v$ and $w$; so we 
	        can confidently state that for any given $v$ and $w$ 
	        there is only one $x$ that satisfies the equation $v + 
	        3x = w$.  
			\end{proof}
			
		\ee	
	\item[1C.]
		\be 
			\item[10.]
			Suppose $U_1$ and $U_2$ are subspaces in $V$. Show that 
			$U_1 \cap U_2$ is a subspace of $V$.
			\begin{proof}
			First, we will make the assumption that $U_1 \cap U_2 
			\neq \emptyset$ and denote $U_1 \cap U_2 = W$.
			To show that $W$ is a subspace of $V$, it will be shown 
			that $W$ 
			satisfies the following conditions.\\\\
			\textbf{Additive Identity}:\\
			As $U_1$ and $U_2$ are subspaces of $V$, it follows that
			$0 \in U_1$ and $0 \in U_2$. Since they both contain $0$,
			this implies that their intersection does as well; and 
			so: $0 \in W$.
			
			\textbf{Closure under Addition}:\\
			Let $w_1, w_2 \in W$, then we also know that $w_1, w_2 
			\in U_1$ and that $w_1 + w_2 \in U_1$. By that same 
			token, this must imply that $w_1, w_2 \in U_2$ and 
			$w_1 + w_2 \in U_2$. As such, we may conclude that 
			$w_1 + w_2 \in W$; and so $W$ is closed under addition. 
			
			\textbf{Closure under Scalar Multiplication}:\\
			Let $w \in W$, and $\alpha$ be an element of the field
			that $V$ is over; so $w \in U_1$ and, as $U_1$ is a
			subspace of $V$, $\alpha w \in U_1$ as well. Likewise,
			$w \in U_2$ and $\alpha w \in U_2$ too. So we have that 
			$\alpha w \in W$, and $W$ is closed under scalar 
			multiplication.\\\\
			So $W = U_1 \cap U_2$ is a subspace of $V$. 
			\end{proof}
			
			 
			\item[20.]
			Suppose
			$$U = \bracks{(x, x, y, y) \in \F^4 \big|\ x,y
			\in \F}$$
			Find a subspace $W$ of $\F^4$ such that 
			$\F^4 = U \oplus W$.
			\begin{proof}[Solution]
			Theorem 1.45:\\Suppose $U$ and $W$ are subspaces of a
			vector space 
			$V$, then $$U + W \text{ is a direct sum} \iff U \cap 
			W = \bracks{0}$$
			So we need only find $W \subset \F^4$ such that 
			$U \cap W = \bracks{0}$.\\
			By construction, let 
			$$W = \bracks{\parens{x,0,y,0}\in \F^4 \big|\ x, y
			\in \F}$$
			Clearly $0 \in U \cap W$, by letting $x=0, 
			y=0$ in each subspaces' definition; but no other vectors
			exist in their intersection because of the 0's in each 
			element of $W$ by default. So $U + W$ is a direct sum, 
			and
			$$U \oplus W = \F^4$$   
			\end{proof}
			
		\ee 
\ee
\noindent\makebox[\linewidth]{\rule{\paperwidth}{0.4pt}}
	
\end{document}
