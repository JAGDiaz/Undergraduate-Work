\documentclass{article}
\usepackage{amsmath}
\usepackage{amssymb}
\usepackage{bm}
\usepackage{amsthm}
\usepackage{enumerate}
\usepackage{graphicx}
\usepackage{psfrag}
\usepackage{color}
\usepackage{url}
\usepackage{listings}
\usepackage{xcolor}
\usepackage{tikz}
\usetikzlibrary{positioning}
\tikzset{main node/.style={circle,fill=gray!20,draw,minimum size=.5cm,inner sep=0pt},}

\definecolor{codegreen}{rgb}{0,0.5,0}
\definecolor{codewhite}{rgb}{1,1,1}
\definecolor{codegray}{rgb}{0.5,0.5,0.5}
\definecolor{codepurple}{rgb}{0.58,0,0.82}
\definecolor{codeblack}{rgb}{0,0,0}
\definecolor{codeorange}{rgb}{0.8,0.4,0}

\lstdefinestyle{mystyle}{
    backgroundcolor=\color{codewhite},   
    commentstyle=\color{codegray},
    keywordstyle=\color{codegreen},
    numberstyle=\color{codegray},
    stringstyle=\color{codeorange},
    basicstyle=\ttfamily ,
    breakatwhitespace=false,         
    breaklines=true,                 
    captionpos=b,                    
    keepspaces=true,                 
    numbers=left,                    
    numbersep=5pt,                  
    showspaces=false,                
    showstringspaces=false,
    showtabs=false,                  
    tabsize=4
}
\lstset{style=mystyle}


\setlength{\hoffset}{-1in}
\addtolength{\textwidth}{1.5in}
\setlength{\voffset}{-1in}
\addtolength{\textheight}{1.5in}
\newcommand{\be}{\begin{enumerate}}
\newcommand{\ee}{\end{enumerate}}
\newcommand{\BigO}[1]{\ensuremath\mathcal{O}\left(#1\right)}
\newcommand{\il}[1]{\lstinline!#1!}
\newcommand{\norm}[1]{\left|\left|#1\right|\right|}
\newcommand{\abs}[1]{\left|#1\right|}
\newcommand{\parens}[1]{\left(#1\right)}
\newcommand{\bracks}[1]{\left\{#1\right\}}
\newcommand{\sqbracks}[1]{\left[#1\right]}
\newcommand{\vep}{\varepsilon}
\newcommand{\ceiling}[1]{\left\lceil#1\right\rceil}
\newcommand{\R}{\mathbb{R}}
\newcommand{\N}{\mathbb{N}}
\newcommand{\Z}{\mathbb{Z}}
\newcommand{\F}{\mathbb{F}}
\newcommand{\LT}[1]{\mathcal{L}\parens{#1}}
\newcommand{\poly}[2]{\mathcal{P}_{#1}\parens{#2}}
\newcommand{\range}[1]{\text{range }#1}
\renewcommand{\null}[1]{\text{null }#1}
\newcommand{\distrib}[2]{\text{#1}\left(#2\right)}
\newcommand{\dd}[1]{\frac{d}{d#1}}
\newcommand{\abracks}[1]{\left< #1\right>}

\begin{document}
	\begin{center}
		\textbf{Fall 2020, Math 524:\ Homework 7.1} \\
		\textbf{Due:\ Monday, November 16th, 2020} \\
		\textbf{Joseph Diaz: 819947915}
	\end{center}
\noindent\makebox[\linewidth]{\rule{\paperwidth}{0.4pt}}

\subsection*{7A}
\be
    \item[1.] Suppose $n \in \N$, and define $T \in \LT{\F^n}$ by 
    $$T\parens{z_1,\ \cdots\, z_n} = \parens{0,\ z_1,\ \cdots\, z_{n-1}}$$
    Find a formula for $T^*\parens{z_1,\ \cdots\, z_n}$.
    \begin{proof}[Answer]
    Let $x = (x_1, \cdots, x_n), y = (y_1, \cdots, y_n) \in \F^n$; 
    then, using the same process as example 7.13 in the text, we have that
    \begin{align*}
    \abracks{x, T^*y} &= \abracks{Tx, y} = \abracks{\parens{0, x_1,\cdots, x_{n-1}}, 
    \parens{y_1, y_2, \cdots, y_n}} \\
    &= (x_1y_2,\ x_2y_3,\ \cdots,\ x_{n-1}y_n) \\
    &= \abracks{(x_1, x_2, \cdots, x_n), 
    (y_2, y_3, \cdots, y_n, 0)} \\
    &= \abracks{x, (y_2, y_3, \cdots, y_n, 0)}
    \end{align*}
    This implies that
    $$T^*(y_1, y_2,\cdots, y_n) = (y_2, y_3, \cdots, y_n, 0)$$
    \end{proof} 

    \item[4.] Suppose $T \in \LT{V, W}$. Prove that
    \be[(a)]
        \item $T$ is injective if and only if $T^*$ is surjective.
        \begin{proof}
        We will prove this from the left and the right.
        \be
        \item[$\Longrightarrow$:] 
        Suppose that $T$ in injective, this means that
        $\null{T} = \bracks{0}$; then
        by theorem 7.7 in the text, we have that
        $$\null{T} = \parens{\range{T^*}}^\perp$$
        taking these together implies that
        $$\parens{\range{T^*}}^\perp = \bracks{0}$$
        Now, by the properties of orthongonal complements it is the 
        case that 
        $$\range{T^*} = V$$
        which means that $T^*$ is surjective, as it's image is $V$.

        \item[$\Longleftarrow$:] 
        Suppose that $T^*$ is surjective, this means that 
        $\range{T^*} = V$; then by the same theorem, we have 
        $$\range{T^*} = \parens{\null{T}}^\perp$$
        Again, taking these together gives 
        $$\parens{\null{T}}^\perp = V$$
        and, using the properties of orthongonal complements once again 
        $$\null{T} = \bracks{0}$$
        which mean that $T$ is injective.
        \ee
        \end{proof}

        \item $T$ is surjective if and only if $T^*$ is injective.
        \begin{proof}
        The proof for this is entirely identical to that of part ~(a); other than
        swapping $T$ and $T^*$, and swapping $\Longrightarrow$ and $\Longleftarrow$.
        \end{proof}
    \ee

    \item[6.] Make $\poly{2}{\R}$ into an inner product space by defining 
    $$\forall p, q \in \poly{2}{\R},\ \abracks{p,q} = \int_0^1 p(x)q(x)\ dx$$
    Define $T \in \LT{\poly{2}{\R}}$ by $T\parens{a_0 + a_1x + a_2x^2} = a_1x$.
    \be[(a)]
        \item Show that $T$ is not self-adjoint.
        \begin{proof}
        To show that $T$ is not self-adjoint, we'll find 2 polynomials $p, q \in 
        \poly{2}{\R}$ that do not satisfy
        $$\abracks{Tp, q} = \abracks{p, Tq}$$
        Let $p,q \in \poly{2}{\R}$, such that
        $$p(x) = 1 + x \qquad q(x) = 1 + x^2$$
        Now, for the left hand side we have
        $$\abracks{Tp, q} = \int_0^1 x(1+x^2)\ dx = \left.\frac{x^2}{2} + 
        \frac{x^4}{4}\right|_0^1 = \frac{3}{4}$$
        and for the right
        $$\abracks{p, Tq} = \int_0^1 (1+x)0\ dx = \int_0^1 0\ dx = 0$$
        Clearly, $0 \neq 3/4$; so $T$ cannot be a self-adjoint operator on 
        $\poly{2}{\R}$ with the current definition of the inner product.
        \end{proof}

        \item The matrix of $T$ with respect to $(1,x,x^2)$ is 
        $$\parens{\begin{matrix}
            0 & 0 & 0 \\
            0 & 1 & 0 \\
            0 & 0 & 0
        \end{matrix}}$$
        This matrix equals its conjugate transpose, even though $T$ is not self-adjoint.
        Explain why this is not a contradiction.
        \begin{proof}[Explanation]
        This not a contradiction, because the implication between the matrix of the 
        transformation and the self-adjoint nature of the operator
        is in the other direction. A self-adjoint
        operator will \emph{always} have a hermitian matrix of the transformation; but the 
        reverse isn't necessarily true and the $T$ given in part ~(a) is a good example of
        this. Further, the main reason that $T$ is not self-adjoint is because of the 
        definition of the inner product on $\poly{2}{\R}$. Given a different definition 
        of the inner product, $T$ \emph{could} be self-adjoint but would still have the
        same matrix of the transformation.
        \end{proof}
    \ee
    
    \item[14.] Suppose $T \in \LT{V}$ is a normal operator. Suppose also that $v,w \in V$ 
    satisfy the equations
    $$\norm{v} = \norm{w} = 2,\ Tv = 3v,\ Tw = 4w$$
    Show that $\norm{T(v+w)} = 10$.
    \begin{proof}
    We have that $Tv = 3v,\ Tw = 4w$. This means that $v$ and $w$
    are eigenvectors of $T$, and, due to $T$ being normal, $v$ and $w$ are 
    orthongonal. Using this fact, we get
    \begin{align*}
    \norm{T(v+w)}^2 &= \norm{Tv+Tw}^2 = \norm{3v+4w}^2 = \norm{3v}^2 + \norm{4w}^2 \\
    &= 3^2\norm{v}^2 + 4^2\norm{w}^2 = 9\cdot2^2 + 16\cdot2^2 = 100   
    \end{align*}
    by the Pythagorean Theorem; finally,
    $$\norm{T(v+w)}^2 = 100 \implies \norm{T(v+w)} = 10$$
    as desired.
    \end{proof}

\ee

\subsection*{7B}
\be
    \item[2.] Suppose that $T$ is a self-adjoint operator on a finite-dimensional inner
    product space and that 2 and 3 are the only eigenvalues of $T$. Prove that
    $T^2 - 5T +  6I = 0$.
    \begin{proof}
    Let $u$ and $v$ be the eigenvectors of $T$ corresponding to 2 and 3, respectively. This means that 
    $$(T - 2I)u = 0 \qquad (T-3I)v=0$$
    So neither $(T -2I)$ nor $(T-3I)$ are invertible. Notice also that
    $$(T-2I)(T-3I) = T^2 - 5T + 6I$$
    By theorem 7.26 in the text, we have that $T^2 - 5T + 6I$ is not invertible either 
    because $25 \not< 24$. So there exist non-zero vectors $x$ such that
    $$(T^2 - 5T + 6I)x = 0$$
    $x$ can be written as a linear combination of the eigenvectors of $T$, since it 
    is self-adjoint. This implies that any given $x$ is in $\null{\parens{T^2-5T+6I}}$, so 
    $T^2-5T+6I = 0$.
    \end{proof}  

    \item[6.] Prove that a normal operator on a complex inner product space is self-adjoint
    if and only if all its eigenvalues are real.
    \begin{proof}
    We will prove this from the left and from the right, but first we denote the complex
    inner product space as $\parens{V, \abracks{\cdot,\cdot}}$ where $\dim V = n \in \N$.
    \be
        \item[$\Longrightarrow$:]
        Since $T$ is self-adjoint, by theorem 7.13 in the text, all of it's eigenvalues 
        must be real.

        \item[$\Longleftarrow$:]
        Suppose that all of the eigenvalues of $T$ are real. Since $T$ is normal, 
        there exists an orthonormal basis of $V$ consisting of the eigenvectors of 
        $T$; by the Complex Spectral Theorem. Denote this basis of eigenvectors as
        $\bracks{e_1,\ \cdots,\ e_n}$ and their corresponding eigenvalues as 
        $\bracks{\lambda_1,\ \cdots,\ \lambda_n}$. By assumption, each of these 
        eigenvalues is real; and so the diagonal matrix that $T$ has with respect to
        $\bracks{e_1,\ \cdots,\ e_n}$ must have the eigenvalues on the main diagonal.
        As this matrix is a diagonal matrix with real entries, it equals it's own 
        conjugate transpose; and so $T$ must be self-adjoint.
    \ee
    \end{proof}
        
\ee

\noindent\makebox[\linewidth]{\rule{\paperwidth}{0.4pt}}
	
\end{document}
 