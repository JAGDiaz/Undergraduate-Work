\documentclass{article}
\usepackage{amsmath}
\usepackage{amssymb}
\usepackage{bm}
\usepackage{amsthm}
\usepackage{enumerate}
\usepackage{graphicx}
\usepackage{psfrag}
\usepackage{color}
\usepackage{url}
\usepackage{listings}
\usepackage{xcolor}
\usepackage{tikz}
\usetikzlibrary{positioning}
\tikzset{main node/.style={circle,fill=gray!20,draw,minimum size=.5cm,inner sep=0pt},}

\definecolor{codegreen}{rgb}{0,0.5,0}
\definecolor{codewhite}{rgb}{1,1,1}
\definecolor{codegray}{rgb}{0.5,0.5,0.5}
\definecolor{codepurple}{rgb}{0.58,0,0.82}
\definecolor{codeblack}{rgb}{0,0,0}
\definecolor{codeorange}{rgb}{0.8,0.4,0}

\lstdefinestyle{mystyle}{
    backgroundcolor=\color{codewhite},   
    commentstyle=\color{codegray},
    keywordstyle=\color{codegreen},
    numberstyle=\color{codegray},
    stringstyle=\color{codeorange},
    basicstyle=\ttfamily ,
    breakatwhitespace=false,         
    breaklines=true,                 
    captionpos=b,                    
    keepspaces=true,                 
    numbers=left,                    
    numbersep=5pt,                  
    showspaces=false,                
    showstringspaces=false,
    showtabs=false,                  
    tabsize=4
}
\lstset{style=mystyle}


\setlength{\hoffset}{-1in}
\addtolength{\textwidth}{1.5in}
\setlength{\voffset}{-1in}
\addtolength{\textheight}{1.5in}
\newcommand{\be}{\begin{enumerate}}
\newcommand{\ee}{\end{enumerate}}
\newcommand{\BigO}[1]{\ensuremath\mathcal{O}\left(#1\right)}
\newcommand{\il}[1]{\lstinline!#1!}
\newcommand{\gnorm}[1]{\left|\left|#1\right|\right|}
\newcommand{\abs}[1]{\left|#1\right|}
\newcommand{\parens}[1]{\left(#1\right)}
\newcommand{\bracks}[1]{\left\{#1\right\}}
\newcommand{\sqbracks}[1]{\left[#1\right]}
\newcommand{\vep}{\varepsilon}
\newcommand{\ceiling}[1]{\left\lceil#1\right\rceil}
\newcommand{\R}{\mathbb{R}}
\newcommand{\N}{\mathbb{N}}
\newcommand{\Z}{\mathbb{Z}}
\newcommand{\F}{\mathbb{F}}
\newcommand{\LT}{\mathcal{L}}
\newcommand{\poly}{\mathcal{P}}
\newcommand{\distrib}[2]{\text{#1}\left(#2\right)}
\newcommand{\dd}[1]{\frac{d}{d#1}}
\newcommand{\abracks}[1]{\left< #1\right>}

\begin{document}
	\begin{center}
		\textbf{Fall 2020, Math 524:\ Homework 5} \\
		\textbf{Due:\ Monday, October 26th, 2020} \\
		\textbf{Joseph Diaz: 819947915}
	\end{center}
\noindent\makebox[\linewidth]{\rule{\paperwidth}{0.4pt}}

\subsection*{5A}
\be
    \item[4.] Suppose that $T \in \LT\parens{V}$ and $U_1,\ \cdots,\
    U_m$ are subspaces of $V$ invariant under $T$. Prove that $U_1 +
    \cdots + U_m$ is invariant under $T$.
    \begin{proof}
    Suppose that $u \in U_1 + \cdots + U_m$, then there exists $u_1 \in U_1,
    \ \cdots,\ u_m \in U_m$ such that
    $$u = u_1 + \cdots + u_m$$
    Now, the application of $T$ on $u$ is
    \begin{align*}
    Tu &= T\parens{u_1 + \cdots + u_m} \\
    &= Tu_1 + \cdots + Tu_m
    \end{align*}
    Each $U_i$ is invariant under $T$, this means that $Tu_i \in U_i$, for 
    $1 \leq i \leq m$. This implies that $Tu \in U_1 + \cdots + U_m$, and 
    so $U_1 + \cdots + U_m$ is invariant under $T$.

    \end{proof}
    
    \item[8.] Define $T \in \LT\parens{\F^2}$ by
    $$T(w,z) = (z,w)$$
    Find all eigenvalues and eigenvectors of $T$.
    \begin{proof}[Solution]
    Let $\lambda$ be an eigenvalue of $T$, now we want to solve the system
    \begin{gather*}
    T(w,z) = \lambda(w,z) \\
    (z,w) = \lambda(w,z) \\
    \implies \begin{array}{c}
        z = \lambda w \\
        w = \lambda z
    \end{array}
    \end{gather*}
    Substituting $\lambda z$ for $w$ in the first equation gives $z = 
    \lambda^2 z$, then after dividing by $z$ we have that $\lambda = \pm 1$.
    For this reason we ignore the trivial case where $(w,z) = (0,0)$. Now, for the eigenvectors for each value:
    \be
        \item[$\lambda = 1$:]
        $$(z,w) = 1\cdot(w,z) \implies z = w$$
        So the corresponding eigenvectors are 
        $$\bracks{(a, a)\ \big|\ a \in \F}$$ 
        \item[$\lambda = -1$:] 
        $$(z,w) = -1\cdot(w,z) \implies z = -w$$
        So the corresponding eigenvectors are 
        $$\bracks{(a, -a)\ \big|\ a \in \F}$$ 
    \ee
    \end{proof}
    
    \item[10.](a) 
    Define $T \in \LT\parens{\F^n}$ by 
    $$T\parens{x_1,x_2, \cdots,x_{n-1},x_n} = 
    \parens{x_1, 2x_2,\cdots,(n-1)x_{n-1},nx_n}$$
    Find all eigenvalues and eigenvectors of $T$.
    \begin{proof}[Solution]
    To find the eigenvalues and vectors of $T$, we'll find the eigenvalues
    and vectors of $A = \mathcal{M}(T)$. Based on the definition of $T$;
    $$A = \sqbracks{\begin{matrix}
    1 & 0 & 0 & \cdots & 0 \\
    0 & 2 & 0 & \cdots & 0 \\
    \vdots & \ddots & \ddots & \ddots & \vdots \\
    0 & \ddots & \ddots & n-1 & 0 \\
    0 & \cdots & \cdots & 0 & n
    \end{matrix}}_{n \times n}$$
    $A$ is a diagonal matrix so solving $\det\parens{A-\lambda I_n} = 0$ is 
    simply 
    $$(1 - \lambda)(2-\lambda)\cdots(n-1 - \lambda)(n-\lambda) = 0$$
    and we have that eigenvalues of $T$ are $\lambda \in \bracks{1,\ \cdots,
    \ n}$. Now evaluating $\parens{A - \lambda_i I_n}\vec{u}_i = \vec{0}$
    to find the $i$-th eigenvector corresponding to the $i$-th eigenvalue; 
    we find that $\vec{u}_i = a \vec{e}_i$ where $a \in \F$ and 
    $\vec{e}_i$ is the $i$-th vector of the standard basis of $\F^n$.
    \end{proof}
    
        
\ee

\subsection*{5B}
\be
    \item[1.](a) 
    Suppose $T \in \LT\parens{V}$ and there exists a positive integer
    $n$ such that $T^n = 0$. Prove that $I - T$ is invertible and that
    $$(I-T)^{-1} = I + T + \cdots + T^{n-1}$$
    \begin{proof}
    First we consider
    \begin{align*}
    \parens{I + T + T^2 + \cdots + T^{n-1}}\parens{I-T} &= I + T + T^2 + 
    \cdots + T^{n-1} - T - T^2 - \cdots - T^{n-1} - T^n \\
    &= I - T^n \\
    &= I
    \end{align*}
    and also
    \begin{align*}
    \parens{I-T}\parens{I + T + T^2 + \cdots + T^{n-1}} &= I + T + T^2 + 
    \cdots + T^{n-1} - T - T^2 - \cdots - T^{n-1} - T^n \\
    &= I - T^n \\
    &= I
    \end{align*}
    This implies that $I - T$ is invertible and that
    $$(I-T)^{-1} = I + T + \cdots + T^{n-1}$$
    \end{proof}  
    
    \item[7.] 
    Suppose $T \in \LT\parens{V}$. Prove that $9$ is an eigenvalue of $T^2$
    if and only if $3$ or $-3$ is an eigenvalue of $T$.
    \begin{proof}
    We'll prove this from the left and right.
    \be 
        \item[$\Longrightarrow$:]
        We have that $9$ is eigenvalue of $T^2$, which is the same as 
        saying that there exists an eigenvector $\vec{v} \in V$ such that
        $$T^2\vec{v} = 9\vec{v}$$
        Rearranging the equations we get 
        $$\parens{T^2 - 9I}\vec{v} = \vec{0}$$
        By the definition of eigenvalues and vectors, this implies that 
        $\parens{T^2 - 9I}$ is not injective. Factoring this as
        $$\parens{T^2 - 9I} = \parens{T - 3I}\parens{T + 3I}$$
        indicates that either $\parens{T - 3I}$ or $\parens{T + 3I}$
        is not injective; which is equivalent to saying that $3$ or $-3$ 
        are eigennvalues of $T$.
        \item[$\Longleftarrow$:] 
        We have that $3$ or $-3$ are eigenvalues of $T$; so there 
        exists eigenvectors in $V$ corresponding to those that satisfy
        $$T\vec{v} = \lambda \vec{v}$$
        Applying $T$ to both sides of the equations gives
        $$T^2\vec{v} = T\parens{\lambda \vec{v}} = \lambda T\vec{v} = 
        \lambda^2 \vec{v}$$
        by the properties of eigenvectors. This implies that $3^2$ or $(-3)
        ^2$ are eigenvalues of $T^2$; both which equal 9.
    \ee
    \end{proof}
    
    \item[14.]
    Give an example of an operator whose matrix with respect to some basis
    contains only 0’s on the diagonal, but the operator is invertible.
    \begin{proof}[Solution]
    From \textbf{5A} problem 8, the operator $T \in \LT\parens{\F^2}$ such 
    that $T(w,z) = (z,w)$ has this property. 
    The matrix of the transformation with respect to the standard basis
    $$\mathcal{M}\parens{T} = \sqbracks{
        \begin{matrix}
            0 & 1 \\
            1 & 0
        \end{matrix}
    }$$
    has only 0's on it's main diagonal, but it is invertible. In fact:
    $$\sqbracks{
        \begin{matrix}
            0 & 1 \\
            1 & 0
        \end{matrix}}\cdot\sqbracks{
        \begin{matrix}
            0 & 1 \\
            1 & 0
        \end{matrix}} = \sqbracks{
        \begin{matrix}
            1 & 0 \\
            0 & 1
        \end{matrix}}$$
        which means that $T$ is it's own inverse.
    \end{proof}

    \item[15.]
    Give an example of an operator whose matrix with respect to some basis
    contains only nonzero numbers on the diagonal, but the operator is not
    invertible.
    \begin{proof}[Solution]
    Let $T \in \LT\parens{\F^3}$ be defined by $T(x,y,z) = 
    (x+y+z, x+y+z, z)$; this transformation has the desired property.
    The matrix of the transformation with respect to the standard basis
    $$\mathcal{M}\parens{T} = \sqbracks{
        \begin{matrix}
        1 & 1 & 1 \\
        1 & 1 & 1 \\
        0 & 0 & 1 \\
        \end{matrix}
    }$$
    has only non-zero elements on it's main diagonal; but it isn't 
    invertible because $T(-1,1,0) = \vec{0}$, which is to say that $T$ is 
    not injective.
    \end{proof}
        
\ee

\subsection*{5C}
\be
    \item[1.] 
    Suppose $T \in \LT\parens{V}$ is diagonalizable. Prove that $V = 
    \text{null }T \oplus \text{range }T$.
    \begin{proof}
    If $T$ is invertible, then we immediately have that $\text{null }T = 
    \{\vec{0}\}$ and $\text{range }T = V$; so $V = \text{null }T \oplus 
    \text{range }T$. For the case where $T$ isn't invertible; let 
    $\lambda_1,\ \cdots,\ \lambda_m$ be the distinct eigenvalues of $T$. 
    By 5.41 in the textbook, this means that 
    $$V=E\parens{\lambda_1,T} \oplus \cdots \oplus E\parens{\lambda_m,T}$$
    As $T$ is not invertible, one of the $\lambda$'s, say $\lambda_1$, must 
    be zero. So $E\parens{\lambda_1, T} = E\parens{0, T} = \text{null }T$.
    Further, for all $\vec{v}_i \in E\parens{\lambda_i, T},\ i \in 
    \bracks{2,\ \cdots,\ m}$, we have 
    $$T\parens{\frac{1}{\lambda_i}\vec{v}_i} = 
    \frac{1}{\lambda_i}T\vec{v}_i = \frac{\lambda_i}{\lambda_i}\vec{v}_i = 
    \vec{v}_i$$
    which is to say that $E\parens{\lambda_i, T} \subset \text{range }T$ 
    and
    $$E\parens{\lambda_2,T} \oplus \cdots \oplus E\parens{\lambda_m,T} 
    \subset \text{range }T$$
    because $\text{range }T$ is closed as a subspace of $V$. Conversely, 
    every $\vec{v} \in V$ can be written as
    $$\vec{v} = \vec{v}_1 + \vec{v}_2 + \cdots + \vec{v}_m$$
    where $\vec{v}_i \in E\parens{\lambda_i, T}$. Now, applying $T$ to 
    $\vec{v}$ gives 
    \begin{align*}
        T\vec{v} &= T\parens{\vec{v}_1 + \vec{v}_2 + \cdots + \vec{v}_m} \\
        &= T\vec{v}_1 + T\vec{v}_2 + \cdots + T\vec{v}_m \\
        &= \lambda_2\vec{v}_2 + \cdots + \lambda_m\vec{v}_m
    \end{align*}
    This means that $\text{range }T \subset E\parens{\lambda_2,T} \oplus 
    \cdots \oplus E\parens{\lambda_m,T}$ and that 
    $$\text{range }T = E\parens{\lambda_2,T} \oplus \cdots \oplus 
    E\parens{\lambda_m,T}$$
    So, finally, from the equation from 5.41
    \begin{align*}
    V &= E\parens{\lambda_1,T} \oplus E\parens{\lambda_2, T} \oplus \cdots 
    \oplus E\parens{\lambda_m,T}\\
    &= \text{null }T \oplus \text{range }T
    \end{align*}
    As desired.
    \end{proof}
    
    \item[2.] 
    Prove that the converse of the statement in Exercise 1 or give a 
    counterexample to the converse.
    \begin{proof}[Solution]
    Counterexample; let $T \in \LT\parens{\F^2}$ such that 
    $T(x,y) = (x+y, y)$, $T$ is invertible because
    $\text{null }T = \{\vec{0}\}$. This also implies that 
    $\text{range }T = \F^2$, so we have that $\F^2 = \text{null }T \oplus 
    \text{range }T$. However, there is no basis for $\F^2$ consisting of the
    eigenvectors of $T$ because $T$ only has a single eigenvector 
    $\vec{e}=(1,0)$. This violates one of the equivalent conditions for 
    diagonalizability in 5.41, so $T$ is not diagonalizable.
    \end{proof}
        
\ee
\noindent\makebox[\linewidth]{\rule{\paperwidth}{0.4pt}}
	
\end{document}
