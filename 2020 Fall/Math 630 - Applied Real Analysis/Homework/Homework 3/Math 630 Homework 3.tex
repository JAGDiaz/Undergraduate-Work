\documentclass{article}
\usepackage{amsmath}
\usepackage{amssymb}
\usepackage{bbm}
\usepackage{bm}
\usepackage{amsthm}
\usepackage{enumerate}
\usepackage{graphicx}
\usepackage{psfrag}
\usepackage{color}
\usepackage{url}
\usepackage{listings}
\usepackage{xcolor}
\usepackage{tikz}
\usetikzlibrary{positioning}
\tikzset{main node/.style={circle,fill=gray!20,draw,minimum size=.5cm,inner sep=0pt},}

\definecolor{codegreen}{rgb}{0,0.5,0}
\definecolor{codewhite}{rgb}{1,1,1}
\definecolor{codegray}{rgb}{0.5,0.5,0.5}
\definecolor{codepurple}{rgb}{0.58,0,0.82}
\definecolor{codeblack}{rgb}{0,0,0}
\definecolor{codeorange}{rgb}{0.8,0.4,0}

\lstdefinestyle{mystyle}{
    backgroundcolor=\color{codewhite},   
    commentstyle=\color{codegray},
    keywordstyle=\color{codegreen},
    numberstyle=\color{codegray},
    stringstyle=\color{codeorange},
    basicstyle=\ttfamily ,
    breakatwhitespace=false,         
    breaklines=true,                 
    captionpos=b,                    
    keepspaces=true,                 
    numbers=left,                    
    numbersep=5pt,                  
    showspaces=false,                
    showstringspaces=false,
    showtabs=false,                  
    tabsize=4
}
\lstset{style=mystyle}


\setlength{\hoffset}{-1in}
\addtolength{\textwidth}{1.5in}
\setlength{\voffset}{-1in}
\addtolength{\textheight}{1.5in}
\newcommand{\be}{\begin{enumerate}}
\newcommand{\ee}{\end{enumerate}}
\newcommand{\BigO}[1]{\ensuremath\mathcal{O}\left(#1\right)}
\newcommand{\il}[1]{\lstinline!#1!}
\newcommand{\norm}[1]{\left|\left|#1\right|\right|}
\newcommand{\abs}[1]{\left|#1\right|}
\newcommand{\parens}[1]{\left(#1\right)}
\newcommand{\bracks}[1]{\left\{#1\right\}}
\newcommand{\sqbracks}[1]{\left[#1\right]}
\newcommand{\vep}{\varepsilon}
\newcommand{\ceiling}[1]{\left\lceil#1\right\rceil}
\newcommand{\R}{\mathbb{R}}
\newcommand{\N}{\mathbb{N}}
\newcommand{\Z}{\mathbb{Z}}
\newcommand{\one}{\mathbbm{1}}
\newcommand{\A}{\mathcal{A}}
\newcommand{\distrib}[2]{\text{#1}\left(#2\right)}
\newcommand{\dd}[1]{\frac{d}{d#1}}
\newcommand{\abracks}[1]{\left< #1\right>}

\newtheorem*{proposition}{Proposition}

\begin{document}
	\begin{center}
		\textbf{Fall 2020, Math 630:\ Homework 3} \\
		\textbf{Due:\ Friday, September 18th, 2020} \\
		\textbf{Joseph Diaz: 819947915}
	\end{center}
\noindent\makebox[\linewidth]{\rule{\paperwidth}{0.4pt}}

\be[(E1)]
    \item 
    Let $\parens{X,d}$ be a metric space with $x,y,z \in X$. 
    Show that 
    $$\abs{d(x,z) - d(y,z)} \leq d(x,y)$$
    \begin{proof}
    From the triangle inequality on $d(x,y)$, we have
    $$d(x,y) \leq d(x,z) + d(z,y)$$
    now with some algebra,
    $$d(x,y) - d(z,y) \leq d(x,z)$$
    and again we use the triangle inequality, but on $d(x,z)$
    $$d(x,y) - d(z,y) \leq d(x,z) \leq d(x,y) + d(z,y)$$
    With some more algebra, we have
    \begin{gather*}
    \begin{array}{ccccc}
    d(x,y) - d(z,y) & \leq & d(x,z) & \leq & 
    d(x,y) + d(z,y) \\
    d(x,y) - 2d(z,y) & \leq & d(x,z) - d(z,y) & \leq & 
    d(x,y)  \\
    -d(x,y) + 2d(z,y) & \leq & d(x,z) - d(z,y) & \leq & 
    d(x,y)  \\
    -d(x,y) & \leq & d(x,z) - d(z,y) & \leq & 
    d(x,y)  \\
    \end{array}\\
    \implies \abs{d(x,z) - d(y,z)} \leq d(x,y)
    \end{gather*}
    which is what we wanted to show.
    \end{proof}
    
    \item Let $\parens{X, d_X}$ and $\parens{Y, d_Y}$ be metric 
    spaces. Show that the product space $X \times Y$ is a metric 
    space with metric $d$ defined by
    $$d(z_1, z_2) = d_X(x_1, x_2) + d_Y(y_1,y_2)$$
    where $z_1 = (x_1, y_1),\ z_2 = (x_2, y_2)$.
    \begin{proof}
    To show that $\parens{X\times Y, d}$ is a metric space, we'll 
    show that $d$ satisfies all of the properties of a metric. Let
    $z_1, z_2, z_2 \in X\times Y$, where $z_i = \parens{x_i, y_i},\
    x_i \in X, y_i \in Y$, then\\\\
    \textbf{Non-negativity and Identity}:\\
    The non-negativity property is trivial to show, 
    because $d$ is defined
    as the sum of $d_X$ and $d_Y$ which are metrics and so
    $$d_X(x_1, x_2) \geq 0,\ d_Y(y_1, y_2) \geq 0 \implies
    d(z_1, z_2) = d_X(x_1, x_2) + d_Y(y_1, y_2) \geq 0$$
    as desired. But the identity property will take more work, we'll
    show it from the left and right.
    \be
        \item[$\Longrightarrow$:]
        We have that $d(z_1, z_2) = 0$, which means that
        $d_X(x_1, x_2) + d_Y(y_1, y_2) = 0$. Both components of the
        sum are strictly non-negative so this implies that
        $$d_X(x_1, x_2) =0,\ d_Y(y_1, y_2) = 0$$
        and as they're both metrics this means that $x_1 = x_2,\ y_1
        = y_2$ and $z_1 = \parens{x_1,y_1} = \parens{x_2, y_2}
        = z_2$; which is what we wanted to show.
        
        \item[$\Longleftarrow$:]
        We have that $z_1 = \parens{x_1,y_1} = \parens{x_2, y_2}
        = z_2$, so $x_1 = x_2,\ y_1 = y_2$ as well. This means that 
        $d_X(x_1, x_2) =0,\ d_Y(y_1, y_2) = 0$. So, clearly
        $$d(z_1,z_2) = d_X(x_1, x_2) + d_Y(y_1, y_2) = 0$$
        as desired.
    \ee
    
    \textbf{Symmetry}:\\
    By the definition of $d$, we have
    $$d(z_1, z_2) = d_X(x_1, x_2) + d_Y(y_1, y_2)$$
    As $d_X$ and $d_Y$ are metrics, they have the symmetry property:
    $$d_X(x_1, x_2) = d_X(x_2, x_1),\ d_Y(y_1, y_2) = d_Y(y_2, y_1)$$
    So, this implies
    \begin{align*}
    d(z_1, z_2) &= d_X(x_1, x_2) + d_Y(y_1, y_2) \\
    &= d_X(x_2, x_1) + d_Y(y_2, y_1) \\
    &= d(z_2, z_1)
    \end{align*}
    which is what we wanted to show. 
    
    \textbf{Triangle Inequality}:\\
    By the definition of $d$, we have
    $$d(z_1, z_2) = d_X(x_1, x_2) + d_Y(y_1, y_2)$$
    As $d_X$ and $d_Y$ are metrics, they respect the triangle 
    inequality; $\forall x_1, x_2, x_3 \in X,\ y_1, y_2, y_3 \in Y$:
    \begin{gather*}
    d_X(x_1, x_2) \leq d_X(x_1, x_3) + d_X(x_3, x_2)\\ 
    d_Y(y_1, y_2) \leq d_Y(y_1, y_3) + d_Y(y_3, y_2) 
    \end{gather*}
    From this we have the following
    \begin{align*}
    d(z_1, z_2) &= d_X(x_1, x_2) + d_Y(y_1, y_2) \\
    &\leq d_X(x_1, x_3) + d_X(x_3, x_2) + d_Y(y_1, y_3) + 
    d_Y(y_3, y_2) \\
    &= \underbrace{d_X(x_1, x_3) + d_Y(y_1, y_3)}_{d(z_1, z_3)} + 
    \underbrace{d_X(x_3, x_2) + d_Y(y_3, y_2)}_{d(z_3, z_2)} \\
    &= d(z_1, z_3) + d(z_3, z_2)
    \end{align*}
    So $d$ also has the triangle inequality.\\\\
    $d$ satisfies the properties of a metric on $X \times Y$, so 
    $\parens{X \times Y,\ d}$ is a metric space.
    \end{proof}
        
    \item
    \be[1.] 
        \item Prove the following proposition:
        \begin{proposition}
        Let $f: [0, \infty) \to [0, \infty)$ be a continuously 
        differentiable function such that $f(0) = 0$ and $f'$ is
        non-negative and monotone decreasing. Then we have that
        $$\forall s,t \geq 0,\ 0 \leq f(s+t) \leq f(s) + f(t)$$
        and 
        $$\forall s,t \in [0, \infty), 0 \leq s \leq t,\ 0 \leq
        f(s) \leq f(t)$$ 
        \end{proposition}
        \begin{proof}
        We will prove the first statement, and then the second.\\\\
        \textbf{First}:\\
        By the definition of $f$, we have that $\forall s \in 
        [0, \infty),\ f(s) \geq 0$; and $f$ 
        is continuously differentiable on $[0,\infty)$, so by the 
        first fundamental theorem of calculus we have that
        $$0 \leq f(s) = \int_0^s f'(x)\ dx$$
        Now let $t \in [0, \infty)$, and we have
        \begin{align*}
        f(s+t) &= \int_0^{s+t} f'(x)\ dx \\
        &= \int_0^s f'(x)\ dx + \int_s^{s+t} f'(x)\ dx \\
        &= \int_0^s f'(x)\ dx + \int_0^{t} f'(x + s)\ dx 
        \end{align*}
        $f'$ is monotonically decreasing, so $f'(x + s) \leq f'(x)$,
        and $\int_a^b f'(x+s)\ dx \leq \int_a^b f'(x)\ dx$,\\ 
        $(a, b)\subseteq [0, \infty)$. Using this we have
        \begin{align*}
        f(s+t) &= \int_0^s f'(x)\ dx + \int_0^{t} f'(x + s)\ dx \\
        &\leq \int_0^s f'(x)\ dx + \int_0^{t} f'(x)\ dx \\
        &= f(s) + f(t)
        \end{align*}
        Consequently,
        $$\forall s,t \geq 0,\ 0 \leq f(s+t) \leq f(s) + f(t)$$
        which is what we wanted to show.
        
        \textbf{Second}:\\
        We have that $f'$ is non-negative on $[0,\infty)$, 
        this directly implies that $f$ is monotonically 
        increasing on
        $[0, \infty)$; in other words, 
        $$\forall s, t \in [0,\infty),\ s \leq t \implies f(s) \leq 
        f(t)$$
        In combination with the constraint $f(0) = 0$ and that 
        $f(s),f(t) \geq 0$, we may conclude that
        $$\forall s, t \in [0,\infty),\ 0 \leq s \leq t \implies 
        0 \leq f(s) \leq f(t)$$
        as desired.
        \end{proof}
        
        \item Let $\parens{X, \norm{\cdot}}$ be a normed linear 
        space. Use the previous proposition to show that
        $$d(x,y) = \frac{\norm{x - y}}{1 + \norm{x-y}}$$
        defines a metric on $X$.
        \begin{proof}
        To use the previous proposition, we'll define a new function 
        $f:[0,\infty) \to [0,\infty)$ such that
        $$f(t) = \frac{t}{1+t}$$
        This functions satisfies all of the desired properties:
        \begin{gather*}
        \forall t \geq 0, f'(t) \text{ exists} \\
        f(0) = 0 \\
        \ f'(t) = \frac{1}{(1+t)^2} \geq 0 \\
        \forall s \geq 0,\ f'(t) \geq f'(s + t)
        \end{gather*}
        Further, by the properties of norms, we can redefine $d$
        as 
        $$d(x,y) = f\parens{\norm{x-y}}$$
        and we'll use this to show that $d$ is a metric. Let
        $x, y, z \in X$, then:\\\\
        \textbf{Non-negativity and Identity}:\\
        The non-negativity property is trivial to show, as we know 
        that $f \geq 0$; but showing the identity property will 
        require more work, we'll show it from the left and right.
        \be 
            \item[$\Longrightarrow$:]
            We have that $d(x,y) = 0$, so:
            $$
            d(x,y) = \frac{\norm{x-y}}{1 + \norm{x-y}} = 0 
            \implies \norm{x-y}=0 \\
            $$
            By the properties of norms, $\norm{x-y} = 0$ implies 
            that $x=y$, as desired.
            
            
            \item[$\Longleftarrow$:]
            We have that $x = y$, so 
            $$
            d(x,y) = d(x,x) = f(\norm{x-x}) = f(0) = 0
            $$
            which is what we wanted to show.
        \ee
        
        \textbf{Symmetry}:\\
        By the definition of $d$, we have
        \begin{align*}
        d(x,y) &= \frac{\norm{x-y}}{1+\norm{x-y}} \\
        &= \frac{\norm{-(y-x)}}{1+\norm{-(y-x)}} \\
        &= \frac{\norm{y-x}}{1+\norm{y-x}} \\
        &= d(y,x)
        \end{align*}
        as desired.
        
        \textbf{Triangle Inequality}:\\
        To show this, we'll use the properties of $f$ from the
        proposition. Let $r = \norm{x-y}$, then by the triangle 
        inequality on norms we have that 
        $$r = \norm{x-y} \leq \norm{x-z} + \norm{z-y}$$
        Now let $s = \norm{x-z},\ t = \norm{z-y}$, so clearly
        $r \leq s + t$ and, by the second property of $f$, 
        $f(r) \leq f(s + t)$. Further, 
        $$
        f(r) \leq f\parens{s + t} \leq f(s) + f(t)
        $$
        by the first property. Now:
        \begin{align*}
        f(r) &\leq f(s)+ f(t) \\
        f\parens{\norm{x-y}} &\leq f\parens{\norm{x-z}} + 
        f\parens{\norm{z-y}} \\
        d(x,y) &\leq d(x,z) + d(z,y)
        \end{align*}
        Which means that $d$ satisfies the triangle inequality.\\\\
        So $d$ satisfies the definition of a metric, and $(X, d)$ is
        a metric space.
        
        \end{proof}
    \ee
    
    
    \item \textbf{British Railway non-stop metric}:\\
    Let $\parens{\R^n, \norm{\cdot}_2}$ be a normed 
    vectorspace. Show that $\parens{\R^n, d}$ is a metric space, with 
    $$\forall x,y \in \R^n,\ d(x,y) = \left\{\begin{array}{lc}
    \norm{x}_2 + \norm{y}_2 & \text{if }x \neq y \\
    0 & \text{if }x = y
    \end{array}
    \right.$$
    \begin{proof}
    We'll show that $d$ satisfies the properties of a metric on 
    $\R^n$, let $x,y,z \in \R^n$.\\\\
    \textbf{Non-negativity and Identity}:\\
    The non-negativity property is trivial to show, as we have that
    $d(x,y) = 0$ for $x=y$ and $d(x,y) = \norm{x}_2 + \norm{x}_2 
    > 0$ for $x \neq y$. Showing the identity property is also
    simple, we will show it from the left and right. 
    \be 
        \item[$\Longrightarrow$:]
        We have that $d(x,y) = 0$, by the definition of $d$
        $$d(x,y) = \norm{x}_2 + \norm{y}_2 = 0$$
        by the properties of norms, this implies that 
        $x=0$ and $y=0$; so $x=y$.
            
        \item[$\Longleftarrow$:]
        We have that $x=y$; so, by it's definition, 
        $d(x,y) = 0$.         
    \ee
        
    \textbf{Symmetry}:\\
    We'll consider 2 cases here. If $x = y$, then 
    $d(x,y) = d(y,x) = 0$. If $x \neq y$, then 
    \begin{align*}
    d(x,y) &= \norm{x}_2 + \norm{y}_2 \\
    &= \norm{y}_2 + \norm{x}_2 \\
    &= d(y,x)
    \end{align*}
    as desired.
    
    \textbf{Triangle Inequality}:\\
    We'll consider some cases here, as well.
    If $x = y$, then for all $z$ we have that
    $$d(x,y) = 0 \leq d(x,z) + d(z,y)$$
    by the properties shown above. In the case that $x \neq y$, we 
    have that
    \begin{align*}
    d(x,y) &= \norm{x}_2 + \norm{y}_2 \\
    &= \norm{x-z+z}_2 + \norm{y}_2 \\
    &\leq \norm{x}_2+\norm{z+z}_2 + \norm{y}_2 \\
    &= \norm{x}_2+\norm{2z}_2 + \norm{y}_2 \\
    &= \norm{x}_2+2\norm{z}_2 + \norm{y}_2 \\
    &= \norm{x}_2+\norm{z}_2 + \norm{z}_2 + \norm{y}_2 \\
    &= d(x,z) + d(z,y)
    \end{align*}     
    So, $d$ respects the triangle inequality.\\\\
    $d$ satisfies the properties of a metric on $\R^n$, 
    so $\parens{\R^n,d}$ is a metric space.
        
    \end{proof}
    
\ee
\noindent\makebox[\linewidth]{\rule{\paperwidth}{0.4pt}}
	
\end{document}
