\documentclass{article}
\usepackage{amsmath}
\usepackage{amssymb}
\usepackage{bm}
\usepackage{amsthm}
\usepackage{enumerate}
\usepackage{graphicx}
\usepackage{psfrag}
\usepackage{color}
\usepackage{url}
\usepackage{listings}
\usepackage{xcolor}
\usepackage{tikz}
\usepackage{verbatim}
\usetikzlibrary{positioning}
\tikzset{main node/.style={circle,fill=gray!20,draw,minimum size=.5cm,inner sep=0pt},}

% In line code stuff%
\definecolor{codegreen}{rgb}{0,0.5,0}
\definecolor{codewhite}{rgb}{1,1,1}
\definecolor{codegray}{rgb}{0.5,0.5,0.5}
\definecolor{codepurple}{rgb}{0.58,0,0.82}
\definecolor{codeblack}{rgb}{0,0,0}
\definecolor{codeorange}{rgb}{0.8,0.4,0}

\lstdefinestyle{mystyle}{
    backgroundcolor=\color{codewhite},   
    commentstyle=\color{codegray},
    keywordstyle=\color{codegreen},
    numberstyle=\color{codegray},
    stringstyle=\color{codeorange},
    basicstyle=\ttfamily ,
    breakatwhitespace=false,         
    breaklines=true,                 
    captionpos=b,                    
    keepspaces=true,                 
    numbers=left,                    
    numbersep=5pt,                  
    showspaces=false,                
    showstringspaces=false,
    showtabs=false,                  
    tabsize=4
}
\lstset{style=mystyle}
\setlength{\hoffset}{-1in}
\addtolength{\textwidth}{1.5in}
\setlength{\voffset}{-1in}
\addtolength{\textheight}{1.5in}

% Custom commands%
\newcommand{\be}{\begin{enumerate}}
\newcommand{\ee}{\end{enumerate}}
\newcommand{\BigO}[1]{\ensuremath\mathcal{O}\left(#1\right)}
\newcommand{\il}[1]{\lstinline!#1!}
\newcommand{\norm}[1]{\left|\left|#1\right|\right|}
\newcommand{\abs}[1]{\left|#1\right|}
\newcommand{\parens}[1]{\left(#1\right)}
\newcommand{\bracks}[1]{\left\{#1\right\}}
\newcommand{\sqbracks}[1]{\left[#1\right]}
\newcommand{\vep}{\varepsilon}
\newcommand{\ceiling}[1]{\left\lceil#1\right\rceil}
\newcommand{\R}{\mathbb{R}}
\newcommand{\N}{\mathbb{N}}
\newcommand{\Z}{\mathbb{Z}}
\newcommand{\F}{\mathbb{F}}
\newcommand{\C}{\mathbb{C}}
\newcommand{\A}{\mathcal{A}}
\newcommand{\poly}[2]{\mathcal{P}_{#1}\parens{#2}}
\newcommand{\cont}[3]{\mathcal{C}_{#1}^{#2}\parens{#3}}
\newcommand{\LT}[1]{\mathcal{L}\parens{#1}}
\newcommand{\distrib}[2]{\text{#1}\left(#2\right)}
\newcommand{\dd}[2]{\frac{d#1}{d#2}}
\newcommand{\abracks}[1]{\left< #1\right>}
\newcommand{\nullspace}[1]{\textup{null}\parens{#1}}
\newcommand{\range}[1]{\textup{range}\parens{#1}}
\newcommand{\one}{\mathbbm{1}}

\newtheorem*{theorem}{Theorem}

\begin{document}
	\begin{center}
		\textbf{Fall 2020, Math 630:\ Exam 2} \\
        \textbf{Due:\ Wednesday, November 25th, 2020} \\
		\textbf{Joseph Diaz: 819947915}
	\end{center}
\noindent\makebox[\linewidth]{\rule{\paperwidth}{0.4pt}}
\be[(E1)] 
    \item 
    Let $\parens{\cont{b}{}{\R}, \norm{\cdot}_\infty}$ be the set of continuous
    bounded functions over $\R$. Let the linear operator $T: \cont{b}{\null}{\R} 
    \to \cont{b}{\null}{\R}$ be defined as 
    $$\forall f \in \cont{b}{\null}{\R},\ \forall x \in \R,\ \parens{Tf}(x)
    = 3f(x) - 2f(x+4)$$
    Show that $T$ is continuous and find $\norm{T}$.
    \begin{proof}
    To show that $T$ is continuous, we'll show that $T$ bounded. To show that,
    we'll find $\norm{T}$. By Proposition 5.2.2, we have
    \begin{align*}
    \norm{T} &= \sup_{\substack{f \in \cont{b}{\null}{\R}\\ \norm{f}_\infty = 1}} 
    \norm{Tf}_\infty = \sup_{*} \max_{x\in \R}\abs{3f(x) - 2f(x+4)} \\
    &\leq \sup_{*} \max_{x\in\R}\parens{\abs{3f(x)} + \abs{2f(x+4)}} = \sup_{*}
    \parens{\max_{x\in\R}\abs{3f(x)} + \max_{x\in\R}\abs{2f(x+4)}} \\
    &= \sup_{*}\parens{3\max_{x\in\R}\abs{f(x)} + 2\max_{y\in\R}\abs{f(y)}} \qquad 
    \parens{\text{Letting }y = x+4}\\
    &= \sup_{*}\parens{3\norm{f}_\infty + 2\norm{f}_\infty} = \sup_{*}(3+2) \\
    &= 5
    \end{align*}
    where we use $*$ in place of $f\in\cont{b}{\null}{\R}$ and $\norm{f}_\infty = 1$.
    So we have that $\norm{T} \leq 5$. Now, let $g \in \cont{b}{\null}{\R}$ be defined 
    as
    $$g(x) = \left\{
    \begin{array}{ccl}
    1 & &\textup{if }x < -1\\
    -x & &\textup{if }-1 \leq x \leq 1 \\
    -1 & &\textup{if }x > 1
    \end{array}
    \right.$$
    We have that $\norm{g}_\infty = 1$; so for $g$ in particular, $\norm{T}$ is
    \begin{align*}
    \norm{T} &= \norm{Tg}_\infty = \max_{x\in\R}\abs{3g(x) - 2g(x+4)} \\
    &= \abs{3(1) - 2(-1)} \qquad (\textup{For }x \in (-3,-1)) \\
    &= \abs{3+2} = 5
    \end{align*}
    So we have that $\norm{T} = 5$. This also implies that 
    $$\forall f \in \cont{b}{\null}{\R},\ \norm{Tf}_\infty \leq 5\norm{f}_\infty$$
    and that $T$ is bounded. By Theorem 5.2.3, this means that $T$ is also
    continuous.
    \end{proof}

    \item Prove the following theorem.
    \begin{theorem}
    Let $\parens{M_1, \norm{\cdot}_{M_1}}$ and $\parens{M_2, \norm{\cdot}_{M_2}}$
    be two linear normed spaces and let $T: M_1 \to M_2$ be a linear operator.
    If $\exists C > 0,\ \forall x \in M_1,\ \norm{Tx}_{M_2} \geq C\norm{x}_{M_1}$,
    then $T$ has a continuous inverse $T^{-1}: \range{T} \to M_1$ and 
    $$\forall y \in \range{T},\ \norm{T^{-1}y}_{M_1} \leq \frac{1}{C}\norm{y}_{M_2}$$
    Furthermore:
    $$\norm{T^{-1}} \leq \frac{1}{C}$$
    \end{theorem}
    \begin{proof}
    First, we'll start by showing that $T$ injective. By way of contradiction,
    suppose that $T$ is not injective. This means that there exists $x\in M_1,\ 
    x \neq 0$ such that $Tx = 0$. By the condition we have on $T$, this means 
    that
    $$\norm{Tx}_{M_2} = \norm{0}_{M_2} = 0 \geq C\norm{x}_{M_1}$$
    But $C$ is strictly greater than $0$, so the only way that this inequality
    holds is when $x=0$, which contradicts our original assumption about $x$ 
    and that $T$ is not injective, so $T$ is injective. \\Now, $\range{T}$ is
    a linear subspace of $M_2$, so we can re-express $T$ as $T: M_1 \to 
    \range{T}$. Clearly, any operator is surjective onto it's own image,
    so $T$ is surjective as well. This means that $T$ is invertible and 
    there exists $T^{-1}: \range{T} \to M_1$ and 
    \begin{gather*}
    \forall x \in M_1,\ \exists!y\in \range{T},\ Tx=y \\
    \forall y \in \range{T},\ \exists! x\in M_1,\ T^{-1}y = x
    \end{gather*}
    Using this with our condition on $T$ we have:
    \begin{align*}
    \norm{Tx}_{M_2} &\geq C\norm{x}_{M_1} \\
    \norm{y}_{M_2} & \geq C\norm{T^{-1}y}_{M_1} \\
    \frac{1}{C}\norm{y}_{M_2} & \geq \norm{T^{-1}y}_{M_1}
    \end{align*}
    So we have that 
    $$\forall y \in \range{T},\ \norm{T^{-1}y}_{M_1} \leq \frac{1}{C}\norm{y}_{M_2}$$
    from which it immediately follows that $T^{-1}$ is bounded, and therefore continuous,
    and that $\norm{T^{-1}}\leq \frac{1}{C}$.
\end{proof}

    \item Let $M_1 = \R^2$ and $M_2 = \bracks{\bm{x} = (0, x_2)\ \big|\ x_2 \in \R} \subset M_1$,
    let $M_1$ and $M_2$ be equipped with $\norm{\cdot}_\infty$. Let the linear functionals $f: 
    M_2 \to \R$ and $f_{\bm{a}}: M_1 \to \R$ defined by 
    \begin{align*}
    \forall \bm{x} \in M_2,\ & f\parens{\bm{x}} = 2x_2 \\
    \forall \bm{a} \in \R^2,\ \forall \bm{x} \in M_1,\ & f_{\bm{a}}\parens{\bm{x}} 
    = a_1x_1 + a_2x_2
    \end{align*}
    where $\bm{x} = (x_1, x_2),\ \bm{a} = (a_1, a_2) \in \R^2$.
    \be[1.]
        \item Show that $\norm{f} = 2$ and $\norm{f_{\bm{a}}} = \norm{\bm{a}}_1$, where 
        $\forall \bm{x} \in \R^2,\ \norm{\bm{x}}_1 = \abs{x_1} + \abs{x_2}$.
        \begin{proof}
        We have that 
        \begin{align*}
        \norm{f} &= \sup_{\substack{\bm{x}\in M_2\\ \bm{x} \neq \bm{0}}}
        \frac{\abs{f(\bm{x})}}{\norm{\bm{x}}_\infty} = \sup_{*}
        \frac{\abs{2x_2}}{\max\bracks{0, \abs{x_2}}} \\
        &= \sup_{*}\frac{2\abs{x_2}}{\abs{x_2}} = 2\sup_{*} \\
        &= 2
        \end{align*}
        where we use $*$ in place of $\bm{x} \in M_2$ and $\bm{x}\neq\bm{0}$. So 
        $\norm{f} = 2$. Now, for $\norm{f_{\bm{a}}}$:
        \begin{align*}
        \norm{f_{\bm{a}}} &= \sup_{\substack{\bm{x}\in M_1 \\ \bm{x} \neq \bm{0}}}
        \frac{\abs{f_{\bm{a}}(\bm{x})}}{\norm{\bm{x}}_\infty} = \sup_{*}
        \frac{\abs{a_1x_1 + a_2x_2}}{\max\bracks{\abs{x_1},\abs{x_2}}} \\
        &= \sup_{*}\frac{\abs{a_1x_1 + a_2x_2}}{\abs{x_2}} \qquad 
        \parens{\text{Supposing }\abs{x_2} \geq \abs{x_1}} \\
        &\leq \sup_{*}\parens{\frac{\abs{a_1x_1}}{\abs{x_2}} + 
        \frac{\abs{a_2x_2}}{\abs{x_2}}} = \sup_{*}
        \parens{\frac{\abs{a_1}\abs{x_1}}{\abs{x_2}} + \abs{a_2}} \\
        &\leq \sup_{*}\parens{\abs{a_1} + \abs{a_2}} \qquad 
        \parens{\text{Because }\abs{x_1}/\abs{x_2} \leq 1} \\
        &= \abs{a_1} + \abs{a_2} = \norm{\bm{a}}_1
        \end{align*}
        So it is the case that $\norm{f_{\bm{a}}} \leq \norm{\bm{a}}_1$, now for 
        $\bm{x} = \parens{\text{sgn}(a_1),\ \text{sgn}(a_2)}$ we get 
        $$\norm{f_{\bm{a}}} = \frac{\abs{f_{\bm{a}}(\bm{x})}}{\norm{\bm{x}}_\infty} 
        = \frac{\abs{a_1\text{sgn}(a_1) + a_2\text{sgn}(a_2)}}
        {\max\bracks{\abs{\text{sgn}(a_1)}, \abs{\text{sgn}(a_2)}}}$$
        We have that $\bm{x} \neq \bm{0}$, so 
        $\max\bracks{\abs{\text{sgn}(a_1)}, \abs{\text{sgn}(a_2)}} = 1$ and
        \begin{align*}
        \norm{f_{\bm{a}}} &= \frac{\abs{a_1\text{sgn}(a_1) + 
        a_2\text{sgn}(a_2)}}{\max\bracks{\abs{\text{sgn}(a_1)}, 
        \abs{\text{sgn}(a_2)}}} \\
        &= \frac{\abs{a_1} + \abs{a_2}}{1} = \abs{a_1} + \abs{a_2} \\
        &= \norm{\bm{a}}_1
        \end{align*}
        So, $\norm{f_{\bm{a}}} = \norm{\bm{a}}_1$.
        \end{proof}

        \item Find $\bm{a}$ which provides that $f_{\bm{a}}$ is the unique Hahn-Banach extension
        of $f$.
        \begin{proof}
        We wish to find $\bm{a}$ such that $\forall \bm{x} \in M_2,\ f_{\bm{a}}(\bm{x}) = 
        f(\bm{x})$ and 
        $\norm{f_{\bm{a}}} = \norm{f}$. Clearly, $\bm{a} = (0,2)$ satisfies this; since:
        \begin{gather*}
        \forall \bm{x} \in M_2, f_{\bm{a}}(\bm{x}) = 0\cdot x_1 + 2x_2 = 2x_2 = f(\bm{x}) \\
        \norm{f_{\bm{a}}} = \abs{0} + \abs{2} = 2 = \norm{f}
        \end{gather*}
        \end{proof}
    \ee

    \item Let $M$ be a vector space defined by 
    $$M = \bracks{f: \R \to \C\ \middle|\ f\text{ is continuous, and }\sum_{k \in \Z} 
    \norm{f\chi_{[k, k+1)}}_\infty < \infty}$$
    where $\chi_\Omega(x)$ is the indicator function on $\Omega$.
    \be[1.]
        \item Show that the quantity $\norm{\cdot}_M$ defined by 
        $$\forall f \in M,\ \norm{f}_M = \sum_{k\in\Z}\norm{f\chi_{[k, k+1)}}_\infty$$
        defines a norm on $M$.
        \begin{proof}
        We'll show that $\norm{\cdot}_M$ satisfies the properties of a norm 
        on $M$, let $f,g \in M$, $\alpha \in \C$, and, for brevity, denote $\forall k\in \Z,\ 
        I_k = [k,k+1)$.\\\\
        \textbf{Non-negativity and Identity}:\\
        As a sum of infinity norms, clearly $\norm{\cdot}_M$ is non-negative. Now,
        we'll show the identity property from the left and from the right:
        \be
            \item[$\Longrightarrow$:].
            Suppose that $\forall x\in  \R,\ f(x) = 0$, then 
            \begin{align*}
            \norm{f}_M &= \sum_{k\in\Z}\norm{f\chi_{I_k}}_\infty = 
            \sum_{k\in\Z}\max_{x\in \R}\abs{f(x)\chi_{I_k}(x)} \\
            &= \sum_{k\in\Z}\max_{x\in\R}\abs{0\chi_{I_k}} = 
            \sum_{k\in\Z}\max_{x\in\R}\abs{0} \\
            &= \sum_{k\in\Z}0 = 0
            \end{align*}
            so $\norm{f}_M = 0$.

            \item[$\Longleftarrow$:] 
            Suppose that $\norm{f}_M = 0$, then
            \begin{align*}
            0 &= \sum_{k\in\Z}\norm{f\chi_{I_k}}_\infty = \sum_{k\in\Z}
            \max_{x\in\R}\abs{f(x)\chi_{I_k}(x)} \\
            &= \sum_{k\in\Z}\sup_{x\in I_k}\abs{f(x)}
            \end{align*}
            We go from the end of the first line to the second by noticing that for 
            $x \notin I_k, \chi_{I_k}(x) = 0$, so we are essentially trying to find 
            $\max_{x\in I_k}\abs{f(x)}$, but since each $I_k$ is not a closed interval, 
            it isn't necessarily the case that $f$ attains it's maximum on $I_k$, so 
            $$\forall k \in \Z,\ \max_{x\in \R}\abs{f(x)\chi_{I_k}(x)} = 
            \sup_{x\in I_k}\abs{f(x)}$$
            As a sum of non-negative real numbers, this means that
            $$\sum_{k\in\Z}\sup_{x\in I_k}\abs{f(x)} = 0 \implies 
            \forall k \in \Z,\ \sup_{x\in I_k}\abs{f(x)} = 0$$
            which ultimately implies that $\forall x\in\R,\ f(x) = 0$.
        \ee

        \textbf{Homogeneity}:\\
        We have that 
        \begin{align*}
        \norm{\alpha f}_M &= \sum_{k\in\Z}\norm{\alpha f\chi_{I_k}}_\infty = 
        \sum_{k\in\Z}\max_{x\in \R}\abs{\alpha f(x)\chi_{I_k}(x)} \\
        &= \sum_{k\in\Z}\max_{x\in\R}\abs{\alpha}\abs{f(x)\chi_{I_k}(x)} =
        \sum_{k\in\Z}\abs{\alpha}\max_{x\in\R}\abs{f(x)\chi_{I_k}(x)} \\
        &= \abs{\alpha}\sum_{k\in\Z}\max_{x\in\R}\abs{f(x)\chi_{I_k}(x)} = 
        \abs{\alpha}\sum_{k\in\Z}\norm{f\chi_{I_k}}_\infty \\
        &= \abs{\alpha}\norm{f}_M
        \end{align*}

        \textbf{Triangle Inequality}:\\
        We have that
        \begin{align*}
        \norm{f+g}_M &= \sum_{k\in\Z}\norm{(f+g)\chi_{I_k}}_\infty = 
        \sum_{k\in\Z}\max_{x\in\R}\abs{(f+g)(x)\chi_{I_k}(x)} \\
        &= \sum_{k\in\Z}\max_{x\in\R}\abs{f(x)\chi_{I_k}(x) + g(x)\chi_{I_k}(x)} \\
        &\leq \sum_{k\in\Z}\max_{x\in\R}\parens{\abs{f(x)\chi_{I_k}(x)} + 
        \abs{g(x)\chi_{I_k}(x)}} \\
        &= \sum_{k\in\Z}\parens{\max_{x\in\R}\abs{f(x)\chi_{I_k}(x)} + 
        \max_{x\in\R}\abs{g(x)\chi_{I_k}(x)}} \\
        &= \sum_{k\in\Z}\max_{x\in\R}\abs{f(x)\chi_{I_k}(x)} + 
        \sum_{k\in\Z}\max_{x\in\R}\abs{g(x)\chi_{I_k}(x)} \\
        &= \sum_{k\in\Z}\norm{f\chi_{I_k}}_\infty + 
        \sum_{k\in\Z}\norm{g\chi_{I_k}}_\infty \\
        &= \norm{f}_M + \norm{g}_M
        \end{align*}
        So $\norm{\cdot}_M$ respects the triangle inequality.\\\\
        Therefore, $\norm{\cdot}_M$ satisfies the properties of a norm on $M$.
        \end{proof}

        \item Does the function $\forall x \in \R,\ f(x) = e^x$ belong to $M$?
        \begin{proof}[Answer]
        No, $f(x) = e^x \notin M$; because it doesn't satisfy the property that 
        $\norm{f}_M$ is finite. 
        \begin{align*}
        \norm{f}_M &= \sum_{k\in\Z}\norm{f\chi_{I_k}}_\infty = 
        \sum_{k\in\Z}\max_{x\in\R}\abs{f(x)\chi_{I_k}(x)} \\
        &= \sum_{k\in\Z}\max_{x\in\R}\abs{e^x\chi_{I_k}(x)} =
        \sum_{k\in\Z}\sup_{x\in I_k}\abs{e^x} \\
        &= \sum_{k\in\Z}\sup_{x\in I_k}e^x
        \end{align*}
        $e^x$ is a monotonically increasing function, so it's supremum is at the 
        right most endpoint of the interval of interest; this gives
        $$\norm{f}_M = \sum_{k\in\Z}\sup_{x\in I_k}e^x = \sum_{k\in\Z}e^{k+1}$$
        However, $\lim_{k\to\infty}e^{k+1} \to \infty$; so clearly this sum diverges 
        to $\infty$ even before considering negative values of $k$. Therefore, 
        $f(x) = e^x \notin M$.
        \end{proof}
    \ee
\ee
\noindent\makebox[\linewidth]{\rule{\paperwidth}{0.4pt}}
	
\end{document}
