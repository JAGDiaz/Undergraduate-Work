\documentclass{article}
\usepackage{amsmath}
\usepackage{amssymb}
\usepackage{bbm}
\usepackage{bm}
\usepackage{amsthm}
\usepackage{enumerate}
\usepackage{graphicx}
\usepackage{psfrag}
\usepackage{color}
\usepackage{url}
\usepackage{listings}
\usepackage{xcolor}
\usepackage{tikz}
\usetikzlibrary{positioning}
\tikzset{main node/.style={circle,fill=gray!20,draw,minimum size=.5cm,inner sep=0pt},}

\definecolor{codegreen}{rgb}{0,0.5,0}
\definecolor{codewhite}{rgb}{1,1,1}
\definecolor{codegray}{rgb}{0.5,0.5,0.5}
\definecolor{codepurple}{rgb}{0.58,0,0.82}
\definecolor{codeblack}{rgb}{0,0,0}
\definecolor{codeorange}{rgb}{0.8,0.4,0}

\lstdefinestyle{mystyle}{
    backgroundcolor=\color{codewhite},   
    commentstyle=\color{codegray},
    keywordstyle=\color{codegreen},
    numberstyle=\color{codegray},
    stringstyle=\color{codeorange},
    basicstyle=\ttfamily ,
    breakatwhitespace=false,         
    breaklines=true,                 
    captionpos=b,                    
    keepspaces=true,                 
    numbers=left,                    
    numbersep=5pt,                  
    showspaces=false,                
    showstringspaces=false,
    showtabs=false,                  
    tabsize=4
}
\lstset{style=mystyle}


\setlength{\hoffset}{-1in}
\addtolength{\textwidth}{1.5in}
\setlength{\voffset}{-1in}
\addtolength{\textheight}{1.5in}
\newcommand{\be}{\begin{enumerate}}
\newcommand{\ee}{\end{enumerate}}
\newcommand{\BigO}[1]{\ensuremath\mathcal{O}\left(#1\right)}
\newcommand{\il}[1]{\lstinline!#1!}
\newcommand{\norm}[1]{\left|\left|#1\right|\right|}
\newcommand{\abs}[1]{\left|#1\right|}
\newcommand{\parens}[1]{\left(#1\right)}
\newcommand{\bracks}[1]{\left\{#1\right\}}
\newcommand{\sqbracks}[1]{\left[#1\right]}
\newcommand{\vep}{\varepsilon}
\newcommand{\ceiling}[1]{\left\lceil#1\right\rceil}
\newcommand{\R}{\mathbb{R}}
\newcommand{\N}{\mathbb{N}}
\newcommand{\Z}{\mathbb{Z}}
\newcommand{\one}{\mathbbm{1}}
\newcommand{\A}{\mathcal{A}}
\newcommand{\distrib}[2]{\text{#1}\left(#2\right)}
\newcommand{\dd}[1]{\frac{d}{d#1}}
\newcommand{\abracks}[1]{\left< #1\right>}

\newtheorem*{proposition}{Proposition}

\begin{document}
	\begin{center}
		\textbf{Fall 2020, Math 630:\ Homework 5} \\
		\textbf{Due:\ Friday, October 16th, 2020} \\
		\textbf{Joseph Diaz: 819947915}
	\end{center}
\noindent\makebox[\linewidth]{\rule{\paperwidth}{0.4pt}}

\be[(E1)]
    \item Let $A = \bracks{\frac{1}{n}\ \big|\ n \in \N}$
    \be[1.]
        \item Find the cluster points of $A$.
        \begin{proof}[Answer]
        We know that $\bracks{\frac{1}{n}}_{n\in\N}$ converges to 0, 
        so any convergent subsequence of $A$ must also converge to 0;
        so 0 is the only cluster point that $A$ has.
        \end{proof}
        
        \item Is $A$ closed? Give the closure of $A$.
        \begin{proof}[Answer]
        $A$ is not closed because 0 is a cluster point of $A$; but 
        $0 \notin A$. Therefore, the closure of $A$ is simply
        $$\overline{A} = A \cup \bracks{0}$$
        \end{proof}
        
        \item Is $A$ dense in $\R$ (equipped with the usual metric)?
        \begin{proof}
        For $A$ to be dense in $\R$, it would have to be the case 
        that $\overline{A} = \R$; which is equivalent to saying that 
        every element of $\R$ has
        a corresponding sequence in $A$ that converges to that 
        element. That isn't the case because every sequence of 
        $A$ converges to 0, so $\overline{A} \neq \R$ and $A$ is not
        dense in $\R$.
        \end{proof}
    
    \ee
    
        
    \item Are the following sets $A$ compact in $X$?
    \be[1.]
        \item $A = \sqbracks{a,b},\ a,b \in \R$ and $X = \R$ is 
        equipped with the discrete metric 
        $$d(x,y) = \left\{\begin{array}{ccl}
        0 & & \text{if } x = y\\
        1 & & \text{otherwise}        
        \end{array}\right.$$
        \begin{proof}
        $A$ is not compact in $X$, because is not totally bounded. 
        Take an arbitrary open ball centered at $x \in A$:
        $$B_\vep(x) = \bracks{y \in A\ \big|\ d(x,y) < \vep}$$
        For an arbitrary $\vep$, the open ball $B_\vep(x)$ must only 
        contain $x$. So the collection of open balls 
        $\bracks{\bracks{x}\ \big|\
        x \in A}$ is an open cover of $A$; but since every open ball
        is just a singleton, it's impossible to get a \emph{finite} 
        subcover $\mathfrak{F}$ of $A$. Any finite collection 
        would be missing elements of $A$ and so it's not totally 
        bounded with respect to the discrete metric and also not 
        compact in $X$.
        \end{proof}
        
        \item $A = \bracks{\parens{x,y} \in \R^2\ \big|\ x^2 + y^2 
        = 1}$ and $X = \R^2$ equipped with the euclidean metric.
        \begin{proof}
        $A$ is compact in $X$. To show this, we'll show that $A$ 
        is both closed and bounded.\\
        \textbf{Closedness}:\\
        To show that $A$ is closed, we'll show that $A^c = \R^2 
        \backslash A$ is open. We have that
        $$A^c = \bracks{\parens{x,y} \in \R^2\ \big|\ x^2 + y^2 
        \neq 1}$$
        $A^c$ is the union of 2 sets $C_1,\ C_2$ such that
        \begin{gather*}
        C_1 = \bracks{\parens{x,y} \in \R^2\ \big|\ x^2 + y^2 
        < 1}\\
        C_2 = \bracks{\parens{x,y} \in \R^2\ \big|\ x^2 + y^2 
        > 1}
        \end{gather*}
        We're using the euclidean metric $d(u, v) = \norm{u - v}_2,\ 
        u,v \in \R^2$, so $C_1 = B_1\parens{(0,0)}$ the open ball 
        centered at $(0,0)$ of radius 1. Therefore, $C_1$ is open. 
        $C_2$ is also open, because we have that
        $$\forall w \in C_2,\ \vep = \norm{w}_2 - 1,\ B_\vep
        \parens{w}\subset C_2$$
        The union of open sets is an open set, so $A^c = C_1 \cup 
        C_2$ is open and $A$ is closed.\\
        \textbf{Boundedness}:\\
        By it's definition, $A$ is bounded because 
        $$\forall u,v \in A, d(u, v) \leq 2$$
        as $A$ is the unit circle centered at the
        origin.\\\\
        $A$ is both closed and bounded, therefore $A$ is compact in 
        $X$.
        \end{proof}
        
        \item $A = \bracks{\parens{x,y} \in \R^2\ \big|\ x^2 + y^2 
        > 1}$ and $X = \R^2$ equipped with the euclidean metric.
        \begin{proof}
        $A$ is not compact in $X$, because $A$ is the ``area'' 
        outside of the unit circle centered at the origin; therefore,
        it is unbounded.
        \end{proof}
    
    \ee
    
    
    \item Show that $A = \bracks{\frac{1}{n}\ \big|\ n \in \N}$ is 
    not compact \underline{by using open covers.}
    \begin{proof}
    To show that $A$ is not compact, we'll find an open cover of $A$ 
    that has no finite subcover. Let $\bracks{C_n}$ be a collection 
    of open intervals in $\R$ such that 
    $$C_n = \parens{\frac{9}{10}\cdot\frac{1}{n},\ 
    \frac{11}{10}\cdot\frac{1}{n}}$$
    The midpoint of each interval is
    $$\frac{\frac{9}{10}\cdot\frac{1}{n} + 
    \frac{11}{10}\cdot\frac{1}{n}}{2} = \frac{1}{n}\cdot
    \frac{\frac{9}{10}+\frac{11}{10}}{2} = \frac{1}{n}\cdot
    \frac{\frac{20}{10}}{2} = \frac{1}{n}$$
    In other words, 
    $$\forall n \in \N,\ \frac{1}{n} \in C_n$$
    This means that the union of all of the $C$'s is an open cover
    of $A$:
    $$A \subset \bigcup_{n=1}^\infty C_n$$
    However, each $C$ contains exactly one element of $A$ so any
    subcover made up of a finite number of $C$'s would necessarily be
    missing elements of $A$. This open cover has no finite subcover, 
    so $A$ is not compact.  
    \end{proof}
    
    
    \item Show that the union of two compact sets is compact.
    \begin{proof}
    Let $A,\ B$ be two compact sets; as compact sets they are 
    both closed and totally bounded. This implies that there exist 
    finite subcovers $\mathfrak{F}_A = \{A_i\}_{i=1}^n$ and 
    $\mathfrak{F}_B = \{B_i\}_{i=1}^m$ of $A$ and $B$ 
    such that
    $$A \subset \bigcup_{i=1}^n A_i,\ B \subset 
    \bigcup_{i=1}^m B_i$$
    Let $C = A \cup B$; as $C$
    is the union of a finite collection of closed sets, $C$ is 
    closed. Further, from above we have that
    $$C = A \cup B \subset \bigcup_{i=1}^n A_i \cup
    \bigcup_{i=1}^m B_i$$
    As $\mathfrak{F}_A$ and $\mathfrak{F}_B$ are finite subcovers 
    the union of these collections, which we'll denote 
    $\mathfrak{F}_C$, is a finite subcover of $C$. This means that 
    $C$ is closed and totally bounded, so $C$ is compact.
    \end{proof}
    
    
    \item Let $\R$ be equipped with the metric $\forall x,y \in \R,\ 
    d(x,y) = \abs{\arctan x - \arctan y}$. Show that $\R$ is not 
    sequentially compact with respect to that metric.
    \begin{proof}
    To show that $\parens{\R, d}$ is not sequentially compact, we'll 
    show that there exists a sequence in $\R$ that does not have 
    any converging subsequences. Let $\bracks{x_n}$ such that 
    $x_n = n,\ \forall n \in \N$ be a sequence in $\R$, then let 
    $\bracks{x_{n_k}}$ be a subsequence of $\bracks{x_n}$; then we 
    have that 
    $$\forall y \in \R, \lim_{k\to\infty}d\parens{x_{n_k}, y} = 
    \lim_{k\to\infty}d\parens{n_k, y} = \lim_{k\to\infty}
    \abs{\arctan n_k - \arctan y}$$
    $x_n \to\infty$ as $n \to \infty$, so $n_k \to \infty$ as $k \to
    \infty$ and 
    $$\lim_{k\to\infty}\abs{\arctan n_k - \arctan y} = 
    \lim_{k\to\infty}\abs{\frac{\pi}{2} - \arctan y} > 0$$
    This implies that $\bracks{x_n}$ has no convergent subsequences, 
    and therefore $\parens{\R, d}$ is not sequentially compact.
    \end{proof}
    
\ee
\noindent\makebox[\linewidth]{\rule{\paperwidth}{0.4pt}}
	
\end{document}
