\documentclass{article}
\usepackage{amsmath}
\usepackage{amssymb}
\usepackage{bbm}
\usepackage{bm}
\usepackage{amsthm}
\usepackage{enumerate}
\usepackage{graphicx}
\usepackage{psfrag}
\usepackage{color}
\usepackage{url}
\usepackage{listings}
\usepackage{xcolor}
\usepackage{tikz}
\usetikzlibrary{positioning}
\tikzset{main node/.style={circle,fill=gray!20,draw,minimum size=.5cm,inner sep=0pt},}

\definecolor{codegreen}{rgb}{0,0.5,0}
\definecolor{codewhite}{rgb}{1,1,1}
\definecolor{codegray}{rgb}{0.5,0.5,0.5}
\definecolor{codepurple}{rgb}{0.58,0,0.82}
\definecolor{codeblack}{rgb}{0,0,0}
\definecolor{codeorange}{rgb}{0.8,0.4,0}

\lstdefinestyle{mystyle}{
    backgroundcolor=\color{codewhite},   
    commentstyle=\color{codegray},
    keywordstyle=\color{codegreen},
    numberstyle=\color{codegray},
    stringstyle=\color{codeorange},
    basicstyle=\ttfamily ,
    breakatwhitespace=false,         
    breaklines=true,                 
    captionpos=b,                    
    keepspaces=true,                 
    numbers=left,                    
    numbersep=5pt,                  
    showspaces=false,                
    showstringspaces=false,
    showtabs=false,                  
    tabsize=4
}
\lstset{style=mystyle}


\setlength{\hoffset}{-1in}
\addtolength{\textwidth}{1.5in}
\setlength{\voffset}{-1in}
\addtolength{\textheight}{1.5in}
\newcommand{\be}{\begin{enumerate}}
\newcommand{\ee}{\end{enumerate}}
\newcommand{\BigO}[1]{\ensuremath\mathcal{O}\left(#1\right)}
\newcommand{\il}[1]{\lstinline!#1!}
\newcommand{\norm}[1]{\left|\left|#1\right|\right|}
\newcommand{\abs}[1]{\left|#1\right|}
\newcommand{\parens}[1]{\left(#1\right)}
\newcommand{\bracks}[1]{\left\{#1\right\}}
\newcommand{\sqbracks}[1]{\left[#1\right]}
\newcommand{\vep}{\varepsilon}
\newcommand{\ceiling}[1]{\left\lceil#1\right\rceil}
\newcommand{\R}{\mathbb{R}}
\newcommand{\N}{\mathbb{N}}
\newcommand{\Z}{\mathbb{Z}}
\newcommand{\C}{\mathbb{C}}
\newcommand{\mC}{\mathcal{C}}
\newcommand{\hilbert}{\mathcal{H}}
\newcommand{\one}{\mathbbm{1}}
\newcommand{\A}{\mathcal{A}}
\newcommand{\distrib}[2]{\text{#1}\left(#2\right)}
\newcommand{\dd}[1]{\frac{d}{d#1}}
\newcommand{\tr}[1]{\text{tr}\parens{#1}}
\newcommand{\LT}{\mathcal{L}}
\newcommand{\abracks}[1]{\left< #1\right>}


\begin{document}
	\begin{center}
		\textbf{Fall 2020, Math 630:\ Homework 9} \\
		\textbf{Due:\ Friday, December 4th, 2020} \\
		\textbf{Joseph Diaz: 819947915}
	\end{center}
\noindent\makebox[\linewidth]{\rule{\paperwidth}{0.4pt}}

\be[(E1)]
    \item Let $\hilbert_1$ and $\hilbert_2$ be Hilbert spaces, and define $\hilbert$
    as 
    $$\hilbert = \hilbert_1 \oplus \hilbert_2 = \bracks{x = (x_1, x_2)\ \Big|\ 
    x_1 \in \hilbert_1,\ x_2 \in \hilbert_2}$$
    with 
    $$\forall x,y \in \hilbert,\ x=(x_1,x_2), y=(y_1,y_2),\ \abracks{x,y}_\hilbert =
    \abracks{x_1,y_1}_{\hilbert_1} + \abracks{x_2,y_2}_{\hilbert_2}$$
    \be[1.]
        \item Show that $\hilbert$ is a Hilbert space.
        \begin{proof}
        $\hilbert$ is defined as the direct of vector spaces, so it is itself a 
        vector space. Now, we show that $\abracks{\cdot,\cdot}_\hilbert$ is an 
        inner product, let $x,y,z \in \hilbert,\ \alpha, \beta \in \C$:
        \begin{itemize}
            \item \textbf{Non-negativity and positive definiteness}:\\
            We have that 
            $$\abracks{x,x}_\hilbert = \abracks{x_1,x_1}_{\hilbert_1} + 
            \abracks{x_2,x_2}_{\hilbert_2}$$
            $\hilbert_1$ and $\hilbert_2$ are Hilbert spaces, so 
            $\abracks{\cdot, \cdot}_{\hilbert_1}$ and $\abracks{\cdot, \cdot}_{\hilbert_2}$
            satisfy the non-negativiy and positive definiteness property; so
            $$\abracks{x_1,x_1}_{\hilbert_1} \geq 0,\ 
            \abracks{x_2,x_2}_{\hilbert_2} \geq 0$$
            which means that 
            $$\abracks{x,x}_\hilbert = \abracks{x_1,x_1}_{\hilbert_1} + 
            \abracks{x_2,x_2}_{\hilbert_2} \geq 0$$
            and $\abracks{\cdot,\cdot}_\hilbert$ satisfies this property.

            \item \textbf{Linearity in first argument}:\\
            By leveraging that $\abracks{\cdot,\cdot}_{\hilbert_1}$ and 
            $\abracks{\cdot,\cdot}_{\hilbert_2}$ are themselves inner products, we have that
            \begin{align*}
            \abracks{\alpha x + \beta y, z}_\hilbert &= \abracks{\alpha(x_1, x_2) + 
            \beta(y_1,y_2), z}_\hilbert = 
            \abracks{(\alpha x_1 + \beta y_1, \alpha x_2 + \beta y_2), z}_\hilbert \\
            &= \abracks{\alpha x_1 + \beta y_1, z_1}_{\hilbert_1} + 
            \abracks{\alpha x_2 + \beta y_2, z_2}_{\hilbert_2} \\
            &= \abracks{\alpha x_1,z_1}_{\hilbert_1} + \abracks{\beta y_1, z_1}_{\hilbert_1} + 
            \abracks{\alpha x_2,z_2}_{\hilbert_2} + \abracks{\beta y_2, z_2}_{\hilbert_2} \\
            &= \abracks{\alpha x_1,z_1}_{\hilbert_1} + \abracks{\alpha x_2,z_2}_{\hilbert_2} + 
            \abracks{\beta y_1, z_1}_{\hilbert_1} +\abracks{\beta y_2, z_2}_{\hilbert_2} \\
            &= \alpha\abracks{x_1,z_1}_{\hilbert_1} +  \alpha\abracks{x_2,z_2}_{\hilbert_2} + 
            \beta\abracks{y_1, z_1}_{\hilbert_1} + \beta\abracks{y_2, z_2}_{\hilbert_2} \\
            &= \alpha\parens{\abracks{x_1,z_1}_{\hilbert_1} + \abracks{x_2,z_2}_{\hilbert_2}} + 
            \beta\parens{\abracks{y_1, z_1}_{\hilbert_1} + \abracks{y_2, z_2}_{\hilbert_2}} \\
            &= \alpha\abracks{x, z}_\hilbert + \beta\abracks{y, z}_\hilbert
            \end{align*}
            So $\abracks{\cdot, \cdot}_\hilbert$ satisfies this property.

            \item \textbf{Hermitian Symmetry}:\\
            Again, leveraging that $\abracks{\cdot,\cdot}_{\hilbert_1}$ and  
            $\abracks{\cdot,\cdot}_{\hilbert_2}$ are themselves inner products, we 
            have that
            \begin{align*}
            \abracks{x,y}_\hilbert &= \abracks{x_1, y_1}_{\hilbert_1} + 
            \abracks{x_2,y_2}_{\hilbert_2} = \overline{\abracks{y_1, x_1}}_{\hilbert_1} + 
            \overline{\abracks{y_2,x_2}}_{\hilbert_2} \\
            &= \overline{\abracks{y_1, x_1}_{\hilbert_1} + \abracks{y_2,x_2}_{\hilbert_2}} 
            = \overline{\abracks{y, x}}_\hilbert 
            \end{align*}
            and $\abracks{\cdot, \cdot}_\hilbert$ satisfies this property.
        \end{itemize}

        Therefore, we conclude that $\abracks{\cdot,\cdot}_\hilbert$ satisfies the 
        properties of an inner product on $\hilbert$.\\\\

        Now, we show that $\hilbert$ is complete. Let $\norm{\cdot}_\hilbert,\ 
        \norm{\cdot}_{\hilbert_1}$ and $\norm{\cdot}_{\hilbert_2}$  
        be the norms induced on $\hilbert,\ \hilbert_1$ and $\hilbert_2$ by their respective 
        inner products such that
        \begin{gather*}
        \forall x \in \hilbert,\ \norm{x}_\hilbert = \sqrt{\abracks{x,x}_\hilbert} \\
        \forall x_1 \in \hilbert_1,\ \norm{x_1}_{\hilbert_1} = 
        \sqrt{\abracks{x_1,x_1}_{\hilbert_1}} \\
        \forall x_2 \in \hilbert_2,\ \norm{x_2}_{\hilbert_2} = 
        \sqrt{\abracks{x_2,x_2}_{\hilbert_2}}
        \end{gather*}
        Let $\bracks{x_n}$ be a cauchy sequence in $\hilbert$ such that 
        $$\forall n \in \N,\ x_n = (\chi_n, \gamma_n)$$
        Where $\bracks{\chi_n}$ and $\bracks{\gamma_n}$ are sequences in 
        $\hilbert_1$ and $\hilbert_2$, respectively. This means that
        $$\forall \vep > 0,\ \exists N \in \N,\ \forall n,k \in \N, n \geq N,\ 
        \norm{x_{n+k} - x_n}_\hilbert < \vep$$
        So we have that 
        \begin{align*}
        \norm{x_{n+k} - x_n}_\hilbert &< \vep \\
        \norm{x_{n+k} - x_n}_\hilbert^2 &< \vep^2 \\
        \abracks{x_{n+k} - x_n, x_{n+k} - x_n}_\hilbert &< \vep^2 \\
        \abracks{(\chi_{n+k},\gamma_{n+k}) - (\chi_n, \gamma_n), 
        (\chi_{n+k},\gamma_{n+k}) - (\chi_n, \gamma_n)}_\hilbert &< \\
        \abracks{(\chi_{n+k}-\chi_n,\gamma_{n+k} - \gamma_n), 
        (\chi_{n+k} - \chi_n,\gamma_{n+k} -\gamma_n)}_\hilbert &< \\
        \abracks{\chi_{n+k} - \chi_n, \chi_{n+k} - \chi_n}_{\hilbert_1} +  
        \abracks{\gamma_{n+k} - \gamma_n, \gamma_{n+k} - \gamma_n}_{\hilbert_2} &< \\
        \norm{\chi_{n+k}- \chi_n}_{\hilbert_1}^2 + \norm{\gamma_{n+k}- \gamma_n}_{\hilbert_2}^2
        &< \vep^2
        \end{align*}
        That last line implies that both $\norm{\chi_{n+k}- \chi_n}_{\hilbert_1}^2$ and  
        $\norm{\gamma_{n+k}- \gamma_n}_{\hilbert_2}^2$ are individually less than $\vep^2$
        and 
        $$\norm{\chi_{n+k}- \chi_n}_{\hilbert_1} < \vep,\  
        \norm{\gamma_{n+k}- \gamma_n}_{\hilbert_2} < \vep$$
        This means that 
        \begin{align*}
            \forall \vep > 0,\ \exists N \in \N,\ \forall n,k \in \N, n \geq N,\ 
            &\norm{\chi_{n+k} - \chi_n}_{\hilbert_1} < \vep \\
            \forall \vep > 0,\ \exists N \in \N,\ \forall n,k \in \N, n \geq N,\ 
            &\norm{\gamma_{n+k} - \gamma_n}_{\hilbert_2} < \vep
        \end{align*}
        and both $\bracks{\chi_n}$ and $\bracks{\gamma_n}$ meet the Cauchy criterion
        and Cauchy sequences themselves. As $\hilbert_1$ and $\hilbert_2$ are both 
        Hilbert spaces, this means they are complete and that there exists $\chi \in 
        \hilbert_1,\ \gamma \in \hilbert_2$ such that 
        $$\lim_{n\to\infty}\chi_n = \chi,\ \lim_{n\to\infty}\gamma_n = \gamma$$
        From this we can deduce
        $$\lim_{n\to\infty}x_n = \lim_{n\to\infty}(\chi_n, \gamma_n) = (\chi, \gamma)$$
        Denote $x = (\chi, \gamma)$; clearly $x \in \hilbert$, so $\bracks{x_n}$ converges
        in $\hilbert$. $\bracks{x_n}$ is an arbitrary cauchy sequence, so we have that 
        $\hilbert$ is complete and, consequently, a Hilbert space.
        \end{proof}
        \pagebreak
        
        \item Let 
        $$\widetilde{\hilbert} = \bracks{x = (x_1, 0)\ \Big|\ x_1 \in \hilbert_1}$$
        be a subspace of $\hilbert$. Find the orthogonal complement of $\widetilde{\hilbert}$.
        \begin{proof}[Answer]
        We wish to find 
        $$\widetilde{\hilbert}^\perp = \bracks{y \in \hilbert\ \Big|\ \forall x \in 
        \widetilde{\hilbert},\ \abracks{x,y}_\hilbert = 0}$$
        Let $x \in \widetilde{\hilbert},\ y \in \widetilde{\hilbert}^\perp$, then  
        by the definition of $\abracks{\cdot, \cdot}_\hilbert$:
        $$\abracks{x,y}_\hilbert = \abracks{x_1,y_1}_{\hilbert_1} + 
        \abracks{0, y_2}_{\hilbert_2} = 0$$
        $\abracks{0, y_2}_{\hilbert_2}$ will equal $0$ for every $y_2 \in \hilbert_2$, 
        so it is essentially free to vary; now we focus on $\abracks{x_1, y_1}_{\hilbert_1}$.
        The only $y_1 \in \hilbert_1$ that satisfies 
        $$\abracks{x_1,y_1}_{\hilbert_1} = 0$$
        for every $x_1 \in \hilbert_1$ is $y_1 = 0$. So this means that 
        $$\widetilde{\hilbert}^\perp = \bracks{y = (0, y_2)\ \Big|\ y_2 \in \hilbert_2}$$
        as desired.
        \end{proof}
    \ee

    \item Let $\hilbert$ be the Hilbert space of functions $f: [-1,1] \to \C$ equipped with
    inner product defined by 
    $$\forall f,g \in \hilbert,\ \abracks{f,g} = \int_{-1}^1 \frac{f(x)\overline{g(x)}}
    {\sqrt{1-x^2}}\ dx$$
    and let the Tchebyshev polynomials be defines as
    $$\forall n \in \N,\ T_n(x) = \cos(n\theta), \text{ where }
    \cos\theta = x, \text{ and } 0 \leq \theta \leq \pi$$
    \be[1.] 
        \item Show that the set of Tchebyshev polynomials form an orthogonal set of $\hilbert$
        (you must investigate the different cases: $m \neq n,\ m = n \neq 0$ and $m = n = 0$).
        \begin{proof}
        Let $m,n \in \N$, and we have that
        \begin{align*}
        \abracks{T_m, T_n} &= \int_{-1}^1 \frac{T_m(x)\overline{T_n(x)}}{\sqrt{1-x^2}}\ dx \\
        &= \int_{-1}^1 \frac{\cos(m\theta)\overline{\cos(n\theta)}}{\sqrt{1-x^2}}\ dx \\
        &= \int_{-1}^1 \frac{\cos(m\theta)\cos(n\theta)}{\sqrt{1-x^2}}\ dx \\
        \end{align*}
        Now we make a trigonometric substition; let $x = \cos\theta$, this also gives
        $dx = -\sin\theta\ d\theta$ and
        \begin{align*}
        \abracks{T_m, T_n} &= \int_{-1}^1 \frac{\cos(m\theta)\cos(n\theta)}{\sqrt{1-x^2}}\ dx \\
        &= -\int_{\pi}^0 \frac{\cos(m\theta)\cos(n\theta)\sin\theta}
        {\sqrt{1-\cos^2(\theta)}}\ d\theta \\
        &= \int_0^\pi \frac{\cos(m\theta)\cos(n\theta)\sin\theta}
        {\sqrt{\sin^2(\theta)}}\ d\theta \\
        &= \int_0^\pi \frac{\cos(m\theta)\cos(n\theta)\sin\theta}
        {\abs{\sin\theta}}\ d\theta \\
        &= \int_0^\pi \frac{\cos(m\theta)\cos(n\theta)\sin\theta}
        {\sin\theta}\ d\theta \quad (\forall \theta \in [0,\pi],\ \sin\theta \geq 0)\\
        &=\int_0^\pi \cos(m\theta)\cos(n\theta)\ d\theta
        \end{align*}
        Then, using integration by parts, we have 
        \begin{align*}
        \int_0^\pi \cos(m\theta)\cos(n\theta)\ d\theta &= \frac{1}{m}\cos(n\theta)
        \sin(m\theta)\Big|_0^\pi + \frac{n}{m}\int_0^\pi \sin(m\theta)\sin(n\theta)\ d\theta \\
        &= \frac{n}{m}\int_0^\pi \sin(m\theta)\sin(n\theta)\ d\theta \quad (\forall m \in \N,\ 
        \sin(m\pi) = 0) \\
        &= \frac{n}{m}\parens{-\frac{1}{m}\sin(n\theta)\cos(m\theta)\Big|_0^\pi + 
        \frac{n}{m}\int_0^\pi \cos(m\theta)\cos(n\theta)\ d\theta} \\
        &= \frac{n^2}{m^2}\int_0^\pi \cos(m\theta)\cos(n\theta)\ d\theta
        \end{align*}
        So we have that 
        $$\int_0^\pi \cos(m\theta)\cos(n\theta)\ d\theta = 
        \frac{n^2}{m^2}\int_0^\pi \cos(m\theta)\cos(n\theta)\ d\theta$$
        which implies that
        $$\int_0^\pi \cos(m\theta)\cos(n\theta)\ d\theta\parens{1-\parens{\frac{n}{m}}^2} = 0$$
        Therefore, in the case that $m \neq n,\ 1 - (n/m)^2 \neq 0$, so it must be the case 
        $$\int_0^\pi \cos(m\theta)\cos(n\theta)\ d\theta = 0 \iff \abracks{T_m,T_n} = 0$$
        So for $m \neq n,\ T_m \perp T_n$.\\\\
        For the case that $m = n \neq 0$, we have 
        \begin{align*}
        \abracks{T_m, T_n} &= \abracks{T_n, T_n} \\
        &= \int_0^\pi \cos^2(n\theta)\ d\theta \\
        &= \frac{1}{2}\int_0^\pi (1 + \cos(2n\theta))\ d\theta \\
        &= \frac{1}{2}\sqbracks{\theta + \frac{1}{2n}\sin\theta}_0^\pi \\
        &= \frac{1}{2}(\pi - 0) = \frac{\pi}{2}
        \end{align*}
        So for $m = n \neq 0,\ T_m \not\perp T_n$.\\\\
        Finally, we examine the case where $m = n = 0$; this gives $T_0(x) = \cos(0\cdot\theta)
        = \cos 0 = 1$ and:
        $$\abracks{T_0, T_0} = \int_0^\pi 1\bar{1}\ d\theta = \int_0^\pi\ d\theta = 
        \theta\Big|_0^\pi = \pi$$
        So for $m=n=0,\ T_m \not\perp T_n$.\\\\
        This covers all cases, so $\bracks{T_n}$ forms an orthogonal set in $\hilbert$.
        \end{proof}

        \item Show that $\norm{T_0}_\hilbert = \sqrt{\pi}$ and $\norm{T_n}_\hilbert = 
        \sqrt{\frac{\pi}{2}}$ for $n \geq 1$.
        \begin{proof}
        Let $\norm{\cdot}_\hilbert$ be the norm on $\hilbert$, induced by the inner
        product, such that
        $$\forall f \in \hilbert,\ \norm{f}_\hilbert = \sqrt{\abracks{f,f}}$$
        From part 1 of this exercise, we already have that 
        $$\abracks{T_0, T_0} = \pi,\quad \forall n \geq 1,\ \abracks{T_n, T_n} = \frac{\pi}{2}$$
        so this gives
        $$\norm{T_0} = \sqrt{\abracks{T_0, T_0}} = \sqrt{\pi},\quad \norm{T_n} = 
        \sqrt{\abracks{T_n, T_n}} = \sqrt{\frac{\pi}{2}}$$
        As desired.
        \end{proof}
    \ee

\ee
\noindent\makebox[\linewidth]{\rule{\paperwidth}{0.4pt}}
	
\end{document}
