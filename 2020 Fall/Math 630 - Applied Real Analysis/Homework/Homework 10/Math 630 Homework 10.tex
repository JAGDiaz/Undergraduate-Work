\documentclass{article}
\usepackage{amsmath}
\usepackage{amssymb}
\usepackage{bm}
\usepackage{amsthm}
\usepackage{enumerate}
\usepackage{graphicx}
\usepackage{psfrag}
\usepackage{color}
\usepackage{url}
\usepackage{listings}
\usepackage{xcolor}
\usepackage{tikz}
\usetikzlibrary{positioning}
\tikzset{main node/.style={circle,fill=gray!20,draw,minimum size=.5cm,inner sep=0pt},}

% In line code stuff%
\definecolor{codegreen}{rgb}{0,0.5,0}
\definecolor{codewhite}{rgb}{1,1,1}
\definecolor{codegray}{rgb}{0.5,0.5,0.5}
\definecolor{codepurple}{rgb}{0.58,0,0.82}
\definecolor{codeblack}{rgb}{0,0,0}
\definecolor{codeorange}{rgb}{0.8,0.4,0}

\lstdefinestyle{mystyle}{
    backgroundcolor=\color{codewhite},   
    commentstyle=\color{codegray},
    keywordstyle=\color{codegreen},
    numberstyle=\color{codegray},
    stringstyle=\color{codeorange},
    basicstyle=\ttfamily ,
    breakatwhitespace=false,         
    breaklines=true,                 
    captionpos=b,                    
    keepspaces=true,                 
    numbers=left,                    
    numbersep=5pt,                  
    showspaces=false,                
    showstringspaces=false,
    showtabs=false,                  
    tabsize=4
}
\lstset{style=mystyle}
\setlength{\hoffset}{-1in}
\addtolength{\textwidth}{1.5in}
\setlength{\voffset}{-1in}
\addtolength{\textheight}{1.5in}

% Custom commands%
\newcommand{\be}{\begin{enumerate}}
\newcommand{\ee}{\end{enumerate}}
\newcommand{\BigO}[1]{\ensuremath\mathcal{O}\left(#1\right)}
\newcommand{\il}[1]{\lstinline!#1!}
\newcommand{\norm}[1]{\left|\left|#1\right|\right|}
\newcommand{\abs}[1]{\left|#1\right|}
\newcommand{\parens}[1]{\left(#1\right)}
\newcommand{\bracks}[1]{\left\{#1\right\}}
\newcommand{\sqbracks}[1]{\left[#1\right]}
\newcommand{\vep}{\varepsilon}
\newcommand{\ceiling}[1]{\left\lceil#1\right\rceil}
\newcommand{\R}{\mathbb{R}}
\newcommand{\N}{\mathbb{N}}
\newcommand{\Z}{\mathbb{Z}}
\newcommand{\F}{\mathbb{F}}
\newcommand{\C}{\mathbb{C}}
\newcommand{\A}{\mathcal{A}}
\newcommand{\hilbert}{\mathcal{H}}
\newcommand{\distrib}[2]{\text{#1}\left(#2\right)}
\newcommand{\dd}[2]{\frac{d#1}{d#2}}
\newcommand{\abracks}[1]{\left< #1\right>}
\newcommand{\nullspace}[1]{\text{null }#1}
\newcommand{\LT}[1]{\mathcal{L}\parens{#1}}
\newcommand{\poly}[2]{\mathcal{P}_{#1}\parens{#2}}
\newcommand{\cont}[3]{\mathcal{C}_{#1}^{#2}\parens{#3}}
\newcommand{\range}[1]{\text{range }#1}
\newcommand{\one}{\mathbbm{1}}

\begin{document}
	\begin{center}
		\textbf{Fall 2020, Math 630:\ Homework 10} \\
		\textbf{Due:\ December 11th, 2020} \\
		\textbf{Joseph Diaz: 819947915}
	\end{center}
\noindent\makebox[\linewidth]{\rule{\paperwidth}{0.4pt}}
\be[(E1)]
    \item Consider the linear space $\cont{}{}{[0,1]}$ with norm
    $$\forall p \in \N,\ \forall f \in \cont{}{}{[0,1]},\ \norm{f}_{L^p} =
    \parens{\int_0^1 \abs{f(t)}^p\ dt}^{1/p}$$
    As seen in class, we know that when $p = 2$ this norm is induced by the inner 
    product
    $$\forall f,g \in \cont{}{}{[0,1]},\ \abracks{f,g} = \int_0^1 f(t)\overline{g(t)}\ dt$$
    The goal of this exercise is to prove that if $p \neq 2$, then $\norm{\cdot}_{L^p}$ is 
    \textit{not} induced by an inner product (i.e. $L^2$ is the \textit{only} Hilbert space
    among the $L^p$ spaces). To do so, study when the parallelogram law holds. Hint: consider 
    the functions
    $$f(t) = \frac{1}{2} - t, \qquad g(t) = \left\{\begin{array}{lcr}
    \frac{1}{2} - t & & 0 \leq t \leq \frac{1}{2} \\
    t - \frac{1}{2} & & \frac{1}{2} \leq t \leq 1    
    \end{array}\right.$$
    \begin{proof}
    For $\norm{\cdot}_{L^p},\ 1 < p < \infty$, on $\cont{}{}{[0,1]}$ to be induced by an inner
    product it must satisfy the parallelogram law; which is 
    $$\forall f,g \in \cont{}{}{[0,1]},\ \norm{f+g}_{L^p}^2 + \norm{f-g}_{L^p}^2 = 
    2\parens{\norm{f}_{L^p}^2 + \norm{g}_{L^p}^2}$$
    We know that this is satisfied for $p = 2$; now suppose, by way of contradiction, 
    that for $p \neq 2$ the parallelogram law still holds. That means that it will
    hold for $f, g \in \cont{}{}{[0,1]}$ such that  
    $$f(t) = \frac{1}{2} - t, \qquad g(t) = \left\{\begin{array}{lcr}
    \frac{1}{2} - t & & 0 \leq t \leq \frac{1}{2} \\
    t - \frac{1}{2} & & \frac{1}{2} \leq t \leq 1    
    \end{array}\right.$$

    Now, we're going to find the ``pieces'' of the parallelogram law in turn:\\
    \begin{align*}
    \norm{f+g}_{L^p} &= \parens{\int_0^1 \abs{f(t) + g(t)}^p\ dt}^{1/p} \\
    &= \parens{\int_0^{1/2} \abs{\frac{1}{2} -t + \frac{1}{2} - t}^p\ dt + 
    \int_{1/2}^1 \abs{\frac{1}{2} -t +t - \frac{1}{2}}^p\ dt}^{1/p} \\
    &= \parens{\int_0^{1/2} \abs{2\parens{\frac{1}{2} -t}}^p\ dt + 
    \int_{1/2}^1 \abs{0}^p\ dt}^{1/p} \\
    &= \parens{\int_0^{1/2} \abs{2}^p\abs{\frac{1}{2} -t}^p\ dt}^{1/p} \\
    &= \parens{2^p\int_0^{1/2} \abs{\frac{1}{2} -t}^p\ dt}^{1/p} \\
    &= 2\parens{\int_0^{1/2} \parens{\frac{1}{2} -t}^p\ dt}^{1/p} \qquad 
    \parens{\forall t \in \sqbracks{0, \frac{1}{2}},\ \frac{1}{2}- t \geq 0}\\
    \end{align*}
    \begin{align*}
    &= 2\parens{-\int_{1/2}^0 u^p\ du}^{1/p} \qquad 
    \parens{\text{Letting } u = \frac{1}{2} - t} \\
    &= 2\parens{\int_0^{1/2} u^p\ du}^{1/p} \\ 
    &= 2\parens{\frac{u^{p+1}}{p+1}\Bigg|_0^{1/2}}^{1/p} \\ 
    &= \parens{\frac{1}{2(p+1)}}^{1/p}
    \end{align*}
    \begin{align*}
    \norm{f-g}_{L^p} &= \parens{\int_0^1 \abs{f(t) - g(t)}^p\ dt}^{1/p} \\
    &= \parens{\int_0^{1/2} \abs{\frac{1}{2} -t - \frac{1}{2} + t}^p\ dt + 
    \int_{1/2}^1 \abs{\frac{1}{2} -t -t + \frac{1}{2}}^p\ dt}^{1/p} \\
    &= \parens{\int_0^{1/2} \abs{0}^p\ dt + 
    \int_{1/2}^1 \abs{2\parens{\frac{1}{2} -t}}^p\ dt}^{1/p} \\
    &= \parens{\int_{1/2}^1 \abs{2}^p\abs{\frac{1}{2} -t}^p\ dt}^{1/p} \\
    &= \parens{2^p\int_{1/2}^1 \abs{\frac{1}{2} -t}^p\ dt}^{1/p} \\
    &= 2\parens{\int_{1/2}^1 \parens{t-\frac{1}{2}}^p\ dt}^{1/p} \qquad 
    \parens{\forall t \in \sqbracks{\frac{1}{2}, 1},\ \frac{1}{2}- t \leq 0}\\
    &= 2\parens{\int_0^{1/2} u^p\ du}^{1/p} \qquad 
    \parens{\text{Letting } u = t - \frac{1}{2}} \\
    &= 2\parens{\frac{u^{p+1}}{p+1}\Bigg|_0^{1/2}}^{1/p} \\ 
    &= \parens{\frac{1}{2(p+1)}}^{1/p}
    \end{align*}
    \begin{align*}
    \norm{f}_{L^p} &= \parens{\int_0^1 \abs{f(t)}^p\ dt}^{1/p} \\
    &= \parens{\int_0^1 \abs{\frac{1}{2}-t}^p\ dt}^{1/p} \\
    &= \parens{\int_0^{1/2} \parens{\frac{1}{2} -t}^p\ dt + 
    \int_{1/2}^1 \parens{t-\frac{1}{2}}^p\ dt}^{1/p} \\
    &= \parens{-\int_{1/2}^0 u^p\ du + \int_0^{1/2} u^p\ du}^{1/p} \\
    \end{align*}
    \begin{align*}
    &= \parens{2\int_0^{1/2} u^p\ du}^{1/p} \\
    &= \parens{2\frac{u^{p+1}}{p+1}\Bigg|_0^{1/2}}^{1/p} \\
    &= \parens{\frac{1}{2^p(p+1)}}^{1/p} \\
    \end{align*}
    \begin{align*}
    \norm{g}_{L^p} &= \parens{\int_0^1 \abs{g(t)}^p\ dt}^{1/p} \\
    &= \parens{\int_0^{1/2} \abs{\frac{1}{2} -t}^p\ dt + 
    \int_{1/2}^1 \abs{t-\frac{1}{2}}^p\ dt}^{1/p} \\
    &= \parens{\int_0^{1/2} \parens{\frac{1}{2} -t}^p\ dt + 
    \int_{1/2}^1 \parens{t-\frac{1}{2}}^p\ dt}^{1/p} \\
    &= \parens{-\int_{1/2}^0 u^p\ du + \int_0^{1/2} u^p\ du}^{1/p} \\
    &= \parens{2\int_0^{1/2} u^p\ du}^{1/p} \\
    &= \parens{2\frac{u^{p+1}}{p+1}\Bigg|_0^{1/2}}^{1/p} \\
    &= \parens{\frac{1}{2^p(p+1)}}^{1/p} \\
    \end{align*}
    Finally, we have that 
    \begin{align*}
    \norm{f+g}_{L^p}^2 + \norm{f-g}_{L^p}^2 &= 2\parens{\norm{f}_{L^p}^2 + \norm{g}_{L^p}^2} \\
    \parens{\frac{1}{2(p+1)}}^{2/p} + \parens{\frac{1}{2(p+1)}}^{2/p} &=
    2\parens{\parens{\frac{1}{2^p(p+1)}}^{2/p} + \parens{\frac{1}{2^p(p+1)}}^{2/p}} \\
    2\parens{\frac{1}{2(p+1)}}^{2/p} &= 2\parens{2\parens{\frac{1}{2^p(p+1)}}^{2/p}} \\
    \parens{\frac{1}{2(p+1)}}^{2/p} &= 2\parens{\frac{1}{2^p(p+1)}}^{2/p} \\
    \frac{1}{2(p+1)} &= 2^{p/2}\parens{\frac{1}{2^p(p+1)}} \\
    \frac{1}{2(p+1)} &= \frac{1}{2^{-p/2}(p+1)} \\
    2^{p/2} &= 2 \\
    p/2 &= 1 \implies p = 2
    \end{align*}
    \pagebreak

    This contradicts our assumption that $p \neq 2$ and that $\norm{\cdot}_{L^p}$ still 
    satisfies the parallelogram law for all functions in $\cont{}{}{[0,1]}$.
    So $\norm{\cdot}_{L^p}$ does not satisfy the parallelogram law for $p \neq 2$ and 
    those $L^p$ spaces are not Hilbert Spaces.
    \end{proof}

    \item Let $\bracks{u_n}_{n=1}^\infty$ be an orthonormal set in an inner product space 
    $\parens{M, \abracks{\cdot, \cdot}}$. Prove that 
    $$\forall x,y \in M,\ \sum_{n=1}^\infty \abs{\abracks{x,u_n}\abracks{y, u_n}} 
    \leq \norm{x}\norm{y}$$
    Hint: Use H{\"o}lder's inequality in $\ell^2$, i.e $\forall x, y \in \ell^2,\ x = 
    \parens{x_1,\ x_2,\ \cdots },\ y = \parens{y_1,\ y_2,\ \cdots }$:
    $$\sum_{n=1}^\infty \abs{x_ny_n} \leq \parens{\sum_{n=1}^\infty \abs{x_n}^2}^{1/2}
    \parens{\sum_{n=1}^\infty \abs{y_n}^2}^{1/2}$$
    \begin{proof}
    $U = \bracks{u_n}$ is an orthonormal set, so $\forall x, y \in M$ there exists 
    $x_U, y_U \in \sqbracks{U}$ such that
    $$x_U = \sum_{n=1}^\infty \abracks{x,u_n}u_n,\quad y_U = \sum_{n=1}^\infty 
    \abracks{y, u_n}u_n$$
    are convergent sums.
    Denote $\bracks{x_n},\ \bracks{y_n}$ such that $x_n = \abracks{x, u_n}$ and 
    $y_n = \abracks{y, u_n}$. From Bessel's inequality, we know that $\sum_{n=1}^\infty
    \abs{\abracks{x,u_n}}^2$ and $\sum_{n=1}^\infty \abs{\abracks{y,u_n}}^2$ are convergent sums 
    as well; so $\bracks{x_n}, \bracks{y_n} \in \ell^2$.
    From this, we have 
    $$\sum_{n=1}^\infty \abs{\abracks{x,u_n}\abracks{y, u_n}} = \sum_{n=1}^\infty\abs{x_ny_n}$$
    Denote $x, y \in \ell^2$ such that $x = (x_1,\ x_2,\ \cdots),\ y = (y_1,\ y_2,\ \cdots)$, 
    then by H{\"o}lder's inequality in $\ell^2$ this implies 
    $$\sum_{n=1}^\infty \abs{\abracks{x,u_n}\abracks{y, u_n}} = \sum_{n=1}^\infty\abs{x_ny_n}
    \leq \parens{\sum_{n=1}^\infty\abs{x_n}^2}^{1/2}\parens{\sum_{n=1}^\infty\abs{y_n}^2}^{1/2} 
    = \norm{x}_{\ell^2}\norm{y}_{\ell^2}$$
    So 
    $$\forall x,y \in M,\ \sum_{n=1}^\infty \abs{\abracks{x,u_n}\abracks{y, u_n}} 
    \leq \norm{x}_{\ell^2}\norm{y}_{\ell^2}$$
    As desired.
    \end{proof}
    \pagebreak
    
    \item Let $\hilbert$ be a complex Hilbert space. Show that 
    $$\forall x,y \in \hilbert,\ \abracks{x,y} = \frac{1}{2\pi}\int_0^{2\pi}\norm{x + 
    e^{i\theta}y}^2 e^{i\theta}\ d\theta$$
    \begin{proof}
    We'll start by looking at the right hand side of the equation; let $x,y \in \hilbert$.
    \begin{align*}
    \frac{1}{2\pi}\int_0^{2\pi}\norm{x + e^{i\theta}y}^2 e^{i\theta}\ d\theta &= 
    \frac{1}{2\pi}\int_0^{2\pi}\abracks{x + e^{i\theta}y, x+e^{i\theta}y}e^{i\theta}\ d\theta \\
    &= \frac{1}{2\pi}\int_0^{2\pi}\parens{\abracks{x,x + e^{i\theta}y} + \abracks{e^{i\theta}y,
    x+e^{i\theta}y}}e^{i\theta}\ d\theta \\
    &= \frac{1}{2\pi}\int_0^{2\pi}\parens{\abracks{x,x} + \abracks{x, e^{i\theta}y} + 
    \abracks{e^{i\theta}y, x} +\abracks{e^{i\theta}y,e^{i\theta}y}}e^{i\theta}\ d\theta \\
    &= \frac{1}{2\pi}\int_0^{2\pi}\parens{\abracks{x,x} + \overline{e^{i\theta}}\abracks{x, y} + 
    e^{i\theta}\abracks{y, x} +e^{i\theta}\overline{e^{i\theta}}\abracks{y,y}}e^{i\theta}
    \ d\theta \\
    &= \frac{1}{2\pi}\int_0^{2\pi}\parens{\abracks{x,x} + e^{-i\theta}\abracks{x, y} + 
    e^{i\theta}\abracks{y, x} + \abracks{y,y}}e^{i\theta}\ d\theta \\
    &= \frac{1}{2\pi}\int_0^{2\pi}e^{i\theta}\abracks{x,x} + \abracks{x, y} + 
    e^{2i\theta}\abracks{y, x} + e^{i\theta}\abracks{y,y}\ d\theta \\
    &= \frac{1}{2\pi}\parens{\int_0^{2\pi}e^{i\theta}\abracks{x,x}\ d\theta + 
    \int_0^{2\pi}\abracks{x, y}\ d\theta + \int_0^{2\pi} e^{2i\theta}\abracks{y, x}\ d\theta 
    + \int_0^{2\pi} e^{i\theta}\abracks{y,y}\ d\theta} \\
    \end{align*}
    To continue, we'll solve each of these integrals individually:
    $$\int_0^{2\pi}e^{i\theta}\abracks{x,x}\ d\theta = \abracks{x,x}\parens{-ie^{i\theta}}
    \Big|_0^{2\pi} = \abracks{x,x}\parens{-ie^{2\pi i} +ie^0} = \abracks{x,x}(-i+i) = 0$$
    From this we can also deduce that
    $$\int_0^{2\pi}e^{i\theta}\abracks{y,y}\ d\theta = 0$$
    because it's the same integral, only with a different ``scalar''. Next, we have
    $$\int_0^{2\pi}\abracks{x, y}\ d\theta = \abracks{x,y} \int_0^{2\pi}\ d\theta =
    \abracks{x,y} \theta\Big|_0^{2\pi} = 2\pi\abracks{x,y}$$
    Lastly, we have
    $$\int_0^{2\pi} e^{2i\theta}\abracks{y, x}\ d\theta = 
    \abracks{y,x}\parens{-\frac{i}{2}e^{2i\theta}}
    \Bigg|_0^{2\pi} = \abracks{y,x} \parens{-\frac{i}{2}e^{4\pi i} + \frac{i}{2}e^0} = 
    \abracks{y,x} \parens{-\frac{i}{2} + \frac{i}{2}} = 0$$
    Hence
    \begin{align*}
    \frac{1}{2\pi}\int_0^{2\pi}\norm{x + e^{i\theta}y}^2 e^{i\theta}\ d\theta &= 
    \frac{1}{2\pi}\parens{0 + 2\pi\abracks{x, y} + 0 + 0} \\
    &= \frac{2\pi}{2\pi}\abracks{x,y} \\
    &= \abracks{x,y}
    \end{align*}
    $x$ and $y$ are arbitrary elements of $\hilbert$; therefore,
    $$\forall x,y \in \hilbert,\ \abracks{x,y} = \frac{1}{2\pi}\int_0^{2\pi}\norm{x + 
    e^{i\theta}y}^2 e^{i\theta}\ d\theta$$
    as desired.
    \end{proof}

\ee
\noindent\makebox[\linewidth]{\rule{\paperwidth}{0.4pt}}
	
\end{document}
