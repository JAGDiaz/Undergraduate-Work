\documentclass{article}
\usepackage{amsmath}
\usepackage{amssymb}
\usepackage{bbm}
\usepackage{bm}
\usepackage{amsthm}
\usepackage{enumerate}
\usepackage{graphicx}
\usepackage{psfrag}
\usepackage{color}
\usepackage{url}
\usepackage{listings}
\usepackage{xcolor}
\usepackage{tikz}
\usetikzlibrary{positioning}
\tikzset{main node/.style={circle,fill=gray!20,draw,minimum size=.5cm,inner sep=0pt},}

\definecolor{codegreen}{rgb}{0,0.5,0}
\definecolor{codewhite}{rgb}{1,1,1}
\definecolor{codegray}{rgb}{0.5,0.5,0.5}
\definecolor{codepurple}{rgb}{0.58,0,0.82}
\definecolor{codeblack}{rgb}{0,0,0}
\definecolor{codeorange}{rgb}{0.8,0.4,0}

\lstdefinestyle{mystyle}{
    backgroundcolor=\color{codewhite},   
    commentstyle=\color{codegray},
    keywordstyle=\color{codegreen},
    numberstyle=\color{codegray},
    stringstyle=\color{codeorange},
    basicstyle=\ttfamily ,
    breakatwhitespace=false,         
    breaklines=true,                 
    captionpos=b,                    
    keepspaces=true,                 
    numbers=left,                    
    numbersep=5pt,                  
    showspaces=false,                
    showstringspaces=false,
    showtabs=false,                  
    tabsize=4
}
\lstset{style=mystyle}


\setlength{\hoffset}{-1in}
\addtolength{\textwidth}{1.5in}
\setlength{\voffset}{-1in}
\addtolength{\textheight}{1.5in}
\newcommand{\be}{\begin{enumerate}}
\newcommand{\ee}{\end{enumerate}}
\newcommand{\BigO}[1]{\ensuremath\mathcal{O}\left(#1\right)}
\newcommand{\il}[1]{\lstinline!#1!}
\newcommand{\norm}[1]{\left|\left|#1\right|\right|}
\newcommand{\abs}[1]{\left|#1\right|}
\newcommand{\parens}[1]{\left(#1\right)}
\newcommand{\bracks}[1]{\left\{#1\right\}}
\newcommand{\sqbracks}[1]{\left[#1\right]}
\newcommand{\vep}{\varepsilon}
\newcommand{\ceiling}[1]{\left\lceil#1\right\rceil}
\newcommand{\R}{\mathbb{R}}
\newcommand{\N}{\mathbb{N}}
\newcommand{\Z}{\mathbb{Z}}
\newcommand{\mC}{\mathcal{C}}
\newcommand{\one}{\mathbbm{1}}
\newcommand{\A}{\mathcal{A}}
\newcommand{\distrib}[2]{\text{#1}\left(#2\right)}
\newcommand{\dd}[1]{\frac{d}{d#1}}
\newcommand{\abracks}[1]{\left< #1\right>}

\newtheorem*{proposition}{Proposition}

\begin{document}
	\begin{center}
		\textbf{Fall 2020, Math 630:\ Homework 6} \\
		\textbf{Due:\ Friday, October 23rd, 2020} \\
		\textbf{Joseph Diaz: 819947915}
	\end{center}
\noindent\makebox[\linewidth]{\rule{\paperwidth}{0.4pt}}

\be[(E1)]
    \item Let $\parens{\mathcal{C}([0,1]), \norm{\cdot}_{L^1}}$ be a space 
    where 
    $$\forall f \in \mathcal{C}([0,1]),\ \norm{f}_{L^1} = 
    \int_0^1 \abs{f(x)}\ dx$$
    \be[1.]
        \item Show that $\parens{\mathcal{C}([0,1]), \norm{\cdot}_{L^1}}$
        is not a Banach Space.  
        \begin{proof}
        To show that $\parens{\mathcal{C}([0,1]), \norm{\cdot}_{L^1}}$ is 
        not a Banach Space, we'll examine the sequence $\bracks{f_n}$ in
        $\mathcal{C}([0,1])$ where
        $$f_n(x) = x^n,\ x \in [0,1]$$
        It's clear that $f_n$ for each $n$ is continuous on $[0,1]$ and 
        now we examine it's point-wise convergence.
        For $x = 0$ and $x = 1$, we have $\forall \vep > 0$
        \begin{gather*}
            f_n(0) = 0 \implies \norm{f_{n+k}(0) - f_n(0)}_{L^1} = 
            \int_0^1 \abs{0 - 0}\ dx = 0 < \vep \\
            f_n(1) = 1^n \implies \norm{f_{n+k}(1) - f_n(1)}_{L^1} = 
            \int_0^1 \abs{1 - 1}\ dx = 0 < \vep \\
        \end{gather*}
        So we have that $f_n(0) \to 0$ and $f_n(1) \to 1$ are cauchy 
        sequences in $\R$. Now we examine $x \in (0,1)$:
        $$\norm{f_{n+k}(x) - f_n(x)}_{L^1} = \int_0^1 \abs{x^{n+k} - x^n}\ 
        dx$$
        For $0 < x < 1,\ x^{n+k} < x^n$, which implies that $x^{n+k} - x^n
        < 0$ and
        \begin{align*}
            \int_0^1 \abs{x^{n+k} - x^n}\ dx &= \int_0^1 -(x^{n+k}- x^n)\ 
            dx \\
            &= \int_0^1 x^{n} - x^{n+k}\ dx \\
            &= \left.\frac{x^{n+1}}{n+1} - \frac{x^{n+k+1}}{n+k+1}
            \right|_0^1 \\
            &= \frac{1}{n+1} - \frac{1}{n+k+1} \\
            &< \frac{1}{n+1} \\
            &< \frac{1}{n} 
        \end{align*}
        Let $\vep = \frac{1}{n}$ and we have that
        $$\forall x \in [0,1],\ \forall \vep > 0, N = 
        \ceiling{1/\vep}, \forall n,k \in \N, n \geq N, 
        \norm{f_{n+k}(x) - f_n(x)}_{L^1} < \vep$$ 
        and $\bracks{f_n}$ is point-wise cauchy. Evaluating the limit
        for $x \in (0,1)$ yields:
        $$\lim_{n\to\infty} x^n = 0$$
        and the function that $f_n$ converges to is 
        $$f(x) = \left\{
        \begin{array}{cl}
            0 & \text{if } 0 \leq x < 1 \\
            1 & \text{if } x = 1 
        \end{array}\right.$$
        But this function that $f_n$ converges to is discontinuous at 
        $x=1$. This implies that $\mathcal{C}([0,1])$ is not 
        complete with respect to $\norm{\cdot}_{L^1}$, as it has a 
        cauchy sequence whose limit is outside of the space. Therefore,
        $\parens{\mathcal{C}([0,1]), \norm{\cdot}_{L^1}}$ is not a Banach
        Space because it isn't complete.
        \end{proof}

        \item Show that if $\forall \bracks{f_n} \in \mathcal{C}([0,1]),\
        \exists f \in \mathcal{C}([0,1]),\ f_n \to f$ with respect to 
        $\norm{\cdot}_\infty$; then $f_n \to f$ with respect to 
        $\norm{\cdot}_{L^1}$.
        \begin{proof}
        We have that 
        \begin{align*}
            \forall x \in [0,1],\ \forall \vep > 0,\ \exists N \in \N, 
            \forall n \geq N, &\norm{f_n - f}_\infty < \vep \\
            \iff &\max_{x\in[0,1]}\abs{f_n(x) - f(x)} < \vep
        \end{align*}
        By the definition of the max norm we have that
        $$\forall x \in [0,1], \forall n \in \N,\ \abs{f_n(x) - f(x)} \leq 
        \max_{x\in[0,1]}\abs{f_n(x) - f(x)}$$
        $$\implies \int_0^1 \abs{f_n(x)- f(x)}\ dx \leq \int_0^1
        \max_{x\in[0,1]}\abs{f_n(x) - f(x)}\ dx$$
        The left hand side of that inequality is just $\norm{f_n- f}_{L^1}
        $, and the right hand side is 
        \begin{align*}
            \int_0^1\max_{x\in[0,1]}\abs{f_n(x) - f(x)}\ dx &= 
            \max_{x\in[0,1]}\abs{f_n(x) - f(x)}\int_0^1\ dx \\
            &= \max_{x\in[0,1]}\abs{f_n(x) - f(x)} \\
        \end{align*}
        because the maximum of $\abs{f_n(x) - f(x)}$ is a constant 
        relative to the integral and $\int_0^1\ dx = 1$. So 
        $$\norm{f_n - f}_{L^1} \leq \max_{x\in[0,1]}\abs{f_n(x) - f(x)} = 
        \norm{f_n - f}_\infty < \vep$$
        which ultimately implies 
        $$\forall x \in [0,1],\ \forall \vep > 0,\ \exists N \in \N, 
        \forall n \geq N, \norm{f_n - f}_{L^1} < \vep$$
        and $f_n$ converges to $f$ with respect to $\norm{\cdot}_{L^1}$
        \end{proof} 
    \ee

    \item Denote 
    $$\mathcal{C}^1([a,b]) = \bracks{f: [a,b] \to \R\ \big|\ f, f' 
    \text{ are continuous}}$$
    the space of continiuously differentiable functions on $[a,b]$. Now 
    define 
    $$\forall f \in \mathcal{C}^1([a,b]),\ \norm{f}_{\mathcal{C}^1} =
    \norm{f}_\infty + \norm{f'}_\infty$$
    \be[1.] 
        \item Show that $\norm{\cdot}_{\mathcal{C}^1}$ defines a norm.
        \begin{proof}
        To show that $\norm{\cdot}_{\mathcal{C}^1}$ is a norm on 
        $\mathcal{C}([0,1])$, we'll show that it satisfies the relevant
        properties. Let $f, g \in \mathcal{C}([0,1]),\ \alpha \in \R$:\\\\
        \textbf{Non-negativity and identity}:\\
        As the sum of max norms on functions, $\norm{\cdot}_{\mC}$ is 
        clearly non-negative. We'll show that it has the identity property
        from the left and the right:
        \be 
            \item[$\Longleftarrow$:]
            If $\norm{f}_{\mC^1} = 0$, then
            $$\norm{f}_{\mC^1} = \norm{f}_\infty + \norm{f'}_\infty = 0$$
            The max norm is non-negative, so this means that
            $$\norm{f}_\infty = 0 \text{ and } \norm{f'}_\infty = 0$$
            As $\max_{x\in[0,1]}\abs{f(x)} = 0$, this implies that $f(x) 
            = 0$ for all $x \in [0,1]$.  

            \item[$\Longrightarrow$:]
            If $f(x) = 0$ for all $x \in [0,1]$, then                 
            \begin{align*}
                \norm{f}_{\mC^1} &= \norm{0}_{\mC^1} \\
                &= \norm{0}_\infty + \norm{0}_{\infty} \\
                &= 2 \max_{x\in[0,1]}\abs{0} \\
                &= 0
            \end{align*}
            and we have that $\norm{f}_{\mC^1} = 0$.
        \ee
        \textbf{Homogeneity}:\\
        We have that            
        \begin{align*}
        \norm{\alpha f}_{\mC^1} &= \norm{\alpha f}_\infty + 
        \norm{\alpha f'}_\infty\\ 
        &= \abs{\alpha}\norm{f}_\infty + 
        \abs{\alpha}\norm{f'}_\infty \\
        &= \abs{\alpha}\parens{\norm{f}_\infty + 
        \norm{f'}_\infty} \\
        &= \abs{\alpha}\norm{f}_{\mC^1}
        \end{align*}
        As desired.\\
        \textbf{Triangle Inequality}:\\
        We have that
        \begin{align*}
            \norm{f + g}_{\mC^1} &= \norm{f + g}_\infty + 
            \norm{\parens{f + g}'}_\infty \\
            &= \norm{f + g}_\infty + 
            \norm{f' + g'}_\infty \\
            &\leq \norm{f}_\infty + \norm{g}_\infty + \norm{f'}_\infty 
            + \norm{g'}_\infty \\
            &= \norm{f}_\infty + \norm{f'}_\infty + \norm{g}_\infty 
            + \norm{g'}_\infty \\
            &= \norm{f}_{\mC^1} + \norm{g}_{\mC^1} 
        \end{align*}
        So $\norm{\cdot}_{\mC^1}$ respects the Triangle Inequality.\\\\
        $\norm{\cdot}_{\mC^1}$ satisfies all the properties of a norm on
        $\mC([0,1])$.
        \end{proof}

        \item Show that $\parens{\mC^1([a,b]), \norm{\cdot}_{\mC^1}}$
        is a Banach Space.
        \begin{proof}
        Let $\bracks{f_n}$ be a Cauchy sequence in $\mC^1([a,b])$, such 
        that
        \begin{align*}
        \forall \vep > 0, \exists N \in \N, 
        \forall n,k \in \N, n\geq N,\ &\norm{f_{n+k}-f_n}_{\mC^1} <\vep \\ 
        \iff &\norm{f_{n+k} - f_n}_\infty + 
        \norm{\parens{f_{n+k} - f_n}'}_\infty < \vep 
        \end{align*}
        This implies that 
        $$\norm{f_{n+k} - f_n}_\infty < \vep \text{ and } 
        \norm{\parens{f_{n+k} - f_n}'}_\infty < \vep$$
        so $f_n$ and $f_n'$ are cauchy sequences with respect to 
        $\norm{\cdot}_\infty$. $[a,b]$ is closed and bounded and, by 
        Heine-Borel, it is compact; this also means that, by Theorem 4.6.1
        in the lecture notes, $\mC([a,b])$ is complete with respect to 
        $\norm{\cdot}_\infty$. So let $f,g \in \mC([a,b])$, 
        such that $f_n \to f$ and $f_n' \to g$. We now 
        wish to show that $f' = g$. Convergence with respect to 
        $\norm{\cdot}_\infty$ is equivalent to uniform convergence for 
        sequences of functions, this implies that $f$ is not just 
        continuous but also differentiable and that $f' = g$.
        Finally, this yields
        \begin{align*}
        \forall \vep > 0, \exists N \in \N, \forall n \in \N, n\geq N,\ 
        &\norm{f_n - f}_\infty < \frac{\vep}{2} \\ 
        \forall \vep > 0, \exists N \in \N, \forall n \in \N, n\geq N,\ 
        &\norm{f_n' - f'}_\infty < \frac{\vep}{2} \\ 
        \end{align*}
        Which is equivalent to saying 
        $$\norm{f_n - f}_{\mC^1} = \norm{f_n - f}_\infty + 
        \norm{f_n' - f'}_\infty < \frac{\vep}{2} + \frac{\vep}{2} = \vep$$
        So $\bracks{f_n}$ converges to a function $f \in \mC^1([a,b])$; 
        and as $\bracks{f_n}$ was arbitrarily chosen, this means that 
        $\parens{\mC^1([a,b]), \norm{\cdot}_{\mC^1}}$ is complete and a 
        Banach Space.
        \end{proof}
    \ee
    
    \item Let $w: [0,1] \to \R$ be a non-negative continuous function. We 
    define the \emph{weighted supremum} norm by 
    $$\forall f \in \mathcal{C}([0,1]),\ \norm{f}_w = \sup_{x\in [0,1]}
    w(x)\abs{f(x)}$$
    \be[1.] 
        \item If $\forall x \in (0,1),\ w(x) > 0$, show that 
        $\norm{\cdot}_w$ is a norm on $\mathcal{C}([0,1])$.
        \begin{proof}
        To show that $\norm{\cdot}_w$ is a norm, we need to show that 
        $\forall f,g,h \in \mathcal{C}\parens{\sqbracks{0,1}},\ \alpha \in 
        \R$, that it satisfies the following properties\\

        \textbf{Non-negativity and identity}:\\
        We have that the norm is the supremum of $w(x)\abs{f(x)}$ over 
        $x\in[0,1]$. Naturally, $\abs{f(x)} \geq 0,\ \forall x \in 
        [0,1]$, but we only have that $w(x) > 0$ for $x\in(0,1)$, so 
        for $x = 0$ or $x = 1,\ w(x)$ may be zero; but the 
        norm is a supremum, so the norm will always take the greatest 
        of the values of $w(x)\abs{f(x)}$ over all $x \in [0,1]$ which 
        will be positive for $f(x) \neq 0$ over all $x$, or zero if 
        $f(x) = 0$ over all $x$. Which means that $\norm{f}_w \geq 0$. 
        This also implies that $f(x) = 0,\ \forall x \in [0,1] 
        \implies \norm{f}_w = 0$. Now, for the converse:
        $$\norm{f}_w = 0 \implies 0 = \sup_{x\in[0,1]}w(x)\abs{f(x)} 
        \implies f(x) = 0,\forall x \in [0,1]$$
        because $w(x)$ cannot be zero everywhere.\\
        
        \textbf{Homogeneity}:\\
        We have that
        $$\norm{\alpha f}_w = \sup_{x\in\sqbracks{0,1}}w(x)\abs{\alpha 
        f(x)} = \sup_{x\in\sqbracks{0,1}}w(x)\abs{\alpha}\abs{f(x)} = 
        \abs{\alpha}\sup_{x\in\sqbracks{0,1}}w(x)\abs{f(x)} = 
        \abs{\alpha}\norm{f}_w$$

        \textbf{Triangle Inequality}:\\
        We have that 
        \begin{align*}
        \norm{f+g}_w &= \sup_{x\in\sqbracks{0,1}}w(x)\abs{f(x) + g(x)} 
        \\
        &\leq \sup_{x\in\sqbracks{0,1}}w(x)\parens{\abs{f(x)} + 
        \abs{g(x)}} \\
        &\leq \sup_{x\in\sqbracks{0,1}}w(x)\abs{f(x)} + 
        \sup_{x\in[0,1]}w(x)\abs{g(x)} \\
        &\leq \norm{f}_w + \norm{g}_w
        \end{align*}
        So
        $$\norm{f+g}_w \leq \norm{f}_w + \norm{g}_w$$

        Therefore, $\norm{\cdot}_w$ satisfies the properties of a norm 
        over $\mC\parens{\sqbracks{0,1}}$.
        \end{proof}

        \item If $\forall x \in [0,1],\ w(x) > 0$, show that 
        $\norm{\cdot}_w$ is a equivalent to $\norm{\cdot}_\infty$.
        \begin{proof}
        As $w(x)$ is 
        continuous on $[0,1]$, by the Extreme Value Theorem $w$ attains 
        its minimum and maximum on $[0,1]$, which we denote 
		$$W_l = \min_{x\in[0,1]}w(x),\  W_g = \max_{x\in[0,1]}w(x)$$
		so that we have 
		$$W_l \leq w(x) \leq W_g,\ \forall x \in [0,1]$$
        Also by the EVT and the definition of $\norm{\cdot}_\infty$, we 
        have that $\exists x_0 \in [0,1]: f(x) \leq \abs{f(x_0)} = 
        \norm{f}_\infty, \forall x \in [0,1]$. Now,
		\begin{align*}
		w(x) &\leq W_g \\
		w(x)\abs{f(x)} &\leq W_g\abs{f(x)} \\
		&\leq W_g\abs{f(x_0)} \\
		w(x)\abs{f(x)} &\leq W_g\norm{f}_\infty \\
        \sup_{x\in[0,1]}w(x)\abs{f(x)} &\leq 
        \sup_{x\in[0,1]}W_g\norm{f}_\infty \\
		\norm{f}_w &\leq W_g\norm{f}_\infty \\
		\Rightarrow \frac{1}{W_g} \norm{f}_w &\leq \norm{f}_\infty
		\end{align*}
		Then, 
		\begin{align*}
		W_l &\leq w(x) \\
		W_l\abs{f(x_0)} &\leq w(x)\abs{f(x_0)} \\
		&\leq \sup_{x\in[0,1]}w(x)\abs{f(x_0)} \\
		W_l\norm{f}_\infty &\leq \norm{f}_w \\
		\Rightarrow \norm{f}_\infty &\leq \frac{1}{W_l}\norm{f}_w
		\end{align*}
		So we have that
        $$\frac{1}{W_g} \norm{f}_w \leq \norm{f}_\infty \leq 
        \frac{1}{W_l}\norm{f}_w$$
        Which implies that $\norm{\cdot}_w$ and $\norm{\cdot}_\infty$ are 
        equivalent on $\mC\parens{\sqbracks{0,1}}$.
        \end{proof}
    \ee

\ee
\noindent\makebox[\linewidth]{\rule{\paperwidth}{0.4pt}}
	
\end{document}
