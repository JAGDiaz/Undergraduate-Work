\documentclass{article}
\usepackage{amsmath}
\usepackage{amssymb}
\usepackage{bbm}
\usepackage{bm}
\usepackage{amsthm}
\usepackage{enumerate}
\usepackage{graphicx}
\usepackage{psfrag}
\usepackage{color}
\usepackage{url}
\usepackage{listings}
\usepackage{xcolor}
\usepackage{tikz}
\usetikzlibrary{positioning}
\tikzset{main node/.style={circle,fill=gray!20,draw,minimum size=.5cm,inner sep=0pt},}

\definecolor{codegreen}{rgb}{0,0.5,0}
\definecolor{codewhite}{rgb}{1,1,1}
\definecolor{codegray}{rgb}{0.5,0.5,0.5}
\definecolor{codepurple}{rgb}{0.58,0,0.82}
\definecolor{codeblack}{rgb}{0,0,0}
\definecolor{codeorange}{rgb}{0.8,0.4,0}

\lstdefinestyle{mystyle}{
    backgroundcolor=\color{codewhite},   
    commentstyle=\color{codegray},
    keywordstyle=\color{codegreen},
    numberstyle=\color{codegray},
    stringstyle=\color{codeorange},
    basicstyle=\ttfamily ,
    breakatwhitespace=false,         
    breaklines=true,                 
    captionpos=b,                    
    keepspaces=true,                 
    numbers=left,                    
    numbersep=5pt,                  
    showspaces=false,                
    showstringspaces=false,
    showtabs=false,                  
    tabsize=4
}
\lstset{style=mystyle}


\setlength{\hoffset}{-1in}
\addtolength{\textwidth}{1.5in}
\setlength{\voffset}{-1in}
\addtolength{\textheight}{1.5in}
\newcommand{\be}{\begin{enumerate}}
\newcommand{\ee}{\end{enumerate}}
\newcommand{\BigO}[1]{\ensuremath\mathcal{O}\left(#1\right)}
\newcommand{\il}[1]{\lstinline!#1!}
\newcommand{\norm}[1]{\left|\left|#1\right|\right|}
\newcommand{\abs}[1]{\left|#1\right|}
\newcommand{\parens}[1]{\left(#1\right)}
\newcommand{\bracks}[1]{\left\{#1\right\}}
\newcommand{\sqbracks}[1]{\left[#1\right]}
\newcommand{\vep}{\varepsilon}
\newcommand{\ceiling}[1]{\left\lceil#1\right\rceil}
\newcommand{\R}{\mathbb{R}}
\newcommand{\N}{\mathbb{N}}
\newcommand{\Z}{\mathbb{Z}}
\newcommand{\K}{\mathbb{K}}
\newcommand{\one}{\mathbbm{1}}
\newcommand{\A}{\mathcal{A}}
\newcommand{\distrib}[2]{\text{#1}\left(#2\right)}
\newcommand{\dd}[1]{\frac{d}{d#1}}
\newcommand{\abracks}[1]{\left< #1\right>}
\newcommand{\lip}[1]{\text{Lip}\parens{#1}}

\newtheorem*{proposition}{Proposition}

\begin{document}
	\begin{center}
		\textbf{Fall 2020, Math 630:\ Exam 1} \\
		\textbf{Due:\ Sunday, October 11th, 2020} \\
		\textbf{Joseph Diaz: 819947915}
	\end{center}
\noindent\makebox[\linewidth]{\rule{\paperwidth}{0.4pt}}

\be[(E1)]
    \item \textbf{Minkowski's Inequality}\\
    We proved in class that $\parens{\K^n, \norm{\cdot}_1}$ and 
    $\parens{\K^n, \norm{\cdot}_\infty}$ are normed vector spaces.
    This problem aims at showing that $\forall p \in \R,\ 1 < p < 
    \infty,\ \parens{\K^n, \norm{\cdot}_p},$ where 
    $$\forall x = \parens{x_1,\ x_2,\ \cdots,\ x_n}\in \K^n,\
    \norm{x}_p = \parens{\sum_{k=1}^n \abs{x_k}^p}^{\frac{1}{p}}$$ 
    is a normed vector space, i.e. show that the above definition 
    defines a norm. The challenging part is to show the triangle 
    inequality, also called Minkowski's inequality in this case. In
    order to prove the triangle inequality, you have to consider the 
    fact that the function $f(t) = t^p$ is convex 
    $\forall t\in \R^+$, i.e.
    $$\forall \alpha \in [0,1],\ \forall u,v \in \R^+,\ 
    \parens{\alpha u + (1- \alpha)v}^p \leq \alpha u^p + 
    (1-\alpha)v^p$$
    An appropriate choice of $\alpha,\ u,$ and $v$ will allow you to 
    show Minkowski's Inequality.
    
    \begin{proof}
    To show that $\forall p \in \R,\ 1< p < \infty,\
    \parens{\K^n,\ \norm{\cdot}_p}$ is a normed vector space, we'll
    show that $\norm{\cdot}_p$ satisfies the properties of a norm on
    $\K^n$, for all $x,y,z \in \K^n$.\\\\
    \textbf{Non-negativity and Identity}:\\
    The non-negativity of $\norm{\cdot}_p$ is obvious because it is
    defined as the $p^{\text{th}}$ root of a sum of positive real
    numbers, so it can never be negative. We'll show the identity
    property from the left and the right.
    \be 
        \item[$\Longrightarrow$:] We have that $\norm{x}_{p} = 0$, 
        this implies that
        $$\sum_{k=1}^n \abs{x_k}^p = 0$$
        As a sum of non-negative elements, $\abs{x_k}^p = 0 
        \implies x_k = 0$ for $1 \leq k \leq n$. So $x = \vec{0}$, 
        i.e.
        $$x = \parens{0,\ 0,\ \cdots,\ 0}$$
        
        \item[$\Longleftarrow$:] We have that $x = \vec{0}$, so 
        $$\norm{x}_p = \parens{\sum_{k=1}^n\abs{0}^p}^{\frac{1}{p}} =
        \sqrt[p]{0} = 0$$
    \ee
    So $\norm{\cdot}_p$ satisfies both of these properties.
    
    \textbf{Homogeneity}:\\
    Let $\alpha \in \K$, then
    \begin{align*}
    \norm{\alpha x}_p &= \parens{\sum_{k=1}^n\abs{\alpha x_k}^p}^{
    \frac{1}{p}} \\
    &= \parens{\sum_{k=1}^n\abs{\alpha}^p \abs{x_k}^p}^{
    \frac{1}{p}} \\
    &= \parens{\abs{\alpha}^p\sum_{k=1}^n \abs{x_k}^p}^{
    \frac{1}{p}} \\
    &= \sqrt[p]{\abs{\alpha}^p}\parens{\sum_{k=1}^n \abs{x_k}^p}^{
    \frac{1}{p}} \\
    &= \abs{\alpha}\parens{\sum_{k=1}^n \abs{x_k}^p}^{
    \frac{1}{p}} \qquad (\abs{\alpha} \geq 0) \\
    &= \abs{\alpha}\norm{x}_p
    \end{align*}
    So $\norm{\cdot}_p$ has the homogeneity property.
    
    \textbf{Triangle Inequality}:\\
    To show that $\norm{\cdot}_p$ respects the triangle inequality,
    we'll use the function $f(t) = t^p$ and 
    exploit the fact that it is convex. From above, we have
    $\forall t\in \R^+$:
    $$\forall \alpha \in [0,1],\ \forall u,v \in \R^+,\ 
    \parens{\alpha u + (1- \alpha)v}^p \leq \alpha u^p + 
    (1-\alpha)v^p$$
    From properties we've already shown, we have
    \begin{gather*}
    0 \leq \norm{x}_p \leq \norm{x}_p + \norm{y}_p \\
    \implies 0 \leq \frac{\norm{x}_p}{\norm{x}_p + \norm{y}_p} \leq 1
    \end{gather*}
    So let $\alpha = \frac{\norm{x}_p}{\norm{x}_p + \norm{y}_p}$, now 
    for $u$ and $v$, we consider that
    $$0 \leq \abs{x_k} \leq \norm{x}_p,\  
    0 \leq \abs{y_k} \leq \norm{y}_p  
    \implies 0 \leq \frac{\abs{x_k}}{\norm{x}_p} \leq 1,\ 
    0 \leq \frac{\abs{y_k}}{\norm{y}_p} \leq 1$$
    So let $u = \frac{\abs{x_k}}{\norm{x}_p},\ 
    v = \frac{\abs{y_k}}{\norm{y}_p}$. Substituting these into our 
    inequality involving the convexity of $f$ gives:
    \begin{align*}
    \parens{\alpha u + (1- \alpha)v}^p &\leq \alpha u^p + 
    (1-\alpha)v^p \\
    \parens{\frac{\norm{x}_p}{\norm{x}_p + \norm{y}_p} 
    \frac{\abs{x_k}}{\norm{x}_p} + \parens{1- \frac{\norm{x}_p}
    {\norm{x}_p + \norm{y}_p}}\frac{\abs{y_k}}{\norm{y}_p}}^p 
    &\leq \frac{\norm{x}_p}{\norm{x}_p + \norm{y}_p} \parens{
    \frac{\abs{x_k}}{\norm{x}_p}}^p + 
    \parens{1-\frac{\norm{x}_p}{\norm{x}_p + \norm{y}_p}}\parens{
    \frac{\abs{y_k}}{\norm{y}_p}}^p \\
    \parens{\frac{\abs{x_k}}{\norm{x}_p + \norm{y}_p} 
    + \frac{\norm{y}_p}
    {\norm{x}_p + \norm{y}_p}\frac{\abs{y_k}}{\norm{y}_p}}^p 
    &\leq \frac{\norm{x}_p}{\norm{x}_p + \norm{y}_p}
    \frac{\abs{x_k}^p}{\norm{x}_p^p} + 
    \frac{\norm{y}_p}{\norm{x}_p + \norm{y}_p}
    \frac{\abs{y_k}^p}{\norm{y}_p^p} \\
    \parens{\frac{\abs{x_k}}{\norm{x}_p + \norm{y}_p} 
    + \frac{\abs{y_k}}
    {\norm{x}_p + \norm{y}_p}}^p 
    &\leq \frac{1}{\norm{x}_p + \norm{y}_p}
    \frac{\abs{x_k}^p}{\norm{x}_p^{p-1}} + 
    \frac{1}{\norm{x}_p + \norm{y}_p}
    \frac{\abs{y_k}^p}{\norm{y}_p^{p-1}} \\
    \parens{\frac{\abs{x_k} + \abs{y_k}}{\norm{x}_p + \norm{y}_p}}^p 
    &\leq \frac{1}{\norm{x}_p + \norm{y}_p}
    \parens{\frac{\abs{x_k}^p}{\norm{x}_p^{p-1}} + 
    \frac{\abs{y_k}^p}{\norm{y}_p^{p-1}}} \\
    \frac{\parens{\abs{x_k} + \abs{y_k}}^p}{
    \parens{\norm{x}_p + \norm{y}_p}^p} 
    &\leq \frac{1}{\norm{x}_p + \norm{y}_p}
    \parens{\frac{\abs{x_k}^p}{\norm{x}_p^{p-1}} + 
    \frac{\abs{y_k}^p}{\norm{y}_p^{p-1}}} \\
    \end{align*}
    Now, we consider the sum of this inequality for all $k \in 
    \bracks{1,\ \cdots,\ n}$:
    \begin{align*}
    \frac{\parens{\abs{x_k} + \abs{y_k}}^p}{
    \parens{\norm{x}_p + \norm{y}_p}^p} 
    &\leq \frac{1}{\norm{x}_p + \norm{y}_p}
    \parens{\frac{\abs{x_k}^p}{\norm{x}_p^{p-1}} + 
    \frac{\abs{y_k}^p}{\norm{y}_p^{p-1}}} \\
    \sum_{k=1}^n\frac{\parens{\abs{x_k} + \abs{y_k}}^p}{
    \parens{\norm{x}_p + \norm{y}_p}^p} 
    &\leq \sum_{k=1}^n\frac{1}{\norm{x}_p + \norm{y}_p}
    \parens{\frac{\abs{x_k}^p}{\norm{x}_p^{p-1}} + 
    \frac{\abs{y_k}^p}{\norm{y}_p^{p-1}}} \\
    \frac{1}{\parens{\norm{x}_p + \norm{y}_p}^p} \sum_{k=1}^n 
    \parens{\abs{x_k} + \abs{y_k}}^p
    &\leq \frac{1}{\norm{x}_p + \norm{y}_p}
    \parens{\frac{1}{\norm{x}_p^{p-1}}\sum_{k=1}^n\abs{x_k}^p + 
    \frac{1}{\norm{y}_p^{p-1}}\sum_{k=1}^n\abs{y_k}^p} \\
    \frac{1}{\parens{\norm{x}_p + \norm{y}_p}^p} \sum_{k=1}^n 
    \parens{\abs{x_k} + \abs{y_k}}^p
    &\leq \frac{1}{\norm{x}_p + \norm{y}_p}
    \parens{\frac{1}{\norm{x}_p^{p-1}}\norm{x}_p^p + 
    \frac{1}{\norm{y}_p^{p-1}}\norm{y}_p^p} \\
    \frac{1}{\parens{\norm{x}_p + \norm{y}_p}^p} \sum_{k=1}^n 
    \parens{\abs{x_k} + \abs{y_k}}^p
    &\leq \frac{1}{\norm{x}_p + \norm{y}_p}
    \parens{\norm{x}_p + \norm{y}_p} \\
    \frac{1}{\parens{\norm{x}_p + \norm{y}_p}^p} \sum_{k=1}^n 
    \parens{\abs{x_k} + \abs{y_k}}^p &\leq 1 \\
    \sum_{k=1}^n 
    \parens{\abs{x_k} + \abs{y_k}}^p &\leq \parens{\norm{x}_p 
    + \norm{y}_p}^p
    \end{align*}
    Now, by the triangle inequality on $\abs{\cdot}$ we know that 
    $$\sum_{k=1}^n\abs{x_k + y_k}^p \leq \sum_{k=1}^n
    \parens{\abs{x_k} 
    + \abs{y_k}}^p$$
    So we can deduce that
    \begin{align*}
    \sum_{k=1}^n\abs{x_k + y_k}^p &\leq \parens{\norm{x}_p 
    + \norm{y}_p}^p \\
    \parens{\sum_{k=1}^n\abs{x_k + y_k}^p}^{\frac{1}{p}} &\leq 
    \norm{x}_p + \norm{y}_p \\
    \norm{x+y}_p &\leq \norm{x}_p + \norm{y}_p
    \end{align*}
    So $\norm{\cdot}_p$ respects the triangle inequality.\\\\
    $\norm{\cdot}_p$ satisfies all of the properties of a norm on
    $\K^n$, so $\parens{\K^n, \norm{\cdot}_p}$ is a normed vector 
    space.
    \end{proof}
    
    
    \item \textbf{H{\"o}lder's Inequality}\\
    This problem will guide you along the process of proving 
    H{\"o}lder's Inequality; using the same definition for
    $\norm{\cdot}_p$ from the previous exercise,
    $$\forall p,q \in \R,\ 1 < p < \infty,\ \frac{1}{p} + \frac{1}{q}
    = 1,\ \forall x,y \in \K^n,\ \sum_{k=1}^n \abs{x_ky_k} \leq 
    \norm{x}_p\norm{y}_q$$
    \be[1.]
        \item Show that 
        $$\forall a,b \geq 0,\ \forall p,q \in \R, 1 < p < \infty,\ 
        \frac{1}{p} + \frac{1}{q} = 1,\ ab \leq \frac{a^p}{p} + 
        \frac{b^q}{q}$$
        (Hint: You can study the function $\forall a \in \R^+,\ 
        f(a) = \frac{a^p}{p} + \frac{b^q}{q} - ab$)
        \begin{proof}
        Let $f(a) = \frac{a^p}{p} + \frac{b^q}{q} - ab$ for all 
        $a \geq 0$ with the same constraints on $p,\ q,$ and $b$ from 
        above, then we have that the derivatives of $f$ with
        respect to $a$ are
        \begin{gather*}
        f'(a) = a^{p-1} - b \\
        f''(a) = (p-1)a^{p-2}
        \end{gather*}
        $f''(a) \geq 0,\ \forall a \geq 0$, this implies 
        that $f$ is convex on it's domain and is, consequently, 
        increasing monotonically. By solving $f'$ for $0$
        we also find that the minimum of $f$ is
        \begin{align*}
        f'(a) &= 0 \\
        a^{p-1} - b &= 0 \\
        a^{p-1} &= b \\
        \implies a &= b^{\frac{1}{p-1}}
        \end{align*}
        Evaluating $f$ at this point gives
        $$f\parens{b^{\frac{1}{p-1}}} = \frac{b^{\frac{p}{p-1}}}{p} 
        + \frac{b^q}{q} - \parens{b^{\frac{1}{p-1}}}b$$
        By the constraints on $p$ and $q$, we have
        $$\frac{1}{p} + \frac{1}{q} = 1 \implies q = \frac{p}{p-1}$$
        Using this, $f\parens{b^{\frac{1}{p-1}}}$ can be rewritten as
        \begin{align*}
        f\parens{b^{\frac{1}{p-1}}} &= \frac{b^{\frac{p}{p-1}}}{p} 
        + \frac{b^q}{q} - \parens{b^{\frac{1}{p-1}}}b \\
        &= \frac{b^q}{p} 
        + \frac{b^q}{q} - b^{\frac{1}{p-1}+1} \\
        &= b^q\parens{\frac{1}{p} + \frac{1}{q}} - 
        b^{\frac{1}{p-1}+1} \\
        &= b^q - b^{\frac{1}{p-1}+1} \\
        &= b^q - b^{\frac{p}{p-1}} \\
        &= b^q - b^q \\
        &= 0
        \end{align*}
        So the minimum of $f$ is $0$, this implies that $f$ is 
        non-negative on it's domain, this yields:
        \begin{gather*}
        f(a) = \frac{a^p}{p} + \frac{b^q}{q} - ab \geq 0 \\
        \implies ab \leq \frac{a^p}{p} + \frac{b^q}{q},\ 
        \forall a,b \geq 0
        \end{gather*}
        Which is what we wanted to show.
        \end{proof}
        
        \item With in an appropriate $a$ and $b$ from the previous 
        inequality. prove H{\"o}lder's Inequality.
        \begin{proof}
        From part 1, we know that $q = \frac{p}{p-1}$, $p > 1$ so we 
        can also deduce from that $q > 1$. To take a page from
        Problem 1, let $$a = \frac{\abs{x_k}}{\norm{x}_p},\ b = 
        \frac{\abs{y_k}}{\norm{y}_q}$$
        Substituting these into the inequality from part 1 gives:
        \begin{align*}
        ab &\leq \frac{a^p}{p} + \frac{b^q}{q} \\
        \frac{\abs{x_k}}{\norm{x}_p}\cdot\frac{\abs{y_k}}{\norm{y}_q}
        &\leq \frac{1}{p}\cdot\frac{\abs{x_k}^p}
        {\norm{x}_p^p} + \frac{1}{q}\cdot\frac{\abs{y_k}^q}
        {\norm{y}_q^q} \\
        \frac{1}{\norm{x}_p\norm{y}_q}\abs{x_ky_k} &\leq 
        \frac{1}{p\norm{x}_p^p}\abs{x_k}^p + 
        \frac{1}{q\norm{y}_q^q}\abs{y_k}^q \\
        \frac{1}{\norm{x}_p\norm{y}_q}\sum_{k=1}^n\abs{x_ky_k} &\leq 
        \frac{1}{p\norm{x}_p^p}\sum_{k=1}^n\abs{x_k}^p + 
        \frac{1}{q\norm{y}_q^q}\sum_{k=1}^n\abs{y_k}^q \\
        \frac{1}{\norm{x}_p\norm{y}_q}\sum_{k=1}^n\abs{x_ky_k} &\leq 
        \frac{1}{p\norm{x}_p^p}\norm{x}_p^p + 
        \frac{1}{q\norm{y}_q^q}\norm{y}_q^q \\
        \frac{1}{\norm{x}_p\norm{y}_q}\sum_{k=1}^n\abs{x_ky_k} &\leq 
        \frac{1}{p} + \frac{1}{q} \\
        \frac{1}{\norm{x}_p\norm{y}_q}\sum_{k=1}^n\abs{x_ky_k} &\leq 
        1 \\
        \implies \sum_{k=1}^n \abs{x_ky_k} &\leq \norm{x}_p\norm{y}_q
        \end{align*}
        As desired.
        \end{proof}
        
        \item Do you recognize this inequality for $p=q=2$?
        \begin{proof}[Answer]
        Yes! For $p=q=2$, H{\"o}lder's Inequality becomes the 
        Cauchy-Schwarz Inequality!
        \end{proof}
    \ee
        
    \item \textbf{Lipschitz Spaces}\\
    Let $(X,d)$ be a metric space. Let a function $f: X \to \R$, $f$
    is said to be Lipschitz if
    $$\exists C > 0,\ \forall x,y \in X,\ \abs{f(x) - f(y)} \leq 
    Cd(x,y)$$
    We denote $\lip{X}$ the set of all Lipschitz functions 
    defined over $X$. Then, $\forall f \in \lip{X}$, we can 
    define the so-called Lipschitz constant of $f$ by
    $$\lip{f} = \sup\bracks{\frac{\abs{f(x) - f(y)}}{d(x,y)}\ 
    \middle|\ \forall x,y \in X,\ x\neq y} = 
    \sup_{x\neq y}\frac{\abs{f(x) - f(y)}}{d(x,y)}$$
    \be[1.]
        \item Show that $\lip{X}$ is a vector space.
        \begin{proof}
        To show that $\lip{X}$ is a vector space, we'll show
        that it satisfies the properties of one. Let $f,g \in 
        \lip{X}$, and $x,y \in X$.\\\\
        \textbf{Closure under addition}:\\
        For $f + g$ to be in $\lip{X}$, there must exist $C \in 
        \R^+$ such that
        $$\abs{(f+g)(x) - (f+g)(y)} \leq Cd(x,y)$$
        To find $C$, we'll start with the left side of the 
        inequality.
        \begin{align*}
        \abs{(f+g)(x) - (f+g)(y)} &= \abs{f(x) + g(x) - f(y)-g(y)} \\
        &= \abs{f(x) - f(y) + g(x) - g(y)} \\
        &\leq \abs{f(x) - f(y)} + \abs{g(x) - g(y)} \\
        \end{align*}
        $f, g \in \lip{X}$, so there exists $C_1, C_2 \in\R^+$ 
        such that
        $$\abs{f(x) - f(y)} \leq C_1d(x,y),\quad 
        \abs{g(x) - g(y)} \leq C_2d(x,y)$$
        Using this and the previous inequality gives
        $$\abs{(f+g)(x) - (f+g)(y)} \leq \abs{f(x) - f(y)} +
        \abs{g(x) - g(y)} \leq C_1d(x,y) + C_2d(x,y)$$
        Which is to say that
        $$\abs{(f+g)(x) - (f+g)(y)} \leq C_1d(x,y) + C_2d(x,y) = 
        \parens{C_1 + C_2}d(x,y)$$
        So letting $C = C_1 + C_2$ gives
        $$\abs{(f+g)(x) - (f+g)(y)} \leq Cd(x,y)$$ 
        and so $f+g \in \lip{X}$.\\\\
        \textbf{Closure under scalar multiplication}:\\
        Let $\alpha \in \R$, for $\alpha f$ to be in $\lip{X}$
        there must exist $C \in \R^+$ such that
        $$\abs{\alpha f(x) - \alpha f(y)} \leq Cd(x,y)$$ 
        Consider
        $$\abs{\alpha f(x) - \alpha f(y)} = \abs{\alpha}
        \abs{f(x) - f(y)}$$
        $f \in \lip{X}$, so there exists $C_1$ such that
        $$\abs{f(x) - f(y)} \leq C_1d(x,y)$$
        Let $C = \abs{\alpha}C_1$ and we have that 
        $$\abs{\alpha f(x) - \alpha f(y)} = \abs{\alpha}
        \abs{f(x) - f(y)} \leq Cd(x,y)$$
        So $\alpha f \in \lip{X}$.\\\\
        \textbf{Commutative addition}:\\
        As Lipschitz functions are real valued, they obey 
        commutativity of addition of real numbers; so
        $$f + g = g + f$$
        \textbf{Associative addition}:\\
        As Lipschitz functions are real valued, they obey 
        associativity of addition of real numbers; let, $h \in 
        \lip{X}$, then
        $$(f + g) + h = f + (g+h)$$
        \textbf{Distributive property}:\\
        As Lipschitz functions are real valued, they obey 
        the distributive property of addition of real numbers; so
        let $\beta \in \R$, then
        $$\alpha(f+g) = \alpha f + \alpha g$$
        and 
        $$(\alpha + \beta)f = \alpha f + \beta f$$ 
        \textbf{Associative scalar multiplication}:\\
        As Lipschitz functions are real valued, they obey 
        the associative property of scalar multiplication; so
        $$\alpha(\beta f) = (\alpha\beta)f$$
        \textbf{Additive identity}:\\
        The all-zero function $f(x) = 0$ is in $\lip{X}$, because
        $$\abs{f(x)-f(y)} = \abs{0-0} = 0 \leq Cd(x,y)$$
        which is true for any choice of $C \in \R^+$. So 
        $\lip{X}$ has an additive identity.\\\\
        So $\lip{X}$ satisfies all of the properties of a 
        vector space.
        
        \end{proof}
        
        \item Let $a \in X$, now we denote $\forall f \in \text{Lip}
        (X)$,
        $$\norm{f}_{\text{Lip},a} = \abs{f(a)} + \text{Lip}(f)$$
        \be[i.]
            \item Show that $\parens{\lip{X}, 
            \norm{\cdot}_{\text{Lip},a}}$ defines a normed vector 
            space.
            \begin{proof}
            Having shown that $\lip{X}$ is a vector space, we need
            only show that $\norm{\cdot}_{\text{Lip},a}$ satisfies 
            the properties of a norm on $\lip{X}$. Let $f,g \in 
            \lip{X},\ a \in X$.\\\\
            \textbf{Non-negativity and Identity}:\\
            By it's own definition, we can can say that 
            $\norm{\cdot}_{\text{Lip},a}$ is non-negative; as 
            $\abs{f(a)}$ is non-negative and so is $\lip{f}$. We'll
            show the identity property is respected from the left and 
            the right.
            \be
            \item[$\Longrightarrow$:]
            Suppose $f(x) = 0$ for all $x \in X$, then 
            $\abs{f(a)}=0$, and the Lipschitz constant is
            $$\lip{f} = \sup_{x\neq y}\frac{\abs{f(x) - f(y)}}
            {d(x,y)} = \sup_{x\neq y}\frac{\abs{0 - 0}}
            {d(x,y)} = 0$$
            so $\norm{f}_{\text{Lip},a} = 0$.
            \item[$\Longleftarrow$:]
            Suppose that $\norm{f}_{\text{Lip},a} = 0$, this implies 
            that
            $$\abs{f(a)} + \lip{f} = 0$$
            As both quantities are non-negative, this means they each
            equal 0. Examining $\lip{f}$ a little more closely, 
            we get that 
            $$\lip{f} = 0 \implies \sup_{x\neq y}
            \frac{\abs{f(x) - f(y)}}{d(x,y)} = 0$$
            So, for any $x, y \in X,\ x \neq y,\ 
            \abs{f(x) - f(y)} = 0$; this is equivalent to $f(x) = 
            f(y)$. Letting $a = y$ then gives $f(x) = f(a) = 0$; 
            and we have that $f(x) = 0$.
            
            \ee
            \textbf{Homogeneity}:\\
            Let $\alpha \in \R$, then 
            \begin{align*}
            \norm{\alpha f}_{\text{Lip},a} &= \abs{\alpha f(a)} +
            \sup_{x\neq y}
            \frac{\abs{\alpha f(x) - \alpha f(y)}}{d(x,y)} \\
            &= \abs{\alpha}\abs{f(a)} + \sup_{x\neq y}
            \frac{\abs{\alpha}\abs{f(x) - f(y)}}{d(x,y)} \\
            &= \abs{\alpha}\abs{f(a)} + \abs{\alpha}\sup_{x\neq y}
            \frac{\abs{f(x) - f(y)}}{d(x,y)} \\ 
            &= \abs{\alpha}\parens{\abs{f(a)} + \sup_{x\neq y}
            \frac{\abs{f(x) - f(y)}}{d(x,y)}} \\ 
            &= \abs{\alpha}\norm{f}_{\text{Lip},a}
            \end{align*}
            \textbf{Triangle Inequality}:\\
            \begin{align*}
            \norm{f + g}_{\text{Lip},a} &= \abs{f(a) + g(a)} +
            \sup_{x\neq y}\frac{\abs{(f+g)(x) - (f+g)(y)}}
            {d(x,y)} \\ 
            &= \abs{f(a) + g(a)} +
            \sup_{x\neq y}\frac{\abs{f(x) + g(x) - f(y) - g(y)}}
            {d(x,y)} \\ 
            &= \abs{f(a) + g(a)} +
            \sup_{x\neq y}\frac{\abs{f(x) - f(y) + g(x) - g(y)}}
            {d(x,y)} \\ 
            &\leq \abs{f(a)} + \abs{g(a)} +  
            \sup_{x\neq y}\frac{\abs{f(x) - f(y)} + 
            \abs{g(x) - g(y)}}{d(x,y)} \\ 
            &= \abs{f(a)} + \abs{g(a)} +  
            \sup_{x\neq y}\frac{\abs{f(x) - f(y)}}{d(x,y)} + 
            \sup_{x\neq y}\frac{\abs{g(x) - g(y)}}{d(x,y)} \\ 
            &= \abs{f(a)} + \sup_{x\neq y}\frac{\abs{f(x) - f(y)}}
            {d(x,y)} + \abs{g(a)} +   
            \sup_{x\neq y}\frac{\abs{g(x) - g(y)}}{d(x,y)} \\ 
            &= \norm{f}_{\text{Lip},a} + \norm{g}_{\text{Lip},a}
            \end{align*}
            So $\norm{\cdot}_{\text{Lip},a}$ satisfies the 
            properties of a norm on $\lip{X}$, and $\parens{\lip{X},
            \norm{\cdot}_{\text{Lip},a}}$ is a normed vector space.
            \end{proof}
            
            \item Let $b \in X,\ b\neq a$. Show that $
            \norm{\cdot}_{\text{Lip},a}$ and 
            $\norm{\cdot}_{\text{Lip},b}$ are equivalent norms.
            \begin{proof}
            We want to show that 
            $$\exists c, c' \in \R^+:\ \forall f \in \lip{X},\
            c'\norm{f}_{\text{Lip},b} \leq
            \norm{f}_{\text{Lip},a} \leq c\norm{f}
            _{\text{Lip},b}$$
            To find these constants, let's start with 
            $\norm{f}_{\text{Lip},a}$:
            \begin{align*}
            \norm{f}_{\text{Lip},a} &= \abs{f(a)} + \lip{f} \\
            &= \abs{f(a) - f(b) + f(b)} + \lip{f} \\
            &\leq \abs{f(a) - f(b)} + \abs{f(b)} + \lip{f} \\
            \end{align*}
            We have that 
            \begin{gather*}
            \frac{\abs{f(a) - f(b)}}{d(a,b)} \leq \sup_{x\neq y}
            \frac{\abs{f(x) - f(y)}}{d(x,y)} = \lip{f} \\
            \implies \abs{f(a) - f(b)} \leq \lip{f}d(x,y)
            \end{gather*}
            Now:
            \begin{align*}
            \norm{f}_{\text{Lip},a} &\leq \abs{f(a) - f(b)} 
            + \abs{f(b)} + \lip{f} \\
            &\leq \lip{f}d(x,y) + \abs{f(b)} + \lip{f} \\
            &= \abs{f(b)} + \parens{1 + d(a,b)}\lip{f}
            \end{align*}
            We also have that
            $$1 \leq 1 + d(a,b) \implies \abs{f(b)} \leq 
            \parens{1 + d(a,b)}\abs{f(b)}$$
            So
            \begin{align*}
            \norm{f}_{\text{Lip},a} &\leq \abs{f(b)} + 
            \parens{1 + d(a,b)}\lip{f} \\ 
            &\leq \parens{1 + d(a,b)}\abs{f(b)} + 
            \parens{1 + d(a,b)}\lip{f} \\
            &= \parens{1 + d(a,b)}\parens{\abs{f(b)} + \lip{f}} \\
            &= \parens{1 + d(a,b)}\norm{f}_{\text{Lip}, b}
            \end{align*}
            and this gives us $c = 1 + d(a,b)$:
            $$\norm{f}_{\text{Lip},a} \leq \parens{1 + d(a,b)}
            \norm{f}_{\text{Lip},b}$$
            A nearly identical argument can be made to show that
            $$\implies \norm{f}_{\text{Lip},b} \leq 
            \parens{1 + d(a,b)}\norm{f}_{\text{Lip},a}$$
            dividing both sides of this equation by $c$ gives us
            $$\frac{1}{1+d(a,b)}\norm{f}_{\text{Lip},b} \leq 
            \norm{f}_{\text{Lip},a}$$
            So let $c' = 1/c$. Finally, this implies
            $$\frac{1}{1+d(a,b)}\norm{f}_{\text{Lip},b} \leq
            \norm{f}_{\text{Lip},a} \leq \parens{1+d(a,b)}\norm{f}
            _{\text{Lip},b}$$
            and we can conclude that $\norm{f}_{\text{Lip},a}$ and
            $\norm{f}_{\text{Lip},b}$ are equivalent norms. 
            \end{proof}
            
        \ee 
        
        \item We will now study the completeness of $\forall a \in X,
        \parens{\text{Lip}(X),\norm{\cdot}_{\text{Lip},a}}$. In the
        following, we consider an arbitrary Cauchy sequence $
        \bracks{f_n}$ in $\lip{X}$, i.e. $\forall n \in \N,
        f_n \in \lip{X}$.
        \be[i.]
            \item Prove that, $\forall x \in X$, the sequence 
            $\bracks{f_n}$ converges pointwise to a function that 
            we will denote $f$. (Hint: Consider the case that 
            $x = a$ first and then the case that $x\neq a$.)
            \begin{proof}
            By assumption, $\bracks{f_n}$ is Cauchy, so 
            \begin{align*}
            \forall \vep,\ \exists N \in \N : \forall n,k \in \N, 
            n \geq N, &\norm{f_n - f_{n+k}}_{\text{Lip},a} < \vep \\
            \implies &\abs{f_n(a) - f_{n+k}(a)} + \lip{f_n - 
            f_{n+k}} < \vep 
            \end{align*}
            Both summands of $\norm{f_n - f_{n+k}}_{\text{Lip},a}$
            are non-negative, So let $\vep_1, \vep_2 >0$ such that
            $\vep_1 + \vep_2 = \vep$ and
            $$\abs{f_n(a) - f_{n+k}(a)} < \vep_1,\qquad \lip{f_n - 
            f_{n+k}} < \vep_2$$
            In the case of $\abs{f_n(a) - f_{n+k}(a)} < \vep_1 < 
            \vep$, this implies that for $x=a,\ \bracks{f_n}$ is a 
            Cauchy sequence in 
            $\R$ and that it converges to a limit that we'll denote 
            as $f(a)$. On the hand, for $\lip{f_n - f_{n+k}} < 
            \vep_2 < \vep$, we have
            $$\lip{f_n - f_{n+k}} = \sup_{x\neq y}
            \frac{\abs{\parens{f_n-f_{n+k}}(x) - 
            \parens{f_n-f_{n+k}}(y)}}{d(x,y)} < \vep_2$$
            Let $y = a$, then for all $x\neq a$,
            \begin{gather*}            
            \frac{\abs{f_n(x)-f_{n+k}(x) - 
            f_n(a)+f_{n+k}(a)}}{d(x,a)} < \vep_2 \\
            \implies \abs{f_n(x)-f_{n+k}(x) - 
            f_n(a)+f_{n+k}(a)} < \vep_2d(x,a)
            \end{gather*}
            With these, we can now show what we wanted to show, for 
            $x\neq a$:
            \begin{align*}
            \abs{f_n(x) - f_{n+k}(x)} &= \abs{f_n(x) - f_{n+k}(x) + 
            f_n(a) - f_n(a) + f_{n+k}(a) - f_{n+k}(a)} \\
            &= \abs{f_n(x) - f_{n+k}(x) - f_n(a) + f_{n+k}(a) + 
            f_n(a) - f_{n+k}(a)} \\
            &= \abs{f_n(x) - f_{n+k}(x) - f_n(a) + f_{n+k}(a)} + 
            \abs{f_n(a) - f_{n+k}(a)} \\
            &< \vep_2d(x,a) + \vep_1
            \end{align*}
            By the previously shown inequalities. We also have that 
            $\vep_1, \vep_2 < \vep$, so
            $$\abs{f_n(x) - f_{n+k}(x)} < \vep_2d(x,a) + \vep_1 < 
            \vep d(x,a) + \vep$$
            and so:
            $$\forall x \in X, \forall \vep > 0, \exists N \in \N : 
            \forall n \geq N, \abs{f_n(x) - f(x)} < 
            \vep(d(x,a) + 1)$$
            Since the convergence of $\bracks{f_n}$ was dependent on 
            the value of $x$, we may conclude that it converges
            \emph{point-wise} to $f$ and that
            $$\lim_{n\to\infty}f_n(x) = f(x)$$
            \end{proof}
            
            \item Prove that $\lim_{n\to\infty}\norm{f_n - f}_{
            \text{Lip},a} = 0$.
            \begin{proof}
            To show this, we'll show that 
            $$\forall \tilde{\vep} >0, \exists N \in \N: \forall n
            \geq N, \norm{f_n - f}_{\text{Lip}, a} < \tilde{\vep}$$
            In 3.3.i, it was shown that $\bracks{f_n}$ converges 
            point-wise to $f$. So, we have that 
            \begin{align*}
            \forall a,x,y \in X, x\neq y, \forall \vep > 0, 
            \exists N \in \N : \forall n \geq N, & \abs{f_n(a) - 
            f(a)} < \vep\\
            & \abs{f_n(x) - f(x)} < \vep d(x,y)\\
            & \abs{f_n(y) - f(y)} < \vep d(x,y)
            \end{align*}
            With this, we have
            \begin{align*}
            \norm{f_n - f}_{\text{Lip},a} &= \abs{f_n(a) - f(a)} +
            \sup_{x\neq y}\frac{\abs{\parens{f_n - f}(x) - 
            \parens{f_n - f}(y)}}{d(x,y)} \\
            &= \abs{f_n(a) - f(a)} +
            \sup_{x\neq y}\frac{\abs{f_n(x) - f(x) + f(y) - f_n(y)}}
            {d(x,y)} \\
            &\leq \abs{f_n(a) - f(a)} +
            \sup_{x\neq y}\parens{\frac{\abs{f_n(x) - f(x)} + 
            \abs{f(y) - f_n(y)}}
            {d(x,y)}} \\
            &< \vep +
            \sup_{x\neq y}\parens{\frac{\vep d(x,y) + \vep d(x,y)}
            {d(x,y)}} \\
            &= \vep + \sup_{x\neq y}\parens{\frac{2\vep d(x,y)}
            {d(x,y)}} \\
            &= \vep + \sup_{x\neq y}2\vep \\
            &= 3\vep
            \end{align*}
            So, we have that
            $$\forall \vep >0, \exists N \in \N: \forall n
            \geq N, \norm{f_n - f}_{\text{Lip}, a} < 3\vep$$
            and we can conclude that $\lim_{n\to\infty}
            \norm{f_n - f}_{\text{Lip},a} = 0$.
            \end{proof}
            
            \item Prove that $f \in \text{Lip}(X)$. Conclude about 
            the completeness of $\parens{\text{Lip}(X), 
            \norm{\cdot}_{\text{Lip},a}}$.
            \begin{proof}
            From 3.3.ii, we have that 
            $$\forall \vep >0, \exists N \in \N: \forall n
            \geq N, \norm{f_n - f}_{\text{Lip}, a} < 3\vep$$
            Particularly, this means that 
            $$\lip{f_n - f} < 3\vep$$
            $\lip{f_n - f}$ is the Lipschitz constant for $f_n - f$, 
            so this implies that 
            $$\forall x,y \in X, \abs{\parens{f_n - f}(x) - 
            \parens{f_n - f}(y)} \leq \lip{f_n - f}d(x,y) < 
            \vep d(x,y)$$
            So $f_n - f \in \lip{X}$ As a vector
            space $\lip{X}$ is closed under addition and so $f \in 
            \lip{X}$.\\\\
            At no point did we deal with specific sequences in 
            $\lip{X}$ or elements of $X$. The Cauchy 
            sequence and it's limit were completely arbitrary and 
            they were both in $\lip{X}$, and the element of $X$ 
            that our norm was defined with respect to was also 
            arbitrary;
            so we can conclude that
            $\parens{\lip{X},\norm{\cdot}_{\text{Lip}, a}}$ is 
            complete because what we have done suffices to show that 
            all Cauchy sequences in $\lip{X}$ converge in $\lip{X}$.
            \end{proof}
        
        \ee
    \ee
    
    
\ee
\noindent\makebox[\linewidth]{\rule{\paperwidth}{0.4pt}}
	
\end{document}
