\documentclass{article}
\usepackage{amsmath}
\usepackage{amssymb}
\usepackage{bbm}
\usepackage{bm}
\usepackage{amsthm}
\usepackage{enumerate}
\usepackage{graphicx}
\usepackage{psfrag}
\usepackage{color}
\usepackage{url}
\usepackage{listings}
\usepackage{xcolor}
\usepackage{tikz}
\usetikzlibrary{positioning}
\tikzset{main node/.style={circle,fill=gray!20,draw,minimum size=.5cm,inner sep=0pt},}

\definecolor{codegreen}{rgb}{0,0.5,0}
\definecolor{codewhite}{rgb}{1,1,1}
\definecolor{codegray}{rgb}{0.5,0.5,0.5}
\definecolor{codepurple}{rgb}{0.58,0,0.82}
\definecolor{codeblack}{rgb}{0,0,0}
\definecolor{codeorange}{rgb}{0.8,0.4,0}

\lstdefinestyle{mystyle}{
    backgroundcolor=\color{codewhite},   
    commentstyle=\color{codegray},
    keywordstyle=\color{codegreen},
    numberstyle=\color{codegray},
    stringstyle=\color{codeorange},
    basicstyle=\ttfamily ,
    breakatwhitespace=false,         
    breaklines=true,                 
    captionpos=b,                    
    keepspaces=true,                 
    numbers=left,                    
    numbersep=5pt,                  
    showspaces=false,                
    showstringspaces=false,
    showtabs=false,                  
    tabsize=4
}
\lstset{style=mystyle}


\setlength{\hoffset}{-1in}
\addtolength{\textwidth}{1.5in}
\setlength{\voffset}{-1in}
\addtolength{\textheight}{1.5in}
\newcommand{\be}{\begin{enumerate}}
\newcommand{\ee}{\end{enumerate}}
\newcommand{\BigO}[1]{\ensuremath\mathcal{O}\left(#1\right)}
\newcommand{\il}[1]{\lstinline!#1!}
\newcommand{\norm}[1]{\left|\left|#1\right|\right|}
\newcommand{\abs}[1]{\left|#1\right|}
\newcommand{\parens}[1]{\left(#1\right)}
\newcommand{\bracks}[1]{\left\{#1\right\}}
\newcommand{\sqbracks}[1]{\left[#1\right]}
\newcommand{\vep}{\varepsilon}
\newcommand{\ceiling}[1]{\left\lceil#1\right\rceil}
\newcommand{\R}{\mathbb{R}}
\newcommand{\N}{\mathbb{N}}
\newcommand{\Z}{\mathbb{Z}}
\newcommand{\mC}{\mathcal{C}}
\newcommand{\one}{\mathbbm{1}}
\newcommand{\A}{\mathcal{A}}
\newcommand{\distrib}[2]{\text{#1}\left(#2\right)}
\newcommand{\dd}[1]{\frac{d}{d#1}}
\newcommand{\abracks}[1]{\left< #1\right>}


\begin{document}
	\begin{center}
		\textbf{Fall 2020, Math 630:\ Homework 7} \\
		\textbf{Due:\ Friday, October 30th, 2020} \\
		\textbf{Joseph Diaz: 819947915}
	\end{center}
\noindent\makebox[\linewidth]{\rule{\paperwidth}{0.4pt}}

\be[(E1)]
    \item Let $\delta : \mC\parens{[0,1]} \to \R$ be an operator which
    evaluates a function $f \in \mC\parens{[0,1]}$ at the origin, i.e.
    $\delta f = f(0)$. Show that
    \be[1.] 
        \item If $\mC\parens{[0,1]}$ is equipped with the max-norm $\norm{f}
        _\infty = \sup_{x\in[0,1]}\abs{f(x)}$, then $\delta$ is a bounded
        operator and then compute it's norm.
        \begin{proof}
        Assuming that the norm on $\R$ is $\abs{\cdot}$; then, 
        for all $f \in \mC\parens{[0,1]}$, 
        $$\norm{\delta f}_\R = \abs{f(0)},\quad\norm{f}_\infty = 
        \sup_{x\in[0,1]}\abs{f(x)}$$
        and 
        $$\abs{f(0)} \leq \sup_{x\in[0,1]}\abs{f(x)}$$
        so $\delta$ is bounded, by the definition of the supremum. 
        As for the norm of $\delta$, we have
        \begin{align*}
            \norm{\delta} = \sup_{\substack{f \in \mC\parens{[0,1]} \\
            f \neq 0}}\frac{\norm{\delta f}_\R}{\norm{f}_\infty} 
            \leq \sup_{\substack{f \in \mC\parens{[0,1]} \\
            f \neq 0}}\frac{\norm{f}_\infty}{\norm{f}_\infty} 
            =\frac{\norm{f}_\infty}{\norm{f}_\infty}= 1
        \end{align*}
        so $\norm{\delta} \leq 1$. Particularly, for $f(x) = a$, where
        $a \in \R\backslash\{0\}$, we have 
        $$\norm{\delta} = \sup_{\substack{f \in \mC\parens{[0,1]} \\ 
        f \neq 0}}\frac{\norm{\delta f}_\R}{\norm{f}_\infty} = 
        \sup_{\substack{f \in \mC\parens{[0,1]} \\ 
        f \neq 0}}\frac{a}{a} = \frac{a}{a} = 1$$
        so, as a supremum, $\norm{\delta} = 1$. 
        \end{proof}

        \item If $\mC\parens{[0,1]}$ is equipped with the one-norm 
        $\norm{f}_1 = \int_0^1 \abs{f(x)}\ dx$, then $\delta$ is unbounded.
        \begin{proof}
        Let $f_n(x) \in \mC\parens{[0,1]}$ defined as
        $$\forall n \in \N,\ f_n(x) = \left\{\begin{array}{cr}
        \parens{x-\frac{1}{n}}^2 & \text{ if } 0 \leq x \leq \frac{1}{n}\\
        0 & \text{ if }\frac{1}{n} < x \leq 1    
        \end{array}\right.$$
        Then, for all $n \in \N,\ \norm{\delta f_n}_\R = \abs{f_n(0)} = 
        \parens{1/n}^2$ and we also have
        \begin{align*}
        \norm{f_n}_1 &= \int_0^1 \abs{f_n(x)}\ dx = \int_0^{1/n} 
        \abs{\parens{x - \frac{1}{n}}^2}\ dx + \int_{1/n}^1 \abs{0}\ dx \\
        &= \int_0^{1/n} \parens{x - \frac{1}{n}}^2\ dx = 
        \left.\frac{\parens{x-\frac{1}{n}}^3}{3}\right|_0^{1/n} = 
        \frac{1}{3n^3}
        \end{align*}
        Finally, we have that 
        $$\frac{\norm{\delta f_n}_\R}{\norm{f_n}_1} = 
        \frac{\frac{1}{n^2}}{\frac{1}{3n^3}} = \frac{3n^3}{n^2} = 3n$$
        This is valid for all $n \in \N$, so this implies that $\delta$ is 
        an unbounded operator because there is no $C > 0$ such that 
        $\norm{\delta f_n}_\R \leq C\norm{f_n}_1$.
    \end{proof}

    \ee

    \item Let $T \in \mathcal{B}\parens{\mC\parens{[0,1]}}$ be defined as 
    $$\forall f \in \mC\parens{[0,1]},\ \parens{Tf}(x) = xf(x)$$
    and consider $\mC\parens{[0,1]}$ equipped with the max-norm. Find 
    $\norm{T}$.
    \begin{proof}[Answer]
    From the other expression of $\norm{T}$, we have 
    \begin{align*}
    \norm{T} &= \sup_{\substack{f \in \mC\parens{[0,1]} \\ 
    \norm{f}_\infty = 1}}\norm{Tf}_\infty 
    = \sup_{\substack{f \in \mC\parens{[0,1]} \\ \norm{f}_\infty = 1}}
    \sup_{x\in[0,1]}\abs{xf(x)} \\
    &\leq \sup_{\substack{f \in \mC\parens{[0,1]} \\ \norm{f}_\infty = 1}}
    \sup_{x\in[0,1]}\abs{f(x)} 
    = \sup_{\substack{f \in \mC\parens{[0,1]} \\ \norm{f}_\infty = 1}}
    \norm{f}_\infty 
    = 1
    \end{align*}
    So we have that $\norm{T} \leq 1$. Particularly, for $f(x) = 1$, 
    we have that
    \begin{align*}
    \norm{T} &= \sup_{\substack{f \in \mC\parens{[0,1]} \\ 
    \norm{f}_\infty = 1}}\norm{Tf}_\infty 
    = \sup_{\substack{f \in \mC\parens{[0,1]} \\ 
    \norm{f}_\infty = 1}}\sup_{x\in[0,1]}\abs{xf(x)} \\
    &= \sup_{\substack{f \in \mC\parens{[0,1]} \\ 
    \norm{f}_\infty = 1}}\sup_{x\in[0,1]}\abs{x}\abs{1} 
    = \sup_{\substack{f \in \mC\parens{[0,1]} \\ 
    \norm{f}_\infty = 1}} 1
    = 1
    \end{align*}
    so, as a supremum, $\norm{T} = 1$. 
    \end{proof}

    \item Let a linear operator $A: \R^n \to \R^m$. $A$ can be expressed as
    a matrix like so:
    $$A = \parens{A_1\ A_2\ \cdots\ A_n}$$
    where $A_i$ is the $i$-th column of $A$, for $1 \leq i \leq n$. 
    Find $\norm{A}$ when $\R^n$ and $\R^m$ are equipped with
    \be[1.] 
        \item $\norm{\cdot}_{\ell^1}$
        \begin{proof}
        Let $x \in \R^n$ such that $x = \parens{x_1,\ \cdots,\ x_n}$ and
        $\norm{x}_{\ell^1} = 1$, then
        \begin{align*}
        \norm{A} &= \sup_{\norm{x}_{\ell^1} = 1} \norm{Ax}_{\ell^1} = 
        \sup_{\norm{x}_{\ell^1} = 1} \norm{\sum_{i=1}^n A_ix_i}_{\ell^1} 
        \leq \sup_{\norm{x}_{\ell^1} = 1} 
        \sum_{i=1}^n \norm{A_ix_i}_{\ell^1} \\
        &= \sup_{\norm{x}_{\ell^1} = 1}\sum_{i=1}^n\abs{x_i}
        \norm{A_i}_{\ell^1} \leq \sup_{\norm{x}_{\ell^1} = 1}\sum_{i=0}^n 
        \abs{x_i}\max_{j}\norm{A_j}_{\ell^1} \\
        &= \max_{j}\norm{A_j}_{\ell^1}
        \sup_{\norm{x}_{\ell^1} = 1}\sum_{i=0}^n \abs{x_i} =
        \max_{j}\norm{A_j}_{\ell^1}
        \sup_{\norm{x}_{\ell^1} = 1}\norm{x}_{\ell^1} \\
        &= \max_{j}\norm{A_j}_{\ell^1}
        \end{align*}
        So we have that $\norm{A} \leq \max_{1\leq j\leq n}
        \norm{A_j}_{\ell^1}$. In particular; for $x = e_j$, where $j$ is 
        the index that maximizes $\norm{A_j}_{\ell^1}$ and $e_j$ is the
        $j$-th vector in the standard basis of $\R^n$, we have that 
        $Ae_j = A_j$, which implies that $\norm{Ae_j}_{\ell^1} = 
        \norm{A_j}_{\ell^1}$ and
        $$\norm{A} = \max_{i\leq j \leq n}\norm{A_j}_{\ell^1}$$
        \end{proof}

        \item $\norm{\cdot}_{\ell^\infty}$
        \begin{proof}
        Let $x \in \R^n$ such that $x = \parens{x_1,\ \cdots,\ x_n}$ and
        $\norm{x}_{\ell^\infty} = 1$, then
        \begin{align*}
        \norm{A} &= \sup_{\norm{x}_{\ell^\infty}=1}\norm{Ax}_{\ell^\infty} 
        = \sup_{\norm{x}_{\ell^\infty}=1}\max_{1\leq j\leq m}
        \abs{\sum_{i=1}^n a_{ji}x_i} 
        \leq \sup_{\norm{x}_{\ell^\infty}=1}\max_{1\leq j\leq m}
        \sum_{i=1}^n \abs{a_{ji}x_i} \\ 
        &\leq \sup_{\norm{x}_{\ell^\infty}=1} 
        \parens{\max_{1\leq j\leq m}\sum_{i=1}^n \abs{a_{ji}}}
        \parens{\max_{1\leq i\leq n}\abs{x_i}} = 
        \sup_{\norm{x}_{\ell^\infty}=1} 
        \parens{\max_{1\leq j\leq m}\sum_{i=1}^n \abs{a_{ji}}}
        \norm{x}_{\ell^\infty} \\
        &= \max_{1\leq j\leq m}\sum_{i=1}^n \abs{a_{ji}}
        \end{align*}    
        so we have that $\norm{A} \leq \max_{j}\sum_{i=1}^n \abs{a_{ji}}$. 
        Now, let the $k$-th row of $A$ be the one that maximizes 
        $\max_{j}\sum_{i=1}^n \abs{a_{ji}}$ and let $x =\parens{x_1,\ 
        \cdots,\ x_n}$ be defined as
        $$\forall i \in \bracks{1,\ \cdots,\ n},\ x_i = 
        \left\{\begin{array}{cr}
            1 & \text{ if }a_{ki} > 0 \\
            0 & \text{ if }a_{ki} = 0 \\
            -1 & \text{ if }a_{ki} < 0
        \end{array}\right.$$
        which is the same as $x_i = \text{sgn}\parens{a_{ki}}$ (the signum 
        function), then we have that $\norm{x}_{\ell^\infty} = 1$ and 
        \begin{align*}
        \norm{A} &= \norm{Ax}_{\ell^\infty} = \max_{1\leq j\leq m}
        \abs{\sum_{i=1}^n a_{ji}x_i} = \abs{\sum_{i=1}^n a_{ki}x_i} \\
        &= \abs{\sum_{i=1}^n a_{ki}\text{sgn}\parens{a_{ki}}} = 
        \abs{\sum_{i=1}^n \abs{a_{ki}}} = \sum_{i=1}^n \abs{a_{ki}} \\
        &= \max_{1\leq j\leq m}\sum_{i=1}^n \abs{a_{ji}}
        \end{align*}
        Finally, this implies that 
        $$\norm{A} = \max_{1\leq j\leq m}\sum_{i=1}^n \abs{a_{ji}}$$
        \end{proof}

    \ee
\ee
\noindent\makebox[\linewidth]{\rule{\paperwidth}{0.4pt}}
	
\end{document}
