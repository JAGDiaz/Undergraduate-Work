\documentclass{article}
\usepackage{amsmath}
\usepackage{amssymb}
\usepackage{bbm}
\usepackage{bm}
\usepackage{amsthm}
\usepackage{enumerate}
\usepackage{graphicx}
\usepackage{psfrag}
\usepackage{color}
\usepackage{url}
\usepackage{listings}
\usepackage{xcolor}
\usepackage{tikz}
\usetikzlibrary{positioning}
\tikzset{main node/.style={circle,fill=gray!20,draw,minimum size=.5cm,inner sep=0pt},}

\definecolor{codegreen}{rgb}{0,0.5,0}
\definecolor{codewhite}{rgb}{1,1,1}
\definecolor{codegray}{rgb}{0.5,0.5,0.5}
\definecolor{codepurple}{rgb}{0.58,0,0.82}
\definecolor{codeblack}{rgb}{0,0,0}
\definecolor{codeorange}{rgb}{0.8,0.4,0}

\lstdefinestyle{mystyle}{
    backgroundcolor=\color{codewhite},   
    commentstyle=\color{codegray},
    keywordstyle=\color{codegreen},
    numberstyle=\color{codegray},
    stringstyle=\color{codeorange},
    basicstyle=\ttfamily ,
    breakatwhitespace=false,         
    breaklines=true,                 
    captionpos=b,                    
    keepspaces=true,                 
    numbers=left,                    
    numbersep=5pt,                  
    showspaces=false,                
    showstringspaces=false,
    showtabs=false,                  
    tabsize=4
}
\lstset{style=mystyle}


\setlength{\hoffset}{-1in}
\addtolength{\textwidth}{1.5in}
\setlength{\voffset}{-1in}
\addtolength{\textheight}{1.5in}
\newcommand{\be}{\begin{enumerate}}
\newcommand{\ee}{\end{enumerate}}
\newcommand{\BigO}[1]{\ensuremath\mathcal{O}\left(#1\right)}
\newcommand{\il}[1]{\lstinline!#1!}
\newcommand{\norm}[1]{\left|\left|#1\right|\right|}
\newcommand{\abs}[1]{\left|#1\right|}
\newcommand{\parens}[1]{\left(#1\right)}
\newcommand{\bracks}[1]{\left\{#1\right\}}
\newcommand{\sqbracks}[1]{\left[#1\right]}
\newcommand{\vep}{\varepsilon}
\newcommand{\ceiling}[1]{\left\lceil#1\right\rceil}
\newcommand{\R}{\mathbb{R}}
\newcommand{\N}{\mathbb{N}}
\newcommand{\Z}{\mathbb{Z}}
\newcommand{\mC}{\mathcal{C}}
\newcommand{\one}{\mathbbm{1}}
\newcommand{\A}{\mathcal{A}}
\newcommand{\distrib}[2]{\text{#1}\left(#2\right)}
\newcommand{\dd}[1]{\frac{d}{d#1}}
\newcommand{\tr}[1]{\text{tr}\parens{#1}}
\newcommand{\LT}{\mathcal{L}}
\newcommand{\abracks}[1]{\left< #1\right>}


\begin{document}
	\begin{center}
		\textbf{Fall 2020, Math 630:\ Homework 8} \\
		\textbf{Due:\ Wednesday, November 11th, 2020} \\
		\textbf{Joseph Diaz: 819947915}
	\end{center}
\noindent\makebox[\linewidth]{\rule{\paperwidth}{0.4pt}}

\be[(E1)]
    \item Let $M_2$ be the space of squared matrices of size $2 \times 2$.
    All linear functionals over $M_2$ are completely defined by the choice
    of a matrix $B \in M_2$ and can be written as 
    $$\forall A \in M_2,\ \varphi\parens{A} = \tr{B^TA}$$
    where tr is the trace of the matrix (no need to prove this result for
    the homework). Show that the canonical basis $\bracks{E_{11},\ E_{12},\ 
    E_{21},\ E_{22}}$ of $M_2$ where 
    $$E_{11} = \parens{\begin{matrix}
        1 & 0 \\
        0 & 0
    \end{matrix}},\ E_{12} = \parens{\begin{matrix}
        0 & 1 \\
        0 & 0
    \end{matrix}},\ E_{21} = \parens{\begin{matrix}
        0 & 0 \\
        1 & 0
    \end{matrix}},\ E_{22} = \parens{\begin{matrix}
        0 & 0 \\
        0 & 1
    \end{matrix}}$$
    is self-dual.
    \begin{proof}
    First, we begin by relabeling $E_{11}, E_{12}, E_{21}, E_{22}$ as 
    $E_{1}, E_{2}, E_{3}, E_{4}$, respectively, then denote 
    $B(A) = \varphi_B(A) = \tr{B^T A}$. Now, we want to 
    find 
    $$B_i = \parens{\begin{matrix}
    a_i & b_i \\
    c_i & d_i    
    \end{matrix}} \in M_2$$
    such that $B_i(E_j) = \delta_{ij},\ i,j \in \bracks{1,2,3,4}$. So, we 
    have
    \begin{align*}
    B_1(E_1) = \tr{\parens{\begin{matrix}
    a_1 & 0 \\ b_1 & 0
    \end{matrix}}} = 1 &\implies a_1 = 1 \\
    B_1(E_2) = \tr{\parens{\begin{matrix}
    0 & a_1 \\ 0 & b_1
    \end{matrix}}} = 0 &\implies b_1 = 0 \\
    B_1(E_3) = \tr{\parens{\begin{matrix}
    c_1 & 0 \\ d_1 & 0
    \end{matrix}}} = 0 &\implies c_1 = 0 \\
    B_1(E_4) = \tr{\parens{\begin{matrix}
    0 & c_1 \\ 0 & d_1
    \end{matrix}}} = 0 &\implies d_1 = 0
    \end{align*}
    So we have that $B_1 = E_1$. Due to $B_i(E_j) = \delta_{ij}$, the only 
    difference bewteen this computation and that of the other $B$'s is
    which pair of $E_i$ and $B_i$ is equal to one under the linear 
    functional. Since this also corresponds to whether $a_i,\ b_i,\ c_i,\ 
    d_i$ will equal 0 or 1, we can deduce that $B_2 = E_2,\ 
    B_3 = E_3,\ B_4 = E_4$ based on this pattern. Therefore, $\bracks{E_1,\ 
    E_2,\ E_3,\ E_4}$ is self-dual.
    \end{proof}

    \item Let $M$ be a linear vector space of dimension $n$, and let 
    $\bracks{v_1,\ v_2,\ \cdots,\ v_n}$ be a basis of $M$. Assume that 
    $\bracks{f_1,\ f_2,\ \cdots,\ f_n}$ is a dual basis of 
    $\bracks{v_1,\ v_2,\ \cdots,\ v_n}$. Find the dual basis
    $\bracks{g_1,\ g_2,\ \cdots,\ g_n}$ of 
    $\bracks{v_1+v_2,\ v_2,\ \cdots,\ v_n}$ (using $\bracks{f_1,\ f_2,\ 
    \cdots,\ f_n}$).
    \begin{proof}
    First, we'll label $v_1 + v_2$ as $u_1$ and $v_i$ as $u_i$ for $i \in 
    \bracks{2,\ \cdots,\ n}$.
    We want to find $\bracks{g_1,\ \cdots,\ g_n}$ such that $g_i(u_j) = 
    \delta_{ij}$. We already have that $f_i(u_j) = \delta_{ij}$ 
    for $i,j > 2$ because $u_i = v_i$ for those values of $i$ and the 
    $f$'s form a dual basis of the $v$'s; so let $g_i = f_i$ for $i > 2$. 
    Now, we examine when $i = 1$ and $j=1,2$:
    \begin{gather*}
    g_1(u_1) = g_1(v_1+v_2) = g_1(v_1) + g_1(v_2) = 1 \\
    g_1(u_2) = g_1(v_2) = 0
    \end{gather*}
    Letting $g_1 = f_1$ satisfies both of these, because
    \begin{gather*}
    g_1(v_1) + g_1(v_2) = f_1(v_1) + f_1(v_2) = 1 + 0 = 1 \\
    g_1(v_2) = f_1(v_2) = 0
    \end{gather*}
    Lastly, we examine when $i = 2$ and $j = 1,2$:
    \begin{gather*}
    g_2(u_1) = g_2(v_1 + v_2) = g_2(v_1) + g_2(v_2) = 0 \\
    g_2(u_2) = g_2(v_2) = 1
    \end{gather*}
    Letting $g_2 = f_2 - f_1$ satisfies both of these, because
    \begin{align*}
    g_2(v_1) + g_2(v_2) &= (f_2 - f_1)(v_1) + (f_2 - f_1)(v_2) \\
    &= f_2(v_1) - f_1(v_1) + f_2(v_2) - f_1(v_2) \\
    &= 0 - 1 + 1 - 0\\\\
    g_2(v_2) &= (f_2 - f_1)(v_2) = f_2(v_2) - f_1(v_2) \\
    &= 1 - 0 = 1
    \end{align*}
    Finally, this means that $\bracks{f_1, f_2- f_1, f_3,\ \cdots,\ f_n}$ 
    forms a dual basis of $\bracks{u_1,\ u_2,\ \cdots,\ u_n} = 
    \bracks{v_1+v_2,\ v_2,\ \cdots,\ v_n}$.
    \end{proof}

    \item Let $M_1$, $M_2$ and $M_3$ be normed linear spaces. 
    Prove the following.
    \begin{itemize}
        \item $\forall T \in \LT\parens{M_1, M_2}$ and $\forall \lambda \in 
        \R$, $\parens{\lambda T}' = \lambda T'$.
        \begin{proof}
        Let $T \in \LT\parens{M_1, M_2},\ T', (\lambda T)' \in \LT\parens{M_2^*, M_1^*}$ 
        and $x \in M_1, x^* \in M_2^*$, then by the definition of dual operators
        $$\parens{(\lambda T)'x^*}(x) = x^*\parens{(\lambda T)x} = \lambda
        x^*(Tx)$$
        Using the definition of dual operators again, we have
        $$\lambda x^*(Tx) = \lambda \parens{T'x^*}(x) = 
        \parens{\lambda T'x^*}(x)$$
        Finally, this means
        $$\parens{(\lambda T)'x^*}(x) = \parens{\lambda T'x^*}(x)$$
        which implies that $\parens{\lambda T}' = \lambda T'$.
        \end{proof}

        \item $\forall S, T \in \LT\parens{M_1, M_2},\ \parens{S+T}' = 
        S'+T'$.
        \begin{proof}
        Let $S,T \in \LT\parens{M_1, M_2},\ S',T',(S+T)' \in \LT\parens{M_2^*,M_1^*}$ 
        and $x\in M_1, x^* \in M_2^*$, then by the definition of dual operators
        $$\parens{(S + T)'x^*}(x) = x^*\parens{(S + T)x} = x^*(Sx) + x^*(Tx)$$
        Using the definition of dual operators again, we have
        $$x^*(Sx) + x^*(Tx) = \parens{S'x^*}(x) + \parens{T'x^*}(x) = 
        \parens{(S' + T')x^*}(x)$$
        Finally, this means
        $$\parens{(S+ T)'x^*}(x) = \parens{(S' + T')x^*}(x)$$
        which implies that $\parens{S + T}' = S' + T'$.
        \end{proof}
        
        \item $\forall T \in \LT\parens{M_1, M_2},\ \forall S \in 
        \LT\parens{M_2, M_3}$; if $ST$ exists, then $\parens{ST}' = T'S'$.
        \begin{proof}
        Let $T \in \LT\parens{M_1, M_2},\ S \in \LT\parens{M_2, M_3}$ and 
        $x \in M_1, x^* \in M_3^*$.
        Suppose that $ST \in \LT\parens{M_1, M_3}$ exists, 
        then by the definition of the dual operator
        $$\parens{(ST)'x^*}(x) = x^*\parens{(ST)x} = x^*\parens{S(Tx)}$$
        We have that $Tx \in M_2$, so using the definition of the dual 
        operator again, we have 
        $$x^*\parens{S(Tx)} = \parens{S'x^*}(Tx)$$
        We now have that $S'x^* \in M_2^*$, so using the definition of the 
        dual operator yet again gives. 
        $$\parens{S'x^*}(Tx) = \parens{T'(S'x^*)}(x) = \parens{T'S'x^*}(x)$$
        Finally, this means 
        $$\parens{(ST)'x^*}(x) = \parens{T'S'x^*}(x)$$
        which implies that $\parens{ST}' = T'S'$.
        \end{proof}

        \item $\forall T \in \LT\parens{M_1}$, if $T^{-1}$ exists, then 
        $\parens{T'}^{-1} \in \LT\parens{M_1^*}$ exists and 
        $\parens{T'}^{-1} = \parens{T^{-1}}'$.
        \begin{proof}
        Let $T \in \LT\parens{M_1}, T' \in \LT\parens{M_1^*}$, 
        $x,y \in M_1, x^*,y^* \in M_1^*$, such that $y = Tx$ and 
        $y^* = T'x^*$. Now 
        suppose that the inverses, $T^{-1}$ and $(T')^{-1}$, exist; then
        \begin{align*}
        x^*\parens{y} &= x^*\parens{T T^{-1}y} = \parens{T'x^*}
        \parens{T^{-1} y} = y^*\parens{T^{-1} y} \\
        &= \parens{(T^{-1})'y^*}(y)
        \end{align*}
        This implies that $(T^{-1})'y^* = x^*$. Applying $T'$ to both sides 
        of this equation gives
        \begin{align*}
        (T^{-1})'y^* &= x^* \\
        T'(T^{-1})'y^* &= T'x^* \\
        T'(T^{-1})'y^* &= y^* \\
        \parens{T^{-1}T}'y^* &= y^* \\
        I'y^* &= y^*
        \end{align*}
        This means that $T'(T^{-1})' = I'$, where $I' \in \LT\parens{M_1^*}$
        is the identity map. $T'$ is invertible, so it must be the case that
        $(T^{-1})' = (T')^{-1}$.
        \end{proof}

    \end{itemize}
    
\ee
\noindent\makebox[\linewidth]{\rule{\paperwidth}{0.4pt}}
	
\end{document}
