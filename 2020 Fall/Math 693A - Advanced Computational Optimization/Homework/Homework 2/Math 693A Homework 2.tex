\documentclass{article}
\usepackage{amsmath}
\usepackage{amssymb}
\usepackage{bm}
\usepackage{amsthm}
\usepackage{enumerate}
\usepackage{graphicx}
\usepackage{psfrag}
\usepackage{color}
\usepackage{url}
\usepackage{listings}
\usepackage{xcolor}
\usepackage{tikz}
\usetikzlibrary{positioning}
\tikzset{main node/.style={circle,fill=gray!20,draw,minimum size=.5cm,inner sep=0pt},}

% In line code stuff%
\definecolor{codegreen}{rgb}{0,0.5,0}
\definecolor{codewhite}{rgb}{1,1,1}
\definecolor{codegray}{rgb}{0.5,0.5,0.5}
\definecolor{codepurple}{rgb}{0.58,0,0.82}
\definecolor{codeblack}{rgb}{0,0,0}
\definecolor{codeorange}{rgb}{0.8,0.4,0}

\lstdefinestyle{mystyle}{
    backgroundcolor=\color{codewhite},   
    commentstyle=\color{codegray},
    keywordstyle=\color{codegreen},
    numberstyle=\color{codegray},
    stringstyle=\color{codeorange},
    basicstyle=\ttfamily ,
    breakatwhitespace=false,         
    breaklines=true,                 
    captionpos=b,                    
    keepspaces=true,                 
    numbers=left,                    
    numbersep=5pt,                  
    showspaces=false,                
    showstringspaces=false,
    showtabs=false,                  
    tabsize=4
}
\lstset{style=mystyle}
\setlength{\hoffset}{-1in}
\addtolength{\textwidth}{1.5in}
\setlength{\voffset}{-1in}
\addtolength{\textheight}{1.5in}

% Custom commands%
\newcommand{\be}{\begin{enumerate}}
\newcommand{\ee}{\end{enumerate}}
\newcommand{\BigO}[1]{\ensuremath\mathcal{O}\left(#1\right)}
\newcommand{\il}[1]{\lstinline!#1!}
\newcommand{\norm}[1]{\left|\left|#1\right|\right|}
\newcommand{\abs}[1]{\left|#1\right|}
\newcommand{\parens}[1]{\left(#1\right)}
\newcommand{\bracks}[1]{\left\{#1\right\}}
\newcommand{\sqbracks}[1]{\left[#1\right]}
\newcommand{\vep}{\varepsilon}
\newcommand{\ceiling}[1]{\left\lceil#1\right\rceil}
\newcommand{\R}{\mathbb{R}}
\newcommand{\N}{\mathbb{N}}
\newcommand{\Z}{\mathbb{Z}}
\newcommand{\F}{\mathbb{F}}
\newcommand{\A}{\mathcal{A}}
\newcommand{\distrib}[2]{\text{#1}\left(#2\right)}
\newcommand{\dd}[1]{\frac{d}{d#1}}
\newcommand{\abracks}[1]{\left< #1\right>}

\begin{document}
	\begin{center}
		\textbf{Fall 2020, Math 693A:\ Homework 2} \\
		\textbf{Due:\ Friday, October 2nd, 2020} \\
		\textbf{Joseph Diaz: 819947915}
	\end{center}
\noindent\makebox[\linewidth]{\rule{\paperwidth}{0.4pt}}
\be 
    \item[1.] 
    State the strategy used used to compute the initial guess 
    in your homework 2 code.
    \begin{proof}[Answer]
    My code uses strategy \#1 for choosing $\alpha_0$:
    $$\alpha_0^{[k]} = \alpha^{[k-1]}\frac{\pmb{\bar{p}}_{k-1}^T
    \nabla f\parens{\pmb{\bar{x}}_{k-1}}}{\pmb{\bar{p}}_{k}^T
    \nabla f\parens{\pmb{\bar{x}}_{k}}}$$
    \end{proof}

    \item[3.] 
    State, in your homework, how you computed $\alpha_{i+1}$ 
    on Line 11 of the LS/Strong Wolfe Conditions.
    \begin{proof}[Answer]
    For choosing the next value of $\alpha_{i+1}$, my code uses 
    a pseudo random number from a uniform distribution with interval
    $[\alpha_i,\ \alpha_{\text{max}}]$. The python code for which is 
    \begin{center}
        \il{a_i = np.random.uniform(a_i, a_max)}
    \end{center}
    In this example, I am reusing the same variable \il{a_i} to
    simplify my code; which is what you'll see in my code.
    \end{proof}

    \item[4.]
    State in your homework the Interpolation method used.
    \begin{proof}[Answer]
    My homework uses the hermite interpolation. The relevant function 
    in my code is called \il{hermite_interpolate}.
    \end{proof}

    \item[5.]
    State in your homework the value of epsilon used.
    \begin{proof}[Answer]
    For both $\vep_1$ and $\vep_2$, I used $10^{-10}$.
    \end{proof}
    
    \item[6.]
    Plot the objective function $f(\bar{\bm{x}})$. 
    On the same figure, plot the $\bar{\bm{x}}_k$ values at 
    different iterates.
    \begin{proof}[Answer]
    After 8.
    \end{proof}

    
    \item[7.]
    Plot the size of the objective function as a function of the
    iteration number. Use semi-log plot.
    \begin{proof}[Answer]
    After 8.
    \end{proof}   
    
    \item[8.]
    Compare the performance for both the Newton and Steepest Descent 
    algorithms; is there a\\significant difference in number of 
    iterations etc.? Discuss this.
    \begin{proof}[Answer]
    Based on the results of the code, it seems to be the case that 
    the Steepest Descent method benefits from exact step length 
    selection significantly more than Newtons Method does. Both
    starting points required at least 6,000 step to reach the minimum
    using the methods of inexact step length selection with Steepest 
    Descent for Homework 1; however, with exact step length selection
    the same starting points require only about 1,000 steps at the
    absolute most to reach the minimum.\\\\
    Contrast this with Newtons Method for either step length 
    selection method. Both methods require about the same number of 
    steps for each starting point. This indicates one of 
    the key differences between Newtons Method and Steepest Descent,
    as far as how the search direction is computed.\\\\In one case, 
    the 
    negative of the gradient has no sense of step length
    which we 
    usually normalize to make it easier to work with. In the other,
    the dot product of the gradient with the hessian carries with it
    some sense of inherent step length that the dot product 
    preserves. So, in the same sense, each $\bar{\bm{p}}$ for Newtons
    method already has a nearly ideal step length built into it.
    Which would explain why the way that $\alpha_k$ is computed is 
    of less importance.
    \end{proof}
    
    
\ee
\begin{figure}[h!]
\includegraphics[width=\linewidth]{HW2SD1Surface.JPG}
\includegraphics[width=\linewidth]{HW2SD2Surface.JPG}
\centering
\end{figure}
\begin{figure}[h!]
\includegraphics[width=\linewidth]{HW2NM1Surface.JPG}
\includegraphics[width=\linewidth]{HW2NM2Surface.JPG}
\centering
\end{figure}
	
\begin{figure}[h!]
\includegraphics[width=\linewidth]{HW2SDObjective.JPG}
\includegraphics[width=\linewidth]{HW2NMObjective.JPG}
\centering
\end{figure}
\noindent\makebox[\linewidth]{\rule{\paperwidth}{0.4pt}}
	
\end{document}
