\documentclass{article}
\usepackage{amsmath}
\usepackage{amssymb}
\usepackage{bm}
\usepackage{amsthm}
\usepackage{enumerate}
\usepackage{graphicx}
\usepackage{psfrag}
\usepackage{color}
\usepackage{url}
\usepackage{listings}
\usepackage{xcolor}
\usepackage{tikz}
\usepackage{mdframed}
\usepackage{multirow}
\usetikzlibrary{positioning}
\tikzset{main node/.style={circle,fill=gray!20,draw,minimum size=.5cm,inner sep=0pt},}

% In line code stuff%
\definecolor{codegreen}{rgb}{0,0.5,0}
\definecolor{codewhite}{rgb}{1,1,1}
\definecolor{codegray}{rgb}{0.85,0.85,0.85}
\definecolor{codepurple}{rgb}{0.58,0,0.82}
\definecolor{codeblack}{rgb}{0,0,0}
\definecolor{codeorange}{rgb}{0.8,0.4,0}

\lstdefinestyle{mystyle}{
    backgroundcolor=\color{codegray},   
    commentstyle=\color{codegray},
    keywordstyle=\color{codegreen},
    numberstyle=\color{codegray},
    stringstyle=\color{codeorange},
    basicstyle=\ttfamily ,
    breakatwhitespace=false,         
    breaklines=true,                 
    captionpos=b,                    
    keepspaces=true,                 
    numbers=left,                    
    numbersep=5pt,                  
    showspaces=false,                
    showstringspaces=false,
    showtabs=false,                  
    tabsize=4
}
\lstset{style=mystyle}
\setlength{\hoffset}{-1in}
\addtolength{\textwidth}{1.5in}
\setlength{\voffset}{-1in}
\addtolength{\textheight}{1.5in}

% Custom commands%
\newcommand{\be}{\begin{enumerate}}
\newcommand{\ee}{\end{enumerate}}
\newcommand{\BigO}[1]{\ensuremath\mathcal{O}\left(#1\right)}
\newcommand{\il}[1]{\lstinline!#1!}
\newcommand{\norm}[1]{\left|\left|#1\right|\right|}
\newcommand{\abs}[1]{\left|#1\right|}
\newcommand{\parens}[1]{\left(#1\right)}
\newcommand{\bracks}[1]{\left\{#1\right\}}
\newcommand{\sqbracks}[1]{\left[#1\right]}
\newcommand{\vep}{\varepsilon}
\newcommand{\ceiling}[1]{\left\lceil#1\right\rceil}
\newcommand{\R}{\mathbb{R}}
\newcommand{\N}{\mathbb{N}}
\newcommand{\Z}{\mathbb{Z}}
\newcommand{\F}{\mathbb{F}}
\newcommand{\A}{\mathcal{A}}
\newcommand{\distrib}[2]{\text{#1}\left(#2\right)}
\newcommand{\dd}[1]{\frac{d}{d#1}}
\newcommand{\abracks}[1]{\left< #1\right>}

\newenvironment{answer}
    {\begin{mdframed}[
    backgroundcolor=lightgray,
    outerlinewidth=0
    ]\emph{Answer.} }
    {\end{mdframed}}

\newenvironment{centeredtable}[1]
    {\begin{center}
    \begin{tabular}{#1}}
    {\end{tabular} 
    \end{center}
    }

\begin{document}
	\begin{center}
		\textbf{Fall 2020, Math 693A:\ Project Proposal} \\
		\textbf{Due:\ Halloween 2020} \\
		\textbf{Joseph Diaz: 819947915}
	\end{center}
\noindent\makebox[\linewidth]{\rule{\paperwidth}{0.4pt}}
\section*{The Proposal}
To solve a combinatorial or mathematical optimization 
problem using the Branch and Bound algorithm design paradigm.
\subsection*{What is Branch and Bound?}
A Branch and Bound algorithm attempts to systematically create a set
of \emph{Candidate solutions} by means of successively trying configurations
of possible solutions until a particular property of these candidates is
satisfied. The set of candidate solutions is usually represented as a 
rooted binary tree, where the root represents the \emph{entire} set of 
candidates and the child nodes represent (usually disjoint) subsets of the
set of solutions. A solution is found by iterating through the branches of
this tree, and, to avoid going down a branch that won't likely have an 
optimum, each branch is checked against estimated bounds of possible 
solutions and previously checked candidate solutions. An exhaustive search
of the candidate space is performed if this branch traversal fails.
\subsection*{The Problem}
This algorithm paradigm has many applications to a variety of problems, including but not limited to:
\begin{itemize}
    \item Integer programming
    \item Maximum satisfiability problem
    \item Nearest neighbor search
    \item Flow shop scheduling
    \item Set inversion
    \item Parameter estimation
\end{itemize}
I haven't decided what problem I am going to solve with BnB; but I'm 
currently leaning towards Nearest Neighbor Search.

\noindent\makebox[\linewidth]{\rule{\paperwidth}{0.4pt}}
	
\end{document}
