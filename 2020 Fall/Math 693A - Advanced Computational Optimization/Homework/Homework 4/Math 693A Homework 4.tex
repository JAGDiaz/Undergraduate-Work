\documentclass{article}
\usepackage{amsmath}
\usepackage{amssymb}
\usepackage{bm}
\usepackage{amsthm}
\usepackage{enumerate}
\usepackage{graphicx}
\usepackage{psfrag}
\usepackage{color}
\usepackage{url}
\usepackage{listings}
\usepackage{xcolor}
\usepackage{tikz}
\usepackage{mdframed}
\usepackage{multirow}
\usetikzlibrary{positioning}
\tikzset{main node/.style={circle,fill=gray!20,draw,minimum size=.5cm,inner sep=0pt},}

% In line code stuff%
\definecolor{codegreen}{rgb}{0,0.5,0}
\definecolor{codewhite}{rgb}{1,1,1}
\definecolor{codegray}{rgb}{0.85,0.85,0.85}
\definecolor{codepurple}{rgb}{0.58,0,0.82}
\definecolor{codeblack}{rgb}{0,0,0}
\definecolor{codeorange}{rgb}{0.8,0.4,0}

\lstdefinestyle{mystyle}{
    backgroundcolor=\color{codegray},   
    commentstyle=\color{codegray},
    keywordstyle=\color{codegreen},
    numberstyle=\color{codegray},
    stringstyle=\color{codeorange},
    basicstyle=\ttfamily ,
    breakatwhitespace=false,         
    breaklines=true,                 
    captionpos=b,                    
    keepspaces=true,                 
    numbers=left,                    
    numbersep=5pt,                  
    showspaces=false,                
    showstringspaces=false,
    showtabs=false,                  
    tabsize=4
}
\lstset{style=mystyle}
\setlength{\hoffset}{-1in}
\addtolength{\textwidth}{1.5in}
\setlength{\voffset}{-1in}
\addtolength{\textheight}{1.5in}

% Custom commands%
\newcommand{\be}{\begin{enumerate}}
\newcommand{\ee}{\end{enumerate}}
\newcommand{\BigO}[1]{\ensuremath\mathcal{O}\left(#1\right)}
\newcommand{\il}[1]{\lstinline!#1!}
\newcommand{\norm}[1]{\left|\left|#1\right|\right|}
\newcommand{\abs}[1]{\left|#1\right|}
\newcommand{\parens}[1]{\left(#1\right)}
\newcommand{\bracks}[1]{\left\{#1\right\}}
\newcommand{\sqbracks}[1]{\left[#1\right]}
\newcommand{\vep}{\varepsilon}
\newcommand{\ceiling}[1]{\left\lceil#1\right\rceil}
\newcommand{\R}{\mathbb{R}}
\newcommand{\N}{\mathbb{N}}
\newcommand{\Z}{\mathbb{Z}}
\newcommand{\F}{\mathbb{F}}
\newcommand{\A}{\mathcal{A}}
\newcommand{\distrib}[2]{\text{#1}\left(#2\right)}
\newcommand{\dd}[1]{\frac{d}{d#1}}
\newcommand{\abracks}[1]{\left< #1\right>}

\newenvironment{answer}
    {\begin{mdframed}[
    backgroundcolor=lightgray,
    outerlinewidth=0
    ]\emph{Answer.} }
    {\end{mdframed}}

\newenvironment{centeredtable}[1]
    {\begin{center}
    \begin{tabular}{#1}}
    {\end{tabular} 
    \end{center}
    }

\begin{document}
	\begin{center}
		\textbf{Fall 2020, Math 693A:\ Homework 4} \\
		\textbf{Due:\ Friday, November 2nd, 2020} \\
		\textbf{Joseph Diaz: 819947915}
	\end{center}
\noindent\makebox[\linewidth]{\rule{\paperwidth}{0.4pt}}
Write a program that implements the dogleg method. Choose $B_k$ to be 
the exact Hessian. Apply it to solve Rosenbrock's function.
\be[1.]
    \item
    Program the trust-region method using the \emph{Steihaug method} (see 
    Lecture 15 and/or Algorithm 7.2 in the text). Choose $B_k$ to be the 
    exact Hessian, and use it to minimize
    $$f(\bm{x}) = 100(x_2 - x_1^2)^2 + (1 - x_1)^2$$
    Use the tolerance $\vep = 10^{-6}$ with 
    $\norm{\nabla f\parens{\bm{x}_k}}< 10^{-8}$ as the stopping criteria 
    for your algorithm. Use an initial trust-region radius of $1$ and a 
    maximum trust-region radius of $300$. Start at the initial points
    $\bm{x}_0 = (-1.2, 1),\ \bm{x}_0 = (2.8, 4)$ and do the following for
    each:
    \be 
        \item Your program should indicate, at every iteration, whether
        Steihaug method encountered negative curvature, reached the 
        trust-region boundary, or met the stopping test. Hand in your 
        output.
        \begin{proof}[Answer]
        This is the output of my program:
\begin{lstlisting}[language=python]
Starting point: x_0 = [-1.2  1. ], Iterations: k = 21
k  | x_k            | Steihaug state at x_k
------------------------------------------------------------------
0  | [-1.200,1.000] | Steihaug met the stopping test.              
1  | [-1.175,1.381] | Steihaug reached the trust region boundary.  
2  | [-1.050,1.095] | Steihaug reached the trust region boundary.  
3  | [-0.776,0.533] | Steihaug met the stopping test.              
4  | [-0.656,0.416] | Steihaug reached the trust region boundary.  
5  | [-0.457,0.175] | Steihaug met the stopping test.              
6  | [-0.272,0.040] | Steihaug met the stopping test.              
7  | [-0.110,-0.014]| Steihaug met the stopping test.              
8  | [0.068,-0.027] | Steihaug met the stopping test.              
9  | [0.195,0.022]  | Steihaug met the stopping test.              
10 | [0.385,0.112]  | Steihaug met the stopping test.              
11 | [0.460,0.206]  | Steihaug reached the trust region boundary.  
12 | [0.572,0.315]  | Steihaug met the stopping test.              
13 | [0.695,0.468]  | Steihaug met the stopping test.              
14 | [0.771,0.588]  | Steihaug met the stopping test.              
15 | [0.877,0.758]  | Steihaug met the stopping test.              
16 | [0.915,0.835]  | Steihaug met the stopping test.              
17 | [0.981,0.958]  | Steihaug met the stopping test.              
18 | [0.991,0.982]  | Steihaug met the stopping test.              
19 | [1.000,1.000]  | Steihaug met the stopping test.              
20 | [1.000,1.000]  | Steihaug met the stopping test.              
------------------------------------------------------------------

Starting point: x_0 = [2.8 4. ], Iterations: k = 20
k  | x_k            | Steihaug state at x_k
------------------------------------------------------------------
0  | [2.800,4.000]  | Steihaug met the stopping test.              
1  | [2.798,7.827]  | Steihaug reached the trust region boundary.  
2  | [2.577,6.596]  | Steihaug met the stopping test.              
3  | [2.422,5.841]  | Steihaug met the stopping test.              
4  | [2.178,4.684]  | Steihaug met the stopping test.              
5  | [2.087,4.346]  | Steihaug reached the trust region boundary.  
6  | [1.939,3.738]  | Steihaug met the stopping test.              
7  | [1.753,3.039]  | Steihaug met the stopping test.              
8  | [1.658,2.739]  | Steihaug reached the trust region boundary.  
9  | [1.565,2.441]  | Steihaug reached the trust region boundary.  
10 | [1.372,1.846]  | Steihaug met the stopping test.              
11 | [1.328,1.761]  | Steihaug reached the trust region boundary.  
12 | [1.217,1.468]  | Steihaug met the stopping test.              
13 | [1.152,1.323]  | Steihaug met the stopping test.              
14 | [1.069,1.137]  | Steihaug met the stopping test.              
15 | [1.040,1.081]  | Steihaug met the stopping test.              
16 | [1.006,1.011]  | Steihaug met the stopping test.              
17 | [1.001,1.002]  | Steihaug met the stopping test.              
18 | [1.000,1.000]  | Steihaug met the stopping test.              
19 | [1.000,1.000]  | Steihaug met the stopping test.              
------------------------------------------------------------------

Starting point: x_0 = [0. 1.], Iterations: k = 10
k  | x_k            | Steihaug state at x_k
------------------------------------------------------------------
0  | [0.000,1.000]  | Steihaug encountered Negative Curvature.     
1  | [0.730,-0.015] | Steihaug met the stopping test.              
2  | [0.733,0.537]  | Steihaug reached the trust region boundary.  
3  | [0.821,0.665]  | Steihaug met the stopping test.              
4  | [0.885,0.779]  | Steihaug met the stopping test.              
5  | [0.949,0.896]  | Steihaug met the stopping test.              
6  | [0.977,0.953]  | Steihaug met the stopping test.              
7  | [0.997,0.993]  | Steihaug met the stopping test.              
8  | [1.000,1.000]  | Steihaug met the stopping test.              
9  | [1.000,1.000]  | Steihaug met the stopping test.              
------------------------------------------------------------------
\end{lstlisting}
        The number of iterations for each is also listed; and the extra 
        starting point, $\bm{x}_0 = (0,1)$, is only shown because the 
        optimization paths from the other starting points do not encounter
        negative curvature anywhere.
        \end{proof}

        \item State the total number of iterations obtained in your 
        optimization algorithm.
        \begin{proof}[Anwser]
        Answered in part ~(a)
        \end{proof}

        \item Plot the objective function $f(\bm{x})$. On the same figure, 
        plot the $\bm{x}_k$ values at the different iterates of your 
        optimization algorithm.
        \begin{proof}[Answer]
        Here are the plots!
        \begin{figure}[h!]
            \centering
            \includegraphics[width=.9\linewidth]{HW4;EasyStart.jpg}
            \includegraphics[width=.9\linewidth]{HW4;HardStart.jpg}
        \end{figure}
        \end{proof}
        \pagebreak

        \item Plot the size of the objective function as a function of the 
        iteration number. Use semi-log plot.
        \begin{proof}[Answer]
        Here are the plots!
        \begin{figure}[h!]
            \centering
            \includegraphics[width=.9\linewidth]{HW4;Objective0.jpg}
            \includegraphics[width=.9\linewidth]{HW4;Objective1.jpg}
        \end{figure}
        \end{proof}

        \item Determine the minimizer of the function $\bm{x}^*$.
        \begin{proof}[Answer]
        The minimizer of the Rosenbrock function is $\bm{x}^* = (1,1)$.
        \end{proof}
    \ee
    \item 
    Implement the standard CG algorithm, and use it to solve linear systems 
    in which $A$ is the Hilbert matrix, whose elements are 
    $a_{ij} = 1/(i + j - 1)$. Set the right-hand-side to be all ones 
    \il{b = ones(n, 1)}, and the initial point to be the origin 
    \il{x_0 = zeros(n,1)}. In the stopping criteria, use 
    $\norm{\bm{r}_k} > 10^{-6}$.
    \be 
        \item For dimensions $n = 5, 8, 12, 20$, plot the norm of the 
        residual as a function of the iteration (on the same figure); stop 
        when the norm is less than $10^{-6}$. Use semi-log plot.
        \begin{proof}[Answer]
        Here is the plot!
        \begin{figure}[h!]
            \centering
            \includegraphics[width=\linewidth]{r_norm_vs_k_one.jpg}
        \end{figure}
        \end{proof}
        \pagebreak

        \item Plot your number of iterations against $n$ for 
        $n = 5, 8, 12, 20$.
        \begin{proof}[Answer]
        Here is the plot!
        \begin{figure}[h!]
            \centering
            \includegraphics[width=.85\linewidth]{iterations_vs_n.jpg}
        \end{figure}
        \end{proof}

        \item Compute the condition number for your Hilbert matrices, 
        generate a plot of the condition number against the matrix size $n$. 
        Use semi-log plot.
        \begin{proof}[Answer]
        Here is the plot!
        \begin{figure}[h!]
            \centering
            \includegraphics[width=.85\linewidth]{condition_number_plot.jpg}
        \end{figure}
        \end{proof}

       \item Plot the eigenvalues for $n = 5, 8, 12, 20$ on the same figure 
        in order to show the spread of the eigenvalues.
        \begin{proof}[Answer]
        Here is the plot!
        \begin{figure}[h!]
            \centering
            \includegraphics[width=.85\linewidth]{eigs_plot.jpg}
        \end{figure}
        \end{proof}


        \item From the formulas provided in the Lecture 11, estimate how 
        many steepest descent iterations you would need to solve the 
        problem to the same precision. What can you say about your 
        estimate, is it a good estimate?
        \begin{proof}[Answer]
        Using the formula for the error bound on Steepest Descent and
        solving $k$ with $\norm{r_k}_2 \approx 10^{-6}$ gives:
        \begin{center}
        \begin{tabular}{|c|c|}
            \hline 
            $n$ & SD iterations \\
            \hline 
            5 & $\sim 7500$\\
            \hline 
            8 & $\sim 1678000$\\
            \hline 
            12 & $\sim 18692410000$\\
            \hline 
            20 & $\sim 42962599000$\\
            \hline
        \end{tabular}
        \end{center}
        However, I don't think its a very good estimate. 
        The condition number $\kappa\parens{H_n}$ blows up pretty 
        quickly, so it's difficult to say exactly how much precision is 
        being lost during computation, floating point round-off 
        notwithstanding. So while some of these numbers are quite high,
        I don't think it would take quite as long with Steepest Descent;
        even know how suboptimal it is as a method.
        \end{proof}
    \ee



\ee
\noindent\makebox[\linewidth]{\rule{\paperwidth}{0.4pt}}
	
\end{document}
