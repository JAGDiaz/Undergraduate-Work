\documentclass{article}
\usepackage{amsmath}
\usepackage{amssymb}
\usepackage{bm}
\usepackage{amsthm}
\usepackage{enumerate}
\usepackage{graphicx}
\usepackage{psfrag}
\usepackage{color}
\usepackage{url}
\usepackage{listings}
\usepackage{xcolor}
\usepackage{tikz}
\usetikzlibrary{positioning}
\tikzset{main node/.style={circle,fill=gray!20,draw,minimum size=.5cm,inner sep=0pt},}

\definecolor{codegreen}{rgb}{0,0.5,0}
\definecolor{codewhite}{rgb}{1,1,1}
\definecolor{codegray}{rgb}{0.5,0.5,0.5}
\definecolor{codepurple}{rgb}{0.58,0,0.82}
\definecolor{codeblack}{rgb}{0,0,0}
\definecolor{codeorange}{rgb}{0.8,0.4,0}

\lstdefinestyle{mystyle}{
    backgroundcolor=\color{codewhite},   
    commentstyle=\color{codegray},
    keywordstyle=\color{codegreen},
    numberstyle=\color{codegray},
    stringstyle=\color{codeorange},
    basicstyle=\ttfamily ,
    breakatwhitespace=false,         
    breaklines=true,                 
    captionpos=b,                    
    keepspaces=true,                 
    numbers=left,                    
    numbersep=5pt,                  
    showspaces=false,                
    showstringspaces=false,
    showtabs=false,                  
    tabsize=4
}
\lstset{style=mystyle}


\setlength{\hoffset}{-1in}
\addtolength{\textwidth}{1.5in}
\setlength{\voffset}{-1in}
\addtolength{\textheight}{1.5in}
\newcommand{\be}{\begin{enumerate}}
\newcommand{\ee}{\end{enumerate}}
\newcommand{\BigO}[1]{\ensuremath\mathcal{O}\left(#1\right)}
\newcommand{\il}[1]{\lstinline!#1!}
\newcommand{\gnorm}[1]{\left|\left|#1\right|\right|}
\newcommand{\abs}[1]{\left|#1\right|}
\newcommand{\parens}[1]{\left(#1\right)}
\newcommand{\bracks}[1]{\left\{#1\right\}}
\newcommand{\sqbracks}[1]{\left[#1\right]}
\newcommand{\vep}{\varepsilon}
\newcommand{\ceiling}[1]{\left\lceil#1\right\rceil}
\newcommand{\R}{\mathbb{R}}
\newcommand{\N}{\mathbb{N}}
\newcommand{\distrib}[2]{\text{#1}\left(#2\right)}
\newcommand{\dd}[1]{\frac{d}{d#1}}
\newcommand{\abracks}[1]{\left< #1\right>}

\begin{document}
	\begin{center}
		\textbf{Spring 2020, Stats 550:\ Homework 17} \\
		\textbf{Due:\ Tuesday, April 28th, 2020} \\
		\textbf{Joseph Diaz: 819947915}
	\end{center}
\noindent\makebox[\linewidth]{\rule{\paperwidth}{0.4pt}}
\subsection*{Chapter 7}
\be 
	\item[1.] Let $X \sim \distrib{Norm}{-4,25}$. Find the following probabilities without using software:
	\be[(a)] 
		\item $P\parens{X > 6}$
		\begin{proof}[Solution]
		Using the transform to \textit{standardized random variable}, we have:
		\begin{align*}
		P\parens{X > 6} &= P\parens{\frac{X-(-4)}{5} > \frac{6-(-4)}{5}} \\
		&= P\parens{Z > 2} \\
		&\approx \frac{1-.95}{2} \\
		&\approx 2.5\%
		\end{align*}
		\end{proof}
		
		\item $P\parens{-9 < X < 1}$
		\begin{proof}[Solution]
		Using the same transform:
		\begin{align*}
		P\parens{-9 < X < 1} &= P\parens{\frac{-9 - (-4)}{5} < \frac{X - (-4)}{5} < \frac{1 -(-4)}{5}} \\
		&= P\parens{-1 < Z < 1} \\
		&\approx 68\%
		\end{align*}
		\end{proof}
		
		\item $P\parens{\sqrt{X} > 1}$
		\begin{proof}[Solution]
		Using the same transform:
		\begin{align*}
		P\parens{\sqrt{X} > 1} &= P\parens{X > 1} \\
		&= P\parens{\frac{X-(-4)}{5} > \frac{1-(-4)}{5}} \\
		&= P\parens{Z > 1} \\
		&\approx \frac{1-.68}{2} \\
		&\approx 16\%
		\end{align*}
		\end{proof}
	\ee
	
	\item[3.] Babies’ birth weights are normally distributed with mean 120 ounces and standard
deviation 20 ounces. \emph{Low birth weight} is an important indicator of a
newborn baby’s chances of survival. One definition of low birth weight is that
it is the fifth percentile of the weight distribution.\\
Let $X \sim \distrib{Norm}{120, 20^2}$ be the weight of newborns.
	\be[(a)] 
		\item Babies who weigh less than what amount would be considered low birth
weight?
		\begin{proof}[Solution]
		We want to find $q_5$ such that 
		$$P\parens{X \leq q_5} = .05$$
		First, we'll transform of question to probability in terms of the standardized normal random variable $Z$
		$$P\parens{X \leq q_5} = P\parens{Z \leq \frac{q_5 - 120}{20}}$$
		using \il{R}, we have that \il{qnorm(.05) = -1.644854}, which yields
		$$-1.644854 \approx \frac{q_5 - 120}{20} \implies q_5 \approx 87.10293$$
		So by this metric, we would consider babies lighter than 87.1 ounces under weight.
		\end{proof}

		\item \emph{Very low birth weight} is used to described babies who are born weighing
less than 52 ounces. Find the probability that a baby is born with very low
birth weight.
		\begin{proof}[Solution]
		We want to find 
		$$P\parens{X < 52}$$
		to find this, we'll use the transform from ~(a):
		$$P\parens{X < 52} = P\parens{Z < \frac{52 - 120}{20}} = P\parens{Z < -3.4}$$
		With \il{R}'s \il{pnorm(-3.4)}, we have 
		$$P\parens{Z < -3.4} \approx 0.03369293\%$$
		So the probability that a new born is severely underweight is about $0.03\%$.
		\end{proof}

		\item Given that a baby is born with low birth weight, what is the probability
that they have very low birth weight?
	\begin{proof}[Solution]
	We want to find
	$$P\parens{X < 52\mid X < 87.1}$$
	We'll find this using the conditional probability formula
	$$
	P\parens{X < 52\mid X < 87.1} = \frac{P\parens{X < 87.1\mid X < 52} P\parens{X < 52}}{P\parens{X < 87.1}}
	$$
	But $P\parens{X < 87.1\mid X < 52} = 1$, because if $X < 52$, then $X < 87.1$. So our formula is
	$$P\parens{X < 52\mid X < 87.1} = \frac{P\parens{X < 52}}{P\parens{X < 87.1}} = \frac{0.03369293}{5} = 0.6738585\%$$
	So the probability that a low weight baby is also a very low weight baby is $0.6738585\%$.
	\end{proof}

	\ee
	
	\item[5.] The rates of return of ten stocks are normally distributed with mean $\mu = 2$
and standard deviation $\sigma = 4$ (units are percentages). Rates of return are
independent from stock to stock. Each stock sells for \$10 a share. Amy, Bill,
and Carrie have \$100 each to spend on their stock portfolio. Amy buys one
share of each stock. Bill buys two shares from five different companies. Carrie
buys ten shares from one company. For each person, find the probability that
their average rate of return is positive.
	\begin{proof}[Solution]
	Let $X_i \sim \distrib{Norm}{2, 4^2}$ be the percentage of returns on the stock market for any given stock. Let $S_A,\ S_B,\ S_C$ be the percentage of returns on the stocks that Amy, Bill, and Carrie buy, respectively. So
	$$S_A = \frac{X_1 + \cdots + X_{10}}{10} \sim \distrib{Norm}{2, \frac{4^2}{10}}$$
	$$S_B = \frac{2X_1 + \cdots + 2X_5}{5} \sim \distrib{Norm}{4, \frac{8^2}{5}}$$
	$$S_C = 10X_1 \sim \distrib{Norm}{20,40^2}$$
	So for each of these we want to find
	\begin{align*}
	P(S_A > 0) &= P\parens{Z > \frac{0-2}{\frac{4}{\sqrt{10}}}}\quad  & P(S_B > 0) &= P\parens{Z > \frac{0-4}{\frac{8}{\sqrt{5}}}}\quad & P(S_C > 0) &= P\parens{Z > \frac{0-20}{40}} \\
	&= P\parens{Z > -1.5114}\quad  & &= P\parens{Z > -1.118}\quad &  &= P\parens{Z > -.5} \\
	&\approx 94.30769\%\quad  & &\approx 86.82238\%\quad &  &\approx 69.14625\% \\
	\end{align*}
	So the probability that Amy's, Bill's, and Carrie's average rate of return is positive are $94.30769\%$,  $86.82238\%$, and $69.14625\%$; respectively.
	\end{proof}
	
	\item[6.] The two main standardized tests in the United States for high school students
are the ACT and SAT. ACT scores are normally distributed with mean 18
and standard deviation 6. SAT scores are normally distributed with mean 500
and standard deviation 100. Suppose Jill takes an SAT exam and scores 680.
Jack plans to take the ACT. What score does Jack need to get so that his
standardized score is the same as Jill’s standardized score? What percentile
do they each score at?
	\begin{proof}[Solution]
	Let $A \sim \distrib{Norm}{18, 6^2}$ be the ACT scores, and $S \sim \distrib{Norm}{500, 100^2}$ be the SAT scores. Jill scored 680, so her percentile is given by
	$$P\parens{S < 680} = P\parens{Z < \frac{680-500}{100}} = P\parens{Z < 9/5} \approx 96.41\%$$
	So we see that Jill's score is at about the $96^{th}$ percentile. To find Jack's equivalent score for the ACT, we need to find $q_{96}$ such that
	$$P\parens{A < q_{96}} = P\parens{Z < \frac{q_{96} - 18}{6}} = P\parens{Z < 9/5}$$
	$$\Rightarrow \frac{q_{96}-18}{6} = \frac{9}{5} \implies q_{96} = 28.8$$
	So a 680 on the SAT is equivalent to a 28.8 on the ACT and both scores are at about the $96^{th}$ percentile.
	\end{proof}
	
	\item[11.] Let $X_1,\ \cdots,\ X_n$ be an i.i.d. sequence of normal random variables with mean
$\mu$ and variance $\sigma^2$. Let $S_n = X_1 + \cdots + X_n$.
	\be 
		\item Suppose $\mu = 5$ and $\sigma^2 = 1$. Find $P\parens{\abs{X_1 - \mu} \leq \sigma}$.
		\begin{proof}[Solution]
		We have that 
		\begin{align*}
		P\parens{\abs{X_1 - \mu} \leq \sigma} &= P\parens{-\sigma \leq X_1 - \mu \leq \sigma} \\
		&= P\parens{-1 \leq \frac{X_1 - \mu}{\sigma} \leq 1} \\
		&= P\parens{-1 \leq Z \leq 1} \\
		&\approx 68\%
		\end{align*}
		\end{proof}
		
		\item Suppose $\mu = 5$, $\sigma^2 = 1$, and $n = 9$. Find $P\parens{\abs{S_n/n - \mu} \leq \sigma}$.
		\begin{proof}[Solution]
		We have that
		\begin{align*}
		P\parens{\abs{S_n / n - \mu} \leq \sigma} &= P\parens{-\sigma \leq S_n / n - \mu \leq \sigma} \\
		&= P\parens{-\sqrt{n} \leq \frac{S_n / n - \mu}{\sigma/\sqrt{n}} \leq \sqrt{n}} \\
		&= P\parens{-\sqrt{n} \leq Z \leq \sqrt{n}}
		\end{align*}
		Here we have that $n = 9$, so:
		$$P\parens{\abs{S_n / n - \mu} \leq \sigma} = P\parens{-3 \leq Z \leq 3} = 99.7\%$$
		\end{proof}
	\ee
	
	\item[17.] If $X,\ Y,\ Z$ are independent standard normal random variables, find the
distribution of $X^2 + Y^2 + Z^2$. (Hint: Think spherical coordinates.)
	\begin{proof}[Solution]
	We have that $X,\ Y,$ and $Z$ are independent standard normal random variables. To find the distribution of $R^2 = X^2 + Y^2 + Z^2$, we'll find the distributions of $X^2,\ Y^2$, and $Z^2$ in turn:
	\begin{align*}
	E\sqbracks{X^2} &= \int_{-\infty}^\infty \frac{x^2 e^{-\frac{x^2}{2}}}{\sqrt{2\pi}}\ dx \\
	&= \frac{1}{\sqrt{2\pi}} \int_{-\infty}^\infty x^2 e^{-\frac{x^2}{2}}\ dx \\
	&= \frac{1}{\sqrt{2\pi}} \left.\parens{-xe^{-\frac{x^2}{2}} + \int e^{-\frac{x^2}{2}}\ dx}\right|_{-\infty}^\infty \\
	&= \frac{1}{\sqrt{2\pi}} \left.\parens{-xe^{-\frac{x^2}{2}} }\right|_{-\infty}^\infty + \int_{-\infty}^\infty \frac{	e^{-\frac{x^2}{2}}}{\sqrt{2\pi}}\ dx \\
	&= 0 + 1 = 1
	\end{align*}
	And as $Y$ and $Z$ have the same distribution as $X$, we may assume
	$$E\sqbracks{X^2} = E\sqbracks{Y^2} = E\sqbracks{Z^2} = 1$$
	As for $V\sqbracks{X^2}$, we'll use the identity
	$$V\sqbracks{X^2} = E\sqbracks{X^4} - E\sqbracks{X^2}^2$$
	and we need to find
	\begin{align*}
	E\sqbracks{X^4} &= \int_{-\infty}^\infty \frac{x^4 e^{-\frac{x^2}{2}}}{\sqrt{2\pi}}\ dx \\
	&= \frac{1}{\sqrt{2\pi}} \int_{-\infty}^\infty x^4 e^{-\frac{x^2}{2}}\ dx \\
	&= \frac{1}{\sqrt{2\pi}} \left.\parens{-x^3e^{-\frac{x^2}{2}} + 3\int x^2e^{-\frac{x^2}{2}}\ dx}\right|_{-\infty}^\infty \\
	&= \frac{1}{\sqrt{2\pi}} \left.\parens{-x^3e^{-\frac{x^2}{2}} + 3\parens{-xe^{-\frac{x^2}{2}} + \int e^{-\frac{x^2}{2}}\ dx}}\right|_{-\infty}^\infty \\
	&= \frac{1}{\sqrt{2\pi}} \left.\parens{-x^3e^{-\frac{x^2}{2}} - 3xe^{-\frac{x^2}{2}} + 3\int e^{-\frac{x^2}{2}}\ dx}\right|_{-\infty}^\infty \\
	&= \frac{1}{\sqrt{2\pi}} \left.\parens{-x^3e^{-\frac{x^2}{2}}}\right|_{-\infty}^\infty - \frac{3}{\sqrt{2\pi}}\left.\parens{xe^{-\frac{x^2}{2}}}\right|_{-\infty}^\infty + 3\int_{-\infty}^\infty \frac{e^{-\frac{x^2}{2}}}{\sqrt{2\pi}}\ dx\\
	&= 0 - 0 + 3 = 3
	\end{align*}
	So $V\sqbracks{X^2} = E\sqbracks{X^4} - E\sqbracks{X^2}^2 = 3 - 1^2 = 2$, similar to the expectation we may conclude
	$$V\sqbracks{X^2} = V\sqbracks{Y^2} = V\sqbracks{Z^2}$$
	The expectation and variance of $R^2$ are
	$$E\sqbracks{R^2} = E\sqbracks{X^2 + Y^2 + Z^2} = E\sqbracks{X^2} + E\sqbracks{Y^2} + E\sqbracks{Z^2} = 3$$
	$$V\sqbracks{R^2} = V\sqbracks{X^2 + Y^2 + Z^2} = V\sqbracks{X^2} + V\sqbracks{Y^2} + V\sqbracks{Z^2} = 6$$
	So the distribution of $R^2$ is
	$$R^2 = X^2 + Y^2 + Z^2 \sim \distrib{Norm}{3, 6}$$
	\end{proof}
	 
\ee
\noindent\makebox[\linewidth]{\rule{\paperwidth}{0.4pt}}
	
\end{document}
