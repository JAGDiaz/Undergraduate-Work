\documentclass{article}
\usepackage{amsmath}
\usepackage{amssymb}
\usepackage{bm}
\usepackage{amsthm}
\usepackage{enumerate}
\usepackage{graphicx}
\usepackage{psfrag}
\usepackage{color}
\usepackage{url}
\usepackage{listings}
\usepackage{xcolor}
\usepackage{tikz}
\usetikzlibrary{positioning}
\tikzset{main node/.style={circle,fill=gray!20,draw,minimum size=.5cm,inner sep=0pt},}

\definecolor{codegreen}{rgb}{0,0.5,0}
\definecolor{codewhite}{rgb}{1,1,1}
\definecolor{codegray}{rgb}{0.5,0.5,0.5}
\definecolor{codepurple}{rgb}{0.58,0,0.82}
\definecolor{codeblack}{rgb}{0,0,0}
\definecolor{codeorange}{rgb}{0.8,0.4,0}

\lstdefinestyle{mystyle}{
    backgroundcolor=\color{codewhite},   
    commentstyle=\color{codegray},
    keywordstyle=\color{codegreen},
    numberstyle=\color{codegray},
    stringstyle=\color{codeorange},
    basicstyle=\ttfamily ,
    breakatwhitespace=false,         
    breaklines=true,                 
    captionpos=b,                    
    keepspaces=true,                 
    numbers=left,                    
    numbersep=5pt,                  
    showspaces=false,                
    showstringspaces=false,
    showtabs=false,                  
    tabsize=4
}
\lstset{style=mystyle}


\setlength{\hoffset}{-1in}
\addtolength{\textwidth}{1.5in}
\setlength{\voffset}{-1in}
\addtolength{\textheight}{1.5in}
\newcommand{\be}{\begin{enumerate}}
\newcommand{\ee}{\end{enumerate}}
\newcommand{\BigO}[1]{\ensuremath\mathcal{O}\left(#1\right)}
\newcommand{\il}[1]{\lstinline!#1!}
\newcommand{\gnorm}[1]{\left|\left|#1\right|\right|}
\newcommand{\abs}[1]{\left|#1\right|}
\newcommand{\parens}[1]{\left(#1\right)}
\newcommand{\bracks}[1]{\left\{#1\right\}}
\newcommand{\vep}{\varepsilon}
\newcommand{\ceiling}[1]{\left\lceil#1\right\rceil}

\begin{document}
	\begin{center}
		\textbf{Spring 2020, Stats 550:\ Homework 12} \\
		\textbf{Due:\ Thursday, April 9th, 2020} \\
		\textbf{Joseph Diaz: 819947915}
	\end{center}
\noindent\makebox[\linewidth]{\rule{\paperwidth}{0.4pt}}
\section*{Chapter 5}
\be 
	\item[20.] There are 500 deer in a wildlife preserve. A sample of 50 deer are caught and
tagged and returned to the population. Suppose that 20 deer are caught.\\
	\be[(a)] 
		\item Find the mean and standard deviation of the number of tagged deer in the
sample.
	\begin{proof}[Solution]
	Let $X$ be the number of tagged deer in the sample of 20, as there are 500 deer in total, and 50 of them are tagged, the expected value and standard deviation of $X$, which has a hypergeometric distribution, is:
	$$E\left[X\right] = \frac{20\cdot50}{500} = 2,\ SD\left[X\right] = \sqrt{V\left[X\right]} = \sqrt{\frac{20\cdot480\cdot50\cdot450}{500^2\cdot499}} \approx 1.316$$
	\end{proof}
		
		\item Find the probability that the sample contains at least three tagged deer.
		\begin{proof}[Solution]
		Given that $X$ has a hypergeometric distribution, we know that the pmf for $X$ is:
		$$P(X=k) = \frac{\binom{50}{k}\cdot\binom{450}{n - k}}{\binom{500}{n}}$$
		where $n$ is the sample size. So the probability we are interested in is given by:
		\begin{align*}
		P(3 \leq X \leq 20) &= \sum_{k=3}^{20}\frac{\binom{50}{k}\cdot\binom{450}{20 - k}}{\binom{500}{20}}\\
		&= \frac{1}{\binom{500}{20}}\sum_{k=3}^{20}\binom{50}{k}\cdot\binom{450}{20 - k}\\
		&\approx .1289 = 12.89\%
		\end{align*}
		So the probability that at least 3 of the recaptured deer are tagged is 12.89\%.
		\end{proof}
	\ee
	
	\item[21.] \textbf{The Lady Tasting Tea.} This is one of the most famous experiments in the
founding history of statistics. In his 1935 book The Design of Experiments
(1935), Sir Ronald A. Fisher writes,
\begin{quotation}
A Lady declares that by tasting a cup of tea made with milk she can discriminate
whether the milk or the tea infusion was first added to the cup. We will consider
the problem of designing an experiment by means of which this assertion can
be tested... Our experiment consists in mixing eight cups of tea, four in one
way and four in the other, and presenting them to the subject for judgment in a
random order.... Her task is to divide the 8 cups into two sets of 4, agreeing, if
possible, with the treatments received.
\end{quotation}
Consider such an experiment. Four cups are poured milk first and four cups
are poured tea first and presented to a friend for tasting. Let $X$ be the number
of milk-first cups that your friend correctly identifies as milk-first.
	\be[(a)] 
		\item Identify the distribution of $X$.\\
		$X$ has a hypergeometric distribution with $N = 8$, sample size of $n = 4$, and the number of milk-first teas $r = 4$. 
		
		\item Find $P(X = k)$ for $k = 0, 1, 2, 3, 4$.
		\begin{proof}[Solution]
		Given that $X$ is has a hypergeometric distribution, the pmf is given by:
		$$P\parens{X = k} = \frac{\binom{4}{k}\cdot\binom{4}{4-k}}{\binom{8}{4}}$$
		Then the probabilities $k = 0, 1, 2, 3, 4$ are:
		$$P\parens{X = 0} = \frac{\binom{4}{0}\cdot\binom{4}{4}}{\binom{8}{4}}= 1.43\%,\ P\parens{X = 1} = \frac{\binom{4}{1}\cdot\binom{4}{3}}{\binom{8}{4}}=22.86\%$$
		$$P\parens{X = 2} = \frac{\binom{4}{2}\cdot\binom{4}{2}}{\binom{8}{4}}=51.43\%,\ P\parens{X = 3} = \frac{\binom{4}{3}\cdot\binom{4}{1}}{\binom{8}{4}}=22.86\%$$
		$$P\parens{X = 4} = \frac{\binom{4}{4}\cdot\binom{4}{0}}{\binom{8}{4}} = 1.43\%$$
		\end{proof}
		
		\item If in reality your friend had no ability to discriminate and actually guessed, what is the probability they would correctly identify all four cups correctly?
		\begin{proof}[Solution]
		This can be calculated if we let $X \sim \text{Binom}\parens{8,.5}$ be the number of correctly identified milk-first teas. Then the probability we're interested in is given by
		$$P\parens{X = 4} = \binom{8}{4}\parens{.5}^8 = 27.34\%$$
		\end{proof}
	\ee
	\item[22.] In a town of 20,000, there are 12,000 voters, of whom 6000 are registered
democrats and 5000 are registered republicans. An exit poll is taken of 200
voters. Assume all registered voters actually voted. Use R to find:
	\be[(i)]
		\item the mean
and standard deviation of the number of democrats in the sample;\\
Let $D$ be a random variable that represents the number of Democratic voters in the sample with Hypergeometric distribution and $N = 12000$, $r = 6000$, and $n = 200$. Using the formulas
\begin{align*}
E[D] = \sum_{k = 0}^{200}kP\parens{D = k}\\
SD[D] = \sqrt{E\left[D^2\right] - E\left[D\right]^2}
\end{align*}
The Mean and Standard Deviation can be calculated using:
\begin{lstlisting}[language=R]
d = 0:200
meanHG = sum(d*dhyper(d, 6000, 6000, 200))

stdHG = sqrt(sum((d^2)*dhyper(k, 6000, 6000, 200)) - meanHG^2)

meanHG
# 100

stdHG
# 7.012187
\end{lstlisting}
		\item the
probability that more than half the sample are republicans.\\
If we let $R$ be a hypergeometrically distributed Random variable that represents the number of Republican voters in the sample with $N = 12000$, $r = 5000$, and $n = 200$; then we may calculate 
$$P\parens{R > 100} = \sum_{k =101}^{200} P\parens{R = k}$$
using this \il{R} code:
\begin{lstlisting}[language=R]
r = 101:200
sum(dhyper(r, 5000, 7000, 200))
# 0.006787477
\end{lstlisting}
	\ee
\ee
\noindent\makebox[\linewidth]{\rule{\paperwidth}{0.4pt}}
	
\end{document}
