\documentclass{article}
\usepackage{amsmath}
\usepackage{amssymb}
\usepackage{bm}
\usepackage{amsthm}
\usepackage{enumerate}
\usepackage{graphicx}
\usepackage{psfrag}
\usepackage{color}
\usepackage{url}
\usepackage{listings}
\usepackage{xcolor}

\definecolor{codegreen}{rgb}{0,0.5,0}
\definecolor{codewhite}{rgb}{1,1,1}
\definecolor{codegray}{rgb}{0.5,0.5,0.5}
\definecolor{codepurple}{rgb}{0.58,0,0.82}
\definecolor{codeblack}{rgb}{0,0,0}
\definecolor{codeorange}{rgb}{0.8,0.4,0}

\lstdefinestyle{mystyle}{
    backgroundcolor=\color{codewhite},   
    commentstyle=\color{codegray},
    keywordstyle=\color{codegreen},
    numberstyle=\color{codegray},
    stringstyle=\color{codeorange},
    basicstyle=\ttfamily ,
    breakatwhitespace=false,         
    breaklines=true,                 
    captionpos=b,                    
    keepspaces=true,                 
    numbers=left,                    
    numbersep=5pt,                  
    showspaces=false,                
    showstringspaces=false,
    showtabs=false,                  
    tabsize=4
}
\lstset{style=mystyle}


\setlength{\hoffset}{-0.5in}
\addtolength{\textwidth}{1.0in}
\setlength{\voffset}{-0.5in}
\addtolength{\textheight}{1.0in}
\newcommand{\be}{\begin{enumerate}}
\newcommand{\ee}{\end{enumerate}}
\newcommand{\BigO}[1]{\ensuremath\mathcal{O}\left(#1\right)}
\newcommand{\il}[1]{\lstinline!#1!}

\begin{document}
	\begin{center}
		\textbf{Spring 2020, Stats 550:\ Homework 6} \\
		\textbf{Due:\ Thursday, March 5th, 2020} \\
		\textbf{Joseph Diaz: 819947915}
	\end{center}
\noindent\makebox[\linewidth]{\rule{\paperwidth}{0.4pt}}

\section*{Chapter 3}
\be
	\item[3.] There is a 70\% chance that a tree is infected with either root rot or bark disease.
The chance that it does not have bark disease is 0.4. Whether or not a tree
has root rot is independent of whether it has bark disease. Find the probability
that a tree has root rot.
	\begin{proof}[Solution]
	Let $R$ be the event that a tree has tree rot, and $B$ be the event that a tree has bark disease,
	then:
	$$P(R\cup B) = .7,\ P(B) = .6,\ P(B^c)=.6$$
	Now, using inclusion-exclusion and the fact that $R$ and $B$ are independent, we have:
	\begin{align*}
	P(R\cup B) &= P(R) + P(B) - P(RB)\\
	P(R\cup B) &= P(R) + P(B) - P(R)P(B)\\
	P(R) - P(R)P(B) &= P(R\cup B) - P(B)\\ 
	P(R)\big(1 - P(B)\big) &= P(R\cup B) - P(B)\\ 
	P(R) &= \frac{P(R\cup B) - P(B)}{P(B^c)}\\
	P(R) &= \frac{.7 - .6}{.4} = .25
	\end{align*}
	So the probability that a tree has tree rot is 25\%.
	\end{proof}
	\item[5.] \textbf{The first probability problem}\\ A gambler’s dispute in 1654 is said to have
led to the creation of mathematical probability. Two French mathematicians,
Blaise Pascal and Pierre de Fermat, considered the probability that in
24 throws of a pair of dice at least one ``double six'' occurs. It was commonly
believed by gamblers at the time that betting on double sixes in 24 throws would be a profitable bet (i.e., greater than 50\% chance of occurring). But
Pascal and Fermat showed otherwise. Find this probability.
	\begin{proof}[Solution]
	Let $A$ be the event that a ``double six'' is rolled when rolling two dice simultaneously, then $P(A) = \frac{1}{36}$, and $P(A^c) = \frac{35}{36}$. Then the probability of getting at least one ``double six'' in 24 rolls is equal to the complement of getting no double sixes in 24 rolls. So the probability of getting at least one in 24 rolls is:
	$$1 - \left(\frac{35}{36}\right)^{24} \approx 1 - 0.508 = 49.1\%$$
	\end{proof}
	\item[6.] A lottery will be held. From 1000 numbers, one will be chosen as the winner.
A lottery ticket is a number between 1 and 1000. How many tickets do you
need to buy in order for the probability of winning to be at least 50\%?
	\begin{proof}[Solution]
	If each number between 1 and 1000 shows up exactly once, then you would need to buy at least half of the tickets to have a 50\% chance of winning.
	\end{proof}
	\item[7.] The original slot machine had 3 reels with 10 symbols on each reel. On each
play of the slot machine, the reels spin and stop at a random position. Suppose
each reel has one cherry on it. Let $X$ be the number of cherries that show up
from one play of the slot machine. Find $P(X = k)$, for $k = 0, 1, 2, 3$. Slot
machines are also known as “one-armed bandits.”
	\begin{proof}[Solution]
	Firstly, we consider that for each spin of the slots there are $10^3$ possible outcomes; then the probability of each outcome involving cherries is:
	\be
		\item[$P(X=0)$:]
		There are $9^3$ ways to not get a single cherry, so:
		$$P(X=0) = \frac{9^3}{10^3} = \frac{729}{1000} = 72.9\%$$
		\item[$P(X=1)$:]
		There are $9^2\cdot \binom{3}{1}$ ways to get exactly one cherry, so:
		$$P(X=1) = \frac{3\cdot9^2}{10^3} = \frac{243}{1000} = 24.3\%$$
		\item[$P(X=2)$:]
There are $9\cdot \binom{3}{2}$ ways to get exactly two cherries, so:
		$$P(X=2) = \frac{3\cdot9}{10^3} = \frac{27}{1000} = 2.7\%$$
		\item[$P(X=3)$:]
There is one way to get exactly three cherries, so:
$$P(X=3) = \frac{1}{10^3} = .1\%$$
	\ee
	\end{proof}
\ee
\noindent\makebox[\linewidth]{\rule{\paperwidth}{0.4pt}}
	
\end{document}
