\documentclass{article}
\usepackage{amsmath}
\usepackage{amssymb}
\usepackage{bm}
\usepackage{amsthm}
\usepackage{enumerate}
\usepackage{graphicx}
\usepackage{psfrag}
\usepackage{color}
\usepackage{url}
\usepackage{listings}

\usepackage{xcolor}
 
\definecolor{codegreen}{rgb}{0,0.6,0}
\definecolor{codegray}{rgb}{0.5,0.5,0.5}
\definecolor{codepurple}{rgb}{0.58,0,0.82}
\definecolor{backcolour}{rgb}{0.95,0.95,0.92}

\lstdefinestyle{mystyle}{
    backgroundcolor=\color{backcolour},   
    commentstyle=\color{codegray},
    keywordstyle=\color{codegreen},
    numberstyle=\tiny\color{magenta},
    stringstyle=\color{codepurple},
    basicstyle=\ttfamily\footnotesize,
    breakatwhitespace=false,         
    breaklines=true,                 
    captionpos=b,                    
    keepspaces=true,                 
    numbers=left,                    
    numbersep=5pt,                  
    showspaces=false,                
    showstringspaces=false,
    showtabs=false,                  
    tabsize=4
}
 
\lstset{style=mystyle}

\setlength{\hoffset}{-0.5in}
\addtolength{\textwidth}{1.0in}
\setlength{\voffset}{-0.5in}
\addtolength{\textheight}{1.0in}
\newcommand{\be}{\begin{enumerate}}
\newcommand{\ee}{\end{enumerate}}

\begin{document}
	\begin{large}
	\begin{center}
		\textbf{Spring 2020, Stats 550:\ Homework 4} \\
		\textbf{Due:\ Tuesday, February 4th, 2020} \\
		\textbf{Joseph Diaz}
	\end{center}
\noindent\makebox[\linewidth]{\rule{\paperwidth}{0.4pt}}
\section*{Chapter 1}
\be[41.]
	\item Modify the code in the R script \textbf{CoinFlip.R} to simulate the probability of
getting exactly one head in four coin tosses.
\begin{proof}[R code]
	The following code simulates the given experiment.
\begin{lstlisting}[language=R]
CoinFlip4 = function(n)
{
	flip = matrix(sample(0:1, 4*n, replace = TRUE), nrow = 4, byrow = FALSE)
	total = flip[1,1:n] + flip[2,1:n] + flip[3,1:n] + flip[4,1:n]
	mean(total == 1)
}

n = 100000

CoinFlip4(n)

# 0.25178
\end{lstlisting}
\end{proof}
	
	\item Modify the code in the R script \textbf{Divisible356.R} to simulate the probability
that a random integer between 1 and 5000 is divisible by 4, 7, or 10. Compare
with your answer in Exercise 1.37.
\begin{proof}[R code] The following code simulates the given experiment.
\begin{lstlisting}[language=R]
Divisible4710 = function(n)
{
  num = sample(1:5000, n, replace = TRUE)
  mean(num%%4==0|num%%7==0|num%%10==0)
}

n = 100000

Divisible4710(n)

# 0.40015
\end{lstlisting}
\end{proof}
\pagebreak
	\item Use R to simulate the probability of getting at least 8 in the sum of two
dice rolls.
\begin{proof}[R code] The following code simulates the given experiment.
\begin{lstlisting}[language=R]
SumTo8 = function(n)
{
  rolls = matrix(sample(1:6, 2*n, replace = TRUE), nrow = 2, byrow = FALSE)
  total = rolls[1,1:n] + rolls[2,1:n]
  mean(total >= 8)
}

n = 100000

SumTo8(n)

# 0.41684
\end{lstlisting}
\end{proof}

	\item Use R to simulate the probability at least one 2 in 5 rolls of a tetrahedron dice.
\begin{proof}[R code] The following code simulates the given experiment.
\begin{lstlisting}[language=R]
TetraDice = function(n)
{
  rolls = matrix(sample(1:4, 5*n, replace = TRUE), nrow = 5, byrow = FALSE)
  success = numeric(n)
  for(i in 1:n)
  {
    success[i] = if(2 %in% rolls[1:5,i]) 1 else 0
  }
  mean(success)
}

n = 100000

TetraDice(n)

# 0.76245
\end{lstlisting}
\end{proof}


\ee
\noindent\makebox[\linewidth]{\rule{\paperwidth}{0.4pt}}
	
	\end{large}
\end{document}
