\documentclass{article}
\usepackage{amsmath}
\usepackage{amssymb}
\usepackage{bm}
\usepackage{amsthm}
\usepackage{enumerate}
\usepackage{graphicx}
\usepackage{psfrag}
\usepackage{color}
\usepackage{url}
\usepackage{listings}
\usepackage{xcolor}
\usepackage{tikz}
\usetikzlibrary{positioning}
\tikzset{main node/.style={circle,fill=gray!20,draw,minimum size=.5cm,inner sep=0pt},}

\definecolor{codegreen}{rgb}{0,0.5,0}
\definecolor{codewhite}{rgb}{1,1,1}
\definecolor{codegray}{rgb}{0.5,0.5,0.5}
\definecolor{codepurple}{rgb}{0.58,0,0.82}
\definecolor{codeblack}{rgb}{0,0,0}
\definecolor{codeorange}{rgb}{0.8,0.4,0}

\lstdefinestyle{mystyle}{
    backgroundcolor=\color{codewhite},   
    commentstyle=\color{codegray},
    keywordstyle=\color{codegreen},
    numberstyle=\color{codegray},
    stringstyle=\color{codeorange},
    basicstyle=\ttfamily ,
    breakatwhitespace=false,         
    breaklines=true,                 
    captionpos=b,                    
    keepspaces=true,                 
    numbers=left,                    
    numbersep=5pt,                  
    showspaces=false,                
    showstringspaces=false,
    showtabs=false,                  
    tabsize=4
}
\lstset{style=mystyle}


\setlength{\hoffset}{-1in}
\addtolength{\textwidth}{1.5in}
\setlength{\voffset}{-1in}
\addtolength{\textheight}{1.5in}
\newcommand{\be}{\begin{enumerate}}
\newcommand{\ee}{\end{enumerate}}
\newcommand{\BigO}[1]{\ensuremath\mathcal{O}\left(#1\right)}
\newcommand{\il}[1]{\lstinline!#1!}
\newcommand{\gnorm}[1]{\left|\left|#1\right|\right|}
\newcommand{\abs}[1]{\left|#1\right|}
\newcommand{\parens}[1]{\left(#1\right)}
\newcommand{\bracks}[1]{\left\{#1\right\}}
\newcommand{\sqbracks}[1]{\left[#1\right]}
\newcommand{\vep}{\varepsilon}
\newcommand{\ceiling}[1]{\left\lceil#1\right\rceil}
\newcommand{\R}{\mathbb{R}}
\newcommand{\N}{\mathbb{N}}
\newcommand{\distrib}[2]{\text{#1}\left(#2\right)}
\newcommand{\dd}[1]{\frac{d}{d#1}}
\newcommand{\abracks}[1]{\left< #1\right>}

\begin{document}
	\begin{center}
		\textbf{Spring 2020, Stats 550:\ Homework 21} \\
		\textbf{Due:\ Tuesday, May 12th, 2020} \\
		\textbf{Joseph Diaz: 819947915}
	\end{center}
\noindent\makebox[\linewidth]{\rule{\paperwidth}{0.4pt}}
\subsection*{Chapter 9}
\be 
	\item[9.] Show how to use Monte Carlo techniques to approximate the following
integrals and sums.
	\be[(a)]
		\item $$I = \int_0^1 \sin(x)e^{-x^2}\ dx$$ 
		\begin{proof}[Solution]
		Let $X_i \sim \distrib{Unif}{0,1},\ 1 \leq i \leq n,\ f(x) = \sin(x)e^{-x^2}$. Then $I$ may be approximated like so:
		$$I \approx \frac{1}{n}\sum_{i=1}^n f\parens{X_i}$$
		And with \il{R}, we have
		\begin{lstlisting}[language=R]
f = function(x)
{
  return(sin(x)*exp(-x^2))
}
n= 100000
mean(f(runif(n)))
# 0.2945254

integrate(f,0,1)
# 0.2946982 with absolute error < 3.3e-15
		\end{lstlisting}
		\end{proof}
		
		\item $$I = \int_0^\infty \sin(x)e^{-x^2}\ dx$$
		\begin{proof}[Solution]
		Before we use Monte Carlo techniques:
		$$\int_0^\infty \sin(x)e^{-x^2}\ dx = \int_0^\infty \sin(x)e^xe^{-x^2}e^{-x}\ dx$$
		So let $f(x) = \sin(x)e^xe^{-x^2},\ g(x) = \sin(x)e^{-x^2}$ and $X_i \sim \distrib{Exp}{1},\ 1 \leq i \leq n$. Then $I$ may be approximated like so:
		$$I \approx \frac{1}{n}\sum_{i=1}^n f\parens{X_i}$$
		And with \il{R}, we have
		\begin{lstlisting}[language=R]
f = function(x)
{
  return(sin(x)*exp(-x^2)*exp(x))
}
n= 100000
mean(f(rexp(n,1)))
# 0.424123

g = function(x)
{
  return(sin(x)*exp(-x^2))
}
integrate(g,0,Inf)
# 0.4244364 with absolute error < 2.3e-06
		\end{lstlisting}
		\end{proof}
		
		\item $$I = \int_{-\infty}^\infty \log\parens{x^2}e^{-x^2}\ dx$$
		\begin{proof}[Solution]
		Before we use Monte Carlo techniques:
		$$\int_{-\infty}^\infty \log\parens{x^2}e^{-x^2}\ dx= \int_{-\infty}^\infty \sqrt{2\pi} \log\parens{x^2}\frac{e^{-x^2}}{\sqrt{2\pi}}\ dx$$
		So let $f(x) = \sqrt{2\pi} \log\parens{x^2},\ g(x) = \log\parens{x^2}e^{-x^2}$ and $X_i \sim \distrib{Norm}{0,1},\ 1 \leq i \leq n$. Then $I$ may be approximated like so:
		$$I \approx \frac{1}{n}\sum_{i=1}^n f\parens{X_i}$$
		And with \il{R}, we have
		\begin{lstlisting}[language=R]
f = function(x)
{
  return(sqrt(2*pi)*log(x^2))
}
n= 1000000
mean(f(rnorm(n,0,1)))
# -3.18667

g = function(x)
{
  return(log(x^2)*exp(-x^2))
}
integrate(g,-Inf,Inf)
# -3.480231 with absolute error < 0.00018
		\end{lstlisting}
		\end{proof}
				
		\item $$S = \sum_{k=1}^\infty \frac{\sin k}{2^k}$$
		\begin{proof}[Solution]
		Before we use Monte Carlo techniques:
		$$\sum_{k=1}^\infty \frac{\sin k}{2^k} = \sum_{k=1}^\infty \sin k \parens{\frac{1}{2}}^{k-1}\parens{\frac{1}{2}} = \sum_{k=1}^\infty \sin k P\parens{X=k}$$
		So let $f(x) = \sin(x),\ g(x) = \frac{\sin(x)}{2^x}$ and $X_i \sim \distrib{Geom}{1/2},\ 1 \leq i \leq n$. Then $S$ may be approximated like so:
		$$S \approx \frac{1}{n}\sum_{i=1}^n f\parens{X_i}$$
		And with \il{R}, we have
		\begin{lstlisting}[language=R]
f = function(x)
{
  return(sin(x))
}
n= 1000000
mean(f(rgeom(n,1/2) + 1))
# 0.5935965

g = function(x)
{
  return(sin(x)/(2^x))
}
x = 0:n
sum(g(x))
# 0.5928376
		\end{lstlisting}
		\end{proof}
		
		\item $$S = \sum_{k=0}^\infty \frac{\cos\cos k}{k!}$$
		\begin{proof}[Solution]
		Before we use Monte Carlo techniques:
		$$\sum_{k=0}^\infty \frac{\cos\cos k}{k!} = \sum_{k=0}^\infty \frac{\cos\cos k}{e^{-1}}\cdot\frac{e^{-1}}{k!} = \sum_{k=0}^\infty e\cos\cos k\cdot P\parens{X = k}$$
		So let $f(x) = e\cos\cos(x),\ g(x) =\frac{\cos\cos x}{x!}$ and $X_i \sim \distrib{Pois}{1},\ 1 \leq i \leq n$. Then $S$ may be approximated like so:
		$$S \approx \frac{1}{n}\sum_{i=1}^n f\parens{X_i}$$
		And with \il{R}, we have
		\begin{lstlisting}[language=R]
f = function(x)
{
  return(exp(1)*cos(cos(x)))
}
n= 1000000
mean(f(rpois(n,1)))
# 1.989062

g = function(x)
{
  return(cos(cos(x))/factorial(x))
}
x = 0:n
sum(g(x))
# 1.988677
		\end{lstlisting}
		\end{proof}
		
	\ee
\ee
\noindent\makebox[\linewidth]{\rule{\paperwidth}{0.4pt}}
	
\end{document}
