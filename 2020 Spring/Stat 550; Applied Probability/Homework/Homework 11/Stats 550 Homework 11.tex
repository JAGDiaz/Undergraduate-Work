\documentclass{article}
\usepackage{amsmath}
\usepackage{amssymb}
\usepackage{bm}
\usepackage{amsthm}
\usepackage{enumerate}
\usepackage{graphicx}
\usepackage{psfrag}
\usepackage{color}
\usepackage{url}
\usepackage{listings}
\usepackage{xcolor}
\usepackage{tikz}
\usetikzlibrary{positioning}
\tikzset{main node/.style={circle,fill=gray!20,draw,minimum size=.5cm,inner sep=0pt},}

\definecolor{codegreen}{rgb}{0,0.5,0}
\definecolor{codewhite}{rgb}{1,1,1}
\definecolor{codegray}{rgb}{0.5,0.5,0.5}
\definecolor{codepurple}{rgb}{0.58,0,0.82}
\definecolor{codeblack}{rgb}{0,0,0}
\definecolor{codeorange}{rgb}{0.8,0.4,0}

\lstdefinestyle{mystyle}{
    backgroundcolor=\color{codewhite},   
    commentstyle=\color{codegray},
    keywordstyle=\color{codegreen},
    numberstyle=\color{codegray},
    stringstyle=\color{codeorange},
    basicstyle=\ttfamily ,
    breakatwhitespace=false,         
    breaklines=true,                 
    captionpos=b,                    
    keepspaces=true,                 
    numbers=left,                    
    numbersep=5pt,                  
    showspaces=false,                
    showstringspaces=false,
    showtabs=false,                  
    tabsize=4
}
\lstset{style=mystyle}


\setlength{\hoffset}{-1in}
\addtolength{\textwidth}{1.5in}
\setlength{\voffset}{-1in}
\addtolength{\textheight}{1.5in}
\newcommand{\be}{\begin{enumerate}}
\newcommand{\ee}{\end{enumerate}}
\newcommand{\BigO}[1]{\ensuremath\mathcal{O}\left(#1\right)}
\newcommand{\il}[1]{\lstinline!#1!}
\newcommand{\gnorm}[1]{\left|\left|#1\right|\right|}
\newcommand{\abs}[1]{\left|#1\right|}
\newcommand{\parens}[1]{\left(#1\right)}
\newcommand{\bracks}[1]{\left\{#1\right\}}
\newcommand{\vep}{\varepsilon}
\newcommand{\ceiling}[1]{\left\lceil#1\right\rceil}

\begin{document}
	\begin{center}
		\textbf{Spring 2020, Stats 550:\ Homework 11} \\
		\textbf{Due:\ Tuesday, April 7th, 2020} \\
		\textbf{Joseph Diaz: 819947915}
	\end{center}
\noindent\makebox[\linewidth]{\rule{\paperwidth}{0.4pt}}
\section*{Chapter 5}
\be 
	\item[11.] A fair coin is tossed until heads appears four times.
	\be[(a)] 
		\item Find the probability that it took exactly 10 flips.
		\begin{proof}[Solution]
		Let $X$ be the number of coin flips until 4 heads appear with $X \sim \text{NegBin(4, .5)}$. So the probability we're interested in is given by:
		$$P\parens{X = 10} = \binom{10-1}{4-1}\parens{.5}^4\parens{1-.5}^{10-4} = \binom{9}{3}\parens{.5}^{10}$$
		$$\approx .08203 = 8.203\%$$
		\end{proof}
		
		\item Find the probability that it took at least 10 flips.
		\begin{proof}[Solution]
		Given the same $X$ as in (a), the probability we're interested in is given by:
		\begin{align*}
		P\parens{X \geq 10} &= 1 - P\parens{X < 10}\\
		&= 1 - \sum_{k = 4}^9 \binom{k-1}{3}\parens{.5}^4\parens{1-.5}^{k-4}\\
		&= 1 - \sum_{k = 4}^9 \binom{k-1}{3}\parens{.5}^k\\
		&\approx .253906 = 25.4\%
		\end{align*}
		\end{proof}
		
		\item Let Y be the number of tails that occur. Find the pmf of $Y$.
		If $Y$ counts the number of tails that occur, then $Y = X -4$, which is equivalent to $$P\parens{Y = k} = P\parens{X = k + 4}$$
		
	\ee
	
	\item[12.] Baseball teams $A$ and $B$ face each other in the World Series. For each game,
the probability that $A$ wins the game is $p$, independent of other games. Find
the expected length of the series.
	\begin{proof}[Solution]
	The World Series is a best of 7 baseball games, so letting $X$ be the number of games until $A$ wins 4 times gives $X \sim \text{NegBin}(4,p)$. Because $X$ has a Negative Binomial Distribution:
	$$E\left[X\right] = \frac{r}{p} = \frac{4}{p}$$
	\end{proof}
	
	\item[15.] People whose blood type is O-negative are universal donors - anyone
can receive a blood transfusion of O-negative blood. In the U.S., 7.2\% of
the people have O-negative blood. A blood donor clinic wants to find 10
O-negative individuals. In repeated screening, what is the chance of finding
such individuals among the first 100 people screened?
	\begin{proof}[Solution]
	Let $X$ be the number of screenings necessary for 10 O-negative blood individuals to be found. So $X \sim \text{NegBinom}\parens{10,.072}$, and the probability we're interested in is given by:
	$$P\parens{X = 100} = \binom{99}{9}\parens{.072}^{10}\parens{1-.072}^{90} = \binom{99}{9}\parens{.072}^{10}\parens{.928}^{90} \approx .0078 = .78\%$$
	\end{proof}
	
	\item[17.] Andy and Beth are playing a game worth \$100. They take turns flipping a
penny. The first person to get 10 heads will win. But they just realized that
they have to be in math class right away and are forced to stop the game. Andy
had four heads and Beth had seven heads. How should they divide the pot?
	\begin{proof}[Solution]
	Given that Andy already had 4 heads and that Beth had had 7, a winner would be guaranteed after 8 additional flips. Let $X$ be the number of flips necessary for Andy to win, with $X \sim \text{NegBin}\parens{6, .5}$. Naturally, there would need to be at least 6 more flips for Andy to win and up to 8 total, so the probability that Andy wins is given by: 
	$$P\parens{6 \leq X \leq 8} = \sum_{k=6}^8\binom{k-1}{5}\parens{.5}^k \approx .1445 = 14.45\%$$
	So that's the probability that Andy wins, so the probability of Beth winning is $1 - .1445 = .8555 = 85.55\%$, so the reasonable way to divide the pot would be \$14.45 to Andy and \$85.55 to Beth.
	\end{proof}

\ee
\noindent\makebox[\linewidth]{\rule{\paperwidth}{0.4pt}}
	
\end{document}
