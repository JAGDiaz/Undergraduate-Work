\documentclass{article}
\usepackage{amsmath}
\usepackage{amssymb}
\usepackage{bm}
\usepackage{amsthm}
\usepackage{enumerate}
\usepackage{graphicx}
\usepackage{psfrag}
\usepackage{color}
\usepackage{url}
\usepackage{listings}
\usepackage{xcolor}
\usepackage{tikz}
\usetikzlibrary{positioning}
\tikzset{main node/.style={circle,fill=gray!20,draw,minimum size=.5cm,inner sep=0pt},}

\definecolor{codegreen}{rgb}{0,0.5,0}
\definecolor{codewhite}{rgb}{1,1,1}
\definecolor{codegray}{rgb}{0.5,0.5,0.5}
\definecolor{codepurple}{rgb}{0.58,0,0.82}
\definecolor{codeblack}{rgb}{0,0,0}
\definecolor{codeorange}{rgb}{0.8,0.4,0}

\lstdefinestyle{mystyle}{
    backgroundcolor=\color{codewhite},   
    commentstyle=\color{codegray},
    keywordstyle=\color{codegreen},
    numberstyle=\color{codegray},
    stringstyle=\color{codeorange},
    basicstyle=\ttfamily ,
    breakatwhitespace=false,         
    breaklines=true,                 
    captionpos=b,                    
    keepspaces=true,                 
    numbers=left,                    
    numbersep=5pt,                  
    showspaces=false,                
    showstringspaces=false,
    showtabs=false,                  
    tabsize=4
}
\lstset{style=mystyle}


\setlength{\hoffset}{-1in}
\addtolength{\textwidth}{1.5in}
\setlength{\voffset}{-1in}
\addtolength{\textheight}{1.5in}
\newcommand{\be}{\begin{enumerate}}
\newcommand{\ee}{\end{enumerate}}
\newcommand{\BigO}[1]{\ensuremath\mathcal{O}\left(#1\right)}
\newcommand{\il}[1]{\lstinline!#1!}
\newcommand{\gnorm}[1]{\left|\left|#1\right|\right|}
\newcommand{\abs}[1]{\left|#1\right|}
\newcommand{\parens}[1]{\left(#1\right)}
\newcommand{\bracks}[1]{\left\{#1\right\}}
\newcommand{\sqbracks}[1]{\left[#1\right]}
\newcommand{\vep}{\varepsilon}
\newcommand{\ceiling}[1]{\left\lceil#1\right\rceil}
\newcommand{\R}{\mathbb{R}}
\newcommand{\N}{\mathbb{N}}
\newcommand{\distrib}[2]{\text{#1}\left(#2\right)}

\begin{document}
	\begin{center}
		\textbf{Spring 2020, Stats 550:\ Homework 15} \\
		\textbf{Due:\ Tuesday, April 21st, 2020} \\
		\textbf{Joseph Diaz: 819947915}
	\end{center}
\noindent\makebox[\linewidth]{\rule{\paperwidth}{0.4pt}}
\section*{Chapter 6}
\be 
	\item[12.] It is 9:00 p.m. The time until Joe receives his next text message has an
exponential distribution with mean 5 minutes.
	\be[(a)]
		\item Find the probability that he will not receive a text in the next 10 minutes.
		\begin{proof}[Solution]
		Let $X \sim \distrib{Exp}{\lambda}$ be the time until Joe gets a text. We have that $E\sqbracks{X} = 5 = \frac{1}{\lambda}$ which implies $\lambda = \frac{1}{5}$. Now, we may use the formula for the tail probability to find the probability that Joe will not get a text in the next 10 minutes:
		$$P(X > 10) = e^{-\frac{1}{5}\cdot 10} = e^{-2} \approx 13.534\%$$
		\end{proof}
		
		\item Find the probability that the next text arrives between 9:07 and 9:10 p.m.
		\begin{proof}[Solution]
		To find this we'll use the cdf of $X$, given by $F(x) = 1 - e^{\frac{1}{5}x}$. So:
		$$P(7 < X < 10) = F(10) - F(7) = 1 - e^{-2} - 1 + e^{-\frac{7}{5}} \approx 11.126\%$$
		\end{proof}
		
		\item Find the probability that a text arrives before 9:03 p.m.
		\begin{proof}[Solution]
		We want to find $P(X < 3)$, which is equivalent to:
		$$P\parens{0 < X < 3} = F(3) - F(0) = 1 - e^{-\frac{3}{5}} \approx 45.119\%$$
		\end{proof}
		
		\item A text has not arrived for 5 minutes. Find the probability that none will
arrive for 7 minutes.
		\begin{proof}[Solution]
		We want to find 
		$$P\parens{X > 7\ |\ X > 5}$$
		and as the exponential distribution is ``memoryless'', this is equivalent to finding
		$$P\parens{X > 7-5} = P(X > 2) = e^{-\frac{2}{5}} \approx 67.032\%$$
		\end{proof}
		
	\ee
	
	\item[14.] Derive the mean of the exponential distribution with parameter $\lambda$.
	\begin{proof}[Derivation]
	We'll derive this using the formula for the expected value of a continuous random variable; let $X \sim \distrib{Exp}{\lambda}$, then:
	\begin{align*}
	E\sqbracks{X} &= \int_0^\infty xf(x)\ dx\\
	&= \int_0^\infty \lambda xe^{-\lambda x}\ dx\\
	&= \frac{1}{\lambda}\parens{-\lambda xe^{-\lambda x} + \int e^{-\lambda x}\ dx}\Big|_0^\infty \\
	&= \frac{1}{\lambda}\parens{-\lambda xe^{-\lambda x} - e^{-\lambda x}}\Big|_0^\infty \\
	&= \frac{1}{\lambda}\parens{-\frac{\lambda \infty}{ e^{\lambda \infty}} - e^{-\lambda \infty} + 0 + e^{0}} \\
	&= \frac{1}{\lambda} \parens{1} \\
	&= \frac{1}{\lambda}
	\end{align*}
	Thus,
	$$E\sqbracks{X} = \frac{1}{\lambda}$$
	\end{proof}
	
	\item[15.] Derive the variance of the exponential distribution with parameter $\lambda$.
	\begin{proof}[Derivation]
	We'll derive this using the formula for the variance of a random continuous variable and the result from 5.14:
	$$V\sqbracks{X} = E\sqbracks{X^2} - E\sqbracks{X}^2, \quad E\sqbracks{X}^2 = \frac{1}{\lambda^2}$$
	So we'll use expected value formula to find $E\sqbracks{X^2}$:
	\begin{align*}
	E\sqbracks{X^2} &= \int_0^\infty x^2f(x)\ dx\\
	&= \int_0^\infty \lambda x^2 e^{-\lambda x}\ dx\\
	&= \lambda\left.\parens{-\frac{x^2e^{-\lambda x}}{\lambda} + \frac{2}{\lambda}\int xe^{-\lambda x}\ dx}\right|_0^\infty\\
	&= \lambda\left.\parens{-\frac{x^2e^{-\lambda x}}{\lambda} + \frac{2}{\lambda^2}\parens{-\lambda xe^{-\lambda x} - e^{-\lambda x}}}\right|_0^\infty\\
	&= \left.\parens{-x^2e^{-\lambda x} + \frac{2}{\lambda}\parens{-\lambda xe^{-\lambda x} - e^{-\lambda x}}}\right|_0^\infty\\
	&= \left.\parens{-x^2e^{-\lambda x}}\right|_0^\infty + \frac{2}{\lambda^2}\\
	&= \parens{\frac{-\infty^2}{e^{\lambda \infty}} - 0} + \frac{2}{\lambda^2}\\
	&= \frac{2}{\lambda^2}
	\end{align*}
	So we have:
	$$V\sqbracks{X} = \frac{2}{\lambda^2} - \frac{1}{\lambda^2} = \frac{1}{\lambda^2}$$
	\end{proof}
	
	\item[17.] The time each student takes to finish an exam has an exponential distribution
with mean 45 minutes. In a class of 10 students, what is the probability that at
least one student will finish in less than 20 minutes? Assume students’ times
are independent.
	\begin{proof}[Solution]
	Let $X_i \sim \distrib{Exp}{\lambda}\ i \in \bracks{1,\ \cdots,\ 10}$ be the time until the $i^{th}$ student finishes the exam. It is given that $E\sqbracks{X_i} = 45$, so we for each we may conclude $\lambda = \frac{1}{45}$. As we assuming that each student's time is independent, the probability that the $i^{th}$ student takes longer than 20 minutes to finish the exam is
	$$P\parens{X_i > 20} = 1 - F(20) = 1 - e^{-\frac{4}{9}}$$
	so the probability that all 10 students take longer than 20 minutes to finish is $\parens{1-e^{-\frac{4}{9}}}^{10}$, to the probability that at least one student finish less than 20 minutes should be the complement of this, so the probability we are interested in is 
	$$1 - \parens{1-e^{-\frac{4}{9}}}^{10} \approx 99.996\%$$
	\end{proof}
	
	\item[21.] Suppose $X \sim \distrib{Exp}{\lambda}$. Find the density of $Y = cX$ for $c > 0$. Describe the
distribution of $Y$.
	\begin{proof}[Solution]
	We have that the cdf for $X$ is $F(x) = 1 - e^{-\lambda x} = P\parens{X \leq x}$ and given the relationship between $X$ and $Y$, we have:
	$$P\parens{Y \leq y} = P\parens{cX \leq y} = P\parens{X \leq \frac{y}{c}}$$
	let $H(y) = F\parens{\frac{y}{c}} = 1 - e^{-\frac{\lambda }{c}y}$ be the cdf of $Y$, then let $h(y) = H'(y)$ be the pdf of $Y$; so:
	$$h(y) = \frac{\lambda}{c}e^{-\frac{\lambda}{c}y},$$
	Which implies $Y \sim \distrib{Exp}{\frac{\lambda}{c}}$.
	\end{proof}
	
\ee
\noindent\makebox[\linewidth]{\rule{\paperwidth}{0.4pt}}
	
\end{document}
