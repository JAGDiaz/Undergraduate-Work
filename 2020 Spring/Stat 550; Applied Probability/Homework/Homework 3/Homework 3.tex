\documentclass{article}
\usepackage{amsmath}
\usepackage{amssymb}
\usepackage{bm}
\usepackage{amsthm}
\usepackage{enumerate}
\usepackage{graphicx}
\usepackage{psfrag}
\usepackage{color}
\usepackage{url}

	
\setlength{\hoffset}{-0.5in}
\addtolength{\textwidth}{1.0in}
\setlength{\voffset}{-0.5in}
\addtolength{\textheight}{1.0in}
\newcommand{\be}{\begin{enumerate}}
\newcommand{\ee}{\end{enumerate}}
%\newcommand{\deg}{\text{deg}}

\begin{document}
	\begin{large}
	\textbf{Joseph Diaz}
	\begin{center}
		\textbf{Spring 2020, Stats 550:\ Homework 3} \\
		\textbf{Due:\ Tuesday, February 11th, 2020} \\
	\end{center}
\noindent\makebox[\linewidth]{\rule{\paperwidth}{0.4pt}}
\section*{Chapter 1}
\be
	\item[10.] A random experiment has three possible outcomes a, b, and c, with
$$P(a) = p, P(b) = p^2, and P(c) = p$$
What choice(s) of $p$ makes this a valid probability model?
	\begin{proof}[Solution]
	As the sum of the probabilities of all possible outcomes of an experiment must equal 1. We conclude:
	\begin{align*}
	P(a) + P(b) + P(c) = 1 \\
	p + p^2 + p = 1 \\
	p^2 + 2p = 1 \\
	p^2 + 2p + 1 = 2 \\
	(p+1)^2 = 2 \\
	p + 1 = \pm\sqrt{2} \\
	p = \pm\sqrt{2} -1
	\end{align*}
	$p$ is a probability, so we'll discard the 
	possibility that $p = -\sqrt{2} - 1$ as probability 
	are defined to be $0 \leq p \leq 1$. So the value of 
	$p$ we're interested in is:
	$$p = \sqrt{2} - 1 \approx .4142$$
	\end{proof}
	
	
	\item[11.] Let $P_1$ and $P_2$ be two probability functions on $\Omega$. Define a new function $P$
such that $P(A) = \frac{P_1(A)+P_2(A)}{2}$. Show the $P$ is a probability function.
	\begin{proof}[Solution]
	We will show that $P$ has the 3 properties of a probability function:
	\be[1)]
		\item $P$ is defined:
		$$P(\omega) = \frac{P_1(\omega)+P_2(\omega)}{2},\ \forall \omega \in \Omega$$
		$P_1, P_2$ are probability functions, so
		$$\forall \omega \in \Omega: P_1(\omega), P_2(\omega) \geq 0$$ Therefore:
		$$P(\omega) = \frac{P_1(\omega)+P_2(\omega)}{2} \geq 0,
\ \forall \omega \in \Omega$$
		
		\item Again by the definition of $P$:
		\begin{align*}
		\sum_{\omega \in \Omega} P(\omega) &= \sum_{\omega \in \Omega} \left(\frac{P_1(\omega)+P_2(\omega)}{2}\right) \\
		&= \frac{1}{2}\sum_{\omega \in \Omega} P_1(\omega)+\frac{1}{2}\sum_{\omega \in \Omega}P_2(\omega) \\
		&= \frac{1}{2} + \frac{1}{2} \\
		&= 1
		\end{align*}
		

		\item Again by the definition of $P$, $\forall A \subseteq \Omega$:
		\begin{align*}
		\sum_{\omega\in A}P(\omega) &= \sum_{\omega \in \Omega} \left(\frac{P_1(\omega)+P_2(\omega)}{2}\right) \\
		 &= \frac{1}{2}\sum_{\omega \in A} P_1(\omega)+\frac{1}{2}\sum_{\omega \in A}P_2(\omega) \\
		 &= \frac{1}{2}P_1(A) + \frac{1}{2}P_2(A)\\
		 &= \frac{P_1(A)+P_2(A)}{2} \\
		 &= P(A)
		\end{align*}
	\ee
	There $P$ is a probability function on $\Omega$
	\end{proof}
	
	\item[14.] A club has 10 members and is choosing a president, vice-president, and
treasurer. All selections are equally likely.
	\be
	\item[(a)]
 What is the probability that Tom is selected president?
 	\begin{proof}[Solution]
 	There are 10 members of the club, so the probability that Tom is chosen as president is $\frac{1}{10} = 10\%$.
 	\end{proof}
 	\item[(b)]
What is the probability that Brenda is chosen president and Liz is chosen treasurer?
	\begin{proof}[Solution]
	Under the assumption that club members can only serve in one position at once: the probability that Brenda is chosen president is $\frac{1}{10}$, and the probability that Liz is then chosen treasurer is $\frac{1}{9}$, so the probability that they're both chosen for these positions is:
	$$\frac{1}{10 \cdot 9} \approx 1.11\%$$
	\end{proof}
	\ee
	
	\item[17.] Suppose you throw five dice and all outcomes are equally likely.
	\be
	\item[(a)] What is the probability that all dice are the same? (In the game of Yahtzee, this is known as a yahtzee.)
	\begin{proof}[Solution]
	There are 6 ways to get the same number on 5 different 
	dice and $6^5$ possible outcomes when rolling 5 dice.
	So the probability that all 5 are the same number is:
	$$\frac{6}{6^5} = \frac{1}{6^4} \approx .08\%$$
	\end{proof}
	\item[(b)] What is the probability of getting at least one 4?
	\begin{proof}[Solution]
	The probability of getting at least one 4 is equal to the complement of getting no 4s at all. The probability of getting no 4s is $\frac{1}{5^5}$, so the probability of getting at least one 4 is:
	$$1 - \frac{1}{5^5} \approx 99\%$$
	
	\end{proof}
	\item[(c)] What is the probability that all the dice are different?
	\begin{proof}[Solution]
	There are $6\cdot5\cdot4\cdot3\cdot2 = 6!$ ways to get a different number on each die, and $6^5$ possible 5 dice rolls so the probability that each die is different is:
	$$\frac{6!}{6^5} \approx 9.26\%$$
	\end{proof}
	\ee	
	
	\item[18.] Amy is picking her fall term classes. She needs to fill three time slots, and there
are 20 distinct courses to choose from, including probability 101, 102, and 103.
She will pick her classes at random so that all outcomes are equally likely.
	\be
	\item[(a)] What is the probability that she will get probability 101?
	\begin{proof}[Solution]
	If she gets into probability 101, then there $\binom{19}{2}$ ways to complete her schedule and $\binom{20}{3}$ possible schedules. So the probability is:
	$$\frac{\binom{19}{2}}{\binom{20}{3}} = .15 = 15\%$$
	\end{proof}
	\item[(b)] What is the probability that she will get probability 101 and
Probability 102?
	\begin{proof}[Solution]
	If she gets into probability 101 and 102, then there $\binom{18}{1}$ ways to complete her schedule and $\binom{20}{3}$ possible schedules. So the probability is:
	$$\frac{\binom{18}{1}}{\binom{20}{3}} \approx .016 = 1.6\%$$
	\end{proof}
	\item[(c)] What is the probability she will get all three probability courses?
	\begin{proof}[Solution]
	If she gets into probability 101, 102 and 103, then there $\binom{17}{0} = 1$ ways to complete her schedule and $\binom{20}{3}$ possible schedules. So the probability is:
	$$\frac{1}{\binom{20}{3}} \approx .00087 = 0.087\%$$
	\end{proof}
	\ee
		
	\item[20.] Suppose $P(A) = 0.40, P(B) = 0.60$, and $P(A \cup B) = 0.80$. Find
	\be
	\item[(a)] $P(A^c \cap B^c)$.
	\begin{proof}[Solution]
	First, $P(A^c \cap B^c) = P\big((A\cup B)^c\big)$, then consider that 
	$$P\big((A\cup B)^c\big) + P\big(A\cup B\big) = 1$$
	Then:
	\begin{align*}
	P\big((A\cup B)^c\big) &= 1 - P\big(A\cup B\big)  \\
	&= 1 - .8
	\end{align*}
	$$\Rightarrow P(A^c \cap B^c) = .2$$
	\end{proof}
	\item[(b)] $P(A \cap B)$.
	\begin{proof}[Solution]
	To find this, we'll use:
	\begin{align*}
	P(A \cup B) = P(A) + P(B) - P(A \cap B) \\
	.8 = .6 + .4 - P(A \cap B) \\
	P(A \cap B) = 1 - .8 
	\end{align*}
	$$\Rightarrow P(A \cap B) = .2$$
	\end{proof}
	\item[(c)] $P(A^c \cup B^c)$.
	\begin{proof}[Solution]
	First, $P(A^c \cup B^c) = P\big((A\cap B)^c\big)$, and $P\big((A\cap B)^c\big) + P(A \cap B) = 1$, so $$P\big((A\cap B)^c\big) = 1 - P(A \cap B)$$
	$P(A \cap B)$ is found in Part~(b), and:
	$$\Rightarrow P(A^c \cup B^c) = .8$$
	\end{proof}
	\ee
	
	\item[21.] Suppose A and B are mutually exclusive, with P(A) = 0.30 and P(B) =
0.60. Find the probability that
	\be
	\item[(a)] At least one of the two events occurs
	\begin{proof}[Solution]
	The probability of least one event occurring is 
	$$P(A \cup B) = P(A) + P(B) = 0.3 + 0.6 = 0.9$$
	\end{proof}
	\item[(b)] Both of the events occur
	\begin{proof}[Solution]
	The events are mutually exclusive, so the probability of both events occurring is
	$$P(A \cap B) = 0$$
	\end{proof}
	\item[(c)] Neither event occurs
	\begin{proof}[Solution]
	The Probability that no events occur
	$$P\big(A^c \cap B^c\big) = P\big((A \cup B)^c\big)$$
	$$\Rightarrow P\big((A\cup B)^c\big) = 1 - P(A\cup B)$$
	$$\Rightarrow P\big((A\cup B)^c\big) = .1$$
	\end{proof}
	\item[(d)] Exactly one of the two events occur
	\begin{proof}[Solution]
	The events are mutually exclusive, so the probability that exactly one event occurs is equal to the probability of at least one happening
	$$P(A \cup B) = .9$$
	\end{proof}
	\ee
		
	\item[22.] Suppose $P(A \cup B) = 0.6$ and $P(A \cup B^c) = 0.8$. Find $P(A)$.
	\begin{proof}[Solution]
	To find $P(A)$, we'll use:
	\begin{align*}
	P(A) + P(A^c) = 1 \\
	P(A \cup B) + P\big((A \cup B)^c\big) = 1 \\
	P(A \cup B^c) + P\big((A \cup B^c)^c\big) = 1
	\end{align*}
	Using our conditions, we have that:
	\begin{align*}
	P\big((A \cup B)^c\big) = 1 - P(A \cup B) = .4 \\
	P\big((A \cup B^c)^c\big) = 1  - P(A \cup B^c) = .2 \\
	P(A) = 1 - P(A^c)
	\end{align*}
	To find $P(A^c)$, we'll consider:
	\begin{align*}
	A^c &= (A\cup B)^c \cup \big(B \cap (A \cap B)^c\big) \\
	&= (A\cup B)^c \cup \big(B \cap (A^c \cup B^c)\big) \\
	&= (A\cup B)^c \cup \big(B \cap A^c \cup B \cap B^c\big) \\
	&= (A\cup B)^c \cup \big(B \cap A^c \cup \emptyset\big) \\
	&= (A\cup B)^c \cup \big(B \cap A^c\big)
	\end{align*}
	The sets $(A\cup B)^c$ and $\big(B \cap A^c\big)$ 
	are disjoint, so:
	$$P(A^c) = P\Big((A\cup B)^c \cup \big(B \cap A^c\big)\Big) = P\big((A\cup B)^c\big) + P\Big(\big(B \cap A^c\big)\Big)$$
	$\big(B \cap A^c\big) = (A \cup B^c)^c$, so:
	$$P(A^c) = .4 + .2 = .6$$
	And finally:
	$$P(A) = 1 - P(A^c) = 1 - .6 = .4$$
	\end{proof}
	
	\item[26.] See the assignment of probabilities to the Venn diagram in Figure 1.4. Find
the following:
	\be
	\item[(a)] $P$(No events occur): h
	\item[(b)] $P$(Exactly one event occurs): a, c, f
	\item[(c)] $P$(Exactly two events occur): b, d, e
	\item[(d)] $P$(Exactly three events occur): g
	\item[(e)] $P$(At least one event occurs): a, c, f, b, d, e, g
	\item[(f)] $P$(At least two events occur): b, d, e, g
	\item[(g)] $P$(At most one event occurs): h, a, c, f
	\item[(h)] $P$(At most two events occur): h, a, c, f, b, d, e
	\ee
		
\ee
\noindent\makebox[\linewidth]{\rule{\paperwidth}{0.4pt}}
	
	\end{large}
\end{document}
