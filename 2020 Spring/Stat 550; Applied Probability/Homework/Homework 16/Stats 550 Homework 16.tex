\documentclass{article}
\usepackage{amsmath}
\usepackage{amssymb}
\usepackage{bm}
\usepackage{amsthm}
\usepackage{enumerate}
\usepackage{graphicx}
\usepackage{psfrag}
\usepackage{color}
\usepackage{url}
\usepackage{listings}
\usepackage{xcolor}
\usepackage{tikz}
\usetikzlibrary{positioning}
\tikzset{main node/.style={circle,fill=gray!20,draw,minimum size=.5cm,inner sep=0pt},}

\definecolor{codegreen}{rgb}{0,0.5,0}
\definecolor{codewhite}{rgb}{1,1,1}
\definecolor{codegray}{rgb}{0.5,0.5,0.5}
\definecolor{codepurple}{rgb}{0.58,0,0.82}
\definecolor{codeblack}{rgb}{0,0,0}
\definecolor{codeorange}{rgb}{0.8,0.4,0}

\lstdefinestyle{mystyle}{
    backgroundcolor=\color{codewhite},   
    commentstyle=\color{codegray},
    keywordstyle=\color{codegreen},
    numberstyle=\color{codegray},
    stringstyle=\color{codeorange},
    basicstyle=\ttfamily ,
    breakatwhitespace=false,         
    breaklines=true,                 
    captionpos=b,                    
    keepspaces=true,                 
    numbers=left,                    
    numbersep=5pt,                  
    showspaces=false,                
    showstringspaces=false,
    showtabs=false,                  
    tabsize=4
}
\lstset{style=mystyle}


\setlength{\hoffset}{-1in}
\addtolength{\textwidth}{1.5in}
\setlength{\voffset}{-1in}
\addtolength{\textheight}{1.5in}
\newcommand{\be}{\begin{enumerate}}
\newcommand{\ee}{\end{enumerate}}
\newcommand{\BigO}[1]{\ensuremath\mathcal{O}\left(#1\right)}
\newcommand{\il}[1]{\lstinline!#1!}
\newcommand{\gnorm}[1]{\left|\left|#1\right|\right|}
\newcommand{\abs}[1]{\left|#1\right|}
\newcommand{\parens}[1]{\left(#1\right)}
\newcommand{\bracks}[1]{\left\{#1\right\}}
\newcommand{\sqbracks}[1]{\left[#1\right]}
\newcommand{\vep}{\varepsilon}
\newcommand{\ceiling}[1]{\left\lceil#1\right\rceil}
\newcommand{\R}{\mathbb{R}}
\newcommand{\N}{\mathbb{N}}
\newcommand{\distrib}[2]{\text{#1}\left(#2\right)}

\begin{document}
	\begin{center}
		\textbf{Spring 2020, Stats 550:\ Homework 16} \\
		\textbf{Due:\ Thursday, April 23rd, 2020} \\
		\textbf{Joseph Diaz: 819947915}
	\end{center}
\noindent\makebox[\linewidth]{\rule{\paperwidth}{0.4pt}}
\section*{Chapter 6}
\be
	\item[23.] Let $X \sim \distrib{Unif}{0,1)}$ Find $E\sqbracks{e^X}$ two ways:
	\be[(a)] 
		\item By finding the density of $e^X$ and then computing the expectation with
respect to that distribution.
		\begin{proof}[Solution]
		First, let $Y = e^X \sim \distrib{Unif}{1,e}$; we know that 
		$$P\parens{X \leq x} = F(x) = \frac{x-0}{1-0} = x$$
		then
		$$P\parens{Y \leq y} = P\parens{e^X \leq y} = P\parens{X \leq \ln y}$$
		Now let $H(y) = F\parens{\ln y} = \ln y$ be the cdf of$Y$, then $h(y) = H'(y) = \frac{1}{y}$ is the pdf of $Y$.\\
		Finally, $E\sqbracks{Y} = E\sqbracks{e^X}$ is given by:
		$$E\sqbracks{e^X} = E\sqbracks{Y} = \int_1^e yh(y)\ dy = \int_1^e 1\ dy = e - 1$$
		\end{proof}
		
		\item By using the law of the unconscious statistician.
		\begin{proof}[Solution]
		The formula we are using is given by
		$$E\sqbracks{g\parens{X}} = \int_a^b g(x)f(x)\ dx$$
		for a continuous random variable $X$ on $(a, b)$, with pdf $f(x)$.\\
		In our case we have:
		$$E\sqbracks{e^X} = \int_0^1 e^x\ dx = e^x \Big|_0^1 = e-1$$
		\end{proof}
		
	\ee
	
	\item[24.] The density of a random variable X is given by
	$$f(x) = \left\{\begin{array}{rl}
	3x^2; & 0 < x < 1 \\
	0; & \text{else}
	\end{array}\right.$$
	Let $Y = e^X$.
	\be[(a)]
		\item Find the density function of $Y$.
		\begin{proof}[Solution]
		First we find that the cdf of $X$ is
		$$F(x) = \int_0^x f(t)\ dt = \int_0^x 3t^2\ dt = t^3\Big|_0^x = x^3,\ 0 < x < 1$$
		Next, $P\parens{X \leq x} = F(x)$; and
		$$P\parens{Y \leq y} = P\parens{e^X \leq y} = P\parens{X \leq \ln y}$$
		Now, let $G(y) = F\parens{\ln y} = \ln^3 y$ be the cdf of $Y$, so $g(y) = G'(y) = \frac{3\ln^2 y}{y}$ is the pdf of $Y$, for $1 < y < e$.
		\end{proof}
		
		\item Find $E\sqbracks{Y}$ two ways:
		\be[(i)]
			\item Using the density of $Y$.
			\begin{proof}[Solution]
			With the pdf of $Y$ being $g(y) = \frac{3\ln^2 y}{y}$, the expected value of $Y$ is:
			\begin{align*}
			E\sqbracks{Y} &= \int_1^e yg(y)\ dy\\
			&= \int_1^e 3\ln^2 y\ dy \\
			&= 3\int_1^e \ln^2 y\ dy \\
			&= 3\parens{y\ln^2 y - 2\int \ln y\ dy}\Big|_1^e \\
			&= 3 \parens{y\ln^2 y - 2y\ln y + 2y}\Big|_1^e \\
			&= 3 \sqbracks{\parens{e - 2e + 2e}- \parens{2}} \\
			& = 3e-6 
			\end{align*}
			\end{proof}
					
			\item Using the density of $X$.
			\begin{proof}[Solution]
			To find this, we'll use the Law of the Unconscious Statistician:
			\begin{align*}
			E\sqbracks{Y} &= E\sqbracks{e^X} \\
			&= \int_0^1 e^x f(x)\ dx \\
			&= 3\int_0^1 x^2 e^x\ dx \\
			&= 3\parens{x^2 e^x - 2\int xe^x\ dx}\Big|_0^1 \\
			&= 3\parens{x^2 e^x - 2\parens{xe^x - \int e^x\ dx}}\Big|_0^1 \\
			&= 3\parens{x^2 e^x - 2xe^x + 2e^x}\Big|_0^1 \\
			&= 3\sqbracks{\parens{e - 2e + 2e}-\parens{2}} \\
			&= 3e-6
			\end{align*}
			\end{proof}
		\ee
	\ee
	
	\item[27.] Let $X \sim \distrib{Unif}{a, b}$. Suppose $Y$ is a linear function of $X$. That is $Y = mX +
n$. Where $m$ and $n$ are constants. Assume also that $m > 0$. Show that $Y$ is
uniformly distributed on the interval $\parens{ma + n, mb + n}$.
	\begin{proof}[Solution]
	$X$ has a uniform distribution, so
	$$P\parens{X \leq x} = F(x) = \frac{x-a}{b-a}$$
	for $Y$ we have
	$$P\parens{Y leq y} = P\parens{mX+n \leq y} = P\parens{X \leq \frac{y-n}{m}} = F\parens{\frac{y-n}{m}}$$
	Let $G(y) = F\parens{\frac{y-n}{m}}$ be the cdf of $Y$, then the pdf of $Y$ is $g(y) = G'(y)$:
	$$G'(y) = \parens{\frac{\frac{y-n}{m} - a}{b-a}}' = \parens{\frac{y - (n+am)}{m(b-a)}}' = \frac{1}{m(b-a)}= \frac{1}{mb - ma}$$
	$$\Rightarrow = \frac{1}{mb + n - ma -n} = \frac{1}{(mb+n) - (ma + n)} = g(y)$$
	and given that $g(y)$ is the pdf for $Y$, this implies that $Y \sim \distrib{Unif}{ma+n, mb+n}$.
	
	\end{proof}
	
\ee
\noindent\makebox[\linewidth]{\rule{\paperwidth}{0.4pt}}
	
\end{document}
