\documentclass{article}
\usepackage{amsmath}
\usepackage{amssymb}
\usepackage{bm}
\usepackage{amsthm}
\usepackage{enumerate}
\usepackage{graphicx}
\usepackage{psfrag}
\usepackage{color}
\usepackage{url}
\usepackage{listings}
\usepackage{xcolor}
\usepackage{tikz}
\usetikzlibrary{positioning}
\tikzset{main node/.style={circle,fill=gray!20,draw,minimum size=.5cm,inner sep=0pt},}

\definecolor{codegreen}{rgb}{0,0.5,0}
\definecolor{codewhite}{rgb}{1,1,1}
\definecolor{codegray}{rgb}{0.5,0.5,0.5}
\definecolor{codepurple}{rgb}{0.58,0,0.82}
\definecolor{codeblack}{rgb}{0,0,0}
\definecolor{codeorange}{rgb}{0.8,0.4,0}

\lstdefinestyle{mystyle}{
    backgroundcolor=\color{codewhite},   
    commentstyle=\color{codegray},
    keywordstyle=\color{codegreen},
    numberstyle=\color{codegray},
    stringstyle=\color{codeorange},
    basicstyle=\ttfamily ,
    breakatwhitespace=false,         
    breaklines=true,                 
    captionpos=b,                    
    keepspaces=true,                 
    numbers=left,                    
    numbersep=5pt,                  
    showspaces=false,                
    showstringspaces=false,
    showtabs=false,                  
    tabsize=4
}
\lstset{style=mystyle}


\setlength{\hoffset}{-1in}
\addtolength{\textwidth}{1.5in}
\setlength{\voffset}{-1in}
\addtolength{\textheight}{1.5in}
\newcommand{\be}{\begin{enumerate}}
\newcommand{\ee}{\end{enumerate}}
\newcommand{\BigO}[1]{\ensuremath\mathcal{O}\left(#1\right)}
\newcommand{\il}[1]{\lstinline!#1!}
\newcommand{\gnorm}[1]{\left|\left|#1\right|\right|}


\begin{document}
	\begin{center}
		\textbf{Spring 2020, Stat 550:\ Homework 10} \\
		\textbf{Due:\ Tuesday, March 26th, 2020} \\
		\textbf{Joseph Diaz: 819947915}
	\end{center}
\noindent\makebox[\linewidth]{\rule{\paperwidth}{0.4pt}}
\section*{Chapter 5}
\be 
	\item[1.] Tom is playing poker with his friends. What is the expected number of hands
it will take before he gets a full house?
\begin{proof}[Solution]
Let $X$ be number of hands before  full house is drawn.
The probability of getting a full house is 
$$p = \frac{\binom{13}{1}\binom{4}{3}\binom{12}{1}\binom{4}{2}}{\binom{52}{5}} = .1441\%$$
So the expected number of trials before Tom gets a full house is
$$E\left[X\right] = \frac{1}{p} = \frac{\binom{52}{5}}{\binom{13}{1}\binom{4}{3}\binom{12}{1}\binom{4}{2}} = 694.1\bar{6}$$
So the expected number of hands before a full house is 694.
\end{proof}
	
	\item[2.] What is the probability that it takes an even number of die rolls to get a four?
	\begin{proof}[Solution]
	Let $A$ be the event that a 4 is rolled on a six-sided die, so $P(A) = \frac{1}{6},\ P\left(A^c\right) = \frac{5}{6}$. We are interested in the probability that it takes an even number of throws for a 4 to be rolled. Let $X$ be the number of throws until a 4 is rolled, so the probability we are after is given by:
	$$P\left(X=2k\right) = P\left(A^c\right)^{2k-1}P(A),\ \forall k \in \{1,2, \cdots, n\}$$
	As the actual even number of throws that it takes is unimportant, we can express this as:
	$$P(X=2) + P(X=4) + \cdots + P(X=2n)$$
	Which, using sigma notation becomes:
	\begin{align*}
	\sum_{k=1}^n P\left(A^c\right)^{2k-1}P(A) &= \sum_{k=1}^n\left(\frac{5}{6}\right)^{2k-1}\left(\frac{1}{6}\right)\\
	&= \sum_{k=1}^n\left(\frac{5}{6}\right)^{2k}\left(\frac{6}{5}\right)\left(\frac{1}{6}\right)\\
	&= \frac{1}{5}\sum_{k=1}^n\left(\frac{5}{6}\right)^{2k}\\
	&= \frac{1}{5}\sum_{k=1}^n\left(\frac{25}{36}\right)^{k}\\
	&= \frac{1}{5}\left[\sum_{k=0}^n\left(\frac{25}{36}\right)^{k}-1\right]\ (\text{The sum is a geometric series.})\\
	&= \frac{1}{5}\left[\frac{1-\left(\frac{25}{36}\right)^{n+1}}{1-\frac{25}{36}}-1\right]\\
	&= \frac{1}{5}\left[\frac{1-\left(\frac{25}{36}\right)^{n+1}- 1 + \frac{25}{36}}{1-\frac{25}{36}}\right]\\
	&= \frac{1}{5}\left[\frac{\frac{25}{36}-\left(\frac{25}{36}\right)\left(\frac{25}{36}\right)^{n}}{1-\frac{25}{36}}\right]\\
	&= \frac{5}{36}\left[\frac{1-\left(\frac{25}{36}\right)^{n}}{1-\frac{25}{36}}\right]\\
	\end{align*}
	So
	$$\sum_{k=1}^n P\left(A^c\right)^{2k-1}P(A) = \frac{5}{36}\left[\frac{1-\left(\frac{25}{36}\right)^{n}}{1-\frac{25}{36}}\right]$$
	and we now have an expression in terms of $n$ for the probability that it takes an even number of rolls to get a 4. For $n \in \{1,\cdots, 10\}$, the probabilities are:
	$$\left[13.9\%,\ 23.5\%,\ 30.2\%,\ 34.9\%,\ 38.1\%,\ 40.4\%,\
 41.9\%,\ 42.9\%,\  43.7\%,\ 44.3\%\right]$$
 And if we let $n \to \infty$, then our probability becomes:
 $$\lim_{n\to\infty}\frac{5}{36}\left[\frac{1-\left(\frac{25}{36}\right)^{n}}{1-\frac{25}{36}}\right] = \frac{5}{11} = 45.\bar{45}\%$$
	\end{proof}
	
	\item[4.] A manufacturing process produces components which have a 1\% chance of
being defective. Successive components are independent.
	\be[(a)] 
		\item Find the probability that it takes exactly 110 components to be produced
before a defective one occurs.\\
Let $X$ be the number of components produced until a defective one is made, with $X \sim \text{Geom}(.01)$. The probability that it takes exactly 110 components to produce a defective one is:
$$P(X=110) = (1-.01)^{110-1}\cdot0.01 = .003344 \approx .33\%$$
		
		\item Find the probability that it takes at least 110 components to be produced
before a defective one occurs.\\
With the same $X$ as ~(a), the probability we're interested in is:
$$P(X \geq 110) = P(X > 109) = (1-.01)^{109} = .3344 \approx 33.44\%$$
		
		\item What is the expected number of components that will be produced before
a defective one occurs?\\
Given that $X$ has a Geometric Distribution, so
$$E\left[X\right] =\frac{1}{p} = \frac{1}{.01} = 100$$
So at that rate defective components, you would expect to have to produce 100 components before you get a defective one.
		
	\ee
	
	\item[5.] Danny is applying to college and sending out many applications. He estimates
there is a 25\% chance that an application will be successful. How many applications
should he send out so that the probability of at least one acceptance is
at least 95\%?
\begin{proof}[Solution]
We are interested in the number of applications that must be sent out by Daniel to have at least a 95\% chance of at least one successful application, where each individual application has a 25\% chance of success. In other words, we want to find $k$ such that:
$$\sum_{i=1}^k (1-.25)^{i-1}\cdot.25^i \geq .95$$
where $i$ represents a successful application on the $i^{\text{th}}$ trial. By testing values of $k$, we find that $k = 11$ satisfies:
$$\sum_{i=1}^{11} (1-.25)^{i-1}\cdot.25^i \geq .95$$
So Danny must send out 11 applications for a 95\% chance of at least one acceptance.
\end{proof}
	
\ee

\noindent\makebox[\linewidth]{\rule{\paperwidth}{0.4pt}}
	
\end{document}
