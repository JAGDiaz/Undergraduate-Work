\documentclass{article}
\usepackage{amsmath}
\usepackage{amssymb}
\usepackage{bm}
\usepackage{amsthm}
\usepackage{enumerate}
\usepackage{graphicx}
\usepackage{psfrag}
\usepackage{color}
\usepackage{url}

\setlength{\hoffset}{-0.5in}
\addtolength{\textwidth}{1.0in}
\setlength{\voffset}{-0.5in}
\addtolength{\textheight}{1.0in}
%\newcommand{\deg}{\text{deg}}

\begin{document}
	\begin{large}
	\textbf{Joseph Diaz}
	\begin{center}
		\textbf{Spring 2020, Stats 550:\ Homework 1} \\
		\textbf{Due:\ Tuesday, February 4th, 2020} \\
	\end{center}
	\noindent\makebox[\linewidth]{\rule{\paperwidth}{0.4pt}}
	\section*{Chapter 1}
	\begin{enumerate}[1.]
		\item Your friend was sick and unable to make today’s class. Explain to your friend,
using your own words, the meaning of the terms (i) random experiment, (ii)
sample space, (iii) event, and (iv) random variable.
		\begin{enumerate}[i.]
			\item A \textit{Random Experiment} is an activity or process whose outcome is uncertain.
		
			\item The \textit{Sample Space} is the set of all possible outcomes from a Random Experiment.
		
			\item An \textit{Event} is particular sequence of outcomes from a Random Experiment. Any given Event is a subset of the Sample Space of the associated Random Experiment.
		
			\item A \textit{Random Variable} is a way of expressing the possible outcomes of any single Random Experiment numerically.
			
		\end{enumerate}
		

		\textit{For the following problems 1.2–1.5, identify (i) the random experiment, (ii)
sample space, (iii) event, and (iv) random variable. Express the probability
in question in terms of the defined random variable, but do not compute the
probability.	}
		
		\item Roll four dice. Consider the probability of getting all fives.
		\begin{enumerate}[i.]
			\item Random Experiment: Rolling all 4 dice.
		
			\item Sample space: The outcome after rolling all four dice:$$S = \{(a,b,c,d) :\ 1 \leq a,b,c,d \leq 6\}$$
		
			\item Event: Getting all fives after rolling all four dice:$$E = (5,5,5,5)$$
		
			\item Random Variable: In the rolling of 4 dice, let $X$ be the number of fives rolled. Then the probability of rolling four fives is: $$P(X = 4)$$
		
		\end{enumerate}
		
		\item A pizza shop offers three toppings: pineapple, peppers, and pepperoni. A
pizza can have 0, 1, 2, or 3 toppings. Consider the probability that a random
customer asks for two toppings.	
		\begin{enumerate}[i.]
			\item Random Experiment: A customer coming in and ordering a pizza with 1, 2, 3, or no toppings.
		
			\item Sample Space: All possible pizzas that customers can order.
		
			\item Event: Having a customer come in and order a pizza that has 2 toppings.
		
			\item Random Variable: Let $X$ be the number of topping that a customer wants on their pizza, then the probability of a customer ordering a pizza with 2 toppings is:
			$$P(X=2)$$
		
		\end{enumerate}		
		
		\item Bored one day, you decide to play the video game Angry Birds until you win.
Every time you lose, you start over. Consider the probability that you win in
less than 1000 tries.
		\begin{enumerate}[i.]
			\item Random Experiment: Each round of game play.
		
			\item Sample Space: Winning or losing a round of game play.
		
			\item Event: Winning in fewer than 1000 tries.
		
			\item Random Variable: Let $X$ be the number of losses before the first win, then the probability of winning in fewer than 1000 tries is $$P(X < 1000)$$
 		
		\end{enumerate}		
		
		\item In Angel’s garden, there is a $3$\% chance that a tomato will be bad. Angel
harvests 100 tomatoes and wants to know the probability that at most five
tomatoes are bad.
		\begin{enumerate}[i.]
			\item Random Experiment: Checking each tomato to determine if it is rotten or not.
		
			\item Sample Space: The state of rotten or not for each of the 100 harvested tomatoes.
		
			\item Event: Having at most 5 rotten tomatoes. 
		
			\item Random Variable: Let $X$ be the number of rotten tomatoes, then the probability of having at most 5 rotten tomatoes is:
			$$P(X \leq 5)$$
		
		\end{enumerate}
		
		\item In two dice rolls, let $X$ be the outcome of the first die, and Y the outcome of
the second die. Then $X+Y$ is the sum of the two dice. Describe the following
events in terms of simple outcomes of the random experiment:
		\begin{enumerate}[(a)]
			\item $\{X + Y = 4\}$\\
			The simple events are rolling a 2 for each die, rolling a 3 and then a 1, or rolling a 1 and then a 3.
			
			\item $\{X + Y = 9\}$\\
			The simple events are rolling a 3 and then a 6, rolling a 6 and then a 3, or the reverse of the previous 2 simple events.
		
			\item $\{Y = 3\}$\\
			The simple event is rolling a 3 on the second die.
			\item $\{X = Y\}$\\
			The simple event is rolling the first die and then getting the same thing after rolling the second die.
			
			\item $\{X > 2Y\}$\\
			The simple event is rolling the first die and getting a number that is less than half of the first on the second.
		
		\end{enumerate}
		
		\item A bag contains r red and b blue balls. You reach into the bag and take k balls.
Let R be the number of red balls you take. Let B be the number of blue balls.
Express the following events in terms of the random variables R and B:
		\begin{enumerate}
			\item You pick no red balls.\\
			The Random Variable expression is $$R = 0$$
		
			\item You pick one red and two blue balls.\\
			The Random Variable expression is $$R = 1 \text{ and } B=2$$
		
			\item You pick four balls.\\
			The Random Variable expression is $$R + B = 4$$
		
			\item You pick twice as many red balls as blue balls.\\
			The Random Variable expression is $$R = 2B$$
		\end{enumerate}
		
		\item A couple plans to continue having children until they have a girl or until
they have six children, whichever comes first. Describe a sample space and
a reasonable random variable for this random experiment.\\\\
The Sample Space is $$S = \{(0,1), (1,1), (2,1), (3,1), (4,1), (5,1), (6,0)\}$$ where the first number in each ordered pair is the number of boys and the second is the number of girls. An appropriate Random Variable would be to let $B$ be the number of boys among the children. Because even though were interested in the cases in which a daughter is born or we get to six children, calculating the complement ($P(B \leq 6)$) would be easier. 
		
	\end{enumerate}
\noindent\makebox[\linewidth]{\rule{\paperwidth}{0.4pt}}
	
	\end{large}
\end{document}
