\documentclass{article}
\usepackage{amsmath}
\usepackage{amssymb}
\usepackage{bm}
\usepackage{amsthm}
\usepackage{enumerate}
\usepackage{graphicx}
\usepackage{psfrag}
\usepackage{color}
\usepackage{url}
\usepackage{listings}
\usepackage{xcolor}
\usepackage{tikz}
\usetikzlibrary{positioning}
\tikzset{main node/.style={circle,fill=gray!20,draw,minimum size=.5cm,inner sep=0pt},}

\definecolor{codegreen}{rgb}{0,0.5,0}
\definecolor{codewhite}{rgb}{1,1,1}
\definecolor{codegray}{rgb}{0.5,0.5,0.5}
\definecolor{codepurple}{rgb}{0.58,0,0.82}
\definecolor{codeblack}{rgb}{0,0,0}
\definecolor{codeorange}{rgb}{0.8,0.4,0}

\lstdefinestyle{mystyle}{
    backgroundcolor=\color{codewhite},   
    commentstyle=\color{codegray},
    keywordstyle=\color{codegreen},
    numberstyle=\color{codegray},
    stringstyle=\color{codeorange},
    basicstyle=\ttfamily ,
    breakatwhitespace=false,         
    breaklines=true,                 
    captionpos=b,                    
    keepspaces=true,                 
    numbers=left,                    
    numbersep=5pt,                  
    showspaces=false,                
    showstringspaces=false,
    showtabs=false,                  
    tabsize=4
}
\lstset{style=mystyle}


\setlength{\hoffset}{-1in}
\addtolength{\textwidth}{1.5in}
\setlength{\voffset}{-1in}
\addtolength{\textheight}{1.5in}
\newcommand{\be}{\begin{enumerate}}
\newcommand{\ee}{\end{enumerate}}
\newcommand{\BigO}[1]{\ensuremath\mathcal{O}\left(#1\right)}
\newcommand{\il}[1]{\lstinline!#1!}
\newcommand{\gnorm}[1]{\left|\left|#1\right|\right|}


\begin{document}
	\begin{center}
		\textbf{Spring 2020, Stat 550:\ Homework 9} \\
		\textbf{Due:\ Tuesday, March 24th, 2020} \\
		\textbf{Joseph Diaz: 819947915}
	\end{center}
\noindent\makebox[\linewidth]{\rule{\paperwidth}{0.4pt}}
\section*{Chapter 3}
\be 
	\item[31.] The number of eggs a chicken hatches is a Poisson random variable. The
probability that the chicken hatches no eggs is 0.10. What is the probability
that she hatches at least two eggs?
\begin{proof}[Solution]
Let $X$ represent the number of eggs a chicken hatches. It is given that $X$ has a Poisson distribution and $P(X=0) = .1$. With this, we may deduce $\lambda$ to be:
$$P(X=0) = .1 = \frac{e^{-\lambda}\lambda^0}{0!} \implies \frac{1}{10} = e^{-\lambda} $$
$$\Rightarrow \lambda = \ln(10)$$
With this, the probability that a chicken hatches at least 2 eggs is given by:
\begin{align*}
P(X\geq 2) &= 1 - P(X < 2)\\
&= 1 - \sum_{k=0}^1 \frac{e^{-\ln(10)}\ln^k(10)}{k!}\\
&= 1 - .1 - .2303\\
&= .66974 \approx 67\%
\end{align*}
So the probability that a chicken hatches at least 2 eggs is about 67\%.
\end{proof}
	
	\item[33.] Cars pass a busy intersection at a rate of approximately 16 cars per minute.
What is the probability that at least 1000 cars will cross the intersection in the
next hour? (Hint: What is the rate per hour?)
\begin{proof}[Solution]
Let $X$ be the number of cars that pass the intersection. It is given that expected number of cars per minute is 16, so the expected number of cars per hour is $\lambda = 16\cdot 60 = 960$. So the probability that at least 1000 cars pass the intersection in an hour is given by:
$$P(X \geq 1000)$$
Which, when approximated with R, is
$$P(X \geq 1000) \approx 9.621202\%$$

\end{proof}
	
	\item[36.] If you take the red pill, the number of colds you get next winter will have
a Poisson distribution with $\lambda = 1$. If you take the blue pill, the number of
colds will have a Poisson distribution with $\lambda = 4$. Each pill is equally likely.
Suppose you get three colds next winter. What is the probability you took the
blue pill?
\begin{proof}[Solution]
Let $R$ and $B$ be the events that the red and blue pills are taken with $P(R) = P(R^c) = P(B) = \frac{1}{2}$, and $X$ be the number of colds next winter. Then the probability we are interested in is given by:
\begin{align*}
P(B|X=3) &= \frac{P(X=3|B)P(B)}{P(X=3)}\\
&= \frac{P(X=3|B)P(B)}{P(X=3|R)P(R) + P(X=3|B)P(B)}
\end{align*}
$X$ has a Poisson Distribution with $\lambda_R = 1$ and $\lambda_B  = 4$, so:
\begin{align*}
P(B|X=3) &= \frac{P(X=3|B)P(B)}{P(X=3|R)P(R) + P(X=3|B)P(B)}\\
&= \frac{\frac{e^{-4}4^3}{3!}\cdot\frac{1}{2}}{\frac{e^{-1}1^3}{3!}\cdot\frac{1}{2} + \frac{e^{-4}4^3}{3!}\cdot\frac{1}{2}}\\
&= \frac{\frac{e^{-4}4^3}{3!}}{\frac{e^{-1} + e^{-4}4^3}{3!}}\\
&= \frac{e^{-4}4^3}{e^{-1} + e^{-4}4^3}\\
&= .76113 \approx 76.11\%
\end{align*}
So the probability that you took the blue pill given that you had the cold 3 times is 
$$P(B|X=3) \approx 76.11\%$$
\end{proof}
	
	\item[37.] A physicist estimated that the probability of a U.S. nickel landing on its edge
is one in 6000. Suppose a nickel is flipped 10,000 times. Let $X$ be the number
of times it lands on its edge. Find the probability that $X$ is between one and
three using
	\be[(a)] 
		\item The exact distribution of $X$.\\
		As $X$ only records the number of ``edge landings'', each individual flip is a Bernoulli trial with $p = \frac{1}{6000}$; and with 10,000 nickel flips, $X \sim \text{Binom}\left(10000, \frac{1}{6000}\right)$.
		
		\item An approximate distribution of $X$.\\
		Given that the probability of an ``edge landing'' is $\frac{1}{6000}$, then for 6,000 nickel flips we have $X \sim \text{Pois}(1)$. For 10,000 flips, $X \sim \text{Pois}(1.\bar{6})$.
	\ee
	
	\item[45.] Simulate a Prussian soldier’s death by horse kick as in table 3.8. Create a
histogram based on 10,000 repetitions. Compare to a histogram of the Poisson
distribution using the command \il{rpois(10000,0.61)}.\\
The following code simulates the desired experiment:
\begin{lstlisting}[language=R]
PrussianHorse=function(n)
{
  x = 1:5
  sim = replicate(n, mean(rpois(x, .61)))
  return(sim)
}

k = seq(0,3,.1)

hist(PrussianHorse(10000), breaks = k)
\end{lstlisting}
\pagebreak
which yields the histogram.
\begin{figure}[h]
	\centering
  	\includegraphics[width=\linewidth]{"PrussianHorse".jpg}
	\end{figure}
	
\ee

\noindent\makebox[\linewidth]{\rule{\paperwidth}{0.4pt}}
	
\end{document}
