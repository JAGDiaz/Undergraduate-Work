% Options for packages loaded elsewhere
\PassOptionsToPackage{unicode}{hyperref}
\PassOptionsToPackage{hyphens}{url}
%
\documentclass[
]{article}
\usepackage{lmodern}
\usepackage{amssymb,amsmath}
\usepackage{ifxetex,ifluatex}
\ifnum 0\ifxetex 1\fi\ifluatex 1\fi=0 % if pdftex
  \usepackage[T1]{fontenc}
  \usepackage[utf8]{inputenc}
  \usepackage{textcomp} % provide euro and other symbols
\else % if luatex or xetex
  \usepackage{unicode-math}
  \defaultfontfeatures{Scale=MatchLowercase}
  \defaultfontfeatures[\rmfamily]{Ligatures=TeX,Scale=1}
\fi
% Use upquote if available, for straight quotes in verbatim environments
\IfFileExists{upquote.sty}{\usepackage{upquote}}{}
\IfFileExists{microtype.sty}{% use microtype if available
  \usepackage[]{microtype}
  \UseMicrotypeSet[protrusion]{basicmath} % disable protrusion for tt fonts
}{}
\makeatletter
\@ifundefined{KOMAClassName}{% if non-KOMA class
  \IfFileExists{parskip.sty}{%
    \usepackage{parskip}
  }{% else
    \setlength{\parindent}{0pt}
    \setlength{\parskip}{6pt plus 2pt minus 1pt}}
}{% if KOMA class
  \KOMAoptions{parskip=half}}
\makeatother
\usepackage{xcolor}
\IfFileExists{xurl.sty}{\usepackage{xurl}}{} % add URL line breaks if available
\IfFileExists{bookmark.sty}{\usepackage{bookmark}}{\usepackage{hyperref}}
\hypersetup{
  pdftitle={Lab 1},
  pdfauthor={Joseph Diaz},
  hidelinks,
  pdfcreator={LaTeX via pandoc}}
\urlstyle{same} % disable monospaced font for URLs
\usepackage[margin=1in]{geometry}
\usepackage{color}
\usepackage{fancyvrb}
\newcommand{\VerbBar}{|}
\newcommand{\VERB}{\Verb[commandchars=\\\{\}]}
\DefineVerbatimEnvironment{Highlighting}{Verbatim}{commandchars=\\\{\}}
% Add ',fontsize=\small' for more characters per line
\usepackage{framed}
\definecolor{shadecolor}{RGB}{248,248,248}
\newenvironment{Shaded}{\begin{snugshade}}{\end{snugshade}}
\newcommand{\AlertTok}[1]{\textcolor[rgb]{0.94,0.16,0.16}{#1}}
\newcommand{\AnnotationTok}[1]{\textcolor[rgb]{0.56,0.35,0.01}{\textbf{\textit{#1}}}}
\newcommand{\AttributeTok}[1]{\textcolor[rgb]{0.77,0.63,0.00}{#1}}
\newcommand{\BaseNTok}[1]{\textcolor[rgb]{0.00,0.00,0.81}{#1}}
\newcommand{\BuiltInTok}[1]{#1}
\newcommand{\CharTok}[1]{\textcolor[rgb]{0.31,0.60,0.02}{#1}}
\newcommand{\CommentTok}[1]{\textcolor[rgb]{0.56,0.35,0.01}{\textit{#1}}}
\newcommand{\CommentVarTok}[1]{\textcolor[rgb]{0.56,0.35,0.01}{\textbf{\textit{#1}}}}
\newcommand{\ConstantTok}[1]{\textcolor[rgb]{0.00,0.00,0.00}{#1}}
\newcommand{\ControlFlowTok}[1]{\textcolor[rgb]{0.13,0.29,0.53}{\textbf{#1}}}
\newcommand{\DataTypeTok}[1]{\textcolor[rgb]{0.13,0.29,0.53}{#1}}
\newcommand{\DecValTok}[1]{\textcolor[rgb]{0.00,0.00,0.81}{#1}}
\newcommand{\DocumentationTok}[1]{\textcolor[rgb]{0.56,0.35,0.01}{\textbf{\textit{#1}}}}
\newcommand{\ErrorTok}[1]{\textcolor[rgb]{0.64,0.00,0.00}{\textbf{#1}}}
\newcommand{\ExtensionTok}[1]{#1}
\newcommand{\FloatTok}[1]{\textcolor[rgb]{0.00,0.00,0.81}{#1}}
\newcommand{\FunctionTok}[1]{\textcolor[rgb]{0.00,0.00,0.00}{#1}}
\newcommand{\ImportTok}[1]{#1}
\newcommand{\InformationTok}[1]{\textcolor[rgb]{0.56,0.35,0.01}{\textbf{\textit{#1}}}}
\newcommand{\KeywordTok}[1]{\textcolor[rgb]{0.13,0.29,0.53}{\textbf{#1}}}
\newcommand{\NormalTok}[1]{#1}
\newcommand{\OperatorTok}[1]{\textcolor[rgb]{0.81,0.36,0.00}{\textbf{#1}}}
\newcommand{\OtherTok}[1]{\textcolor[rgb]{0.56,0.35,0.01}{#1}}
\newcommand{\PreprocessorTok}[1]{\textcolor[rgb]{0.56,0.35,0.01}{\textit{#1}}}
\newcommand{\RegionMarkerTok}[1]{#1}
\newcommand{\SpecialCharTok}[1]{\textcolor[rgb]{0.00,0.00,0.00}{#1}}
\newcommand{\SpecialStringTok}[1]{\textcolor[rgb]{0.31,0.60,0.02}{#1}}
\newcommand{\StringTok}[1]{\textcolor[rgb]{0.31,0.60,0.02}{#1}}
\newcommand{\VariableTok}[1]{\textcolor[rgb]{0.00,0.00,0.00}{#1}}
\newcommand{\VerbatimStringTok}[1]{\textcolor[rgb]{0.31,0.60,0.02}{#1}}
\newcommand{\WarningTok}[1]{\textcolor[rgb]{0.56,0.35,0.01}{\textbf{\textit{#1}}}}
\usepackage{longtable,booktabs}
% Correct order of tables after \paragraph or \subparagraph
\usepackage{etoolbox}
\makeatletter
\patchcmd\longtable{\par}{\if@noskipsec\mbox{}\fi\par}{}{}
\makeatother
% Allow footnotes in longtable head/foot
\IfFileExists{footnotehyper.sty}{\usepackage{footnotehyper}}{\usepackage{footnote}}
\makesavenoteenv{longtable}
\usepackage{graphicx,grffile}
\makeatletter
\def\maxwidth{\ifdim\Gin@nat@width>\linewidth\linewidth\else\Gin@nat@width\fi}
\def\maxheight{\ifdim\Gin@nat@height>\textheight\textheight\else\Gin@nat@height\fi}
\makeatother
% Scale images if necessary, so that they will not overflow the page
% margins by default, and it is still possible to overwrite the defaults
% using explicit options in \includegraphics[width, height, ...]{}
\setkeys{Gin}{width=\maxwidth,height=\maxheight,keepaspectratio}
% Set default figure placement to htbp
\makeatletter
\def\fps@figure{htbp}
\makeatother
\setlength{\emergencystretch}{3em} % prevent overfull lines
\providecommand{\tightlist}{%
  \setlength{\itemsep}{0pt}\setlength{\parskip}{0pt}}
\setcounter{secnumdepth}{-\maxdimen} % remove section numbering

\title{Lab 1}
\author{Joseph Diaz}
\date{February 16, 2020}

\begin{document}
\maketitle

\hypertarget{task-1-a-basic-simulation-event}{%
\subsection{Task 1: A Basic Simulation
Event}\label{task-1-a-basic-simulation-event}}

Much of our applied probability computing work in this class will be
simulating events. This means that we generate an event at random. The R
function for simulating random numbers is sample; check out the help
screen.

\begin{Shaded}
\begin{Highlighting}[]
\KeywordTok{help}\NormalTok{(sample)}
\end{Highlighting}
\end{Shaded}

\hypertarget{code-set-up}{%
\subsubsection{Code set-up}\label{code-set-up}}

Let us try simulating a die roll: The parameter replace = TRUE is
important here as we are rolling the die over and over again, not
drawing marbles out of a bag. Here is how to roll a 6-sided die five
times in R, and then compute the average of the rolls. Try running it!

\begin{Shaded}
\begin{Highlighting}[]
\NormalTok{x =}\StringTok{ }\DecValTok{1}\OperatorTok{:}\DecValTok{6}  \CommentTok{# sides of the die}
\NormalTok{roll =}\StringTok{ }\KeywordTok{sample}\NormalTok{(x, }\DecValTok{5}\NormalTok{, }\DataTypeTok{replace =} \OtherTok{TRUE}\NormalTok{)  }\CommentTok{# tell R how many sides (x) and how many rolls (5)}
\KeywordTok{mean}\NormalTok{(roll) }\CommentTok{# average of the 5 rolls}
\end{Highlighting}
\end{Shaded}

\begin{verbatim}
## [1] 2.6
\end{verbatim}

Note that the \texttt{echo\ =\ FALSE} parameter was added to the code
chunk to prevent printing of the R code that generated the plot.

\hypertarget{the-problem}{%
\subsubsection{The problem}\label{the-problem}}

A tetrahedron die is a four-sided die with labels \{1, 2, 3, 4\}. Have R
make 10 rolls of the tetrahedron die and compute the average. (Can think
of this as rolling 10 different tetrahedron dice as well.) Keep the
output in this RMarkdown file for grading purposes.

\begin{Shaded}
\begin{Highlighting}[]
\NormalTok{tetraRoll=}\ControlFlowTok{function}\NormalTok{(n)}
\NormalTok{\{}
\NormalTok{  rolls =}\StringTok{ }\KeywordTok{sample}\NormalTok{(}\DecValTok{1}\OperatorTok{:}\DecValTok{4}\NormalTok{, n, }\DataTypeTok{replace =} \OtherTok{TRUE}\NormalTok{)}
  \KeywordTok{return}\NormalTok{(}\KeywordTok{mean}\NormalTok{(rolls))}
\NormalTok{\}}
\end{Highlighting}
\end{Shaded}

\hypertarget{question}{%
\subsubsection{Question:}\label{question}}

What value do you expect to get for the average of 10 rolls?\\
Calling \texttt{tetraRoll(10)}: \emph{1.9}

\hypertarget{task-2-playing-with-for-loops}{%
\subsection{Task 2: Playing with
for-loops}\label{task-2-playing-with-for-loops}}

For-loops are central to the simulation studies we will be performing in
this class. In these experiments, simulation tasks are repeated over and
over again. The for-loop can easily perform this replication for us. The
trick is appropriately storing your results for analysis. The syntax for
a for-loop in R is \texttt{for(var\ in\ seq)\{task\}}, read ``for a
given variable in a specified sequence.'' The for-loop steps that
variable through the sequence and performs the task each time.

\hypertarget{code-set-up-1}{%
\subsubsection{Code set-up}\label{code-set-up-1}}

Let us apply a for-loop for simulating a 6-sided die roll. That is,
repeat 1000 times the experiment of rolling a 6-sided die five times and
computing the average.

\begin{Shaded}
\begin{Highlighting}[]
\NormalTok{S =}\StringTok{ }\DecValTok{1000} \CommentTok{# number of simulation experiments performed}
\CommentTok{# store results in a (1000-dimensional) vector called rolls.avgs}
\NormalTok{rolls.avgs =}\StringTok{ }\KeywordTok{vector}\NormalTok{(}\DataTypeTok{length=}\NormalTok{S)  }
\CommentTok{# setting up the variable rolls.avgs to store the average roll for each experiment}
\CommentTok{# this for-loop steps the variable simnum through the sequence 1 to 1000,}
\CommentTok{# repeating 1000 times the die rolling tasks inside the curly brackets \{...\}.}
\ControlFlowTok{for}\NormalTok{(simnum }\ControlFlowTok{in} \DecValTok{1}\OperatorTok{:}\NormalTok{S)}
\NormalTok{\{}
  \CommentTok{# Use our die rolling code from Task 1!}
\NormalTok{  x =}\StringTok{ }\DecValTok{1}\OperatorTok{:}\DecValTok{6} \CommentTok{# sides of the die}
\NormalTok{  roll =}\StringTok{ }\KeywordTok{sample}\NormalTok{(x, }\DecValTok{5}\NormalTok{, }\DataTypeTok{replace =} \OtherTok{TRUE}\NormalTok{)  }\CommentTok{# simulate a die roll}
\NormalTok{  rolls.avgs[simnum] =}\StringTok{ }\KeywordTok{mean}\NormalTok{(roll) }\CommentTok{# store the average roll}
\NormalTok{\}}
\CommentTok{# take a look at the first 6 simulation results}
\KeywordTok{head}\NormalTok{(rolls.avgs)}
\end{Highlighting}
\end{Shaded}

\begin{verbatim}
## [1] 3.8 2.8 2.2 3.4 5.2 3.8
\end{verbatim}

\begin{Shaded}
\begin{Highlighting}[]
\CommentTok{# compute the mean of the 1000 experiments}
\KeywordTok{mean}\NormalTok{(rolls.avgs)}
\end{Highlighting}
\end{Shaded}

\begin{verbatim}
## [1] 3.4928
\end{verbatim}

\hypertarget{the-problem-1}{%
\subsubsection{The problem}\label{the-problem-1}}

Repeat 1000 times the experiment you performed in Task 1, that is
rolling a tetrahedron die 10 times and computing the average. Report the
average and standard deviation of the 1000 experiments. The standard
deviation function in R is \texttt{sd(x)}.

\begin{Shaded}
\begin{Highlighting}[]
\NormalTok{simlist =}\StringTok{ }\KeywordTok{numeric}\NormalTok{(}\DecValTok{1000}\NormalTok{)}
\ControlFlowTok{for}\NormalTok{(i }\ControlFlowTok{in} \DecValTok{1}\OperatorTok{:}\DecValTok{1000}\NormalTok{)}
\NormalTok{\{}
\NormalTok{  simlist[i] =}\StringTok{ }\KeywordTok{tetraRoll}\NormalTok{(}\DecValTok{10}\NormalTok{)}
\NormalTok{\}}
\KeywordTok{mean}\NormalTok{(simlist)}
\end{Highlighting}
\end{Shaded}

\begin{verbatim}
## [1] 2.4951
\end{verbatim}

\begin{Shaded}
\begin{Highlighting}[]
\KeywordTok{sd}\NormalTok{(simlist)}
\end{Highlighting}
\end{Shaded}

\begin{verbatim}
## [1] 0.3604663
\end{verbatim}

\hypertarget{questions}{%
\subsubsection{Questions:}\label{questions}}

\begin{itemize}
\tightlist
\item
  Is the mean closer to the value you would expect than the average you
  had in Task 1? Why?\\
  \emph{Yes, because the multiples experiments `even out' the
  distribution you'd expect from a single experiment and you essentially
  zero in on the true average.}
\item
  How do you interpret the standard deviation in this problem?\\
  \emph{The standard deviation gives you an idea of the spread around
  the mean that you might get out of any single experiment. Some may be
  more or less than the mean, but the standard deviation tells about how
  far they can be.}
\end{itemize}

\hypertarget{task-3-presenting-tables-in-rmarkdown}{%
\subsection{Task 3: Presenting tables in
RMarkdown}\label{task-3-presenting-tables-in-rmarkdown}}

Let us present a table of our die rolls. We will use \texttt{xtable} and
\texttt{pander} R packages. Make sure to install the \texttt{pander}
package prior to running the code chunk. In this task, we will also try
the \texttt{replicate()} function in R to replace the for-loop.

Have R make 5 rolls of the tetrahedron die and repeat that 4 times.
Present the results in a table.

\hypertarget{rmarkdown}{%
\subsubsection{RMarkdown}\label{rmarkdown}}

The exact code is provided for you below. In this way you can
cut-and-paste this code for table-making in future labs. Three
parameters were added to the code chunk: The \texttt{echo\ =\ FALSE}
parameter was added to prevent printing of the R code that generated the
table. The \texttt{results=\textquotesingle{}asis\textquotesingle{}}
parameter was added to have R present the results as is for the table
generation. The \texttt{warning=FALSE} parameter suppresses warning
messages from R that are often presented when loading packages.\\
As an aside, a fourth common parameter is \texttt{include=FALSE}, which
prevents R from printing output when running the code chunk.

\begin{longtable}[]{@{}ccccc@{}}
\caption{Replicate 5 rolls of a tetrahedron die two
times}\tabularnewline
\toprule
\begin{minipage}[b]{0.17\columnwidth}\centering
Replicate 1\strut
\end{minipage} & \begin{minipage}[b]{0.17\columnwidth}\centering
Replicate 2\strut
\end{minipage} & \begin{minipage}[b]{0.17\columnwidth}\centering
Replicate 3\strut
\end{minipage} & \begin{minipage}[b]{0.17\columnwidth}\centering
Replicate 4\strut
\end{minipage} & \begin{minipage}[b]{0.17\columnwidth}\centering
Replicate 5\strut
\end{minipage}\tabularnewline
\midrule
\endfirsthead
\toprule
\begin{minipage}[b]{0.17\columnwidth}\centering
Replicate 1\strut
\end{minipage} & \begin{minipage}[b]{0.17\columnwidth}\centering
Replicate 2\strut
\end{minipage} & \begin{minipage}[b]{0.17\columnwidth}\centering
Replicate 3\strut
\end{minipage} & \begin{minipage}[b]{0.17\columnwidth}\centering
Replicate 4\strut
\end{minipage} & \begin{minipage}[b]{0.17\columnwidth}\centering
Replicate 5\strut
\end{minipage}\tabularnewline
\midrule
\endhead
\begin{minipage}[t]{0.17\columnwidth}\centering
4\strut
\end{minipage} & \begin{minipage}[t]{0.17\columnwidth}\centering
2\strut
\end{minipage} & \begin{minipage}[t]{0.17\columnwidth}\centering
4\strut
\end{minipage} & \begin{minipage}[t]{0.17\columnwidth}\centering
2\strut
\end{minipage} & \begin{minipage}[t]{0.17\columnwidth}\centering
1\strut
\end{minipage}\tabularnewline
\begin{minipage}[t]{0.17\columnwidth}\centering
2\strut
\end{minipage} & \begin{minipage}[t]{0.17\columnwidth}\centering
2\strut
\end{minipage} & \begin{minipage}[t]{0.17\columnwidth}\centering
3\strut
\end{minipage} & \begin{minipage}[t]{0.17\columnwidth}\centering
2\strut
\end{minipage} & \begin{minipage}[t]{0.17\columnwidth}\centering
4\strut
\end{minipage}\tabularnewline
\begin{minipage}[t]{0.17\columnwidth}\centering
2\strut
\end{minipage} & \begin{minipage}[t]{0.17\columnwidth}\centering
3\strut
\end{minipage} & \begin{minipage}[t]{0.17\columnwidth}\centering
4\strut
\end{minipage} & \begin{minipage}[t]{0.17\columnwidth}\centering
1\strut
\end{minipage} & \begin{minipage}[t]{0.17\columnwidth}\centering
3\strut
\end{minipage}\tabularnewline
\begin{minipage}[t]{0.17\columnwidth}\centering
4\strut
\end{minipage} & \begin{minipage}[t]{0.17\columnwidth}\centering
2\strut
\end{minipage} & \begin{minipage}[t]{0.17\columnwidth}\centering
1\strut
\end{minipage} & \begin{minipage}[t]{0.17\columnwidth}\centering
2\strut
\end{minipage} & \begin{minipage}[t]{0.17\columnwidth}\centering
4\strut
\end{minipage}\tabularnewline
\bottomrule
\end{longtable}

\hypertarget{question-1}{%
\subsubsection{Question:}\label{question-1}}

What do you observe across the replicates?\\
\emph{Seeing as the ``experiment'' is repeated multiple times using
replicate, the results were different every time.}

\hypertarget{task-4-presenting-graphs-in-rmarkdown}{%
\subsection{Task 4: Presenting graphs in
RMarkdown}\label{task-4-presenting-graphs-in-rmarkdown}}

Graphs are easy to display in an RMarkdown file.

\hypertarget{code-set-up-2}{%
\subsubsection{Code set-up}\label{code-set-up-2}}

Let us draw a histogram of our 1000 die rolls from earlier.

\begin{Shaded}
\begin{Highlighting}[]
\KeywordTok{hist}\NormalTok{(rolls.avgs, }\DataTypeTok{main=}\StringTok{""}\NormalTok{, }\DataTypeTok{xlab=}\StringTok{"Average of set of 10 rolls"}\NormalTok{)}
\end{Highlighting}
\end{Shaded}

\includegraphics{Lab-1_files/figure-latex/unnamed-chunk-7-1.pdf}

\hypertarget{the-problem-2}{%
\subsubsection{The problem}\label{the-problem-2}}

Let us add a normal approximation (bell curve) to the histogram. We will
cover the normal distribution later in the course. But hopefully you
recall it from your Statistics course! To add a density curve to the
plot, need to change the y-axis to a `density' scale. This is done by
setting the parameter prob = TRUE. The curve function addes a curve to
the plot. We will use a normal distribution with mean and standard
deviation set at the values obtained in Task 2. Here is the code

\begin{Shaded}
\begin{Highlighting}[]
\KeywordTok{hist}\NormalTok{(rolls.avgs, }\DataTypeTok{prob =}\NormalTok{ T, }\DataTypeTok{main=}\StringTok{""}\NormalTok{, }\DataTypeTok{xlab=}\StringTok{"Average of set of 10 rolls"}\NormalTok{) }\CommentTok{# histogram}
\KeywordTok{curve}\NormalTok{(}\KeywordTok{dnorm}\NormalTok{(x, }\DataTypeTok{mean=}\KeywordTok{mean}\NormalTok{(rolls.avgs), }\DataTypeTok{sd=}\KeywordTok{sd}\NormalTok{(rolls.avgs)), }\DataTypeTok{add=}\OtherTok{TRUE}\NormalTok{, }\DataTypeTok{col=}\StringTok{"green"}\NormalTok{) }\CommentTok{# normal approximation}
\end{Highlighting}
\end{Shaded}

Add these to the code chunk to present a histogram with a normal
approximation

\begin{Shaded}
\begin{Highlighting}[]
\KeywordTok{hist}\NormalTok{(rolls.avgs, }\DataTypeTok{prob =}\NormalTok{ T, }\DataTypeTok{main=}\StringTok{""}\NormalTok{, }\DataTypeTok{xlab=}\StringTok{"Average of set of 10 rolls"}\NormalTok{) }\CommentTok{# histogram}
\KeywordTok{curve}\NormalTok{(}\KeywordTok{dnorm}\NormalTok{(x, }\DataTypeTok{mean=}\KeywordTok{mean}\NormalTok{(rolls.avgs), }\DataTypeTok{sd=}\KeywordTok{sd}\NormalTok{(rolls.avgs)), }\DataTypeTok{add=}\OtherTok{TRUE}\NormalTok{, }\DataTypeTok{col=}\StringTok{"green"}\NormalTok{) }\CommentTok{# normal approximation}
\end{Highlighting}
\end{Shaded}

\includegraphics{Lab-1_files/figure-latex/unnamed-chunk-9-1.pdf}

\hypertarget{question-2}{%
\subsubsection{Question:}\label{question-2}}

\begin{itemize}
\tightlist
\item
  Interpret the histogram's shape, skew, spread, and center. \emph{The
  histgram's shape is indicatve of a normal distribution, with most
  occurrences being in the middle and the likelyhood of extreme values
  dropping off as you get away from the average. It is not exact due to
  the discrete nature and small number of ``Buckets'' to place
  occurrence into, though the general shape is preserved and the
  distribution is not skewed left or right.}
\item
  Does this follow what you would expect to see? \emph{Yes, given that
  the outcome of the dice rolls in each experiment is equally likely, we
  should expect the average to be around the same value each time; with
  values far from the overall mean being very unlikely.} \#\# Task 5:
  Boolean expressions Another useful task is making logical statements
  in R.
\end{itemize}

\hypertarget{code-set-up-3}{%
\subsubsection{Code set-up}\label{code-set-up-3}}

Let us first make 10 rolls of a die and see how often a 6 is rolled.

\begin{Shaded}
\begin{Highlighting}[]
\NormalTok{x =}\StringTok{ }\DecValTok{1}\OperatorTok{:}\DecValTok{6}  \CommentTok{# 6-sided die}
\NormalTok{rolls =}\StringTok{ }\KeywordTok{sample}\NormalTok{(x, }\DecValTok{10}\NormalTok{, }\DataTypeTok{replace =} \OtherTok{TRUE}\NormalTok{)  }\CommentTok{# roll the die five times}
\CommentTok{# Boolean expression: how often is the roll EXACTLY 6, use double-equals sign}
\KeywordTok{sum}\NormalTok{(rolls }\OperatorTok{==}\StringTok{ }\DecValTok{6}\NormalTok{)}
\end{Highlighting}
\end{Shaded}

\begin{verbatim}
## [1] 0
\end{verbatim}

Now repeat 1000 times the experiment of rolling a die 10 times as in
Task 2. We will see how many times a six occurs at least once out of ten
rolls across all these experiments. The code for this counting exercise
is sum(roll==6)\textgreater0 since a ``success'' is an experiment where
the total number of sixes showing on ten rolls is more than zero!

\begin{Shaded}
\begin{Highlighting}[]
\NormalTok{six =}\StringTok{ }\DecValTok{0} \CommentTok{# start a counter for number of times at least one six shows in 5 rolls}
\NormalTok{S =}\StringTok{ }\DecValTok{1000} \CommentTok{# number of experiments}
\CommentTok{# for-loop to repeat die rolling experiment 1000 times}
\ControlFlowTok{for}\NormalTok{(simnum }\ControlFlowTok{in} \DecValTok{1}\OperatorTok{:}\NormalTok{S)}
\NormalTok{\{}
\NormalTok{  x =}\StringTok{ }\DecValTok{1}\OperatorTok{:}\DecValTok{6} \CommentTok{# 6-sided die}
\NormalTok{  roll =}\StringTok{ }\KeywordTok{sample}\NormalTok{(x, }\DecValTok{10}\NormalTok{, }\DataTypeTok{replace =} \OtherTok{TRUE}\NormalTok{) }\CommentTok{# roll the die five times}
  \CommentTok{# two ways to count: with an if-then statement, or more elegantly }
  \CommentTok{# with a Boolean computation}
  \CommentTok{# if(sum(roll==2) > 0)\{st = st + 1\}  # if-then statement}
\NormalTok{  six =}\StringTok{ }\NormalTok{six }\OperatorTok{+}\StringTok{ }\NormalTok{(}\KeywordTok{sum}\NormalTok{(roll}\OperatorTok{==}\DecValTok{6}\NormalTok{)}\OperatorTok{>}\DecValTok{0}\NormalTok{) }
  \CommentTok{# Boolean computation: add one to the counter only if at least one 6 shows.}
\NormalTok{\}}
\NormalTok{six}
\end{Highlighting}
\end{Shaded}

\begin{verbatim}
## [1] 844
\end{verbatim}

\hypertarget{the-problems}{%
\subsubsection{The problems:}\label{the-problems}}

\hypertarget{first-problem}{%
\subsubsection{First problem:}\label{first-problem}}

How often in 5 rolls of a tetrahedron die is a two rolled?

\begin{Shaded}
\begin{Highlighting}[]
\NormalTok{howManyTwos =}\StringTok{ }\ControlFlowTok{function}\NormalTok{(n)}
\NormalTok{\{}
\NormalTok{  twos =}\StringTok{ }\DecValTok{0}
\NormalTok{  rolls =}\StringTok{ }\KeywordTok{matrix}\NormalTok{(}\KeywordTok{sample}\NormalTok{(}\DecValTok{1}\OperatorTok{:}\DecValTok{4}\NormalTok{, }\DecValTok{5}\OperatorTok{*}\NormalTok{n, }\DataTypeTok{replace =} \OtherTok{TRUE}\NormalTok{), }\DataTypeTok{nrow=}\NormalTok{n)}
  \ControlFlowTok{for}\NormalTok{(i }\ControlFlowTok{in} \DecValTok{1}\OperatorTok{:}\NormalTok{n)}
\NormalTok{  \{}
\NormalTok{    twos =}\StringTok{ }\NormalTok{twos }\OperatorTok{+}\StringTok{ }\NormalTok{(}\KeywordTok{sum}\NormalTok{(rolls[i, }\DecValTok{1}\OperatorTok{:}\DecValTok{5}\NormalTok{] }\OperatorTok{==}\StringTok{ }\DecValTok{2}\NormalTok{) }\OperatorTok{>}\StringTok{ }\DecValTok{0}\NormalTok{)}
\NormalTok{  \}  }
  \KeywordTok{return}\NormalTok{(twos)}
\NormalTok{\}}
\end{Highlighting}
\end{Shaded}

\hypertarget{questions-1}{%
\subsubsection{Questions:}\label{questions-1}}

\begin{itemize}
\item
  Run the code multiples times. What values do you get?\\
  \emph{The number of 2s in 10 sets of 1000 trials:}\\
  \emph{775, 759, 780, 750, 744, 746, 772, 734, 752, 776}
\item
  Are the values different? Is that what you expect? \emph{The values
  are different, which is contrary to expectation as you'd expect that
  each individual roll outcome is equally likely. Then again, when you
  recognize that in one case we are using a six sided die and in the
  other case we are using a four sided die. So the difference in total
  outcomes makes sense.}
\end{itemize}

\hypertarget{second-problems}{%
\subsubsection{Second problems:}\label{second-problems}}

This is heading towards a probability calculation. Roll the tetrahedron
die 5 times and repeat this experiment 1000 times as in Task 2. Report
the proportion of 1000 simulations where a two occurred. (This derives
from Dobrow problem 1.44: Probability of rolling at least one 2 in five
rolls of a tetrahedron die.)

\begin{Shaded}
\begin{Highlighting}[]
\NormalTok{areThereTwos =}\StringTok{ }\ControlFlowTok{function}\NormalTok{(n)}
\NormalTok{\{}
\NormalTok{  twos =}\StringTok{ }\KeywordTok{numeric}\NormalTok{(n)}
\NormalTok{  rolls =}\StringTok{ }\KeywordTok{matrix}\NormalTok{(}\KeywordTok{sample}\NormalTok{(}\DecValTok{1}\OperatorTok{:}\DecValTok{4}\NormalTok{, }\DecValTok{5}\OperatorTok{*}\NormalTok{n, }\DataTypeTok{replace =} \OtherTok{TRUE}\NormalTok{), }\DataTypeTok{nrow=}\NormalTok{n)}
  \ControlFlowTok{for}\NormalTok{(i }\ControlFlowTok{in} \DecValTok{1}\OperatorTok{:}\NormalTok{n)}
\NormalTok{  \{}
\NormalTok{    twos[i] =}\StringTok{ }\ControlFlowTok{if}\NormalTok{(}\DecValTok{2} \OperatorTok\StringTok{ }\NormalTok{rolls[i, }\DecValTok{1}\OperatorTok{:}\DecValTok{5}\NormalTok{]) }\DecValTok{1} \ControlFlowTok{else} \DecValTok{0}
\NormalTok{  \}  }
  \KeywordTok{return}\NormalTok{(}\KeywordTok{mean}\NormalTok{(twos))}
\NormalTok{\}}
\end{Highlighting}
\end{Shaded}

\hypertarget{questions-2}{%
\subsubsection{Questions:}\label{questions-2}}

\begin{itemize}
\tightlist
\item
  Is the value you get close to the truth (0.7627)?\\
  \emph{Probability of getting at least 1 two: 0.758}\\
\item
  How can we modify the simulation experiement to get a value even close
  to the truth?\\
  \emph{Do more trials in our simulation, as more trials will even out
  the variance from individual outcomes.}\\
  \#\# Task 6: Finalize your output for submitting to Blackboard As we
  noted, you can run each code chunk using the ``play'' button on the
  top right corner of the chunk. You may also run individual lines of
  code in the RStudio console for debugging purposes.\\
  To run the whole document and preview the Word document, press the
  ``Knit'' button in the menu bar underneath the tabs. This should
  present a preview of the Word file and save a .docx file in your
  working directory.\\
  Alternatively, you may render the Word file in the R console using the
  render command.
\item
  Save your file as a .rmd file to your desired working directory.
\item
  Then in the R console, place the following:
\end{itemize}

\begin{Shaded}
\begin{Highlighting}[]
\KeywordTok{library}\NormalTok{(rmarkdown)}
\KeywordTok{render}\NormalTok{(}\StringTok{"filename.rmd"}\NormalTok{)}
\end{Highlighting}
\end{Shaded}

This will save the .docx file of the report to your working directory.

\end{document}
