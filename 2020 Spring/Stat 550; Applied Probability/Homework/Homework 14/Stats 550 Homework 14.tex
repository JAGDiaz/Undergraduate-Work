\documentclass{article}
\usepackage{amsmath}
\usepackage{amssymb}
\usepackage{bm}
\usepackage{amsthm}
\usepackage{enumerate}
\usepackage{graphicx}
\usepackage{psfrag}
\usepackage{color}
\usepackage{url}
\usepackage{listings}
\usepackage{xcolor}
\usepackage{tikz}
\usetikzlibrary{positioning}
\tikzset{main node/.style={circle,fill=gray!20,draw,minimum size=.5cm,inner sep=0pt},}

\definecolor{codegreen}{rgb}{0,0.5,0}
\definecolor{codewhite}{rgb}{1,1,1}
\definecolor{codegray}{rgb}{0.5,0.5,0.5}
\definecolor{codepurple}{rgb}{0.58,0,0.82}
\definecolor{codeblack}{rgb}{0,0,0}
\definecolor{codeorange}{rgb}{0.8,0.4,0}

\lstdefinestyle{mystyle}{
    backgroundcolor=\color{codewhite},   
    commentstyle=\color{codegray},
    keywordstyle=\color{codegreen},
    numberstyle=\color{codegray},
    stringstyle=\color{codeorange},
    basicstyle=\ttfamily ,
    breakatwhitespace=false,         
    breaklines=true,                 
    captionpos=b,                    
    keepspaces=true,                 
    numbers=left,                    
    numbersep=5pt,                  
    showspaces=false,                
    showstringspaces=false,
    showtabs=false,                  
    tabsize=4
}
\lstset{style=mystyle}


\setlength{\hoffset}{-1in}
\addtolength{\textwidth}{1.5in}
\setlength{\voffset}{-1in}
\addtolength{\textheight}{1.5in}
\newcommand{\be}{\begin{enumerate}}
\newcommand{\ee}{\end{enumerate}}
\newcommand{\BigO}[1]{\ensuremath\mathcal{O}\left(#1\right)}
\newcommand{\il}[1]{\lstinline!#1!}
\newcommand{\gnorm}[1]{\left|\left|#1\right|\right|}
\newcommand{\abs}[1]{\left|#1\right|}
\newcommand{\parens}[1]{\left(#1\right)}
\newcommand{\bracks}[1]{\left\{#1\right\}}
\newcommand{\sqbracks}[1]{\left[#1\right]}
\newcommand{\vep}{\varepsilon}
\newcommand{\ceiling}[1]{\left\lceil#1\right\rceil}
\newcommand{\R}{\mathbb{R}}
\newcommand{\N}{\mathbb{N}}
\newcommand{\distrib}[2]{\text{#1}\left(#2\right)}

\begin{document}
	\begin{center}
		\textbf{Spring 2020, Stats 550:\ Homework 14} \\
		\textbf{Due:\ Thursday, April 16th, 2020} \\
		\textbf{Joseph Diaz: 819947915}
	\end{center}
\noindent\makebox[\linewidth]{\rule{\paperwidth}{0.4pt}}
\section*{Chapter 6}
\be 
	\item[1.] A random variable $X$ has density function
	$$f(x) = ce^x,\ \text{for}\ -2 < x < 2$$
	\be[(a)]
		\item Find $c$.
		\begin{proof}[Solution]
		To find $c$, we'll use $\int_{-2}^2f(x)\ dx = 1$:
		\begin{align*}
		\int_{-2}^2 ce^x\ dx = 1 \\
		ce^x\Big|_{-2}^2 = 1 \\
		c\parens{e^2 - e^{-2}} = 1 \\
		c \parens{\frac{e^4 - 1}{e^2}} = 1\\
		\Rightarrow c = \frac{e^2}{e^4 - 1}
		\end{align*}
		\end{proof}
		
		\item Find $P\parens{X < -1}$.
		\begin{proof}[Solution]
		$$P(X < -1) = \int_{-2}^{-1}\frac{e^2}{e^4 - 1}\cdot e^x\ dx = \frac{e^2}{e^4 - 1}\cdot e^x \Big|_{-2}^{-1} = \frac{e^2}{e^4 - 1}\parens{e^{-1} - e^{-2}}$$
		$$= \frac{e^2}{e^4 - 1}\cdot\frac{e-1}{e^2} = \frac{e-1}{e^4 - 1}$$
		\end{proof}
		
		\item Find $E\sqbracks{X}$.
		\begin{proof}[Solution]
		$$E\sqbracks{X} = \int_{-2}^2xf(x)\ dx$$
		\begin{align*}
		\Rightarrow E\sqbracks{X} &= \frac{e^2}{e^4 - 1}\int_{-2}^2xe^x\ dx \\
		&= \frac{e^2}{e^4 - 1} \parens{xe^x - e^x}\Big|_{-2}^2 \\
		&= \frac{e^2}{e^4 - 1} \parens{2e^2 - e^2 + 2e^{-2} + e^{-2}} \\
		&= \frac{e^2}{e^4 - 1}\parens{e^2 + 3e^{-2}} \\
		&= \frac{e^2}{e^4 - 1}\cdot\frac{e^4 + 3}{e^2} \\
		E\sqbracks{X} &= \frac{e^4 + 3}{e^4 - 1}
		\end{align*}
		\end{proof}
		
	\ee
		
	\item[2.] A random variable $X$ has density function proportional to $x^{-5}$ for $x > 1$.
	\be[(a)]
		\item Find the constant of proportionality.
		\begin{proof}[Solution]
		Let $f$ be the pdf of $X$, so 
		$$f(x) = \frac{c}{x^5},\ x > 1$$
		and $f$ must satisfy $\int_S f(x)\ dx = 1$, which yields
		\begin{align*}
		\int_1^\infty \frac{c}{x^5}\ dx = 1\\
		-\frac{c}{4x^4} \Big|_1^\infty = 1 \\
		-\frac{c}{4}\parens{\frac{1}{\infty} - \frac{1}{1}} = 1	 \\
		\frac{c}{4} = 1\\
		\Rightarrow c = 4
		\end{align*}
		So $$f(x) = \frac{4}{x^5}$$
		\end{proof}
		
		\item Find and graph the cdf of $X$.
		\begin{proof}[Solution]
		By the definition of the cdf we have:
		$$F(x) = \int_1^x \frac{4}{t^5}\ dt,\ x > 1$$
		Which after evaluating the integral is:
		$$F(x) = 1 - \frac{1}{x^4}$$
		and here is the cdf plotted for $1 < x < 5$.
		\pagebreak
		\begin{figure}[h!]
		\centering
		\includegraphics[width=.9\linewidth]{"Stats14".jpg}
		\end{figure}
		\end{proof}
		
		\item Use the cdf to find $P\parens{2 < X < 3}$.
		\begin{proof}[Solution]
		We may calculate $P\parens{2 < X < 3}$ using the cdf we found in (b):
		$$P\parens{2 < X < 3} = F(3) - F(2) = \parens{1 - \frac{4}{81}} - \parens{1 - \frac{4}{16}} \approx .201$$
		\end{proof}
		\item Find the mean and variance of $X$.
		\begin{proof}[Solution]
		For the expected value of $X$ we have
		\begin{align*}
		E\sqbracks{X} &= \int_1^\infty xf(x)\ dx \\
		&= \int_1^\infty \frac{4}{x^4}\ dx \\
		&= -\frac{4}{3}\parens{\frac{1}{x^3}}\Big|_1^\infty \\
		&= -\frac{4}{3}\parens{\frac{1}{\infty} - 1} \\
		E\sqbracks{X} &= \frac{4}{3}
		\end{align*}
		And for the variance of $X$ we have
		\begin{align*}
		V\sqbracks{X} &= \int_1^\infty \parens{x - E\sqbracks{X}}^2f(x)\ dx \\
		&= \int_1^\infty \parens{x - \frac{4}{3}}^2\cdot\frac{4}{x^5}\ dx\\
		&= 4 \int_1^\infty \parens{x^2 - \frac{8}{3}x + \frac{16}{9}}\cdot\frac{1}{x^5}\ dx \\
		&= 4 \int_1^\infty \parens{\frac{1}{x^3} - \frac{8}{3x^4} + \frac{16}{9x^5}}\ dx \\
		&= 4 \sqbracks{-\frac{1}{2x^2} + \frac{8}{9x^3} - \frac{4}{9x^4}}_1^\infty \\
		&= -4\parens{-\frac{1}{2} + \frac{8}{9} - \frac{4}{9}} \\
		V\sqbracks{X} &= \frac{2}{9}
		\end{align*}			 
		\end{proof}
		
	\ee
	
	\item[3.] The cumulative distribution function for a random variable $X$ is
	$$F(x) = \left\{\begin{array}{rl}
	0, & \text{if}\ x \leq 0\\
	\sin(x), & \text{if}\ 0 < x \leq \frac{\pi}{2}\\
	1, & \text{if}\ x > \frac{\pi}{2}
	\end{array}\right.$$
	\be[(a)]
		\item Find $P\parens{0.1 < X < 0.2}$.
		\begin{proof}[Solution]
		We have that
		\begin{align*}
		P\parens{0.1 < X < 0.2} &= P\parens{0.1 < X \leq \frac{\pi}{2}} + P\parens{\frac{\pi}{2} < X < 0.2}\\
		&= F\parens{\frac{\pi}{2}} - F\parens{0.1} + F\parens{0.2} - F\parens{\frac{\pi}{2}} \\
		&= F\parens{0.2} - F\parens{0.1}
		&= 1 - \sin\parens{0.1} \\
		&\approx .9002
		\end{align*}
		\end{proof}
		
		\item Find $E\sqbracks{X}$.
		\begin{proof}[Solution]
		To find the expected value of $X$, we'll use the following identity
		$$F'(x) = f(x)$$
		which gives 
		$$f(x) = \left\{\begin{array}{rl}
	0, & \text{if}\ x \leq 0\\
	\cos(x), & \text{if}\ 0 < x \leq \frac{\pi}{2}\\
	0, & \text{if}\ x > \frac{\pi}{2}
	\end{array}\right.$$
		
		So to find $E\sqbracks{X}$, we need only integrate from $0$ to $\frac{\pi}{2}$:
		\begin{align*}
		E\sqbracks{X} &= \int_0^{\frac{\pi}{2}} xf(x)\ dx\\
		&= \int_0^{\frac{\pi}{2}} x\cos(x)\ dx\\
		&= \parens{x\sin(x) + \cos(x)}\Big|_0^{\frac{\pi}{2}}\\
		&= \parens{\frac{\pi}{2} + 0} - \parens{0 + 1}\\
		E\sqbracks{X} &= \frac{\pi}{2}-1
		\end{align*}
		\end{proof}
	\ee
	
	\item[7.] Let $X \sim \distrib{Unif}{a,b}$. Find a general expression for the $k^{th}\ moment\ E\sqbracks{X^k}$.
	\begin{proof}[Solution]
	We have that the pdf of $X$ is
	$$f(x) = \frac{1}{b-a}$$
	So the expected value of $X^k$ is given by
	\begin{align*}
	E\sqbracks{X^k} &= \int_a^b x^kf(x)\ dx\\
	&= \frac{1}{b-a}\int_a^b x^k\ dx\\
	&=\frac{1}{b-a}\parens{\frac{x^{k+1}}{k+1}}\Big|_a^b\\
	E\sqbracks{X^k} &= \frac{b^{k+1} - a^{k+1}}{(b-a)(k+1)}
	\end{align*}
	\end{proof}
	
	\item[8.] An isosceles right triangle has side length uniformly distributed on $\parens{0,1}$. Find the expectation and variance of the length of the hypotenuse.
	\begin{proof}[Solution]
	Let $L \sim \distrib{Unif}{0,1}$ be the lengths of the legs of an isosceles right triangle, then 
	$$L^2 + L^2 = H^2 \implies H = \sqrt{2}L$$
	is the length of the hypotenuse of the triangle. Then the expected value of $H$ is
	\begin{align*}
	E\sqbracks{H} &= \sqrt{2}E\sqbracks{L} \\
	&= \sqrt{2}\cdot\frac{1 + 0}{2}\\
	&= \frac{\sqrt{2}}{2}
	\end{align*}
	and the variance of $H$ is
	\begin{align*}
	V\sqbracks{H} &= 2V\sqbracks{L}\\
	&= 2\cdot\frac{(1-0)^2}{12}\\
	&= 2\cdot\frac{1}{12}\\
	&= \frac{1}{6}
	\end{align*}
	So:
	$$E\sqbracks{H} = \frac{\sqrt{2}}{2},\ V\sqbracks{H} = \frac{1}{6}$$
	\end{proof}
	
	\item[11.] For continuous random variable $X$ and constants $a$ and $b$, prove that
$E\sqbracks{aX + b} = aE\sqbracks{X} + b$.
	\begin{proof}
	Let $f(x)$ be the pdf for $X$, then
	\begin{align*}
	E\sqbracks{aX+b} &= \int_{-\infty}^\infty \parens{ax+b}f(x)\ dx \\
	&= \int_{-\infty}^\infty axf(x)\ dx + \int_{-\infty}^\infty bf(x)\ dx \\
	&= a\int_{-\infty}^\infty xf(x)\ dx + b\int_{-\infty}^\infty f(x)\ dx
	\end{align*}
	$f(x)$ is the pdf for $X$, so
	$$\int_{-\infty}^\infty xf(x)\ dx = E\sqbracks{X} \text{ and }\int_{-\infty}^\infty f(x)\ dx = 1$$
	So we may conclude
	$$E\sqbracks{aX + b} = a\int_{-\infty}^\infty xf(x)\ dx + b\int_{-\infty}^\infty f(x)\ dx = aE\sqbracks{X} + b$$
	\end{proof}
	
\ee
\noindent\makebox[\linewidth]{\rule{\paperwidth}{0.4pt}}
	
\end{document}
