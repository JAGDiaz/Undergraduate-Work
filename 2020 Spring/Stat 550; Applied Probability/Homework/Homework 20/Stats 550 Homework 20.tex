\documentclass{article}
\usepackage{amsmath}
\usepackage{amssymb}
\usepackage{bm}
\usepackage{amsthm}
\usepackage{enumerate}
\usepackage{graphicx}
\usepackage{psfrag}
\usepackage{color}
\usepackage{url}
\usepackage{listings}
\usepackage{xcolor}
\usepackage{tikz}
\usetikzlibrary{positioning}
\tikzset{main node/.style={circle,fill=gray!20,draw,minimum size=.5cm,inner sep=0pt},}

\definecolor{codegreen}{rgb}{0,0.5,0}
\definecolor{codewhite}{rgb}{1,1,1}
\definecolor{codegray}{rgb}{0.5,0.5,0.5}
\definecolor{codepurple}{rgb}{0.58,0,0.82}
\definecolor{codeblack}{rgb}{0,0,0}
\definecolor{codeorange}{rgb}{0.8,0.4,0}

\lstdefinestyle{mystyle}{
    backgroundcolor=\color{codewhite},   
    commentstyle=\color{codegray},
    keywordstyle=\color{codegreen},
    numberstyle=\color{codegray},
    stringstyle=\color{codeorange},
    basicstyle=\ttfamily ,
    breakatwhitespace=false,         
    breaklines=true,                 
    captionpos=b,                    
    keepspaces=true,                 
    numbers=left,                    
    numbersep=5pt,                  
    showspaces=false,                
    showstringspaces=false,
    showtabs=false,                  
    tabsize=4
}
\lstset{style=mystyle}


\setlength{\hoffset}{-1in}
\addtolength{\textwidth}{1.5in}
\setlength{\voffset}{-1in}
\addtolength{\textheight}{1.5in}
\newcommand{\be}{\begin{enumerate}}
\newcommand{\ee}{\end{enumerate}}
\newcommand{\BigO}[1]{\ensuremath\mathcal{O}\left(#1\right)}
\newcommand{\il}[1]{\lstinline!#1!}
\newcommand{\gnorm}[1]{\left|\left|#1\right|\right|}
\newcommand{\abs}[1]{\left|#1\right|}
\newcommand{\parens}[1]{\left(#1\right)}
\newcommand{\bracks}[1]{\left\{#1\right\}}
\newcommand{\sqbracks}[1]{\left[#1\right]}
\newcommand{\vep}{\varepsilon}
\newcommand{\ceiling}[1]{\left\lceil#1\right\rceil}
\newcommand{\R}{\mathbb{R}}
\newcommand{\N}{\mathbb{N}}
\newcommand{\distrib}[2]{\text{#1}\left(#2\right)}
\newcommand{\dd}[1]{\frac{d}{d#1}}
\newcommand{\abracks}[1]{\left< #1\right>}

\begin{document}
	\begin{center}
		\textbf{Spring 2020, Stats 550:\ Homework 20} \\
		\textbf{Due:\ Thursday, May 7th, 2020} \\
		\textbf{Joseph Diaz: 819947915}
	\end{center}
\noindent\makebox[\linewidth]{\rule{\paperwidth}{0.4pt}}
\subsection*{Chapter 9}
\be 
	\item[3.] Let $S$ be the sum of 100 fair dice rolls. Use (i) Markov’s inequality and (ii)
Chebyshev’s inequality to bound $P\parens{S \geq 380}$.
	\be[(i)]
		\item Using Markov's Inequality\\
		We have that $S = X_1 + \cdots + X_{100}$ with $X_k \sim \distrib{Unif}{\bracks{1,\ 2,\ 3,\ 4,\ 5,\ 6}}$ for $k = 1,\ \cdots,\ 100$. We know that 
		$$\forall k,\ E\sqbracks{X_k} = 3.5 \implies E\sqbracks{S} = E\sqbracks{X_1 + \cdots + X_{100}} = E\sqbracks{X_1} + \cdots + E\sqbracks{X_{100}}$$
		$$\implies E\sqbracks{S} = 100E\sqbracks{X_1} = 350$$
		Now let $\vep = 380$, so with Markov's inequality we have:
		$$P\parens{S \geq 380} \leq \frac{350}{380} \approx 92.12\%$$
		
		\item Using Chebyshev's Inequality\\
		With the same definitions from (i), we have 
		$$\forall k,\ V\sqbracks{X_k} = \frac{35}{12} \implies V\sqbracks{S} = V\sqbracks{X_1 + \cdots + X_{100}} = V\sqbracks{X_1} + \cdots + V\sqbracks{X_{100}}$$
		$$\implies V\sqbracks{S} = 100V\sqbracks{X_1} = \frac{3500}{12}$$
		Now with Chebyshev's Inequality we have:
		$$P\parens{S \geq 380} = P\parens{S - 350 \geq 30} = P\parens{\abs{S-350}\geq 30} \leq \frac{\frac{3500}{12}}{30^2} = \frac{35}{108} \approx 32.41\%$$
		With $\vep = 30$.
		
	\ee
	
	\item[4.] Find the best value of $c$ so that $P\parens{X \geq 5} \leq c$ using Markov’s and Chebyshev’s
inequalities, filling in the subsequent table. Compare with the exact
probabilities.
\begin{proof}[Solution]
The best values of $c$ for the Markov and Chebyshev inequalities are
$$c = \frac{E\sqbracks{X}}{5},\quad c = \frac{V\sqbracks{X}}{\parens{5 - E\sqbracks{X}}^2}$$
respectively. And so the filled table is:
\begin{center}
\begin{tabular}{|l|c|c|c|}
\hline 
\textbf{Distribution} & \textbf{Markov} & \textbf{Chebyschev} & \textbf{Exact Probability} \\
\hline
$\distrib{Pois}{2}$ & $40\%$ & $22.\bar{2}\%$ & $1.656361\%$ \\
\hline
$\distrib{Exp}{1/2}$ & $40\%$ & $44.\bar{4}\%$ & $8.2085\%$ \\
\hline
$\distrib{Norm}{2,4}$ & $40\%$ & $44.\bar{4}\%$ & $22.66274\%$ \\
\hline
$\distrib{Geom}{1/2}$ & $40\%$ & $22.\bar{2}\%$ & $1.5625\%$ \\
\hline
\end{tabular}
\end{center}
\end{proof}
	
	\item[5.] Let $X$ be a positive random variable with $\mu = 50$ and $\sigma^2 = 25$.
	\be[(a)] 
		\item What can you say about $P\parens{X \geq 60}$ using Markov’s inequality?
		\begin{proof}[Solution]
		Using Markov's Inequality we can say that the probability is bounded above and below:
		 $$0 \leq P\parens{X \geq 60} \leq \frac{50}{60} = \frac{5}{6}$$

		\end{proof}
		
		\item What can you say about $P\parens{X \geq 60}$ using Chebyshev’s inequality?
		\begin{proof}[Solution]
		Using Chebyshev's Inequality we can say that the probability is bounded above and below:
		 $$0 \leq P\parens{X \geq 60} = P\parens{X-50 \geq 10} = P\parens{\abs{X-50} \geq 10} \leq \frac{25}{100} = 25\%$$
		\end{proof}
		
	\ee
	
\ee
\noindent\makebox[\linewidth]{\rule{\paperwidth}{0.4pt}}
	
\end{document}
