\documentclass{article}
\usepackage{amsmath}
\usepackage{amssymb}
\usepackage{bm}
\usepackage{amsthm}
\usepackage{enumerate}
\usepackage{graphicx}
\usepackage{psfrag}
\usepackage{color}
\usepackage{url}
\usepackage{listings}
\usepackage{xcolor}

\definecolor{codegreen}{rgb}{0,0.5,0}
\definecolor{codewhite}{rgb}{1,1,1}
\definecolor{codegray}{rgb}{0.5,0.5,0.5}
\definecolor{codepurple}{rgb}{0.58,0,0.82}
\definecolor{codeblack}{rgb}{0,0,0}
\definecolor{codeorange}{rgb}{0.8,0.4,0}

\lstdefinestyle{mystyle}{
    backgroundcolor=\color{codewhite},   
    commentstyle=\color{codegray},
    keywordstyle=\color{codegreen},
    numberstyle=\color{codegray},
    stringstyle=\color{codeorange},
    basicstyle=\ttfamily ,
    breakatwhitespace=false,         
    breaklines=true,                 
    captionpos=b,                    
    keepspaces=true,                 
    numbers=left,                    
    numbersep=5pt,                  
    showspaces=false,                
    showstringspaces=false,
    showtabs=false,                  
    tabsize=4
}
\lstset{style=mystyle}


\setlength{\hoffset}{-1in}
\addtolength{\textwidth}{1.5in}
\setlength{\voffset}{-1in}
\addtolength{\textheight}{1.5in}
\newcommand{\be}{\begin{enumerate}}
\newcommand{\ee}{\end{enumerate}}
\newcommand{\BigO}[1]{\ensuremath\mathcal{O}\left(#1\right)}
\newcommand{\il}[1]{\lstinline!#1!}

\begin{document}
	\begin{center}
		\textbf{Spring 2020, Stats 550:\ Homework 7} \\
		\textbf{Due:\ Thursday, March 12th, 2020} \\
		\textbf{Joseph Diaz: 819947915}
	\end{center}
\noindent\makebox[\linewidth]{\rule{\paperwidth}{0.4pt}}
\section*{Chapter 4}
\be
	\item[3.] The following dice game costs \$10 to play. If you roll 1, 2, or 3, you lose
your money. If you roll 4 or 5, you get your money back. If you roll a 6, you
win \$24.
	\be[(a)]
		\item Find the distribution of your winnings $W$.
		\begin{proof}[Solution]
		$$P(W) = P(X=6) = \frac{1}{6}$$
		\end{proof}
		\item Find the expected value of the game.
		\begin{proof}[Solution]
		Let $S = \{1,2,3,4,5,6\}$, and $f(1) = f(2) = f(3) = -10, f(4) = f(5) = 0, f(6) = 24$; then:
		\begin{align*}
		E[X] &= \sum_{x\in S}f(x)P(X=x)\\
		&= \frac{1}{6}\sum_{x\in S}f(x)\\
		&= \frac{1}{6}(-10-10-10+0+0+24)\\
		&= -\frac{6}{6}= -1
		\end{align*}
		So you should expect to lose \$1. 
		\end{proof}
	\ee
	
	\item[4.] Let $X\ \char`\~\ \text{Unif}\{-2,-1, 0, 1, 2\}$.
	\be[(a)]
		\item Find $E[X]$
		\begin{proof}[Solution]
		\begin{align*}
		E[X] &= \sum_{x\in S}xP(X=x)\\
		&= \frac{1}{5}\sum_{x\in S}x\\
		&= \frac{1}{5}(-2 -1 + 0 + 1 + 2)\\
		&= 0
		\end{align*}
		\end{proof}
		
		\item Find $E[e^X]$
		\begin{proof}[Solution]
		\begin{align*}
		E[e^X] &= \sum_{x\in S}e^xP(X=x)\\
		&= \frac{1}{5}\sum_{x\in S}e^x\\
		&= \frac{1}{5}\big(e^{-2} + e^{-1} + 1 + e^1 + e^2\big)\\
		&= \frac{1 + e + e^2 + e^3 + e^4}{5e^2}
		\end{align*}
		\end{proof}
		
		\item Find $E[1/(X + 3)]$
		\begin{proof}[Solution]
		\begin{align*}
		E[1/(X + 3)] &= \sum_{x\in S}\frac{1}{x+3}P(X=x)\\
		&= \frac{1}{5}\sum_{x\in S}\frac{1}{x+3}\\
		&= \frac{1}{5}\left(1 + \frac{1}{2} + \frac{1}{3} + \frac{1}{4} + \frac{1}{5}\right)\\
		&= \frac{137}{300}
		\end{align*}				
		\end{proof}
		
	\ee
	
	\item[6.] Suppose $E[X^2] = 1, E[Y^2] = 2$, and $E[XY] = 3$. Find $E[(X + Y)^2]$.
	\begin{proof}[Solution]
	First we notice:
	$$(X+Y)^2 = X^2 + 2XY + Y^2$$
	So
	$$E[(X + Y)^2] = E[X^2 + 2XY + Y^2]$$
	and by the linearity of the Expected Value:
	\begin{align*}
	E[X^2 + 2XY + Y^2] &= E[X^2] + 2E[XY] + E[Y^2]\\
	&= 1 + 2\cdot3 + 2\\
	&= 9
	\end{align*}
	
	
	\end{proof}
	\item[8.] You are dealt five cards from a standard deck. Let $X$ be the number of aces in
your hand. Find $E[X]$.
	\begin{proof}[Solution]
	\begin{align*}
	E[X] &= \sum_{k=0}^4kP(X=k)\\
	&= \sum_{k=0}^4 k\frac{\binom{4}{k}\cdot\binom{48}{5-k}}{\binom{52}{5}}\\
	&= \frac{1}{\binom{52}{5}}\sum_{k=0}^4 k\binom{4}{k}\cdot\binom{48}{5-k}\\
	&= \frac{1}{\binom{52}{5}}\left(0 + \binom{4}{1}\cdot\binom{48}{4} + 2\binom{4}{2}\cdot\binom{48}{3} + 3\binom{4}{3}\cdot\binom{48}{2} + 4\binom{4}{4}\cdot\binom{48}{1}\right)\\
	&= \frac{5}{13}	
	\end{align*}	
	\end{proof}
	\item[9.] Let $X\ \tilde{}\ \text{Pois}(\lambda)$. Find $E[X!]$. For what values of $\lambda$ does the expectation
not exist?
\begin{proof}[Solution]
$E[X!]$ is given by:
\begin{align*}
E[X!] &= \sum_{k=0}^\infty k!\frac{e^{-\lambda}\lambda^k}{k!}\\
&= \sum_{k=0}^\infty e^{-\lambda}\lambda^k\\
&= e^{-\lambda}\sum_{k=0}^\infty \lambda^k
\end{align*}
$e^{-\lambda}\sum_{k=0}^\infty \lambda^k$ is a geometric series which converges for $|\lambda| < 1$ and equals:
$$e^{-\lambda}\sum_{k=0}^\infty \lambda^k = \frac{e^{-\lambda}}{1-\lambda}$$
so for $|\lambda| < 1$:
$$E[X!] = \frac{e^{-\lambda}}{1-\lambda}$$
\end{proof}
	
	\item[25.] Suppose $X$ takes values $-1, 0$, and $3$, with respective probabilities $0.1, 0.3$,
and $0.6$. Find $V [X]$.
\begin{proof}[Solution]
From the information given, we have:
\begin{align*}
E[X^2] &= \sum_{x\in S}x^2P(X=x) = 1\cdot.1 + 9\cdot.6 = 5.5\\
E[X]^2 &= \left(\sum_{x\in S}xP(X=x)\right)^2 = \left(-1\cdot.1 + 3\cdot.6\right)^2 = 2.89
\end{align*}
And so:
$$V[X] = 5.5 - 2.89 = 2.61$$
\end{proof}
	
	\item[26.] Suppose $E[X] = a, E[Y] = b, V [X] = c$, and $V [Y] = d$. If $X$ and $Y$ are
independent, find $V [2X - 3Y + 4]$.
\begin{proof}[Solution]
As $X$ and $Y$ are independent, we have:
\begin{align*}
V[2X - 3Y + 4] &= 2^2V[X] + (-3)^2V[y]\\
&= 4V[X] + 9V[Y]\\
&= 4c + 9d
\end{align*}
\end{proof}
	
	\item[28.] In a random experiment, let $A$ and $B$ be two independent events with $P(A) =
P(B) = p$. In an outcome of the experiment, let $X$ be the number of these
events that occur $(0, 1, \text{ or } 2)$. Find $E[X]$ and $V [X]$.
	\begin{proof}[Solution]
	$$E[X] = 3p\qquad V[X] = 5p-9p^2$$
	\end{proof}
	\item[29.] Suppose $E[X] = 2$ and $V [X] = 3$.
	\be[(a)]
		\item Find $E[(3 + 2X)^2]$
		\begin{proof}[Solution]
		First:
		$$E[(3 + 2X)^2] = E[4X^2 + 6X + 9] = 4E[X^2] + 6E[X] + 9$$
		To find $E[X^2]$, we'll consider:
		$$V[X] = E[X^2] - E[X]^2 \implies E[X^2] = V[X] + E[X]^2$$
		$$\to E[X^2] = 3 + 4 = 7$$
		So:
		$$E[(3 + 2X)^2] = 4\cdot7 + 6\cdot2 + 9 = 49$$
		\end{proof}
		\item Find $V [4 - 5X]$
		\begin{proof}[Solution]
		Using the properties of the variance, we have:
		$$V [4 - 5X] = 25V[X] = 25\cdot3 = 75$$
		\end{proof}
	\ee

	
	\item[31.] Suppose $A$ and $B$ are events with $P(A) = a, P(B) = b$ and $P(AB) = c$.
Define indicator random variables $I_A$ and $I_B$. Find $V [I_A + I_B]$.
	
	\item[55.] Simulate the probability of obtaining at least one match in the problem of
coincidences (see Exercise 4.54).
\begin{lstlisting}[language=R]
coincidence=function(n,m)
{
  `%!in%` <- Negate(`%in%`)
  v = replicate(n, if(TRUE %!in% (1:m == sample(1:m,m))) 1 else 0)
  return(1-mean(v))
}
coincidence(10000,10000)
# 0.6345
\end{lstlisting}
	\item[56.] Simulate the dice game in Example 4.3. Estimate the expectation and variance
of your winnings.
\begin{lstlisting}[language=R]
DiceGame=function(n)
{
  rolls = sample(1:6, n, replace = TRUE)
  distrib = c(0,0,0)
  for(i in 1:n)
  {
    if(rolls[i] == 6)
    {
      rolls[i] = 24
      distrib[3] = distrib[3] + 1
    }
    else if(rolls[i] %in% c(4,5))
    {
      rolls[i] = 0
      distrib[2] = distrib[2] + 1
    }
    else
    {
      rolls[i] = -10
      distrib[1] = distrib[1] + 1
    }
  }
  return(list("rolls" = rolls, "distrib" = distrib))
}
result = DiceGame(10000)
rolls = result$rolls
distrib = result$distrib
winnings = c(-10, 0, 24)
par(mar=c(10,5,10,5))
barplot(distrib, main = "Frequency of outcomes per 1000 games",
		names.arg = winnings)

mean(rolls)
# -0.8814
var(rolls)
# 145.6973
\end{lstlisting}
\ee
\noindent\makebox[\linewidth]{\rule{\paperwidth}{0.4pt}}
	
\end{document}
