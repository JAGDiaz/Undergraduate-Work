\documentclass{article}
\usepackage{amsmath}
\usepackage{amssymb}
\usepackage{bm}
\usepackage{amsthm}
\usepackage{enumerate}
\usepackage{graphicx}
\usepackage{psfrag}
\usepackage{color}
\usepackage{url}
\usepackage{listings}

\usepackage{xcolor}
 
\definecolor{codegreen}{rgb}{0,0.6,0}
\definecolor{codegray}{rgb}{0.5,0.5,0.5}
\definecolor{codepurple}{rgb}{0.58,0,0.82}
\definecolor{backcolour}{rgb}{0.95,0.95,0.92}

\lstdefinestyle{mystyle}{
    backgroundcolor=\color{backcolour},   
    commentstyle=\color{codegray},
    keywordstyle=\color{codegreen},
    numberstyle=\tiny\color{magenta},
    stringstyle=\color{codepurple},
    basicstyle=\ttfamily\footnotesize,
    breakatwhitespace=false,         
    breaklines=true,                 
    captionpos=b,                    
    keepspaces=true,                 
    numbers=left,                    
    numbersep=5pt,                  
    showspaces=false,                
    showstringspaces=false,
    showtabs=false,                  
    tabsize=4
}
 
\lstset{style=mystyle}

\setlength{\hoffset}{-0.5in}
\addtolength{\textwidth}{1.0in}
\setlength{\voffset}{-0.5in}
\addtolength{\textheight}{1.0in}
\newcommand{\be}{\begin{enumerate}}
\newcommand{\ee}{\end{enumerate}}


\begin{document}
	\begin{large}
	\begin{center}
		\textbf{Spring 2020, Math 543:\ Homework 2} \\
		\textbf{Due:\ Friday, February 14th, 2020} \\
		\textbf{Joseph Diaz}

	\end{center}
\noindent\makebox[\linewidth]{\rule{\paperwidth}{0.4pt}}
\be
	\item[\textbf{4.1}] Determine SVDs for the following matrices:
	$$(\text{a})\ \left[\begin{matrix}
	 3 & 0 \\
	0 & -2
\end{matrix}\right], \ (\text{b})\ \left[\begin{matrix}
	2 & 0 \\
	0 & 3
\end{matrix}\right], \ (\text{c})\ \left[\begin{matrix}
	0 & 2 \\
	0 & 0 \\
	0 & 0
\end{matrix}\right], \ (\text{d})\ \left[\begin{matrix}
	1 & 1 \\
	0 & 0
\end{matrix}\right], \ (\text{e})\ \left[\begin{matrix}
	1 & 1 \\
	1 & 1
\end{matrix}\right]$$
The SVDs shown below are of the form: $A = U\Sigma V^{*}$
	\be[(a)]
	\item $$\left[\begin{matrix}
	 3 & 0 \\
	0 & -2
\end{matrix}\right] = \left[\begin{matrix}
	 1 & 0 \\
	0 & 1
\end{matrix}\right]\left[\begin{matrix}
	 3 & 0 \\
	0 & 2
\end{matrix}\right]\left[\begin{matrix}
	 1 & 0 \\
	0 & -1
\end{matrix}\right]$$
	
	\item $$\left[\begin{matrix}
	2 & 0 \\
	0 & 3
\end{matrix}\right] = \left[\begin{matrix}
	0 & 1 \\
	1 & 0
\end{matrix}\right]\left[\begin{matrix}
	3 & 0 \\
	0 & 2
\end{matrix}\right]\left[\begin{matrix}
	0 & 1 \\
	1 & 0
\end{matrix}\right]$$
	
	\item $$\left[\begin{matrix}
	0 & 2 \\
	0 & 0 \\
	0 & 0
\end{matrix}\right] = \left[\begin{matrix}
	1 & 0 & 0\\
	0 & 1 & 0\\
	0 & 0 & 1
\end{matrix}\right]\left[\begin{matrix}
	2 & 0 \\
	0 & 0 \\
	0 & 0
\end{matrix}\right]\left[\begin{matrix}
	0 & 1 \\
	-1 & 0 \\
\end{matrix}\right]$$
	
	\item $$\left[\begin{matrix}
	1 & 1 \\
	0 & 0
\end{matrix}\right] = \left[\begin{matrix}
	1 & 0 \\
	0 & 1
\end{matrix}\right]\left[\begin{matrix}
	\sqrt{2} & 0 \\
	0 & 0
\end{matrix}\right]\left[\begin{matrix}
	\frac{\sqrt{2}}{2} & -\frac{\sqrt{2}}{2} \\
	\frac{\sqrt{2}}{2} & \frac{\sqrt{2}}{2}
\end{matrix}\right]$$
	
	\item $$\left[\begin{matrix}
	1 & 1 \\
	1 & 1
\end{matrix}\right] = \left[\begin{matrix}
	-\frac{\sqrt{2}}{2} & -\frac{\sqrt{2}}{2} \\
	-\frac{\sqrt{2}}{2} & \frac{\sqrt{2}}{2}
\end{matrix}\right]\left[\begin{matrix}
	2 & 0 \\
	0 & 0
\end{matrix}\right]\left[\begin{matrix}
	-\frac{\sqrt{2}}{2} & -\frac{\sqrt{2}}{2} \\
	-\frac{\sqrt{2}}{2} & \frac{\sqrt{2}}{2}
\end{matrix}\right]$$
	\ee
	 
\noindent\makebox[\linewidth]{\rule{\paperwidth}{0.4pt}}
	\item[\textbf{4.3}] Write a program which, given a real $2 \times 2$ matrix $A$, plots the right singular vectors $\vec{v}_1$ and $\vec{v}_2$ in the unit circle and also the left singular vectors $\vec{u}_1$ and $\vec{u}_2$ in the appropriate ellipse, as in Figure 4.1. Apply your program to the matrix $\left[\begin{matrix}
	1 & 2 \\
	0 & 2
	\end{matrix}\right]$ and also to the $2\times2$ matrices of Exercise 4.1.
	\be
	\pagebreak
	\item[(a)] The singular values of 
	$\left[\begin{matrix}
	 3 & 0 \\
	0 & -2
\end{matrix}\right]$ are $\sigma_1 = 3, \sigma_2 = 2$:
\begin{figure}[h]
  	\includegraphics[width=\linewidth]{H2UnitaryVecs001.jpg}
  	\centering
	\end{figure}
	\pagebreak
	\item[(b)] The singular values of
	$\left[\begin{matrix}
	 2 & 0 \\
	0 & 3
\end{matrix}\right]$ are $\sigma_1 = 3, \sigma_2 = 2$:
\begin{figure}[h]
  	\includegraphics[width=\linewidth]{H2UnitaryVecs002.jpg}
  	\centering
	\end{figure}
	\pagebreak
	\item[(d)] The singular values of
	$\left[\begin{matrix}
	 1 & 1 \\
	 0 & 0
\end{matrix}\right]$ are $\sigma_1 = \sqrt{2}, \sigma_2 = 0$:
\begin{figure}[h]
  	\includegraphics[width=\linewidth]{H2UnitaryVecs003.jpg}
  	\centering
	\end{figure}
	\pagebreak
	\item[(e)] The singular values of
	$\left[\begin{matrix}
	 1 & 1 \\
	 1 & 1
\end{matrix}\right]$ are $\sigma_1 = 2, \sigma_2 = 0$:
\begin{figure}[h]
  	\includegraphics[width=\linewidth]{H2UnitaryVecs004.jpg}
  	\centering
	\end{figure}
	\pagebreak
	\item[(f)] The singular values of
	$\left[\begin{matrix}
	 1 & 2 \\
	 0 & 2
\end{matrix}\right]$ are $\sigma_1 \approx 2.921, \sigma_2 \approx 0.685$:
\begin{figure}[h]
  	\includegraphics[width=\linewidth]{H2UnitaryVecs005.jpg}
  	\centering
	\end{figure}
	\ee
\ee
\noindent\makebox[\linewidth]{\rule{\paperwidth}{0.4pt}}
	\end{large}
\section*{Python Source Code}
\begin{lstlisting}[language=Python]
#! /usr/bin/python3

import numpy as np
from matplotlib import pyplot as plt
from matplotlib.patches import Ellipse 
import matplotlib.axes as AXES
plt.rcParams['text.usetex'] = True
plt.rcParams['savefig.format'] = 'jpg'
plt.rcParams['figure.figsize'] = (12, 12)

a = np.array([[3,0], [0, -2]])

b = np.array([[2, 0], [0, 3]])

c = np.array([[0, 2], [0, 0], [0, 0]])

d = np.array([[1,1], [0, 0]])

e = np.array([[1, 1], [1, 1]])

f = np.array([[1, 2], [0, 2]])

ra = np.array([a, b, d, e, f])

print("The SVD of\n", c, "is:\n")
u, s, vh = np.linalg.svd(c)
print(u, "\n*", s, "\n*", vh)
print("\n")

for i in ra:
	print("The SVD of\n", i, "is:\n")
	u, s, vh = np.linalg.svd(i, full_matrices=True)
	print(u, "\n*", s, "\n*", vh)
	print("\n")
	vh = np.transpose(vh)
	
	fig, ax = plt.subplots(1,1)

	if(s[0] >= 1e-6):
		a1 = ax.arrow(0,0, s[0]*u[0, 0], s[0]*u[1, 0], shape='full', width=.05, 
		color='y', length_includes_head=True, label=r"$\sigma_1\vec{u}_1$")
		
	if(s[1] >= 1e-6):
		a2 = ax.arrow(0,0, s[1]*u[0, 1], s[1]*u[1, 1], shape='full', width=.05, 
		color='b', length_includes_head=True, label=r"$\sigma_2\vec{u}_2$")

	a3 = ax.arrow(0,0, vh[0, 0], vh[1, 0], shape='full', width=.05, 
	color='r', length_includes_head=True, label=r"$\vec{v}_1$")
	a4 = ax.arrow(0,0, vh[0, 1], vh[1, 1], shape='full', width=.05, color='g', 
				length_includes_head=True, label=r"$\vec{v}_2$")
	ax.grid()

	ellipse = Ellipse((0,0), width=2*s[0], height=2*s[1], 
	angle=(180/np.pi)*np.arctan2(u[1,0],u[0,0]), fc='none', ec='cyan')
	
	circle = Ellipse((0,0), width=2, height=2, fc='none', ec='purple')
	
	ax.add_patch(ellipse)
	ax.add_patch(circle)
	ax.set_xlim(-3, 3)
	ax.set_ylim(-3, 3)
	ax.set_title("Unitary Matrices' column vectors", size=20)
	ax.legend(loc='lower right', handles=[a1, a2, a3, a4], fontsize=20)
	ax.set_xlabel("$x$", size = 20)
	ax.set_ylabel("$y$", size = 20)
	fig.tight_layout()
	plt.show()
\end{lstlisting}
\noindent\makebox[\linewidth]{\rule{\paperwidth}{0.4pt}}
\end{document}
