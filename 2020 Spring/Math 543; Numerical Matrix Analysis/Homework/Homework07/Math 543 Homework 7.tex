\documentclass{article}
\usepackage{amsmath}
\usepackage{amssymb}
\usepackage{bm}
\usepackage{amsthm}
\usepackage{enumerate}
\usepackage{graphicx}
\usepackage{subcaption}
\usepackage{psfrag}
\usepackage{color}
\usepackage{url}
\usepackage{listings}
\usepackage{xcolor}
\usepackage{tikz}
\usetikzlibrary{positioning}
\tikzset{main node/.style={circle,fill=gray!20,draw,minimum size=.5cm,inner sep=0pt},}

\definecolor{codegreen}{rgb}{0,0.5,0}
\definecolor{codewhite}{rgb}{1,1,1}
\definecolor{codegray}{rgb}{0.5,0.5,0.5}
\definecolor{codepurple}{rgb}{0.58,0,0.82}
\definecolor{codeblack}{rgb}{0,0,0}
\definecolor{codeorange}{rgb}{0.8,0.4,0}

\lstdefinestyle{mystyle}{
    backgroundcolor=\color{codewhite},   
    commentstyle=\color{codegray},
    keywordstyle=\color{codegreen},
    numberstyle=\color{codegray},
    stringstyle=\color{codeorange},
    basicstyle=\ttfamily ,
    breakatwhitespace=false,         
    breaklines=true,                 
    captionpos=b,                    
    keepspaces=true,                 
    numbers=left,                    
    numbersep=5pt,                  
    showspaces=false,                
    showstringspaces=false,
    showtabs=false,                  
    tabsize=4
}
\lstset{style=mystyle}


\setlength{\hoffset}{-1in}
\addtolength{\textwidth}{1.5in}
\setlength{\voffset}{-1in}
\addtolength{\textheight}{1.5in}
\newcommand{\be}{\begin{enumerate}}
\newcommand{\ee}{\end{enumerate}}
\newcommand{\BigO}[1]{\ensuremath\mathcal{O}\left(#1\right)}
\newcommand{\il}[1]{\lstinline!#1!}
\newcommand{\gnorm}[1]{\left|\left|#1\right|\right|}
\newcommand{\abs}[1]{\left|#1\right|}
\newcommand{\parens}[1]{\left(#1\right)}
\newcommand{\bracks}[1]{\left\{#1\right\}}
\newcommand{\sqbracks}[1]{\left[#1\right]}
\newcommand{\vep}{\varepsilon}
\newcommand{\ceiling}[1]{\left\lceil#1\right\rceil}
\newcommand{\R}{\mathbb{R}}
\newcommand{\N}{\mathbb{N}}
\newcommand{\distrib}[2]{\text{#1}\left(#2\right)}
\newcommand{\dd}[1]{\frac{d}{d#1}}

\begin{document}
	\begin{center}
		\textbf{Spring 2020, Math 543:\ Homework 7} \\
		\textbf{Due:\ Monday, May 4th, 2020} \\
		\textbf{Joseph Diaz: 819947915}
	\end{center}
\noindent\makebox[\linewidth]{\rule{\paperwidth}{0.4pt}}
\subsection*{Trefethen \& Bau}
\be 
	\item Read 26.1, 26.3
	
	\item 24.3\\
	Let $A$ be a $10\times10$ random matrix with entries from the standard normal distribution, minus twice the identity. Write a program to plot $\gnorm{e^{tA}}_2$ against $t$ for $0 \leq t \leq 20$ on a log scale, comparing the result to the straight line $e^{\alpha\parens{A}t}$, where $\alpha\parens{A} = \max_j \text{Re}\parens{\lambda_j}$ is the \textit{spectral abscissa} of $A$. Run the program for ten random matrices $A$ and comment on the results. What property of a matrix leads to a $\gnorm{e^{tA}}_2$ curve that remains oscillatory as $t\to \infty$?
	\begin{proof}[Solution]
	
	\begin{figure}[h!]
	\centering
		\begin{subfigure}[b]{0.3\linewidth}
		\includegraphics[width=\linewidth]{"log2NormVst00".jpg} 
  		\end{subfigure}
  		\begin{subfigure}[b]{0.3\linewidth}
		\includegraphics[width=\linewidth]{"log2NormVst01".jpg} 
  		\end{subfigure}
  		\begin{subfigure}[b]{0.3\linewidth}
		\includegraphics[width=\linewidth]{"log2NormVst02".jpg} 
  		\end{subfigure}
  		\begin{subfigure}[b]{0.3\linewidth}
		\includegraphics[width=\linewidth]{"log2NormVst03".jpg} 
  		\end{subfigure}
  		\begin{subfigure}[b]{0.3\linewidth}
		\includegraphics[width=\linewidth]{"log2NormVst04".jpg} 
  		\end{subfigure}
  		\begin{subfigure}[b]{0.3\linewidth}
		\includegraphics[width=\linewidth]{"log2NormVst05".jpg} 
  		\end{subfigure}
  		\begin{subfigure}[b]{0.3\linewidth}
		\includegraphics[width=\linewidth]{"log2NormVst06".jpg} 
  		\end{subfigure}
  		\begin{subfigure}[b]{0.3\linewidth}
		\includegraphics[width=\linewidth]{"log2NormVst07".jpg} 
  		\end{subfigure}
  		\begin{subfigure}[b]{0.3\linewidth}
		\includegraphics[width=\linewidth]{"log2NormVst08".jpg} 
  		\end{subfigure}
  		\begin{subfigure}[b]{0.3\linewidth}
		\includegraphics[width=\linewidth]{"log2NormVst09".jpg} 
  		\end{subfigure}
	\end{figure}
	The property of the Matrix that leads to the oscillatory motion is the nature the eigenvalues of $A$. $e{tA}$ is similar to the fundamental of a system of ordinary differential with coefficient matrix $A$, and the any oscillations in the 2-norm of $e^{tA}$ is due complex eigenvalues of $A$. We don't have that behavior with $e^{\alpha(A)t}$, because we are only looking at the largest real part of the eigenvalues of $A$.
	\begin{lstlisting}[language=python]
for j in range(10):
	A = np.random.normal(size=(10,10)) - 2*np.diag(np.ones(10))
	w, v = np.linalg.eig(A)
	alpha = np.max(np.real(w))

	t = np.linspace(0, 20, 1001)
	eAlpha = np.exp(t*alpha)

	tA = np.array([i*A for i in t])

	normExptA=np.array([np.linalg.norm(lin.expm(i),2) for i in tA])

	fig, ax = plt.subplots(figsize=(16,12))
	ax.plot(t, np.log(normExptA), 'k-', 
		label=r"$\log\left|\left|e^{tA}\right|\right|_2$")
	ax.plot(t, np.log(eAlpha), 'r-', label =r"$e^{\alpha(A)t}$")
	ax.set_title(r"Random $10 \times 10$ matrix run #%02d" % (j+1),
		size = 25)
	ax.set_ylabel(r"$y$", size = 20)
	ax.set_xlabel(r"$t$", size = 20)
	small=np.log(normExptA).min()<np.log(eAlpha).min()and 
		np.log(normExptA).min()or np.log(eAlpha).min()
	ax.set_xlim(t.min(),)
	ax.legend(loc='best', fontsize = 20)
	ax.set_ylim(small,)
	plt.savefig("log2NormVst%02d" % j)
	\end{lstlisting}
	\end{proof}
	
	
	\item 26.2 (Extra Credit)\\
	Let $A$ be the $32\times 32$ matrix with $-1$ on the main diagonal, 1 on the first and second superdiagonals, and elsewhere.
	\be[(a)]
		\item Using an SVD algorithm and contour plotting software, generate a plot of the boundaries of the 2-norm $\vep$-psuedospectra of $A$ for $\vep \in \bracks{10^{-i}}_{i=1}^8$.
		
		\item Produce a semilogy plot of $\gnorm{e^{tA}}_2$ against $t $ for $0 \leq t \leq 50$. What is the initial growth rate of the curve before the eventual decay sets in? Can you relate this to your plot of psuedospectra?
		
	\ee  
	
	\item Implement \& test Householder Reduction to Hessenberg form.
	\begin{lstlisting}[language=python]
def householderRedux(A):
	for k in range(A.shape[0]-2):
		x = np.array(A[k+1:, k], copy=True)
		e = np.zeros(x.size)
		e[0] = 1
		v = np.transpose(np.array([np.sign(x[0])*np.linalg.norm(x,2)
			*e + x], copy=True))
		v = v/np.linalg.norm(v,2)
		A[k+1:, k:] = A[k+1:, k:]-2*v@(np.transpose(v)@A[k+1:,k:])
		A[: ,k+1: ] = A[: ,k+1: ]-2*(A[:,k+1:]@v)@np.transpose(v)
	return A

def isHessenberg(A, tol):
	for j in range(0,A.shape[0]):
		for i in range(j+2, A.shape[1]):
			if(A[i,j] > tol):
				return False
	return True
	
for i in range(10):
	A = np.random.normal(size=(2**(i+1),2**(i+1)))
	B = householderRedux(A)
	print(isHessenberg(B, 1e-14))
	\end{lstlisting}
	
	\item Implement \& test Rayleigh Quotient Iteration.
	\begin{lstlisting}[language=python]
def RayleighIter(A, vIn, maxiter):
	v = np.array(vIn/np.linalg.norm(vIn,2), copy=True)
	lam = v@A@np.transpose(v)
	k = 0
	while(k < maxiter):
		k = k + 1
		w = np.linalg.solve(A - lam*np.diag(np.ones(A.shape[0])), v)
		v = w/np.linalg.norm(w, 2)
		lam = v@A@np.transpose(v)

	return lam, v
	
for i in range(10):
	A = np.random.normal(size=(10,10))
	x = np.random.normal(size=(10))
	lam, x = RayleighIter(A,x, 1e5)

	u,v = np.linalg.eig(A)
	for s in u:
		if(np.abs(s - lam)<1e-9):
			print(True)
	\end{lstlisting}
	\item Read 27.3
\ee

\noindent\makebox[\linewidth]{\rule{\paperwidth}{0.4pt}}
	
\end{document}
