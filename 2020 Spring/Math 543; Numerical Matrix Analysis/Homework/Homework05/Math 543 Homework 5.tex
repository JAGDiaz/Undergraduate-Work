\documentclass{article}
\usepackage{amsmath}
\usepackage{amssymb}
\usepackage{bm}
\usepackage{amsthm}
\usepackage{enumerate}
\usepackage{graphicx}
\usepackage{subcaption}
\usepackage{psfrag}
\usepackage{color}
\usepackage{url}
\usepackage{listings}
\usepackage{xcolor}

\definecolor{codegreen}{rgb}{0,0.5,0}
\definecolor{codewhite}{rgb}{1,1,1}
\definecolor{codegray}{rgb}{0.5,0.5,0.5}
\definecolor{codepurple}{rgb}{0.58,0,0.82}
\definecolor{codeblack}{rgb}{0,0,0}
\definecolor{codeorange}{rgb}{0.8,0.4,0}

\lstdefinestyle{mystyle}{
    backgroundcolor=\color{codewhite},   
    commentstyle=\color{codegray},
    keywordstyle=\color{codegreen},
    numberstyle=\color{codegray},
    stringstyle=\color{codeorange},
    basicstyle=\ttfamily ,
    breakatwhitespace=false,         
    breaklines=true,                 
    captionpos=b,                    
    keepspaces=true,                 
    numbers=left,                    
    numbersep=5pt,                  
    showspaces=false,                
    showstringspaces=false,
    showtabs=false,                  
    tabsize=4
}
\lstset{style=mystyle}


\setlength{\hoffset}{-1in}
\addtolength{\textwidth}{1.5in}
\setlength{\voffset}{-1in}
\addtolength{\textheight}{1.5in}
\newcommand{\be}{\begin{enumerate}}
\newcommand{\ee}{\end{enumerate}}
\newcommand{\BigO}[1]{\ensuremath\mathcal{O}\left(#1\right)}
\newcommand{\il}[1]{\lstinline!#1!}
\newcommand{\gnorm}[1]{\left|\left|#1\right|\right|}

\begin{document}
	\begin{center}
		\textbf{Spring 2020, Math 543:\ Homework 5} \\
		\textbf{Due:\ Sunday, March 29, 2020} \\
		\textbf{Joseph Diaz: 819947915}
	\end{center}
\noindent\makebox[\linewidth]{\rule{\paperwidth}{0.4pt}}
\subsection*{12.3}
The goal of this problem is to explore some properties of random matrices. Your job is to be a laboratory scientist, performing experiments that lead to conjectures and more refined experiments. Do not try to prove anything. Do produce well-designed plots, which are worth a thousand numbers.\\
Define a \textit{random matrix} to be an $m\times m$ matrix whose entries are independent samples from the real normal distribution with mean zero and standard deviation $m^{-\frac{1}{2}}$. The factor $\sqrt{m}$ is introduced to make the limiting behavior clean as $m \to \infty$.
\be[(a)]
	\item What do the eigenvalues of a random matrix look like? What happens, say, if you take 100 random matrices and superimpose all their eigenvalues in a single plot? If you do this for $m = 8, 16, 32, 64 \cdots$, what pattern is suggested? How does the spectral radius $\rho(A)$ behave as $m \to \infty$?
	
	\begin{proof}[Solution]
	\begin{figure}[h!]
	\centering
	  \begin{subfigure}[b]{0.45\linewidth}
  	\includegraphics[width=\linewidth]{"Eigenvals0008by0008".jpg}
  \end{subfigure}
  \begin{subfigure}[b]{0.45\linewidth}
    \includegraphics[width=\linewidth]{"Eigenvals0016by0016".jpg}
  \end{subfigure}
	\end{figure}
	
	\begin{figure}[h!]
	\centering
	  \begin{subfigure}[b]{0.45\linewidth}
  	\includegraphics[width=\linewidth]{"Eigenvals0032by0032".jpg}
  \end{subfigure}
  \begin{subfigure}[b]{0.45\linewidth}
    \includegraphics[width=\linewidth]{"Eigenvals0064by0064".jpg}
  \end{subfigure}
	\end{figure}
	
	\begin{figure}[h!]
	\centering
	  \begin{subfigure}[b]{0.45\linewidth}
  	\includegraphics[width=\linewidth]{"Eigenvals0128by0128".jpg}
  \end{subfigure}
  \begin{subfigure}[b]{0.45\linewidth}
    \includegraphics[width=\linewidth]{"Eigenvals0256by0256".jpg}
  \end{subfigure}
	\end{figure}
	
	\begin{figure}[h!]
	\centering
	  \begin{subfigure}[b]{0.45\linewidth}
  	\includegraphics[width=\linewidth]{"Eigenvals0512by0512".jpg}
  \end{subfigure}
  \begin{subfigure}[b]{0.45\linewidth}
    \includegraphics[width=\linewidth]{"rhoVsM".jpg}
  \end{subfigure}
	\end{figure} 
	
	As can be seen in the first 7 plots, the concentration of the eigenvalues of our randomly generated matrices seem to approach something resembling the unit circle in the complex plane as $m \to \infty$. This is further reinforced by the eighth plot, which represents the greatest of the $\rho$'s from the previous plots as the radii of circles. $\rho\left(A\right)$ seems to be approaching 1 as $m \to \infty$.
		\end{proof}

	\pagebreak
	\pagebreak
	\item What about norms? How does the 2-norm of a random matrix behave as $m\to\infty$? Of course, we must have $\rho(A) \leq \gnorm{A}$. Does this inequality appear to approach equality as $m \to \infty$?
	\begin{proof}[Solution]
	\begin{figure}[h!]
	\centering
  	\includegraphics[width=\linewidth]{"normAVsM".jpg}
	\end{figure} 
	As the plot shows, the inequality is maintained for large m; but with the 2-norm at least it does not appear to be approaching equality. 
	\end{proof}	
	\pagebreak
	\vfill
	\item What about condition numbers - or more simply, the smallest singular value $\sigma_{\text{min}}$? Even for fixed $m$ this question is interesting. What proportions of random matrices in $\mathbb{R}^{m\times m}$ seem to have $\sigma_{\text{min}} \leq 2^{-1}, 2^{-2}, 2^{-3}, \cdots$? In other words, what does the tail of the probability distribution of smallest singular values look like? How does the scale of all this change with $m$?
	\begin{proof}[Solution]
	\begin{figure}[h!]
	\centering
  \begin{subfigure}[b]{0.45\linewidth}
    \includegraphics[width=\linewidth]{"minSigProp".jpg}
  \end{subfigure}
  \begin{subfigure}[b]{0.45\linewidth}
  	\includegraphics[width=\linewidth]{"MinSigmInv".jpg}
  \end{subfigure}
	\end{figure}
	The first of these plots shows that as $m$ increases, the proportion of random matrices increases as well; while the second shows that the minimum singular value among 100 random matrices for all $m$ is close to 0. So we can expect a condition number for each matrix.
	\end{proof}
 
\ee
\noindent\makebox[\linewidth]{\rule{\paperwidth}{0.4pt}}

\subsection*{Source code for plots:}
\begin{lstlisting}[language=python]
import numpy as np
import matplotlib.pyplot as plt
from matplotlib.patches import Ellipse as ellip
plt.rcParams['text.usetex'] = True
plt.rcParams['savefig.format'] = 'jpg'

#a

ms = np.array([8, 16, 32, 64, 128, 256, 512])
rhoTemp = np.zeros(100)
minsig = np.ones(ms.size)
prob = np.zeros(ms.size)
rho = np.zeros(ms.size)
norm2 = np.zeros(ms.size)
j = 0
for m in ms:
	mNegSqrt = 1./np.sqrt(m)
	circle = ellip((0,0), width=2, height=2, fc='none', ec='blue')
	fig, ax = plt.subplots(figsize=(12,12))
	for i in range(99):
		A = np.random.normal(loc = 0., scale = mNegSqrt, size = (m,m))
		s = np.linalg.svd(A, compute_uv= False)
		prob[j] += 1 if s[-1] < 1./m else 0
		minsig[j] = minsig[j] if minsig[j] < s[-1] else s[-1]
		eVals = np.linalg.eigvals(A)
		norm2[j] = norm2[j] if norm2[j] > np.linalg.norm(A,2) else np.linalg.norm(A,2)
		rhoTemp[i] = np.max(np.absolute(eVals))
		realVals = np.real(eVals)
		imagVals = np.imag(eVals)
		ax.plot(realVals, imagVals, 'ko')
		ax.set_xlabel(r"Real Part", size = 20)
		ax.set_ylabel(r"Imaginary Part", size = 20)
		ax.set_title(r"Eigenvalues of 100 random $A_{%d\times%d}$" %(m,m), size = 20)
	ax.axhline(y=0, color='k')
	ax.axvline(x=0, color='k')
	ax.add_patch(circle)
	ax.grid()
	ax.tick_params(length = 6, width = 2, labelsize = 20)
	fig.savefig("Eigenvals%04dby%04d" % (m, m))
	rho[j] = np.max(rhoTemp)
	prob /= 100.
	j += 1

fig, ax = plt.subplots(figsize=(16,16))
rhoMax = np.max(rho)
colors=['red','blue','yellow','purple','orange','cyan','pink']
merg=tuple(zip(rho,colors,ms))
for r,c,m in merg:
	circle = ellip((0,0), width=2*r, height=2*r, fc='none', ec=c, 
				label=r"$m = %d,\ \rho\left(A\right) = %1.2f$" % (m,r))
	ax.set_xlabel(r"$x$", size=20)
	ax.set_ylabel(r"$y$", size=20)
	ax.add_patch(circle)
	ax.set_xlim(-rhoMax-.1, rhoMax + .1)
	ax.set_ylim(-rhoMax-.1, rhoMax + .1)
ax.set_title(r"Spectral Radius $\rho\left(A\right)$ for random $A_{m \times m}$", size=20)
ax.grid()
ax.tick_params(length = 6, width = 2, labelsize = 20)
ax.legend(loc=('center'), fontsize = 20)
fig.savefig("rhoVsM")

fig, ax = plt.subplots(figsize=(16,16))
ax.plot(np.log2(ms), norm2, 'k-', label=r"$\left|\left| A \right|\right|_2$")
ax.plot(np.log2(ms), rho, 'r-', label=r"$\rho\left(A\right)$")
ax.set_xlabel(r"$\log_2(m)$", size=20)
ax.set_ylabel(r"$y$", size=20)
ax.tick_params(length = 6, width = 2, labelsize = 20)
ax.grid()
ax.legend(loc=('best'), fontsize = 20)
ax.set_title(r"2-norm and sprectral radius of random $A_{m \times m}$", size=20)
fig.savefig("normAVsM")

fig, ax = plt.subplots(figsize=(16,16))
ax.plot(np.log2(ms), minsig, 'r-', label=r"$\sigma_{min}$")
ax.plot(np.log2(ms), 1./ms, 'k-', label=r"$\frac{1}{m}$")
ax.set_xlabel(r"$\log_2(m)$", size=20)
ax.set_ylabel(r"$y$", size=20)
ax.tick_params(length = 6, width = 2, labelsize = 20)
ax.grid()
ax.legend(loc=('best'), fontsize = 20)
ax.set_title(r"$\sigma_{min}$ and $\frac{1}{m}$ for random $A_{m \times m}$", size=20)
fig.savefig("MinSigmInv")

fig, ax = plt.subplots(figsize=(16,16))
ax.plot(np.log2(ms), np.log10(prob), 'r-')
ax.set_xlabel(r"$\log_2(m)$", size=20)
ax.set_ylabel(r"$\log_{10}(p)$", size=20)
ax.tick_params(length = 6, width = 2, labelsize = 20)
ax.grid()
ax.set_title(r"Proportion of $A_{m \times m}$ matrices with $\sigma_{min} < \frac{1}{m}$", size=20)
fig.savefig("minSigProp")

\end{lstlisting}
			
\end{document}
