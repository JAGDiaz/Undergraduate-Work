\documentclass{article}
\usepackage{amsmath}
\usepackage{amssymb}
\usepackage{bm}
\usepackage{amsthm}
\usepackage{enumerate}
\usepackage{graphicx}
\usepackage{psfrag}
\usepackage{color}
\usepackage{url}
\usepackage{listings}
\usepackage{xcolor}
\usepackage{tikz}
\usepackage{gensymb}
\usetikzlibrary{positioning}
\tikzset{main node/.style={circle,fill=gray!20,draw,minimum size=.5cm,inner sep=0pt},}

\definecolor{codegreen}{rgb}{0,0.5,0}
\definecolor{codewhite}{rgb}{1,1,1}
\definecolor{codegray}{rgb}{0.5,0.5,0.5}
\definecolor{codepurple}{rgb}{0.58,0,0.82}
\definecolor{codeblack}{rgb}{0,0,0}
\definecolor{codeorange}{rgb}{0.8,0.4,0}

\lstdefinestyle{mystyle}{
    backgroundcolor=\color{codewhite},   
    commentstyle=\color{codegray},
    keywordstyle=\color{codegreen},
    numberstyle=\color{codegray},
    stringstyle=\color{codeorange},
    basicstyle=\ttfamily ,
    breakatwhitespace=false,         
    breaklines=true,                 
    captionpos=b,                    
    keepspaces=true,                 
    numbers=left,                    
    numbersep=5pt,                  
    showspaces=false,                
    showstringspaces=false,
    showtabs=false,                  
    tabsize=4
}
\lstset{style=mystyle}


\setlength{\hoffset}{-1in}
\addtolength{\textwidth}{1.5in}
\setlength{\voffset}{-1in}
\addtolength{\textheight}{1.5in}
\newcommand{\be}{\begin{enumerate}}
\newcommand{\ee}{\end{enumerate}}
\newcommand{\BigO}[1]{\ensuremath\mathcal{O}\left(#1\right)}
\newcommand{\il}[1]{\lstinline!#1!}
\newcommand{\gnorm}[1]{\left|\left|#1\right|\right|}
\newcommand{\abs}[1]{\left|#1\right|}
\newcommand{\parens}[1]{\left(#1\right)}
\newcommand{\bracks}[1]{\left\{#1\right\}}
\newcommand{\sqbracks}[1]{\left[#1\right]}
\newcommand{\vep}{\varepsilon}
\newcommand{\ceiling}[1]{\left\lceil#1\right\rceil}
\newcommand{\R}{\mathbb{R}}
\newcommand{\N}{\mathbb{N}}

\begin{document}
	\begin{center}
		\textbf{Spring 2020, Math 543:\ Homework 6} \\
		\textbf{Due:\ Friday, April 10th, 2020} \\
		\textbf{Joseph Diaz: 819947915}
	\end{center}
\noindent\makebox[\linewidth]{\rule{\paperwidth}{0.4pt}}
\be
	\item[18.1.] Consider the example
	$$A = \sqbracks{
	\begin{matrix}
	1 & 1\\
	1 & 1.0001\\
	1 & 1.0001
	\end{matrix}}, \quad 	
	b = \sqbracks{
	\begin{matrix}
	2 \\
	0.0001 \\
	4.0001 
	\end{matrix}}$$
	\be[(a)] 
		\item What are the matrices $A^+$ and $P$ for this example? Give an exact answer.
		\begin{proof}[Solution]
		For this example, we have that:
		$$A^+ = \sqbracks{
		\begin{matrix}
		10,001 & -5,000 & -5,000 \\
		-10,000 & 5,000 & 5,000
		\end{matrix}
		}, \quad P = \sqbracks{
		\begin{matrix}
		1 & 0 & 0 \\
		0 & 0.5 & 0.5 \\
		0 & 0.5 & 0.5
		\end{matrix}				
		}$$
		\end{proof}
		
		\item Find the exact solutions $x$ and $y = Ax$ to the least squares problem $Ax\approx b$.
		\begin{proof}[Solution]
		We have that $x = A^{+}b,\ y = Pb$, so:
		$$x = \sqbracks{
		\begin{matrix}
		10,001 & -5,000 & -5,000 \\
		-10,000 & 5,000 & 5,000
		\end{matrix} 			
		} \cdot \sqbracks{
	\begin{matrix}
	2 \\
	0.0001 \\
	4.0001 
	\end{matrix}} = \sqbracks{
	\begin{matrix}
	1 \\
	1
	\end{matrix}		
	}$$
	$$y = \sqbracks{
		\begin{matrix}
		1 & 0 & 0 \\
		0 & 0.5 & 0.5 \\
		0 & 0.5 & 0.5
		\end{matrix}				
		} \cdot \sqbracks{
	\begin{matrix}
	2 \\
	0.0001 \\
	4.0001 
	\end{matrix}} = \sqbracks{
	\begin{matrix}
	2 \\
	2.0001 \\
	2.0001 
	\end{matrix}}
	$$
		\end{proof}
		
		\item What are $\kappa\parens{A},\ \theta,$ and $\eta$? From here on, numerical answers are acceptable.
		\begin{proof}[Solution]
		With \il{Python} the numerical answers are:
		$$\kappa\parens{A} \approx 42429.23541608304, \quad \theta \approx 0.684702873261185 \approx 39\degree, \quad \eta \approx 1.0000000005555367$$
		\end{proof}
		
		\item What are the four condition numbers of Theorem 18.1?
		\begin{proof}[Solution]
		Our 4 condition numbers are:
		$$\begin{matrix}
		\displaystyle\frac{1}{\cos\parens{\theta}} \approx 1.29097724 & \displaystyle\frac{\kappa\parens{A}}{\eta \cos\parens{\theta}} \approx 54,775.1770 \\
		\displaystyle\frac{\kappa\parens{A}}{\cos\parens{\theta}} \approx 54,775.1771 & \displaystyle\kappa\parens{A} + \frac{\kappa\parens{A}^2 \tan\parens{\theta}}{\eta} \approx 1.46988325E09
		\end{matrix}$$
		\end{proof}
		
		\item Give an example of perturbations $\delta b$ and $\delta A$ that approximately attain these four condition numbers.
		\begin{proof}[Solution]
		Let $\delta = 1.23E-9$ and 
		$$\delta A = A + \delta \sqbracks{1}_{3 \times 2},\ \delta b = b + \delta \sqbracks{1}_{3 \times 1}$$
		If we were to calculate the four conditions for these ``perturbed'' $A$ and $b$, we get
		$$\kappa\parens{\delta A} \approx 42429.23546826927, \quad \delta\theta \approx 0.6847028729599098 \approx 39\degree, \quad \delta\eta \approx 1.0000000005556982$$
		with 
		$$\begin{matrix}
		\displaystyle\frac{1}{\cos\parens{\delta\theta}} \approx 1.29097724 & \displaystyle\frac{\kappa\parens{\delta A}}{\delta\eta \cos\parens{\delta\theta}} \approx 54,775.1771 \\
		\displaystyle\frac{\kappa\parens{\delta A}}{\cos\parens{\delta\theta}} \approx 54,775.1771 & \displaystyle\kappa\parens{\delta A} + \frac{\kappa\parens{\delta A}^2 \tan\parens{\delta\theta}}{\delta\eta} \approx 1.46988326E09
		\end{matrix}$$
		Which are basically indistinguishable from the original condition numbers.
		\end{proof}
		
	\ee
	
	\item[14.1.] Consider the vector $\vec{x} \in \R^{101}$ consisting of equi-spaced points in the interval $\sqbracks{0,1}$, e.g. \il{x = linspace(0,1,101)';} and let $A_k \in \R^{101 \times \parens{k+1}}$ be the matrix consisting of columns formed by component-size power $\bracks{0,\ \cdots,\ k}$ of the $x$-values \big(a Vandermonde Matrix\big). Let $c_k = \kappa\parens{A_k}$ be the collection of condition numbers for these matrices.
	\be[(a)]
		\item plot $\vec{c}$ \big(use a log scale\big).
		\begin{figure}[h]
		\centering
		\includegraphics[width=.9\linewidth]{"cVsk".jpg}
		\end{figure}
		
		\item We could use these matrices $\parens{A_k}$ to least-squares-fit polynomials \big(of matching degree $k$\big) to some data-set with 101 measurements. Is it necessarily better to have more model parameters \big( i.e. fitting a higher degree polynomial\big)? - Discuss.
	\begin{proof}[Solution]
	Not necessarily, in the case of the Vandermonde matrices for which we calculated $\kappa$ and plotted $\kappa$ against $k$ the Condition Number eventually levels off. But the value to which is it approached is on the order of $10^{14}$, which is a pretty absurdly high condition number. We may deduce that models with fewer model parameters tend to have smaller Condition numbers, and so are less sensitive to perturbation in the model's inputs; and it might be advantageous to minimize that. 
	\end{proof}
	\ee
\ee

\subsection*{Source Code}
\begin{lstlisting}[language=python]
import numpy as np
import matplotlib.pyplot as plt
plt.rcParams['savefig.format'] = 'jpg'

def get_ep_mach():
	ep = 1
	while(1 + ep != 1):
		ep /= 2
	return ep

A = np.array([[1,1],[1, 1.0001], [1, 1.0001]])
ANorm =  np.linalg.norm(A,2)
b = np.array([2, 0.0001, 4.0001])

Apinv = np.linalg.pinv(A)
x = Apinv@b
xNorm = np.linalg.norm(x, 2)

P = A@Apinv
y = P@b
yNorm = np.linalg.norm(y, 2)

kappa = ANorm*np.linalg.norm(Apinv,2)

theta = np.arccos(yNorm/np.linalg.norm(b, 2))

eta = (ANorm*xNorm)/yNorm

condNums = np.zeros((2,2))
condNums[0, 0] = 1./np.cos(theta)
condNums[0, 1] = (kappa/eta)*condNums[0, 0]
condNums[1, 0] = kappa*condNums[0, 0]
condNums[1, 1] = kappa + ((kappa**2)*np.tan(theta))/eta

ep = 1.23E-9
Aper = A + ep
AperNorm =  np.linalg.norm(Aper,2)
bper = b + ep

Aperinv = np.linalg.pinv(Aper)
xper = Aperinv@bper
xperNorm = np.linalg.norm(xper, 2)

Pper = Aper@Aperinv
yper = Pper@bper
yperNorm = np.linalg.norm(yper, 2)

kappaper = AperNorm*np.linalg.norm(Apinv,2)

thetaper = np.arccos(yperNorm/np.linalg.norm(bper, 2))

etaper = (AperNorm*xperNorm)/yperNorm

print(kappaper)
print(thetaper)
print(etaper)

condNumsper = np.zeros((2,2))
condNumsper[0, 0] = 1./np.cos(thetaper)
condNumsper[0, 1] = (kappaper/etaper)*condNumsper[0, 0]
condNumsper[1, 0] = kappaper*condNumsper[0, 0]
condNumsper[1, 1] = kappaper + ((kappaper**2)*np.tan(thetaper))/etaper

print(condNumsper)
#-------------------------------------------------------#

x = np.linspace(0,1,101)
ks = np.arange(500)
c = np.zeros(ks.size)

A = np.ones((x.size, ks.size))
for k in range(1,ks.size):
	A[:,k] = A[:,k-1]*x

for k in ks:
	Ainv = np.linalg.pinv(A[:,:k+1])
	c[k] = np.linalg.norm(A[:,:k+1], 2)*np.linalg.norm(Ainv, 2)

fig, ax = plt.subplots(figsize=(18,12))
ax.plot(np.log10(ks), np.log10(c), 'r-')
ax.set_ylabel(r"$\log_{10}\left(\kappa\left(A_k\right)\right)$", size=20)
ax.set_xlabel("$\log_{10}(k)$", size=20)
ax.tick_params(length=6, width=2, labelsize=20)
ax.set_title(r"Condition Number of Vandermonde Matrices $A_k \in \mathbb{R}^{101\times(k+1)}$", size = 20)
#ax.legend(loc=('best'))
fig.savefig("cVsk")

\end{lstlisting}
\noindent\makebox[\linewidth]{\rule{\paperwidth}{0.4pt}}
	
\end{document}
