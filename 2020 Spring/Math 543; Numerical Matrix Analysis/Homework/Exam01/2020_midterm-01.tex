\documentclass[12pt]{article}

\usepackage{hyperref}
\usepackage{colordvi}
\usepackage{fancybox}
\usepackage{amsfonts}
\usepackage{fullpage}
\usepackage{epsfig}
\usepackage{linegoal}
\usepackage{amsmath}
\usepackage{amssymb}
\usepackage{bm}
\usepackage{amsthm}
\usepackage{enumerate}
\usepackage{graphicx}
\usepackage{psfrag}
\usepackage{color}
\usepackage{url}
\usepackage{listings}
\usepackage{xcolor}
\usepackage{tikz}
\usetikzlibrary{positioning}
\tikzset{main node/.style={circle,fill=gray!20,draw,minimum size=.5cm,inner sep=0pt},}

\definecolor{codegreen}{rgb}{0,0.5,0}
\definecolor{codewhite}{rgb}{1,1,1}
\definecolor{codegray}{rgb}{0.5,0.5,0.5}
\definecolor{codepurple}{rgb}{0.58,0,0.82}
\definecolor{codeblack}{rgb}{0,0,0}
\definecolor{codeorange}{rgb}{0.8,0.4,0}

\lstdefinestyle{mystyle}{
    backgroundcolor=\color{codewhite},   
    commentstyle=\color{codegray},
    keywordstyle=\color{codegreen},
    numberstyle=\color{codegray},
    stringstyle=\color{codeorange},
    basicstyle=\ttfamily ,
    breakatwhitespace=false,         
    breaklines=true,                 
    captionpos=b,                    
    keepspaces=true,                 
    numbers=left,                    
    numbersep=5pt,                  
    showspaces=false,                
    showstringspaces=false,
    showtabs=false,                  
    tabsize=4
}
\lstset{style=mystyle}



\newcommand{\be}{\begin{enumerate}}
\newcommand{\ee}{\end{enumerate}}
\newcommand{\BigO}[1]{\ensuremath\mathcal{O}\left(#1\right)}
\newcommand{\il}[1]{\lstinline!#1!}
\newcommand{\gnorm}[1]{\left|\left|#1\right|\right|}
\newcommand{\abs}[1]{\left|#1\right|}
\newcommand{\parens}[1]{\left(#1\right)}
\newcommand{\bracks}[1]{\left\{#1\right\}}
\newcommand{\vep}{\varepsilon}
\newcommand{\ceiling}[1]{\left\lceil#1\right\rceil}
\newcommand{\bfit}[1]{\textbf{\textit{#1}}}
\newcommand{\fl}[1]{\text{fl}\left(#1\right)}

\begin{document}

\noindent
\textbf{Math 543, Spring 2020\\
  Midterm \#1, Due Friday April 10 at 11:59pm.}\\
\\
\textbf{Submission:~}~\parbox[t]{\linegoal}{\textbf{UPLOAD ALL PAGES TO
    GRADESCOPE, attach extra page(s) \\AFTER the 10 numbered pages}}\\
\\[0.5em]

\centerline{
\textrm{\textbf{\small
\begin{tabular}{|c|c|c|} \hline
  Problem & Pts Possible & Pts Scored \\ \hline \hline
  1(a)     &   5 & \\ \hline
  1(b)-i   &   5 & \\ \hline
  1(b)-ii  &   5 & \\ \hline
  1(b)-iii &   5 & \\ \hline
  1(b)-iv  &   5 & \\ \hline
  1(c)-i   &   5 & \\ \hline
  1(c)-ii  &   5 & \\ \hline
  1(c)-iii &   5 & \\ \hline \hline
  2(a)     &   5 & \\ \hline
  2(b)     &   5 & \\ \hline
  2(c)     &   5 & \\ \hline
  2(d)     &   5 & \\ \hline
  2(e)     &   5 & \\ \hline
  2(f)-i   &   5 & \\ \hline
  2(f)-ii  &   5 & \\ \hline
  2(f)-iii &   5 & \\ \hline
  2(f)-iv  &   5 & \\ \hline \hline
  3(a)     &   5 & \\ \hline
  3(b)     &   5 & \\ \hline
  3(c)     &   5 & \\ \hline
  3(d)     &   5 & \\ \hline \hline
  4        &   5 & \\ \hline \hline
  5        &   0 & \\ \hline \hline
  Total    & 110 & \\ \hline
\end{tabular}
}}}

\vspace*{2\baselineskip}

\centerline{%
  \parbox{1.15\linewidth}{%
    I, Joseph Diaz, pledge that this exam is
    \bfit{completely my own work,} and that I did not take, borrow
    or steal any portions from any other person, including \emph{\large
      ``Questionable Cousin Chegg.''}  Any and all references I used are
    clearly cited in my solutions. I understand that if I violate this honesty
    pledge, I am subject to disciplinary action pursuant to the appropriate
    sections of the San Diego State University Policies.%
  }%
}

\vspace*{1.0\baselineskip}

\[
  \stackrel{\hbox{\underline{\hspace*{3.5in}}}}{\hbox{\scriptsize
  Signature}}
\]

\vspace*{1.0\baselineskip}

\noindent
\textbf{Rules:} This midterm is open-book, open-notes.  You cannot consult any
other human.  If you refer to results from books (including the class text),
research papers, or the web (other than the class web page(s)) carefully cite
your source(s).



\begin{enumerate}

   \item \bfit{The Relative Condition Number} ---

   \begin{enumerate}

   \item (5 pts.) For the function $f:X\rightarrow Y$, define the
     relative condition number $\kappa(x)$.
     \begin{proof}[Solution]
     Let $x \in X$, then the relative condition number is $\kappa(x)$ is defined as:
     $$\kappa(x) = \sup_{\Delta x}\bracks{\frac{\gnorm{f\parens{x + \Delta x} - f(x)}}{\gnorm{f(x)}}\cdot\frac{\gnorm{x}}{\gnorm{\Delta x}}}$$
     where $\Delta x$ is some small perturbation on $x$.
     \end{proof}

     
   \item
     \begin{enumerate}

     \item (5 pts.) What is the smallest (in magnitude) real perturbation
       $\varepsilon_{\mathbb{R}} \in \mathbb{R} \backslash
       \{0\}$?
       \begin{proof}[Solution]
       There is none. $\forall a \in \mathbb{R}, a > 0,\ \exists b \in \mathbb{R}: 0 < b < a$, for example; $b = a/2$. So there is no smallest magnitude element, as there is always a smaller positive element.
       \end{proof}
        


     \item (5 pts.) What is the smallest (in magnitude) integer perturbation
       $\varepsilon_{\mathbb{Z}} \in \mathbb{Z} \backslash \{0\}$?
       \begin{proof}[Solution]
       The smallest magnitude integers are $-1$ and $1$.
       \end{proof}


     \item (5 pts.) What is the smallest (in magnitude) IEEE 754-1985 64-bit
       floating point $\mathbb{F}_{64}$ perturbation
       $\varepsilon_{\mathbb{F}_{64}} \in \mathbb{F}_{64} \backslash \{0\}$?
       \begin{proof}[Solution]
       IEEE 754-1985 64-bit numbers are represented using:
       $$r = (-1)^s\cdot 2^{c-1023}\cdot(1+f),\ c = \sum_{n=0}^{10}c_n2^n,\ f = \sum_{k=0}^{51}\frac{m_k}{2^{52-k}}$$
       So the smallest positive number that could be represented would be when $s = 0,\ c = 0\ (c_n = 0,\ \forall n \in \bracks{0,\ 1,\ \cdots,\ 10})$, and $f = 2^{-52}\ (m_0 = 1,\ m_k = 0,\ \forall k \in \bracks{1,\ 2, \cdots,\ 51})$; so let $r_{\text{min}}$ be the smallest  IEEE 754-1985 64-bit number, then:
       $$r_{\text{min}} = (-1)^0\cdot 2^{-1023}\cdot \parens{1 + 2^{-52}} = 2^{-1023} + 2^{-1075} \approx 2^{-1023}$$
       There exists $d \in \mathbb{R}: 10^d = 2^{-1023}$. So we now solve for $d$:
       $$10^d = 2^{-1023} \Rightarrow d = \log_{10}\parens{2^{-1023}} = -1023\log_{10}(2) \approx -308$$
       So $r_{\text{min}} \approx 10^{-308}$.
       \end{proof}


     \item (5 pts.) What is the smallest (in magnitude) IEEE 754-1985 128-bit
       floating point $\mathbb{F}_{128}$ perturbation
       $\varepsilon_{\mathbb{F}_{128}} \in \mathbb{F}_{128} \backslash \{0\}$?
       \begin{proof}[Solution]
       IEEE 754-1985 128-bit numbers are represented using:
       $$r = (-1)^s\cdot 2^{c-16383}\cdot(1+f),\ c = \sum_{n=0}^{14}c_n2^n,\ f = \sum_{k=0}^{111}\frac{m_k}{2^{112-k}}$$
       So the smallest positive number that could be represented would be when $s = 0,\ c = 0\ (c_n = 0,\ \forall n \in \bracks{0,\ 1,\ \cdots,\ 14})$, and $f = 2^{-112}\ (m_0 = 1,\ m_k = 0,\ \forall k \in \bracks{1,\ 2, \cdots,\ 111})$; so let $r_{\text{min}}$ be the smallest  IEEE 754-1985 128-bit number, then:
       $$r_{\text{min}} = (-1)^0\cdot 2^{-16383}\cdot \parens{1 + 2^{-112}} = 2^{-16383} + 2^{-16495} \approx 2^{-16383}$$
       There exists $d \in \mathbb{R}: 10^d = 2^{-16383}$. So we now solve for $d$:
       $$10^d = 2^{-16383} \Rightarrow d = \log_{10}\parens{2^{-16383}} = -16383\log_{10}(2) \approx -4931$$
       So $r_{\text{min}} \approx 10^{-4931}$.
       \end{proof}
       
     \end{enumerate}

   \item
     \begin{enumerate}
     \item (5 pts.) For the specific function $f(x) = \left\{
         \begin{array}{rl} -55, & x<0.5 \\ 55, & x \ge 0.5 \end{array}
       \right.$ where $x\in\mathbb{R}$ is a \bfit{real variable} in
       the interval $[0,1]$, what is the relative condition number
       $\kappa(x)$, for all values of $x$ (\emph{i.e.}\/ as a function
       of $x$)?
       \begin{proof}[Solution]
       $\forall x \in [0, 0.5),\ f(x) = -55 \implies \kappa(x) = 0$, and $\forall x \in (0.5, 1],\ f(x) = 55 \implies \kappa(x) = 0$; but at $x = 0.5,\ \kappa(x) \to \infty$, as any small perturbation $\Delta x \to 0$ of $x$ leads to massive changes in $f$. So:
       $$\kappa(x) = \left\{
         \begin{array}{rl} 0, & x \in [0,0.5)\cup(0.5,1] \\ \infty, & x = 0.5 \end{array}
       \right.$$
       \end{proof}


     \item (5 pts.) For the specific function $f(x) = \left\{
         \begin{array}{rl} -55, & x<0.5 \\ 55, & x\ge 0.5 \end{array} \right.$
       where $x\in\mathbb{F}_{64}$ is a \bfit{64-bit floating point variable}
       in the interval $[0,1]$, what is the relative condition number
       $\kappa(x)$, for all values of $x$ (\emph{i.e.}\/ as a function of
       $x$)?
       \begin{proof}[Solution]
       Like (i), $\forall x \in [0, 0.5),\ f(x) = -55 \implies \kappa(x) = 0$, and $\forall x \in (0.5, 1],\ f(x) = 55 \implies \kappa(x) = 0$; but at $x = 0.5$ we need to consider the smallest perturbation that can be made on $x$. As $x \in \mathbb{F}_{64}$, the smallest $\Delta x$ could be is $\Delta x \approx 10^{-308}$, so:
       \begin{align*}
       \kappa\parens{0.5} &\approx \sup_{\Delta x}\bracks{\frac{\abs{f\parens{0.5 + \Delta x} - f(0.5)}}{\abs{f(0.5)}}\cdot\frac{\abs{0.5}}{\abs{\Delta x}}}\\
       &= \frac{\abs{f\parens{0.5 + 10^{-308}} - f(0.5)}}{\abs{f(0.5)}}\cdot\frac{\abs{0.5}}{\abs{10^{-308}}}\\
       &= \frac{\abs{55 - (-55)}}{55}\cdot\frac{0.5}{10^{-308}}\\
       &= \frac{2(55)}{55}\cdot\frac{0.5}{10^{-308}}\\
       &= \frac{55}{55}\cdot\frac{1}{10^{-308}}\\
       &= \frac{1}{10^{-308}}\\
       &= 10^{308}
       \end{align*}
       Finally:
       $$\kappa(x) = \left\{
         \begin{array}{rl} 0, & x \in [0,0.5)\cup(0.5,1] \\ 10^{308}, & x = 0.5 \end{array}
       \right.$$
       \end{proof}



     \item (5 pts.) For the specific function $f(x) = \left\{
         \begin{array}{rl} -55, & x \le 9,989,224 \\ 55, & x\ge
           9,989,225
         \end{array} \right.$ where $x\in\mathbb{Z}$ is an \bfit{integer
         variable}, what is the relative condition number $\kappa(x)$,
       for all values of $x$ (\emph{i.e.}\/ as a function of $x$)?
       \begin{proof}[Solution]
	$\forall x < 9,898,224,\ f(x) = -55 \implies \kappa(x) = 0$, and $\forall x > 9,898,224 ,\ f(x) = 55 \implies \kappa(x) = 0$; but for $x = 9,898,224$ or $x = 9,898,225$ we need to consider the smallest perturbation that can be made on $x$. As $x \in \mathbb{Z}$, the smallest $\Delta x$ could be is $\Delta x = 1$, so:
       \begin{align*}
       \kappa\parens{9,898,224} &\approx \sup_{\Delta x}\bracks{\frac{\abs{f\parens{9,898,224 + \Delta x} - f(9,898,224)}}{\abs{f(9,898,224)}}\cdot\frac{\abs{9,898,224}}{\abs{\Delta x}}}\\
       &= \frac{\abs{f\parens{9,898,225} - f(9,898,224)}}{\abs{f(9,898,224)}}\cdot\frac{\abs{9,898,224}}{\abs{1}}\\
       &= \frac{\abs{55 - (-55)}}{55}\cdot\frac{9,898,224}{1}\\
       &= \frac{2(55)}{55}\cdot9,898,224\\
       &= 2\cdot9,898,224\\
       &= 19,796,448\\
       \end{align*}
       Finally:
       $$\kappa(x) = \left\{
         \begin{array}{rl} 0, & x < 9898224 \text{ or } x > 9898225 \\ 19796448, & x \in \bracks{9898224\ , 9898225} \\
          \end{array}
       \right.$$
       \end{proof}

     \end{enumerate}


   \end{enumerate}

 \item \bfit{Least Squares Problems} --- Derive the matrix least squares
   problem for fitting a data set $\{y(t_i),t_i\}_{i=1,\dots,m}$ by
   \begin{enumerate}
   \item (5 pts.) a constant,
   \begin{proof}[Solution]
   We would have
   $$A = \parens{\begin{matrix}
   1\\
   \vdots\\
   \vdots\\
   1
   \end{matrix}}_{m\times 1},\ \vec{c} = \parens{
   \begin{matrix}
   c_0
   \end{matrix}
   }_{1\times 1},\ \vec{y} = \parens{
   \begin{matrix}
   y(t_1)\\
   \vdots\\
   \vdots\\
   y(t_m)
   \end{matrix}
   }_{m\times 1}
   $$
   And would need to choose $c_0$ to minimize 
   $$\gnorm{A\vec{c} - \vec{y}}_2^2$$
   \end{proof}

     
     
   \item (5 pts.) a straight line $z(t)=a+bt$.
	\begin{proof}[Solution]
   We would have
   $$A = \parens{\begin{matrix}
   1 & t_1\\
   \vdots & \vdots \\
   \vdots & \vdots \\
   1 & t_m
   \end{matrix}}_{m\times 2},\ \vec{c} = \parens{
   \begin{matrix}
   c_0 \\
   c_1
   \end{matrix}
   }_{2\times 1},\ \vec{y} = \parens{
   \begin{matrix}
   y(t_1)\\
   \vdots\\
   \vdots\\
   y(t_m)
   \end{matrix}
   }_{m\times 1}
   $$
   and would need to choose $c_0$ and $c_1$ to minimize 
   $$\gnorm{A\vec{c} - \vec{y}}_2^2$$
   \end{proof}
     
     
   \item (5 pts.) What is the solution in case (a)?
   \begin{proof}[Solution]
   The choice of $c_0$ that would minimize $\gnorm{A\vec{c} - \vec{y}}_2^2$ would be $c_0 = \text{mean}\bracks{y(t_i)}_{i = 1, \cdots, m}$.
   \end{proof}
	 
     
     
   \item (5 pts.) Explain why, in general, solving the least squares
     problem using the normal equations is not such a good idea.
     \begin{proof}[Solution]
     The ``Normal'' way of solving these sorts of problem is generally faster; but due to roundoff errors may not end up giving you an accurate or useful result, especially for large values of $m$ as more iterations include more possibilities for roundoff errors to occur and compound.
     \end{proof}

     
   \item (5 pts.) Suggest one alternative approach, what are its advantages
     and disadvantages?
     \begin{proof}[Solution]
     An SVD approach is preferable to the normal equations and other alternatives for least squares fit problems, as you're usually dealing with over-determined systems. Typically in these cases, $m >> n$ where $m$ is the number of data points with each data point being of dimension $n$ and we have that SVD isn't significantly more expensive computationally than other approcaches as it would be for $m \approx n$. When considered alongside SVD's superior stability, it becomes an attractive choice for least squares fit problems. 
     \end{proof}


   \item Download the file\\
     \texttt{[clickable]}\\
     \hspace*{1em}\parbox[t]{\linegoal}{\texttt{\scriptsize\url{https://github.com/CSSEGISandData/COVID-19/raw/master/csse_covid_19_data/csse_covid_19_time_series/time_series_covid19_confirmed_global.csv}}}\\
     \texttt{[/clickable]}

     Extract the data for the US (it's a single line starting with
     ``\texttt{,US,37.0902,-95.7129,}'' followed by the Confirmed Cases (since
     1/22 until your download date).\\

     \begin{enumerate}

     \item (5 pts.) Plot the data on a log-scale\\

     \item (5 pts.) Identify period of exponential growth (a period where the
       growth looks linear on the log scale)\\

       Specify day range: February 23rd through April 9th
       \vfill

     \item (5 pts.) Find the best log-linear (that is linear fit to the
       $\mathrm{log}_{10}$ of the data,) and plot that fit on the same plot.\\


     \item (5 pts.) Use the model to extrapolate 7 days (add to the plot);
       what is the count?\\

       Specify extrapolated case count: At end of 7 day extrapolation, April 16th, the case count is approximately $1,175,000$.
       
       [Attach plot at the end of the exam]\\[0.5em]
       [Attach code at the end of the exam]\\[0.5em]
       Note: For most meaningful results, plot everything in ONE plot.


     \end{enumerate}
     
   \end{enumerate}

   \item \bfit{Stability and Backward Stability} --- Let $f(x)$ denote
   the exact solution to the exact problem, let $\tilde{f}(x)$ denote
   the solution computed by an algorithm.

   \begin{enumerate}

     \item (5 pts.) Complete the statement: \textit{``We say that an
     algorithm $\tilde{f}$ for a problem $f(x)$ is \emph{stable} if
     for each $x\in X$}
     $$\frac{\gnorm{\tilde{f}\parens{\vec{x}} - f\parens{\tilde{\vec{x}}}}}{\gnorm{f\parens{\tilde{\vec{x}}}}} = \BigO{\vep_{\text{mach}}}$$
     \textit{for some $\tilde{x}$ with}
     $$\frac{\gnorm{\tilde{\vec{x}} - \vec{x}}}{\gnorm{\vec{x}}} = \BigO{\vep_{\text{mach}}}$$
     



     \item (5 pts.)  Complete the statement: \textit{`` We say that an
     algorithm $\tilde{f}$ for a problem $f(x)$ is \emph{backward
     stable} if for each $x\in X$}
	$$\tilde{f}\parens{\vec{x}} = f\parens{\tilde{\vec{x}}}$$
     
     \textit{for some $\tilde{x}$ with}
     $$\frac{\gnorm{\tilde{\vec{x}} - \vec{x}}}{\gnorm{\vec{x}}} = \BigO{\vep_{\text{mach}}}$$
     
     
   \item (5 pts.) Show that the obvious algorithm for solving (for $x$) the
     1-by-1 system $rx=b$ is backward stable if executed in matlab on a
     modern-day computer.  Explain your notation and any results / definitions
     / axioms you are invoking.
     \begin{proof}
     Naturally, we may solve for $x$ like so: $x = \frac{b}{r}$. We may express the computational solution to this as $\tilde{x} = \fl{x} = b \oslash r = (b/r)\cdot(1 + \vep):\abs{\vep} \leq \vep_{\text{mach}}$. So define
     $$x = f(b) = \frac{b}{r},\ \tilde{x} = \tilde{f}(b) = (b/r)\cdot(1+\vep)$$ 
     Now, let $\tilde{b} = b(1 + \vep_{\text{mach}})$, then:
     $$f(\tilde{b}) = \frac{\tilde{b}}{r} = \frac{b}{r}\cdot\parens{1 + \vep_{\text{mach}}} = \tilde{f}\parens{b}$$
     and we see that
     $$\frac{\abs{\tilde{b} - b}}{\abs{b}} = \frac{\abs{b\parens{1+\vep_{\text{mach}}} - b}}{\abs{b}} = \frac{\abs{b}\cdot \abs{1+\vep_{\text{mach}} - 1}}{\abs{b}} = \abs{\vep_{\text{mach}}} = \BigO{\vep_{\text{mach}}}$$
     So we have that $\tilde{f}$ is backwards stable.
     
     \end{proof}


   \item (5 pts.) Explain how backward stability, the condition number, and
     the available ``computational precision'' limit how well we can solve a
     problem numerically.  (You may state a theorem, but it is not necessary
     (but highly recommended)).
     \begin{proof}[Solution]
     Given the definition of Backward stability, we may interpret it as a way of verifying that the output of the algorithm we have created is consistent with the problem we wanted solved up to some small difference in inputs; and an understanding of the available computational precision allows us to quantify that small difference. While it's called the condition number of a algorithm or function, what it really measures is the sensitivity of the algorithm to inputs; where we have (somewhat) arbitrarily decided how much precision is good enough for a given application. Taken together, they ultimately allow us to quantify our ability to solve problems numerically.
     \end{proof}


   \end{enumerate}


\item (5\,pts.)
  The following figure was published in \emph{The Economist} (March
  18, 2006).\\

  \centerline{
    \epsfig{
      file=TheEconomist_2006-03-18.eps,
      width=0.75\linewidth
    }
  }

  The sloped line is most likely a least squares fit (of some kind).
  Do you have any comments?  (Do NOT use more space than this page and
  the back!  Do not feel compelled to fill the space; sometimes less
  is more!)
  \begin{proof}[Opinion]
  While it has the qualities of a least squares fit, the plot also has some extra ``information'' that clouds what it might actually be representing. If the data was simply a set of ordered pairs $\parens{x_i, y_i}$, then fair enough; but each point also has a component of size that represent another piece of data (size proportional to some $z_i$), and it's unclear if that third data point has factored into the least squares fit in some way. The graph is only 2-dimensional after all, and maybe it would be easier to understand (though, maybe harder to print) if the plot had a $z$-axis.
  \end{proof}



\item (0\,pts.) \bfit{Conditioning, Stability, and
    Accuracy of the US Presidential Election} --- Use your toolbox
  from \textbf{***\,this class\,***} to comment on the election
  process.  You may choose to limit the scope to a single-state
  scenario.  Clearly define the concepts you need for your discussion.
  (If you are completely unfamiliar with the US Presidential Election,
  you may comment on \textbf{\textit{any}} democratic election process
  (with more than one candidate), just make sure you clearly define
  the rules of the chosen process.)

  \begin{itemize}
  \item This question is \textbf{not due}, but still worth thinking
    about on a dark and stormy night.
  \end{itemize}
\end{enumerate}

\subsection*{Plots and code for Problem 2}
\begin{figure}[h]
\centering
\includegraphics[width=.9\linewidth]{"Log10CasesVsDate".jpg}
\caption{The Blue shaded region corresponds to the dates for which the number of confirmed cases on the log scale most resemble exponential growth; from February 23rd through April 9th. Extrapolation using a linear best fit predicts about 1.17 million confirmed cases by April 16th.}
\end{figure}
\pagebreak
\begin{lstlisting}[language=python]
import pandas as pd
import numpy as np
import matplotlib.pyplot as plt
plt.rcParams['savefig.format'] = 'jpg'

data = np.array(pd.read_csv(
	"time_series_covid19_confirmed_global.csv"))

fecha = np.array(pd.read_csv(
	"time_series_covid19_confirmed_global.csv", 
									header = None))[0, 4:]
fecha7 = np.append(fecha, ['4/10/20','4/11/20','4/12/20',
	'4/13/20','4/14/20','4/15/20','4/16/20'])
USAdata = data[225,4:]
USAlog = np.array([np.log10(x) for x in USAdata])
date = np.arange(USAdata.shape[0])
date7 = np.arange(date.size + 7)

p = np.polyfit(date, USAlog, 1)

abline = p[0]*date7 + p[1]

print(10**abline[-1])

interval = np.array([True if i > 31 else False for i in date])
print(fecha[32])

fig, ax = plt.subplots(figsize=(18,12))
ax.plot(fecha, USAlog, 'ko-')
ax.plot(fecha7, abline, 'r-', label = "Linear Best Fit")
ax.set_ylabel("$\log_{10}(C)$", size =20)
ax.set_xlabel("Date", size = 20)
ax.fill_between(date, np.zeros(date.size), USAlog,
				where= interval, color ='blue')
ax.tick_params(length=6, width=2, labelsize=20)
ax.set_ylim(0,)
temp = ax.xaxis.get_ticklabels()
temp = list(set(temp) - set(temp[::21]))
for label in temp:
	label.set_visible(False)
ax.set_title("Confirmed COVID-19 cases $(C)$ in the USA",
 	size = 20)
ax.legend(loc=('best'))
fig.savefig("Log10CasesVsDate")
\end{lstlisting}
\end{document}

