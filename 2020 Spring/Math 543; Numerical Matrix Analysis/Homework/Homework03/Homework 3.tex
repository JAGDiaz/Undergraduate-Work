\documentclass{article}
\usepackage{amsmath}
\usepackage{amssymb}
\usepackage{bm}
\usepackage{amsthm}
\usepackage{enumerate}
\usepackage{graphicx}
\usepackage{psfrag}
\usepackage{color}
\usepackage{url}
\usepackage{listings}
\usepackage{xcolor}

\definecolor{codegreen}{rgb}{0,0.5,0}
\definecolor{codewhite}{rgb}{1,1,1}
\definecolor{codegray}{rgb}{0.5,0.5,0.5}
\definecolor{codepurple}{rgb}{0.58,0,0.82}
\definecolor{codeblack}{rgb}{0,0,0}
\definecolor{codeorange}{rgb}{0.8,0.4,0}

\lstdefinestyle{mystyle}{
    backgroundcolor=\color{codewhite},   
    commentstyle=\color{codegray},
    keywordstyle=\color{codegreen},
    numberstyle=\color{codegray},
    stringstyle=\color{codeorange},
    basicstyle=\ttfamily ,
    breakatwhitespace=false,         
    breaklines=true,                 
    captionpos=b,                    
    keepspaces=true,                 
    numbers=left,                    
    numbersep=5pt,                  
    showspaces=false,                
    showstringspaces=false,
    showtabs=false,                  
    tabsize=4
}
\lstset{style=mystyle}

\setlength{\hoffset}{-1in}
\addtolength{\textwidth}{2in}
\setlength{\voffset}{-1in}
\addtolength{\textheight}{1.0in}

\newcommand{\be}{\begin{enumerate}}
\newcommand{\ee}{\end{enumerate}}

\begin{document}
	\begin{large}
	\begin{center}
		\textbf{Spring 2020, Math 543:\ Homework 3} \\
		\textbf{Due:\ Friday, February 28th, 2020} \\
		\textbf{Joseph Diaz: 819947915}
	\end{center}
\noindent\makebox[\linewidth]{\rule{\paperwidth}{0.4pt}}
	\section*{QR-factorization}
	Implement the reduced \textbf{QR-factorization} by classical Gram-Schmidt Orthogonalization.\\
	Write a function which given a matrix $A \in \mathbb{C}^{m\times n}$ computes $Q \in \mathbb{C}^{m\times n}$ and $R \in \mathbb{C}^{n\times n}$. \\
	Validate your function - test that $A - QR \approx 0$, that $Q$ is \textit{unitary} and $R$ \textit{upper triangular}. Compare the result with the built-in version of the QR-factorization for your language. Can you find a matrix where your QR-factorization breaks?\\
\noindent\makebox[\linewidth]{\rule{\paperwidth}{0.4pt}}
Sample output:
\begin{lstlisting}
A is a 120 by 10, with tolerance 1.000e-12

My QR: Infinity norm of A - QR: 0.000e+00
My Q is unitary:  False
My R is upper triangular:  True
--------------------------------------------------------------------
A is a 100 by 100, with tolerance 1.000e-12

My QR: Infinity norm of A - QR: 2.731e-14
My Q is unitary:  True
My R is upper triangular:  True

Python's QR: Infinity norm of A - QR: 8.419e-14
Python's Q is unitary:  True
Python's R is upper triangular:  True
--------------------------------------------------------------------
A is a 54 by 3, with tolerance 1.000e-13

My QR: Infinity norm of A - QR: 4.441e-16
My Q is unitary:  True
My R is upper triangular:  True

Python's QR: Infinity norm of A - QR: 8.882e-16
Python's Q is unitary:  True
Python's R is upper triangular:  True
--------------------------------------------------------------------
A is a 666 by 37, with tolerance 1.000e-14

My QR: Infinity norm of A - QR: 7.799e-15
My Q is unitary:  True
My R is upper triangular:  True

Python's QR: Infinity norm of A - QR: 3.535e-14
Python's Q is unitary:  True
Python's R is upper triangular:  True
--------------------------------------------------------------------
A is a 32 by 100, with tolerance 1.000e-15

My QR: Infinity norm of A - QR: 2.047e-14
My Q is unitary:  False
My R is upper triangular:  True

Python's QR: Infinity norm of A - QR: 7.323e-14
Python's Q is unitary:  True
Python's R is upper triangular:  False
--------------------------------------------------------------------
\end{lstlisting}
As we can see, the infinity-norm of the difference of our $A$ and $QR$ matrices is very low in each case, even on the same order of magnitude as that of Python's QR factorization. In addition, we have Python subroutines to check if $Q$ is unitary and if $R$ is upper triangular. The comparisons in these last 2 subroutines use a tolerance value (some small number on the order $1e-10$) instead of checking equality with zero where it makes sense to do so to avoid problems with floating-point number comparisons. The only matrices for which the QR factorization failed were matrices with more columns than rows and zero matrices of any dimension ($Q$ could never be unitary in these cases). \\
\noindent\makebox[\linewidth]{\rule{\paperwidth}{0.4pt}}
Source code for program:
\begin{lstlisting}[language=python]
#! /usr/bin/python3

import numpy as np
from matplotlib import pyplot as plt

def classic_GS(A):
	r = np.zeros((A.shape[1], A.shape[1]))
	q = np.zeros(A.shape)
	for j in range(A.shape[1]):
		v = np.copy(A[:,j], order='K')
		for i in range(j):
			r[i,j] = np.dot(q[:,i],v)
			v = v - r[i,j]*q[:,i]

		r[j,j] = np.linalg.norm(v)
		if(r[j,j] < 1e-12):
			r[j,j] = 1e-12
		q[:,j] = v/r[j,j]

	return q, r

def matrix_Print(A):
	for i in A:
		print(i)

def is_Up_Tri(A, tol):
	if(A.shape[0] != A.shape[1]):
		return False
	for i in range(A.shape[0]):
		for j in range(i + 1, A.shape[0]):
			if(np.abs(A[j,i]) > tol):
				return False
	return True

def is_Low_Tri(A, tol):
	if(A.shape[0] != A.shape[1]):
		return False
	for i in range(A.shape[0]):
		for j in range(i + 1, A.shape[0]):
			if(np.abs(A[i,j]) > tol):
				return False
	return True

def is_Unitary(A, tol):
	for i in range(A.shape[1]):
		if(np.abs(1-np.linalg.norm(A[:,i], 2)) > tol):
			return False
		for j in range(i+1,A.shape[1]):
			if(np.abs(np.dot(A[:,i],A[:,j])) > tol):
			  	return False
	return True
	
A = np.zeros((120,10))
print("A is a %d by %d, with tolerance %1.3e" 
		% (A.shape[0], A.shape[1], 1e-12))
q,r = classic_GS(A)
print("\nMy QR: Infinity norm of A - QR: %1.3e" 
		% np.linalg.norm(A-q@r, np.inf))
print("My Q is unitary: ", is_Unitary(q,1e-12))
print("My R is upper triangular: ", is_Up_Tri(r, 1e-12))
print("")

for p in pSizes:
	print("A is a %d by %d, with tolerance %1.3e" % (p))

	Ar = np.random.randn(p[0],p[1])
	q,r = classic_GS(Ar)
	q1, r1 = np.linalg.qr(Ar, 'reduced')

	print("\nMy QR: Infinity norm of A - QR: %1.3e" 
			% np.linalg.norm(Ar-q@r, np.inf))
	print("My Q is unitary: ", is_Unitary(q,p[2]))
	print("My R is upper triangular: ", is_Up_Tri(r, p[2]))

	print("\nPython's QR: Infinity norm of A - QR: %1.3e" 
			% np.linalg.norm(Ar-q1@r1, np.inf))
	print("Python's Q is unitary: ", is_Unitary(q1,p[2]))
	print("Python's R is upper triangular: ", is_Up_Tri(r1,p[2]))
	print("")
\end{lstlisting}
\noindent\makebox[\linewidth]{\rule{\paperwidth}{0.4pt}}
	
	\end{large}
\end{document}
