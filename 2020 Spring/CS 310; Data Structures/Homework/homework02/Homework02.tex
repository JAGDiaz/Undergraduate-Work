\documentclass{article}
\usepackage{amsmath}
\usepackage{amssymb}
\usepackage{bm}
\usepackage{amsthm}
\usepackage{enumerate}
\usepackage{graphicx}
\usepackage{psfrag}
\usepackage{color}
\usepackage{url}
\usepackage{listings}
\usepackage{xcolor}

\definecolor{codegreen}{rgb}{0,0.5,0}
\definecolor{codewhite}{rgb}{1,1,1}
\definecolor{codegray}{rgb}{0.5,0.5,0.5}
\definecolor{codepurple}{rgb}{0.58,0,0.82}
\definecolor{codeblack}{rgb}{0,0,0}
\definecolor{codeorange}{rgb}{0.8,0.4,0}

\lstdefinestyle{mystyle}{
    backgroundcolor=\color{codewhite},   
    commentstyle=\color{codegray},
    keywordstyle=\color{codegreen},
    numberstyle=\color{codegray},
    stringstyle=\color{codeorange},
    basicstyle=\ttfamily ,
    breakatwhitespace=false,         
    breaklines=true,                 
    captionpos=b,                    
    keepspaces=true,                 
    numbers=left,                    
    numbersep=5pt,                  
    showspaces=false,                
    showstringspaces=false,
    showtabs=false,                  
    tabsize=4
}
\lstset{style=mystyle}


\setlength{\hoffset}{-0.5in}
\addtolength{\textwidth}{1.0in}
\setlength{\voffset}{-0.5in}
\addtolength{\textheight}{1.0in}
\newcommand{\be}{\begin{enumerate}}
\newcommand{\ee}{\end{enumerate}}
\newcommand{\BigO}[1]{\ensuremath\mathcal{O}\left(#1\right)}
\newcommand{\il}[1]{\lstinline!#1!}

\begin{document}
	\begin{center}
		\textbf{Spring 2020, CS 310:\ Homework 2} \\
		\textbf{Due:\ Friday, February 21st, 2020} \\
		\textbf{Joseph Diaz: 819947915}
	\end{center}
\noindent\makebox[\linewidth]{\rule{\paperwidth}{0.4pt}}
\be[1)]
	\item $\textbf{(2+2)}$ $\BigO{n}$ and Run Time Analysis:\\ Describe the worst case running time of the following pseudocode functions in $\BigO{n}$ notation in terms of the variable $n$. Showing your work is not required
		\be[a)]
		\item \textit{}\\
		\begin{lstlisting}[language=java]
void silly(int n, int x, int y)
{ 
	if (x < y)
	{ 
		for (int i = 0; i < n; ++i)
			for (int j = 0; j < n * i; ++j) 
				System.out.println("y = " + y); 
	} 
	else 
	{ 
		System.out.println("x = " + x); 
	} 
} \end{lstlisting}
$$T(n) = \frac{n^3 + 4n^2 + 3n}{2} + n^3 +n^2 +2 = \BigO{n^3}$$
		\item \textit{}\\
		\begin{lstlisting}[language=java]
void silly(int n)
{ 
	for (int i = 0; i < n * n; ++i) 
	{	 
		for (int j = 0; j < n; ++j) 
		{ 
			for (int k = 0; k < i; ++k) 	
				System.out.println("k = " + k); 
			for (int m = 0; m < 100; ++m) 
				System.out.println("m = " + m); 
		} 
	} 
} \end{lstlisting} 
$$T(n) = \frac{n^5+n^4}{2} + 5151n^3 + 10,310 = \BigO{n^5}$$
		\ee
		
	\item $\textbf{(4+4)}$ Given an \textbf{array implementation} of a STACK that holds integers. (Refer to Array Implementation of Stack on Blackboard)  
Stack is initialized in the Driver class as follows:
	
	\il{StackArray<Integer> intStack = new StackArray<Integer>();}
	
	\be[a)]
	\item Write a new public void member function called \il{countPosNeg()}  

This function counts and displays the number of positive integers and number of negative integers in \il{intStack}. \il{intStack} must be returned to it’s original state after counting.
		\be[i)]
		\item Write the java code for the function.

		\il{countPosNeg()} source code:
\begin{lstlisting}[language=java]
public void countPosNeg()
{
	int temp, posiCount = 0, negaCount = 0;
	StackArray<Integer> tempStack =new StackArray<Integer>();
	while(!this.isEmpty())
	{
		temp = (int)this.pop();
		
		if(temp > 0)
			posiCount++;
		else if(temp < 0)
			negaCount++;
			
		tempStack.push(temp);
	}
	
	while(!tempStack.isEmpty())
		this.push((E)tempStack.pop());
		
	tempStack = null;
	System.out.printf("There are %d positive numbers and %d", posiCount, negaCount);
	System.out.println(" negative numbers in this stack.");
}
\end{lstlisting}

		
		\item Derive the time function and calculate $\BigO{n}$ to the tightest upperbound of the function \il{countPosNeg()}.

		The time function of \il{countPosNeg()} is constrained by the fact that each \il{StackArray} object limits its capacity to just 10 elements. So the time function of \il{countPosNeg()} is:
$$T(n) \approx 80 = \BigO{1}$$
and, therefore, a tight upper bound would be something like $G(n) = 90, \forall n \in \mathbb{N}$.

		
		\item Write a line of code that shows how the \il{main()} would use this new member function. 

	\begin{lstlisting}[language=java]
public static void main(String[] args)
{
	StackArray<Integer> intStack = new StackArray<Integer>();
	.
	. // non-existent code for filling intStack
	.
	intStack.countPosNeg();
}
\end{lstlisting}

		\ee

	\item Write a \textbf{recursive (not iterative)} Java method\\ \il{public boolean sameStack(StackArray<E> s2)}\\ to test whether two stacks contain the same elements. The elements stored in the stack are integers. 
One Stack object calls this method with a second stack object, from inside the \il{main()}\\
\il{stack1.sameStack(stack2);}\\
The function will return true if the stacks contain the same elements and false otherwise.\\
\il{sameStack()} source code:
\begin{lstlisting}[language=java]
public boolean sameStack(StackArray<E> s2)
{
	if(this.top != s2.top)
		return false;

	if(this.isEmpty() && s2.isEmpty())
		return true;
			
	boolean temp;
	if(this.peek() == s2.peek())
	{
		E temp1 = this.pop();
		E temp2 = s2.pop();
		temp = this.sameStack(s2);
		this.push(temp1);
		s2.push(temp2);
	}
	else
		return false;

	return temp;
}
public E peek()
{
	return this.items[this.top];
}
\end{lstlisting}
	\ee
	\item $\textbf{(4+4)}$ Given an array implementation of a CIRCULAR QUEUE that holds integers.  (Refer to Array Implementation of Queue on Blackboard)  
Using the following input in order: \\
\il{4	5	 67	 89	 21	 3	 0	 76	34	12}\\
	The main function in the Driver Class is as follows:
	\begin{lstlisting}[language=java]
public static void main(String args[])
{
	QueueArray<Integer> numqueue = new QueueArray<Integer>();
	int i = 0;

   	Scanner myObj = new Scanner(System.in); 
	do
	{
	    int choice = Integer.parseInt(myObj.next());
	   	if (choice < 35) 
	   		numqueue.enqueue(choice);
	   	i++;
	}while (i<10);

	do
	{
		System.out.println("Dequeued from the Queue: " 
						+ numqueue.dequeue());
	}while (numqueue.isEmpty()==false);
}
	\end{lstlisting}
	\be[a)]
	\item What is the output of the Main function?
	\begin{lstlisting}[language=java]
	4
	5
	67
	89
	21
	3
	0
	76
	34
	12
	Dequeued from the Queue: 4
	Dequeued from the Queue: 5
	Dequeued from the Queue: 21
	Dequeued from the Queue: 3
	Dequeued from the Queue: 0
	Dequeued from the Queue: 34
	Dequeued from the Queue: 12
	\end{lstlisting}
	\item Derive the time function and calculate the $\BigO{n}$ to the tightest upperbound of the function \il{main()}.
	\begin{proof}[Time function]
	Due to the constant number of times that each loop is iterated over, the time function of this program is
	$$T(n) \approx 50 = \BigO{1} $$
	and a tight upper bound would be something like 
	$$$$
	\end{proof}
	\ee
\ee
\noindent\makebox[\linewidth]{\rule{\paperwidth}{0.4pt}}
\pagebreak
\section*{\il{StackArray.java} Source Code}
\begin{lstlisting}[language=java]
package ArrayImplementations;

public class StackArray<E> 
{
			
		private int top=-1;
		private static final int MAX_ITEMS = 10;
		private E items[];

		@SuppressWarnings("unchecked")
		public StackArray() {
			items = (E[]) new Object[MAX_ITEMS];
			System.out.println("Stack Created!");
		}

		public void push(E e) {
			if (isFull()==true) {
			   System.out.println("Stack Full!");
			}
			else{
				top=top+1;
				items[top] = e;
			}
		}


		public E pop() {
			 if (isEmpty()==true) {
				   System.out.println("Stack Empty!");
				}
			 else{
			
			E e = (E) items[top];
			items[top] = null;
			top = top-1;
			return e;
			 }
			return null;
		}

		public boolean isFull() {
			 if (top == items.length-1) {
				 return true;
			 }
			 return false;
		}

		public boolean isEmpty(){
			 if (top==-1) {
				 return true;
			 }
			 return false;
		}

		public E peek()
		{
			return this.items[this.top];
		}

		public void countPosNeg()
		{
			int temp, posiCount = 0, negaCount = 0;
			StackArray<Integer> tempStack = new StackArray<Integer>();
			while(!this.isEmpty())
			{
				temp = (int)this.pop();
				
				if(temp > 0)
					posiCount++;
				else if(temp < 0)
					negaCount++;
					
				tempStack.push(temp);
			}
			
			while(!tempStack.isEmpty())
				this.push((E)tempStack.pop());
				
			tempStack = null;
			System.out.printf("There are %d positive numbers and %d", posiCount, negaCount);
			System.out.println(" negative numbers in this stack.");
		}

		public boolean sameStack(StackArray<E> s2)
		{
			if(this.top != s2.top)
				return false;
				
			if(this.isEmpty() && s2.isEmpty())
				return true;
			
			boolean temp;
			if(this.peek() == s2.peek())
			{
				E temp1 = this.pop();
				E temp2 = s2.pop();
				temp = this.sameStack(s2);
				this.push(temp1);
				s2.push(temp2);
			}
			else
				return false;

			return temp;
		}

		@Override
		public String toString()
		{
			System.out.println("Array:");
			System.out.print("{");
			 for(int i = 0; i < items.length ;i++) {
				 System.out.print(items[i]+" ");
		}
			 System.out.print("}");
			return "";
		}
		public static void main(String[] args) 
		{
			
			// Code reference for countPosNeg method
			StackArray<Integer> intStack = new StackArray<Integer>();
				intStack.push(10);
				intStack.push(10);
				intStack.push(30);
				intStack.push(-40);
			// call countPosNeg here
				intStack.countPosNeg();
				System.out.println("intStack before counting");
				System.out.println(intStack);
			
			 System.out.println("intStack after counting");
				System.out.println(intStack);
				
			// Code reference for sameStack method
		   StackArray<Integer> stack = new StackArray<Integer>();
		   StackArray<Integer> stack2 = new StackArray<Integer>();

			stack.push(10);
			stack.push(20);
			stack.push(30);
			stack.push(40);
			stack2.push(10);
			stack2.push(20);
			stack2.push(30);
			stack2.push(40);
			 
		   
			System.out.println(stack);
			System.out.println(stack2);
			
			//Calling comparison method
		   System.out.println(stack.sameStack(stack2)); 
		}

				
				
}

\end{lstlisting}
\end{document}
