\documentclass{article}
\usepackage{amsmath}
\usepackage{amssymb}
\usepackage{bm}
\usepackage{amsthm}
\usepackage{enumerate}
\usepackage{graphicx}
\usepackage{psfrag}
\usepackage{color}
\usepackage{url}
\usepackage{listings}

\usepackage{xcolor}
 
\definecolor{codegreen}{rgb}{0,0.6,0}
\definecolor{codegray}{rgb}{0.5,0.5,0.5}
\definecolor{codepurple}{rgb}{0.58,0,0.82}
\definecolor{backcolour}{rgb}{0.95,0.95,0.92}
 
\lstdefinestyle{mystyle}{
    backgroundcolor=\color{backcolour},   
    commentstyle=\color{codegray},
    keywordstyle=\color{codegreen},
    numberstyle=\tiny\color{magenta},
    stringstyle=\color{codepurple},
    basicstyle=\ttfamily\footnotesize,
    breakatwhitespace=false,         
    breaklines=true,                 
    captionpos=b,                    
    keepspaces=true,                 
    numbers=left,                    
    numbersep=5pt,                  
    showspaces=false,                
    showstringspaces=false,
    showtabs=false,                  
    tabsize=4
}
 
\lstset{style=mystyle}

\setlength{\hoffset}{-0.5in}
\addtolength{\textwidth}{1.0in}
\setlength{\voffset}{-0.5in}
\addtolength{\textheight}{1.0in}
\newcommand{\BigO}[1]{\ensuremath\mathcal{O}\left(#1\right)}
%\newcommand{\deg}{\text{deg}}

\begin{document}
	\begin{large}
	\begin{center}
		\textbf{Spring 2020, CS 310:\ Homework 1 ReadMe} \\
		\textbf{Due:\ Thursday, February 13th, 2020} \\
		\textbf{Joseph Diaz}
	\end{center}
	\noindent\makebox[\linewidth]{\rule{\paperwidth}{0.4pt}}
\section*{Time Complexity of Interface Methods}
\subsection*{\textit{searchStudent}()}
	Let $n$ be the value stored in \textit{currIndex}.
	\begin{lstlisting}[language=Java, caption=\textit{searchStudent}()]
	public void searchStudent(String id)
	{
		int identity = Integer.parseInt(id); 	// -> 1
		foundFlag = false;						// -> 1
		for(int i = 0; i < currIndex; i++)		// -> 1 + (n + 1) + n
		{
			if(studentArray[i] != null)			// -> n
			{
				if(identity == studentArray[i].getID())	// -> n
				{
					System.out.println("\nStudent found:"); 	// -> 1
					studentArray[i].printInfo(studentArray[i]);	// -> 1
					System.out.println(); 	// -> 1
					foundFlag = true;		// -> 1
					break;					// -> 1
				}
			}
		}
		if(foundFlag == false)				// -> 1
		{
			System.out.printf("Student with ID %d not found.\n", identity);//-> 1
			System.out.print("That ID is available.\n");		// -> 1
		}
	}
\end{lstlisting}
As the code shows, the time function $T(n)$ is given by:
$$T(n) = 4n + 11 = \BigO{n}$$
The executions within the innermost if statement are only counted once, because any situation in which they are executed will lead to the break statement that exits the loop. So the worst case is that is that they are executed on the very last iteration of the loops.
\pagebreak
\subsection*{\textit{dispalAll}()}
Here $n$ is the length of the array \textit{studentArray}.
\begin{lstlisting}[language=Java, caption=\textit{displayAll}()]
public void displayAll()
{
	System.out.println("\nAll Students:"); // -> 1
	for(Student stud : studentArray) // -> n
	{
		if(stud != null) // -> n
		{
			stud.printInfo(stud); // -> n
			System.out.println(); // -> n
		}
	}
}
\end{lstlisting}	
	And so the time complexity of this method is:
	$$T(n) = 4n + 1 = \BigO{n}$$
	This method uses an enhanced for-loop as opposed to a classic one. This allows for fewer operations overall, and simplifies the time complexity.	\\\\
	A tight upper-bound for both time functions would be something like $T(n) = 15n$, as neither time function will ever be greater than that.\\
\noindent\makebox[\linewidth]{\rule{\paperwidth}{0.4pt}}
\section*{Classes}
\subsection*{Person.java}
This class has 2 private fields, for which the class has the appropriate ``getters'' and ``setters''. These are public, unlike the values they access, because otherwise the subclass ``Student'' wouldn't be able to modify or retrieve name or age. In addition, the class has 2 constructors, one of them the default:
\begin{lstlisting}[language=Java]
package homework01.Classes;

public class Person
{
	private String name;
	private int age;
	
	public Person()
	{
		this.name = "";
		this.age = 0;
	}
	public Person(String name, int age)
	{
		this.name = name;
		this.age = age;
	}
	public String getName()
	{
		return this.name;
	}
	public void setName(String name)
	{	
		this.name = name;
	}

	public int getAge()
	{
		return this.age;
	}
	public void setAge(int age)
	{	
		this.age = age;
	}
}

\end{lstlisting}
and these are both used by the subclass for the initialization of a new subclass instance.
\subsection*{Student.java}
This class extends Person and includes another field for an int ID, along with a getter and setter for it. The constructors for Student reuse the Person's constructors in the default and non-default cases:
\begin{lstlisting}[language=Java]
package homework01.Classes;

public class Student extends Person
{
	private int ID;

	public Student()
	{
		super();
		this.ID = 0;
	}
	public Student(String name, int age, int ID)
	{
		super(name, age);
		this.ID = ID;
	}
	public void setID(int ID)
	{
		this.ID = ID;
	}
	public int getID()
	{
		return this.ID;
	}
	public void printInfo(Student a)
	{
		System.out.printf(
		"ID: %-8d Name: %-16s Age: %-4d", a.getID(), a.getName(), a.getAge());
	}	
}

\end{lstlisting}
In addition, the \textit{printInfo} method uses the getters for our member fields to display info on the calling student.
\subsection*{Studentmethods.java}
Studentmethods has 2 methods that MainClass must implement:
\begin{lstlisting}[language=Java]
public interface Studentmethods
{
	public void displayAll();

	public void searchStudent(String id);
}
\end{lstlisting}
and not much else.
\subsection*{MainClass.java}	
This class contains the main method that uses the Student objects and implements Studentmethods. It also contains helper methods for modularization of the program.\\
A do-while loop and switch statement are implemented to drive the main function of the main; the loop is used to re-print the menu and allows for the ability to choose a different input on each go around, while the switch statement allows for branching at different input. Each valid input calls their own helper method or a memeber method from a student object to actually do what the user specified.\\
Below is a synopsis of MainClass.java; some code is omitted for brevity.
\begin{lstlisting}[language=Java]
package homework01.MainClasses;
import homework01.Classes.Student;
import homework01.Interfaces.Studentmethods;
import java.util.Scanner;
import java.lang.Exception;
/* 
Above are the package declaration and imports for the class.
The Student Class and Studentmethods interface are user defined
and used in the execution of the main method in MainClass.
The Scanner is used to accept input, and Exception is used for 
validation.
 */

public class MainClass implements Studentmethods
{
	
	public MainClass(){}
	/*
The methods that we implement from Studentmethods are non-static,
so we must make an instance of the MainClass. To do so we need
the above constructor for the class
	 */

	public static Student[] studentArray = new Student[10];
	public static int currIndex = 0;
	public static boolean foundFlag = false;
	/*
The above static variables are used throughout the class. So in order
to avoid passing them as arguments all the time, we leave them visible 
to all methods within MainClass. studentArray is the more important of
the 3, as it holds our Student objects. 
	 */

	public static void main(String[] args)
	{
		MainClass caller = new MainClass();
		Scanner input = new Scanner(System.in);	
		int idTemp, ageTemp;
		String sTemp = "";	
// The Scanner and MainClass instances are needed for interactivty.
// The temp variables are needed for input.

		do
		{
// This do-while will only exit if a specific value is input,
// and it represents the main looping structure of the program.	

			generateMenu();
// The above method prints the menu of options to the console.
// More on that later.

			idTemp = validateInteger("selection", input);		
// Here we accept input for the variable that our branching 
// switch statement will use, instead of calling input.nextInt()
// we instead call validateInteger(), more on that later.

			switch(idTemp)
			{
// This switch statement handles the branching that is required for
// this program, the previously called validateInteger() ensures 
// that whatever gets put into idTemp is an integer.

				case 1:
						...
						break;
						/*
The above case is where we attempt to add new Students to the 
'Roll'. If the index we intend to insert at equals the length
of the list, then our list is full and inform the user of that.
If not, prompt the user for information regarding the new student
and collect the input with validateInteger(). The addStudent()
method is called with a new Student object to add that student
to our list, and the user is notified. There is also a check for
the insertion index's value, so that the user can be notified 
that it is now full.
						 */						

				case 2:
						...
						break;
						/*
The above case is for attempting to remove students from the list.
First, the insertion index is checked to make sure that there are
students in the list at all. If there are, we prompt the user for
the ID of the student they wish to remove and use validateInteger 
for input. We call removeStudent() using our instance of the 
MainClass object, caller. If the list is empty, we inform then 
that there are no students to look for.
						 */

				case 3:
						...
						break;
						/*
The above case is nearly identical to the previous, except that
we are calling searchStudent() instead of removeStudent(). The
message in case of an empty list is slightly different as well.
						*/

				case 4:
						...
						break;
						/*
 The above case is for displaying the contents of the list using
caller to call displayAll(). Like the previous, we check for an 
empty list beforehand; and let the user know if the list has 
nothing in it. 
						*/

				case 5:
						System.out.println("\nThanks!\nExiting...");
						break;
						/*
The above case is for thanking the user before exiting the program.
						 */
				
				default:
						...
// The default case catches all invalid inputs by the user.
			}
		}while(idTemp != 5);
// End of do-while above, end of main below.	
	}	
	
	public static void generateMenu()
	...
// The above method prints the menu options to the console. Using the method
// helps to modularise the code.

	public static int validateInteger(String data, Scanner input)
	...
	/*
The above method validates input to ensure that the methods expecting integers
will get them. It's used for all cases where integers are necessary and uses
exception handling along Integer.parseInt() to accept String input and determine
whether the string can be parsed to int. It does so, if possible, but as the 
program isn't interested in non-positive integers, we remain in the do-while to
re-prompt for valid input if negative integers are given. The necessary print 
outs to the user are dictated by the String parameter given when validateInteger()
is called. If the String fails to parse to int, an exception is thrown and caught,
and relevant usage information is printed. The local variables out and temp facilitate
the function of the method and out's value is returned at the end of the method.
	 */

	@Override
	public void searchStudent(String id)
	...
	/*
The above method is implemented from Studentmethods, and searches the list of 
students for a student with the matching ID. The argument for the method is 
of type String, so it is parsed to int after being passed in. A local boolean
variable is used to print a message if the ID in question is never found. A 
for-loop is used to iterate through the list and check each element of the list
to determine if an object's reference is stored there. If so, that object's ID
member (retrieved by calling a getter) is compared with the local variable. If 
a match is found, the user is notified, the found flag is set to true, and the
loop is broken out of. If not, the loop continues, until we reach the insertion
index. If nothing was found, the user is notified. 
	 */

	public void removeStudent(String id)
	...
	/*
The above method is identical to the searchStudent method, except that when
there is a match between the local variable and a student object's ID that
object's reference is removed from trhe array, and the object reference 
whose index in the list is highest is moved up to fill the 'gap' in the 
array. currIndex is likewise updated to reflect this.
	 */

	public static void addStudent(Student stud)
	...
	/*
The above method adds the student reference passed in to the list of students
and uses the post-increment to increase the current insertion index, currIndex.
	 */

	@Override
	public void displayAll()
	...
	/*
The above method is implemented from Studentmethods, and is called with 
an instance of MainClass to call each reference's printInfo sequentially.
As the index of each reference is irrelevant (unlike the search and remove
methods) an enhanced of for-each loop is used. The possibility of null 
reference's methods being called is accounted for by checking ahead of 
time.
	 */
}
\end{lstlisting}
\noindent\makebox[\linewidth]{\rule{\paperwidth}{0.4pt}}

\end{large}
\end{document}
