\documentclass[12pt]{article}
\usepackage{amsmath}
\usepackage{amssymb}
\usepackage{bm}
\usepackage{amsthm}
\usepackage{enumerate}
\usepackage{graphicx}
\usepackage{psfrag}
\usepackage{color}
\usepackage{url}
\usepackage{listings}
\usepackage{xcolor}
\usepackage{tikz}
\usetikzlibrary{positioning}
\tikzset{main node/.style={circle,fill=gray!20,draw,minimum size=.5cm,inner sep=0pt},}

\definecolor{codegreen}{rgb}{0,0.5,0}
\definecolor{codewhite}{rgb}{1,1,1}
\definecolor{codegray}{rgb}{0.5,0.5,0.5}
\definecolor{codepurple}{rgb}{0.58,0,0.82}
\definecolor{codeblack}{rgb}{0,0,0}
\definecolor{codeorange}{rgb}{0.8,0.4,0}

\lstdefinestyle{mystyle}{
    backgroundcolor=\color{codewhite},   
    commentstyle=\color{codegray},
    keywordstyle=\color{codegreen},
    numberstyle=\color{codegray},
    stringstyle=\color{codeorange},
    basicstyle=\ttfamily ,
    breakatwhitespace=false,         
    breaklines=true,                 
    captionpos=b,                    
    keepspaces=true,                 
    numbers=left,                    
    numbersep=5pt,                  
    showspaces=false,                
    showstringspaces=false,
    showtabs=false,                  
    tabsize=4
}
\lstset{style=mystyle}


\setlength{\hoffset}{-1in}
\addtolength{\textwidth}{1.5in}
\setlength{\voffset}{-1in}
\addtolength{\textheight}{1.5in}
\newcommand{\be}{\begin{enumerate}}
\newcommand{\ee}{\end{enumerate}}
\newcommand{\BigO}[1]{\ensuremath\mathcal{O}\left(#1\right)}
\newcommand{\il}[1]{\lstinline!#1!}
\newcommand{\gnorm}[1]{\left|\left|#1\right|\right|}
\newcommand{\abs}[1]{\left|#1\right|}
\newcommand{\parens}[1]{\left(#1\right)}
\newcommand{\bracks}[1]{\left\{#1\right\}}
\newcommand{\sqbracks}[1]{\left[#1\right]}
\newcommand{\vep}{\varepsilon}
\newcommand{\ceiling}[1]{\left\lceil#1\right\rceil}
\newcommand{\R}{\mathbb{R}}
\newcommand{\N}{\mathbb{N}}
\newcommand{\distrib}[2]{\text{#1}\left(#2\right)}
\newcommand{\dd}[1]{\frac{d}{d#1}}

\begin{document}
	\begin{center}
		\textbf{Spring 2020, Geology 303} \\
		\textbf{Tectonic Assessment Report} \\
		\textbf{Joseph Diaz: 819947915}
	\end{center}
\noindent\makebox[\linewidth]{\rule{\paperwidth}{0.4pt}}
\be[1.]
	\item Convergent.
	
	\item The tectonic regime we have here is subduction at a convergent plate boundary, the denser oceanic Nazca plate is subducting under the less dense continental South American plate. The map even names the offshore trench where the plates meet: The Peru-Chile Trench. As we have this kind of plate boundary, this area is subject to all four major types of seismically triggered natural disasters, the worst of which being megathrust earthquakes. The slab pull of the subducting plate builds massive energy as the rock of both plates scrape by until the energy is released in an earthquake. \\
	
	 These earthquakes can also move larger masses of water if the quake occurs in the trench itself, which give rise to tsunamis. Ocean water is drawn in while the oceanic plate subducts and as the water is evaporated in the heat and pressure of the Asthenosphere, it helps along the process of convection that drives magma to the surface to form volcanoes in large continental volcanic arcs. Crucially, this area is part of the infamous pacific ring of fire; a series of convergent plate boundaries that surround the Pacific Ocean that are replete with active volcanoes. \\
	 
	 The continental uplift also gives rise to mountain ranges in which the first the 3 disasters may be the triggering event for mass wasting events (like landslides) of many types. These kinds of regions show the greatest elevation gradients on earth; as the very deep trenches where the plate boundaries themselves lies are not very far from the mountain ranges that the convergent plate produce. Most earthquakes (and tectonic activity of any kind) happen along plate boundaries like this one, and the largest magnitude earthquake ever recorded (1960 Chile earthquake) happened along this particular plate boundary.
	 
	\item 
	\begin{itemize}
		\item Earthquakes
		
		\item Volcanoes
		
		\item Tsunamis
		
		\item Landslides
		
	\end{itemize}
	
	\item $\sim$ 11 cm/yr East
	
	\item We see that this eroded roadway is parallel to a beach; so this is likely a case of coastal erosion undercutting the cliff on which the road was built. After enough of the rocks at the base of the cliff was removed, a rock fall occurred which took chunks out of the roadway. The beach looks like it isn't that deep (distance from cliff bottom to water's edge), so the waves don't need to travel very far up the beach to erode the cliff and the water level probably reaches it at high tide. 
	
	\item We see rocks and some pieces of trees strewn everywhere around this house and the houses themselves seem to be caked with dried mud up to about 7 or 8 feet off of the ground; so the likeliest cause is a mud flow instigated by heavy rain. It doesn't look like its raining in the picture, so this was likely taken long after the disaster took place.
	
	
\ee
\noindent\makebox[\linewidth]{\rule{\paperwidth}{0.4pt}}
	
\end{document}
