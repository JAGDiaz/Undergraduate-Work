\documentclass{article}
\usepackage{amsmath}
\usepackage{amssymb}
\usepackage{bm}
\usepackage{amsthm}
\usepackage{enumerate}
\usepackage{graphicx}
\usepackage{psfrag}
\usepackage{color}
\usepackage{url}
\usepackage{listings}
\usepackage{xcolor}

\definecolor{codegreen}{rgb}{0,0.5,0}
\definecolor{codewhite}{rgb}{1,1,1}
\definecolor{codegray}{rgb}{0.5,0.5,0.5}
\definecolor{codepurple}{rgb}{0.58,0,0.82}
\definecolor{codeblack}{rgb}{0,0,0}
\definecolor{codeorange}{rgb}{0.8,0.4,0}

\lstdefinestyle{mystyle}{
    backgroundcolor=\color{codewhite},   
    commentstyle=\color{codegray},
    keywordstyle=\color{codegreen},
    numberstyle=\color{codegray},
    stringstyle=\color{codeorange},
    basicstyle=\ttfamily ,
    breakatwhitespace=false,         
    breaklines=true,                 
    captionpos=b,                    
    keepspaces=true,                 
    numbers=left,                    
    numbersep=5pt,                  
    showspaces=false,                
    showstringspaces=false,
    showtabs=false,                  
    tabsize=4
}
\lstset{style=mystyle}


\setlength{\hoffset}{-0.5in}
\addtolength{\textwidth}{1.0in}
\setlength{\voffset}{-0.5in}
\addtolength{\textheight}{1.0in}
\newcommand{\be}{\begin{enumerate}}
\newcommand{\ee}{\end{enumerate}}
\newcommand{\BigO}[1]{\ensuremath\mathcal{O}\left(#1\right)}
\newcommand{\il}[1]{\lstinline!#1!}

\begin{document}
	\begin{center}
		\textbf{Spring 2020, Math 530:\ Homework 3} \\
		\textbf{Due:\ Monday, March 2nd, 2020} \\
		\textbf{Joseph Diaz: 819947915}
	\end{center}
\noindent\makebox[\linewidth]{\rule{\paperwidth}{0.4pt}}
\be[(E1)]
	\item Assume that the series $\sum_{k=n_0}^\infty a_k$ converges. Prove that $\forall c \in \mathbb{R},\ \sum_{k=n_0}^\infty ca_k$ converges and
$$\sum_{k=n_0}^\infty ca_k = c\sum_{k=n_0}^\infty a_k$$
(hint: express the limit of partial sums)
\begin{proof}
Let $S_n$ be the $n^{th}$ partial sum of $\{a_k\}_{k=n_0}^\infty$. Then, since $\sum_{k=n_0}^\infty a_k$ converges; we have that 
$$\lim_{n \to \infty}S_n = S = \sum_{k=n_0}^\infty a_k$$
Now let $B_n$ be the $n^{th}$ partial sum of the new sequence $\{ca_k\}_{k=n_0}^\infty, \forall c \in \mathbb{R}$. Then:
\begin{align*}
B_n &= ca_{n_0} + ca_{n_0 + 1} + ca_{n_0 + 2} + \cdots + ca_n\\
&= c(a_{n_0} + a_{n_0 + 1} + a_{n_0 + 2} + \cdots + a_n)\\
&= cS_n
\end{align*}
We know that $\lim_{n\to\infty}S_n$ converges, so $\lim_{n\to\infty}B_n$ must as well. Indeed:
$$\lim_{n\to\infty}B_n = \lim_{n\to\infty} cS_n = c\lim_{n\to\infty}S_n = cS$$
This means that $\lim_{n\to\infty}B_n = \sum_{k=n_0}^\infty ca_k$, so we deduce that the series converges and that:
$$\sum_{k=n_0}^\infty ca_k = c\sum_{k=n_0}^\infty a_k$$ 
\end{proof}

	\item Let $\sum_{k=n_0}^\infty a_k $ and 
	$\sum_{k=n_0}^\infty b_k$ be two series. 
	Prove that if both series
	converge then \\$\sum_{k=n_0}^\infty (a_k + b_k)$ converges and 
$$\sum_{k=n_0}^\infty (a_k + b_k) = \sum_{k=n_0}^\infty a_k + \sum_{k=n_0} b_k$$
(hint: express the limit of partial sums)
\begin{proof}
Let $A_n$ and $B_n$ be the $n^{th}$ partial sums of $\{a_k\}_{k=n_0}^\infty$ and $\{b_k\}_{k=n_0}^\infty$, respectively. Given that the series $\sum_{k=n_0}^\infty a_k$ and $\sum_{k=n_0}^\infty b_k$ converge, we have that:
$$\lim_{n\to\infty}A_n = \sum_{k=n_0}^\infty a_k = A \qquad \lim_{n\to\infty}B_n = \sum_{k=n_0}^\infty b_k = B$$
Now, let $C_n$ be the $n^{th}$ partial sum of the sequence $\{c_k\}_{k=n_0}^\infty$, where $c_k = a_k + b_k$. Then:
\begin{align*}
C_n &= c_{n_0} + c_{n_0 + 1} + c_{n_0 + 2} + \cdots + c_{n}\\
&= (a_{n_0} + b_{n_0}) + (a_{n_0 + 1} + b_{n_0 + 1}) + \cdots + (a_{n} + b_{n})\\
&= a_{n_0} + a_{n_0 + 1} + \cdots + a_{n} + b_{n_0} + b_{n_0 + 1} + \cdots + b_{n}\\
&= A_n + B_n
\end{align*}
We know that $\lim_{n\to\infty}A_n$ and $\lim_{n\to\infty}B_n$ exist, and given that
$$\lim_{n\to\infty}A_n + \lim_{n\to\infty}B_n = \lim_{n\to\infty}\left(A_n + B_n\right) = \lim_{n\to\infty} C_n$$
we may conclude that $\lim_{n\to\infty} C_n = C$ exists, and also:
$$\sum_{k=n_0}^\infty (a_k + b_k) = \sum_{k=n_0}^\infty a_k + \sum_{k=n_0} b_k$$
\end{proof}
	\item Show that the series
	$$\sum_{k=1}^\infty k \sin\left(\frac{1}{k}\right)$$
	diverges.
	\begin{proof}
	Let $a_k = k \sin\left(\frac{1}{k}\right)$, then:
	\begin{align*}
	\lim_{k\to\infty} a_k &= \lim_{k\to\infty}k\sin\left(\frac{1}{k}\right)\\
	&= \lim_{k\to\infty} \frac{\sin\left(\frac{1}{k}\right)}{\frac{1}{k}}
	\end{align*}
	Now let $\frac{1}{k} = n$, and observe that as $k \to \infty$, $n \to 0^+$. With this substitution, our limit becomes:
	$$
	\lim_{k\to\infty} \frac{\sin\left(\frac{1}{k}\right)}{\frac{1}{k}} = \lim_{n \to 0^+} \frac{\sin(n)}{n} = 1 \neq 0
	$$
	So the limit of the sequence $\{a_k\}_{k=1}^\infty$ does not equal 0, so we may conclude that $\sum_{k=1}^\infty k \sin\left(\frac{1}{k}\right)$ diverges.	 
	\end{proof}
	\item Use the ratio test to check the convergence or divergence of the following series (eventually with respect to the involved constants).
	\be[a)]
		\item $\sum_{k=1}^\infty \frac{\alpha^k}{k^p}$ \textbf{where} $\alpha > 0,\ p > 0$.
		\begin{proof}[Ratio Test]
		Let $a_k = \frac{\alpha^k}{k^p}$, then:
		\begin{align*}
		\lim_{k\to\infty} \left\vert\frac{a_{k+1}}{a_k}\right\vert &= \lim_{k\to\infty} \left\vert\frac{\frac{\alpha^{k+1}}{(k+1)^p}}{\frac{\alpha^k}{k^p}}\right\vert\\
		&= \lim_{k\to\infty} \frac{\alpha^{k+1}}{(k+1)^p}\cdot\frac{k^p}{\alpha^k}\\
		&= \lim_{k\to\infty} \frac{\alpha k^p}{(k+1)^p}\\
		&= \alpha\lim_{k\to\infty} \frac{k^p}{(k+1)^p}\\
		&= \alpha
		\end{align*}
		So, the series converges for $0 < \alpha < 1$, and diverges for $\alpha > 1$, and if $\alpha = 1$, then the ratio test will be inconclusive; but the series becomes $\sum_{k=1}^\infty \frac{1}{k^p}$, a $p$-series, which will converge for $p > 1$ and diverge for $0 < p \leq 1$. 
		\end{proof}
		
		\item $\sum_{k=1}^\infty \frac{k^\alpha}{e^k}$ \textbf{where} $\alpha > 0$.
		\begin{proof}[Ratio Test]
		Let $a_k = \frac{k^\alpha}{e^k}$, then:
		\begin{align*}
		\lim_{k\to\infty} \left\vert\frac{a_{k+1}}{a_k}\right\vert &= \lim_{k\to\infty} \left\vert\frac{\frac{(k+1)^\alpha}{e^{k+1}}}{\frac{k^\alpha}{e^k}}\right\vert\\
		&= \lim_{k\to\infty} \frac{(k+1)^\alpha}{e^{k+1}}\cdot\frac{e^k}{k^\alpha}\\
		&= \lim_{k\to\infty} \frac{(k+1)^\alpha}{ek^\alpha}\\
		&= \frac{1}{e}\lim_{k\to\infty} \frac{(k+1)^\alpha}{k^\alpha}\\
		&= \frac{1}{e} < 1
		\end{align*}
		So, by the Ratio Test, the series $\sum_{k=1}^\infty \frac{k^\alpha}{e^k}$ converges.
		\end{proof}
		
		\item $\sum_{k=1}^\infty \frac{\sqrt{k^2-1}}{\sqrt{k^5+1}}$.
		\begin{proof}[Ratio Test]
		Let $a_k = \sqrt{\frac{k^2-1}{k^5+1}}$, then:
		\begin{align*}
		\lim_{k\to\infty}\left\vert\frac{a_{k+1}}{a_k}\right\vert &= \lim_{k\to\infty}\left\vert\frac{\sqrt{\frac{(k+1)^2-1}{(k+1)^5+1}}}{\sqrt{\frac{k^2-1}{k^5+1}}}\right\vert\\
		&= \lim_{k\to\infty}\sqrt{\frac{(k+1)^2-1}{(k+1)^5+1}}\cdot\sqrt{\frac{k^5+1}{k^2-1}}\\
		&= \lim_{k\to\infty}\sqrt{\frac{\left((k+1)^2-1\right)\left(k^5+1\right)}{\left((k+1)^5+1\right)\left(k^2-1\right)}}\\
		&= \lim_{k\to\infty}\sqrt{\frac{k^5-k^4+k^3-k^2 + k}{k^5+2k^4+k^3-2k^2-k-1}}\\
		&= \sqrt{\lim_{k\to\infty}\left(\frac{k^5-k^4+k^3-k^2 + k}{k^5+2k^4+k^3-2k^2-k-1}\right)}\\
		&= \sqrt{1} = 1
		\end{align*}
		We have that $\lim_{k\to\infty}\left\vert\frac{a_{k+1}}{a_k}\right\vert = 1$, so the ratio test is inconclusive. 
		\end{proof}
		
	\ee
\ee

\noindent\makebox[\linewidth]{\rule{\paperwidth}{0.4pt}}
	
\end{document}
