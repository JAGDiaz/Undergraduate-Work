\documentclass{article}
\usepackage{amsmath}
\usepackage{amssymb}
\usepackage{bm}
\usepackage{amsthm}
\usepackage{enumerate}
\usepackage{graphicx}
\usepackage{psfrag}
\usepackage{color}
\usepackage{url}
\usepackage{listings}
\usepackage{xcolor}
\usepackage{tikz}
\usetikzlibrary{positioning}
\tikzset{main node/.style={circle,fill=gray!20,draw,minimum size=.5cm,inner sep=0pt},}

\definecolor{codegreen}{rgb}{0,0.5,0}
\definecolor{codewhite}{rgb}{1,1,1}
\definecolor{codegray}{rgb}{0.5,0.5,0.5}
\definecolor{codepurple}{rgb}{0.58,0,0.82}
\definecolor{codeblack}{rgb}{0,0,0}
\definecolor{codeorange}{rgb}{0.8,0.4,0}

\lstdefinestyle{mystyle}{
    backgroundcolor=\color{codewhite},   
    commentstyle=\color{codegray},
    keywordstyle=\color{codegreen},
    numberstyle=\color{codegray},
    stringstyle=\color{codeorange},
    basicstyle=\ttfamily ,
    breakatwhitespace=false,         
    breaklines=true,                 
    captionpos=b,                    
    keepspaces=true,                 
    numbers=left,                    
    numbersep=5pt,                  
    showspaces=false,                
    showstringspaces=false,
    showtabs=false,                  
    tabsize=4
}
\lstset{style=mystyle}


\setlength{\hoffset}{-1in}
\addtolength{\textwidth}{1.5in}
\setlength{\voffset}{-1in}
\addtolength{\textheight}{1.5in}
\newcommand{\be}{\begin{enumerate}}
\newcommand{\ee}{\end{enumerate}}
\newcommand{\BigO}[1]{\ensuremath\mathcal{O}\left(#1\right)}
\newcommand{\il}[1]{\lstinline!#1!}
\newcommand{\gnorm}[1]{\left|\left|#1\right|\right|}
\newcommand{\abs}[1]{\left|#1\right|}
\newcommand{\parens}[1]{\left(#1\right)}
\newcommand{\bracks}[1]{\left\{#1\right\}}
\newcommand{\sqbracks}[1]{\left[#1\right]}
\newcommand{\vep}{\varepsilon}
\newcommand{\ceiling}[1]{\left\lceil#1\right\rceil}
\newcommand{\R}{\mathbb{R}}
\newcommand{\N}{\mathbb{N}}
\newcommand{\distrib}[2]{\text{#1}\left(#2\right)}
\newcommand{\dd}[1]{\frac{d}{d#1}}

\begin{document}
	\begin{center}
		\textbf{Spring 2020, Math 530:\ Take Home 2} \\
		\textbf{Due:\ Friday, April 24th, 2020} \\
		\textbf{Joseph Diaz: 819947915}
	\end{center}
\noindent\makebox[\linewidth]{\rule{\paperwidth}{0.4pt}}
\be[(E1)]
	\item Let $\forall x \in \R,\ \forall n \in \N,\ f_n(x) = \frac{\cos(nx)}{n^{7/3}}$, and denote 
	$$f(x) = \sum_{n=1}^\infty\ f_n(x)$$
	\be[1.	]
		\item Discuss the convergence of $\sum_{n=1}^\infty\ f_n(x)$.
		\begin{proof}[Solution]
		$\sum_{n=1}^\infty f_n(x)$ converges uniformly. We have that
		$$\abs{f_n(x)} = \abs{\frac{\cos(nx)}{n^{7/3}}} \leq \abs{\frac{1}{n^{7/3}}} = \frac{1}{n^{7/3}}$$
		So let $M_n = \frac{1}{n^{7/3}}$, then $\abs{f_n(x)} \leq M_n, \forall n \in \N$; also $\sum_{n=1}^\infty M_n$ converges, as it is a $p$-series with $p = 7/3 > 1$ and by the WeierStrass $M$-test, $\sum_{n=1}^\infty f_n(x)$ converges uniformly to $f(x), \forall x \in \R$.
		\end{proof}
		
		\item Discuss the convergence of $\sum_{n=1}^\infty\ f'_n(x)$.
		\begin{proof}
		$\sum_{n=1}^\infty f'_n(x)$ converges uniformly. We have that
		$$f'_n(x) = \dd{x}\parens{\frac{\cos(nx)}{n^{7/3}}} = -\frac{n\sin(nx)}{n^{7/3}} = -\frac{\sin(nx)}{n^{4/3}}$$
		 and let $g(x) = \sum_{n=1}^\infty f'_n(x)$, then
		$$\abs{f'_n(x)} = \abs{-\frac{\sin(nx)}{n^{4/3}}} \leq \abs{\frac{1}{n^{4/3}}} = \frac{1}{n^{4/3}}$$
		So let $M_n = \frac{1}{n^{4/3}}$, then $\abs{f'_n(x)} \leq M_n, \forall n \in \N$; also $\sum_{n=1}^\infty M_n$ converges, as it is a $p$-series with $p = 4/3 > 1$ and by the WeierStrass $M$-test, $\sum_{n=1}^\infty f'_n(x)$ converges uniformly to $g(x), \forall x \in \R$.
		\end{proof}
		
		\item Without any calculation, while justifying your result, give the expression of $f'(x)$.
		\begin{proof}[Solution]
		From 1.1 and 1.2, we have that $\sum_{n=1}^\infty f_n(x)$ and $\sum_{n=1}^\infty f'_n(x)$ converge uniformly to $f$ and $g$, respectively, for all $x$; so
		$$g(x) = \sum_{n=1}^\infty f'_n(x) = \sum_{n=1}^\infty\dd{x}f_n(x) = \dd{x} \sum_{n=1}^\infty f_n(x) = \dd{x}f(x) = f'(x)$$
		\end{proof}
		
	\ee
	
	\item Let $f(x) = (1+x)^\alpha$ for $\abs{x} < 1$ and $\alpha \in \R$.
	\be[1.]
		\item Prove by induction that the $n^{th}$ derivative $\parens{n \geq 1}$ of $f$ is given by 
		$$f^{(n)}(x) = \alpha\parens{\alpha - 1}\cdots\parens{\alpha - n + 1}\parens{1+x}^{\alpha-n}$$
		\begin{proof}
		First, we will establish the base case, $n = 1$:
		$$f^{(1)}(x) = \parens{\alpha - 1 + 1}\parens{1+x}^{\alpha - 1} = \alpha\parens{1+x}^{\alpha - 1} = \frac{d}{dx}f(x) = f'(x)$$
		then, our inductive hypothesis is
		$$f^{(n)}(x) = \alpha\parens{\alpha - 1}\cdots\parens{\alpha - n + 1}\parens{1+x}^{\alpha-n}$$
		and we want to show
		$$f^{(n+1)}(x) = \alpha\parens{\alpha - 1}\cdots\parens{\alpha - n + 1}\parens{\alpha - (n+1) + 1}\parens{1+x}^{\alpha-(n+1)}$$
		So we have:
		\begin{align*}
		f^{(n+1)}(x) &= \frac{d}{dx}f^{(n)}(x) \\
		&= \frac{d}{dx}\parens{\alpha\parens{\alpha - 1}\cdots\parens{\alpha - n + 1}\parens{1+x}^{\alpha-n}} \\
		&= \alpha\parens{\alpha - 1}\cdots\parens{\alpha - n + 1}\dd{x}\parens{\parens{1+x}^{\alpha - n}}\\
		&=  \alpha\parens{\alpha - 1}\cdots\parens{\alpha - n + 1}\parens{\alpha - n}\parens{1+x}^{\alpha - n - 1}\\
		&=  \alpha\parens{\alpha - 1}\cdots\parens{\alpha - n + 1}\parens{\alpha - n - 1 + 1}\parens{1+x}^{\alpha - n - 1}\\
		f^{(n+1)}(x) &=  \alpha\parens{\alpha - 1}\cdots\parens{\alpha - n + 1}\parens{\alpha - (n + 1) + 1}\parens{1+x}^{\alpha - (n + 1)}\\
		\end{align*}
		Which is what we wanted to show. So
		$$f(x) = (1+x)^\alpha,\ \abs{x} < 1,\ \alpha \in \R \implies f^{(n)}(x) = \alpha\parens{\alpha - 1}\cdots\parens{\alpha - n + 1}\parens{1+x}^{\alpha-n}, \forall n \geq 1$$
		\end{proof}
		
		\item Using the notation of the ``Generalized Binomial Coefficient'', $\binom{\alpha}{n} = \frac{\alpha\parens{\alpha - 1}\cdots\parens{\alpha - n + 1}}{n!}$, give the MacLaurin expansion of $f$.
		\begin{proof}[Solution]
		From 2.1, we have that
		$$f^{(n)}(x) = \alpha\parens{\alpha - 1}\cdots\parens{\alpha - n + 1}\parens{1+x}^{\alpha-n}, \forall n \geq 1$$
		which implies
		$$f^{(n)}(0) = \alpha\parens{\alpha - 1}\cdots\parens{\alpha - n + 1}\parens{1+0}^{\alpha-n}$$
		$$\Rightarrow\ = \alpha\parens{\alpha - 1}\cdots\parens{\alpha - n + 1}$$
		So the MacLaurin series for $f$ is given by 
		\begin{align*}
		f(x) &= \sum_{n=0}^\infty \frac{f^{(n)}(0)}{n!}x^n \\
		&= \frac{f^{(0)}(0)}{n!}x^0 + \sum_{n=1}^\infty \frac{f^{(n)}(0)}{n!}x^n \\
		&= 1 + \sum_{n=1}^\infty \frac{\alpha\parens{\alpha - 1}\cdots\parens{\alpha - n + 1}}{n!}x^n \\
		&= 1 + \sum_{n=1}^\infty \binom{\alpha}{n}x^n \\
		&= \binom{\alpha}{0}x^0 + \sum_{n=1}^\infty \binom{\alpha}{n}x^n \\
		&= \sum_{n=0}^\infty \binom{\alpha}{n}x^n
		\end{align*}
		So the MacLaurin series for $f$ is
		$$f(x) = \parens{1+x}^\alpha = \sum_{n=0}^\infty \binom{\alpha}{n}x^n$$
		\end{proof}			
		
	\ee
	
	\item Denote $\Omega \subset \R,\ \Omega \neq \varnothing
	$ with 
	$$\mathcal{B}\parens{\Omega, \R} = \bracks{f : \Omega \to \R\ \mid f \text{ is bounded}}$$
	Show that $\mathcal{B}\parens{\Omega, \R}$ is a vector space (\emph{Hint: some properties can be justified without calculation}).
	\begin{proof}
	Let $f, g, h \in \mathcal{B}\parens{\Omega, \R},\ \exists \text{ finite }  M_f, M_g, M_h > 0 : \forall x \in \Omega, \abs{f(x)} < M_f,\ \abs{g(x)} < M_g,\ \abs{h(x)} < M_h$, now we will show each property in turn:
	\be[1.]
		\item $\mathcal{B}\parens{\Omega, \R}$ is a set of real valued functions, so automatically we have:
		$$(f+g)(x) = f(x) + g(x),\quad (\alpha f)(x) = \alpha f(x), \alpha \in \R$$
		from this we have:
		\begin{align*}
		(f+g)(x) &= f(x) + g(x) \\
		&\leq \abs{f(x) + g(x)} \\
		&\leq \abs{f(x)} + \abs{g(x)} \\
		&< M_f + M_g 
		\end{align*}
		so $f(x) + g(x) \in \mathcal{B}\parens{\Omega, \R}$, and
		\begin{align*}
		(\alpha f)(x) &= \alpha f(x) \\
		&\leq \abs{\alpha f(x)} \\
		&<\abs{\alpha} M_f 
		\end{align*}
		so $\alpha f(x) \in \mathcal{B}\parens{\Omega, \R}$
		
		\item $f$ and $g$ are functions real valued functions of real numbers, so their addition is commutative, thus:
		$$f(x) + g(x) = g(x) + f(x)$$
		
		\item $f$, $g$, and $h$ real valued functions of real numbers, so their addition is associative, thus:
		$$\parens{f(x) + g(x)} + h(x) = f(x) + \parens{g(x) + h(x)}$$
		
		\item $\alpha, \beta \in \R$, and $f$ is a real valued function of real numbers, so multiplication of $f$ by a sum of constants is distributive:
		$$\parens{\alpha + \beta}f(x) = \alpha f(x) + \beta f(x)$$
		
		\item $\alpha \in \R$, and $f$ and $g$ are real valued functions of real numbers, so multiplication of a constant by a sum of $f$ and $g$ is distributive:
		$$\alpha\parens{f(x) + g(x)} = \alpha f(x) + \alpha g(x)$$
		
		\item $\alpha, \beta \in \R$ and $f$ is a real valued function of real numbers, so multiplication by multiple constants is associative:
		$$\parens{\alpha\beta}f(x) = \alpha\parens{\beta f(x)}$$
		
		\item Let $u(x) \in \mathcal{B}\parens{\Omega, \R}: \forall x \in \Omega, u(x) = 1$ and $f$ is a real valued function of real numbers, then by previous properties:
		$$u(x)f(x) = 1\cdot f(x) = f(x)$$
		
		\item Let $z(x) \in \mathcal{B}\parens{\Omega, \R}: \forall x \in \Omega, z(x) = 0$ and $f$ is a real valued function of real numbers, then by previous properties:
		  $$f(x) + z(x) = f(x) + 0 = f(x)$$
		  
		  \item We have that $f$ is an arbitrary element of $\mathcal{B}\parens{\Omega, \R}$, so we want to find $i(x) \in \mathcal{B}\parens{\Omega, \R}$, such that:
		  $$f(x) + i(x) = z(x) = 0 \Rightarrow i(x) = -f(x)$$
		  so $$\forall f \in \mathcal{B}\parens{\Omega, \R}, \exists!\ i(x) = -f(x): f(x) + i(x) = z(x)$$
	\ee
	
	So $\mathcal{B}\parens{\Omega, \R}$ is a vector space.
	\end{proof}
	
	\item Let
	$$\forall x \in \R,\ f(x) = \sum_{n=1}^\infty \frac{\sin\parens{\parens{2\pi}^n x}}{\parens{4\pi}^n}$$
	\be[1.] 
		\item Does this series converge absolutely and uniformly? Uniformly? Pointwise?
		\begin{proof}[Solution]
		This series converges uniformly. Let 
		$$f_n(x) = \frac{\sin\parens{\parens{2\pi}^n x}}{\parens{4\pi}^n} \implies \abs{f_n(x)} = \abs{\frac{\sin\parens{\parens{2\pi}^n x}}{\parens{4\pi}^n}} \leq \frac{1}{\parens{4\pi}^n}$$
		Let $M_n = \frac{1}{\parens{4\pi}^n}$, then
		$$\forall k \in \N,\ M_k = \frac{1}{\parens{4\pi}^k},\ M_k > 0,\ \forall x \in \R,\ \abs{f_k(x)} \leq M_k$$
		but does the series $\sum_{n=1}^\infty M_n$ converge? 
		With the Ratio Test:
		\begin{align*}
		\lim_{n\to\infty} \abs{\frac{M_{n+1}}{M_n}} &= \lim_{n\to\infty} \abs{\frac{\parens{4\pi}^n}{\parens{4\pi}^{n+1}}} \\
		&= \lim_{n\to\infty} \abs{\frac{\parens{4\pi}^n}{\parens{4\pi}^{n}\parens{4\pi}}} \\
		&= \lim_{n\to\infty} \abs{\frac{1}{4\pi}} \\
		&= \frac{1}{4\pi} < 1
		\end{align*}
		So by the Ratio Test, $\sum_{n=1}^\infty M_n$ converges, and so the Weierstrass $M$-test $\sum_{n=1}^\infty \frac{\sin\parens{\parens{2\pi}^n x}}{\parens{4\pi}^n}$ converges uniformly to $f(x)$ for all $x$.
		\end{proof}
		
		\item Give the series expansion of $\int f(x)\ dx$.
		\begin{proof}[Solution]
		From 4.1 , we know that the $\sum_{n=1}^\infty \frac{\sin\parens{\parens{2\pi}^n x}}{\parens{4\pi}^n}$ converges uniformly to $f(x),\ \forall x \in \R$; so with this property we may rewrite $\int f(x)\ dx$ like so:
		\begin{align*}
		\int f(x)\ dx &= \int \sqbracks{ \sum_{n=1}^\infty\frac{\sin\parens{\parens{2\pi}^n x}}{\parens{4\pi}^n}}\ dx \\
		&= \sum_{n=1}^\infty \sqbracks{\int \frac{\sin\parens{\parens{2\pi}^n x}}{\parens{4\pi}^n}\ dx}\\
		&= \sum_{n=1}^\infty \sqbracks{\frac{1}{\parens{4\pi}^n} \int \sin\parens{\parens{2\pi}^n x}\ dx} \\
		&= \sum_{n=1}^\infty \sqbracks{\frac{1}{\parens{4\pi}^n\parens{2\pi}^n} \parens{-\cos\parens{\parens{2\pi}^n x}	 + C}} \quad \parens{C\in \R} \\
		&= \sum_{n=1}^\infty \sqbracks{\frac{-\cos\parens{\parens{2\pi}^n x}	 + C}{2^n\parens{2\pi}^{2n}}} \\
		&= \sum_{n=1}^\infty \sqbracks{\frac{-\cos\parens{\parens{2\pi}^n x}}{2^n\parens{2\pi}^{2n}} + \frac{C}{2^n\parens{2\pi}^{2n}}} \\
		&= -\sum_{n=1}^\infty \frac{\cos\parens{\parens{2\pi}^n x}}{2^n\parens{2\pi}^{2n}} + C\sum_{n=1}^\infty\frac{1}{2^n\parens{2\pi}^{2n}} \\
		\end{align*}
		So the series expansion of $\int f(x)\ dx$ is
		$$\int f(x)\ dx = -\sum_{n=1}^\infty \frac{\cos\parens{\parens{2\pi}^n x}}{2^n\parens{2\pi}^{2n}} + C\sum_{n=1}^\infty\frac{1}{2^n\parens{2\pi}^{2n}}$$
		\end{proof}
	\ee
\ee
\noindent\makebox[\linewidth]{\rule{\paperwidth}{0.4pt}}
	
\end{document}
