\documentclass{article}
\usepackage{amsmath}
\usepackage{amssymb}
\usepackage{bm}
\usepackage{amsthm}
\usepackage{enumerate}
\usepackage{graphicx}
\usepackage{psfrag}
\usepackage{color}
\usepackage{url}
\usepackage{listings}
\usepackage{xcolor}
\usepackage{tikz}
\usetikzlibrary{positioning}
\tikzset{main node/.style={circle,fill=gray!20,draw,minimum size=.5cm,inner sep=0pt},}

\definecolor{codegreen}{rgb}{0,0.5,0}
\definecolor{codewhite}{rgb}{1,1,1}
\definecolor{codegray}{rgb}{0.5,0.5,0.5}
\definecolor{codepurple}{rgb}{0.58,0,0.82}
\definecolor{codeblack}{rgb}{0,0,0}
\definecolor{codeorange}{rgb}{0.8,0.4,0}

\lstdefinestyle{mystyle}{
    backgroundcolor=\color{codewhite},   
    commentstyle=\color{codegray},
    keywordstyle=\color{codegreen},
    numberstyle=\color{codegray},
    stringstyle=\color{codeorange},
    basicstyle=\ttfamily ,
    breakatwhitespace=false,         
    breaklines=true,                 
    captionpos=b,                    
    keepspaces=true,                 
    numbers=left,                    
    numbersep=5pt,                  
    showspaces=false,                
    showstringspaces=false,
    showtabs=false,                  
    tabsize=4
}
\lstset{style=mystyle}


\setlength{\hoffset}{-1in}
\addtolength{\textwidth}{1.5in}
\setlength{\voffset}{-1in}
\addtolength{\textheight}{1.5in}
\newcommand{\be}{\begin{enumerate}}
\newcommand{\ee}{\end{enumerate}}
\newcommand{\BigO}[1]{\ensuremath\mathcal{O}\left(#1\right)}
\newcommand{\il}[1]{\lstinline!#1!}
\newcommand{\gnorm}[1]{\left|\left|#1\right|\right|}
\newcommand{\abs}[1]{\left|#1\right|}
\newcommand{\parens}[1]{\left(#1\right)}
\newcommand{\bracks}[1]{\left\{#1\right\}}
\newcommand{\sqbracks}[1]{\left[#1\right]}
\newcommand{\vep}{\varepsilon}
\newcommand{\ceiling}[1]{\left\lceil#1\right\rceil}
\newcommand{\R}{\mathbb{R}}
\newcommand{\N}{\mathbb{N}}
\newcommand{\distrib}[2]{\text{#1}\left(#2\right)}
\newcommand{\dd}[1]{\frac{d}{d#1}}
\newcommand{\abracks}[1]{\left< #1\right>}

\begin{document}
	\begin{center}
		\textbf{Spring 2020, Math 530:\ Homework 8} \\
		\textbf{Due:\ Wednesday, April 29th, 2020} \\
		\textbf{Joseph Diaz: 819947915}
	\end{center}
\noindent\makebox[\linewidth]{\rule{\paperwidth}{0.4pt}}
\be[(E1)]
	\item Let $V = \bracks{\parens{x, \frac{1}{2}x} \mid x \in \R}$. Is $V$ a vector space over $\R$?
	\begin{proof}
	Let $x_1, x_2, x_3 \in \R$, then $u = \parens{x_1, \frac{1}{2}x_1}, v = \parens{x_2, \frac{1}{2}x_2}, w = \parens{x_3,\frac{1}{2}x_3} \in V$. Then, $\forall \alpha, \beta \in R$:
	$$u + v = \parens{x_1 + x_2, \frac{1}{2}x_1 + \frac{1}{2}x_2}\quad \text{and}\quad \alpha u = \parens{\alpha x_1, \frac{\alpha}{2}x_1}$$
	We will show that $V$ is vector space by property:
	\begin{itemize}
	\item Commutative Addition: 
	$$u+v = \parens{x_1 + x_2, \frac{1}{2}x_1 + \frac{1}{2}x_2} = \parens{x_2 + x_1, \frac{1}{2}x_2 + \frac{1}{2}x_1} = v + u$$
	
	\item Associative Addition:
	$$(u+v) + w = \parens{x_1 + x_2, \frac{1}{2}x_1 + \frac{1}{2}x_2} + \parens{x_3, \frac{1}{2}x_3} = \parens{x_1 + x_2 + x_3, \frac{1}{2}x_1 + \frac{1}{2}x_2 + \frac{1}{2}x_3}$$
	$$\Rightarrow = \parens{x_1, \frac{1}{2}x_1} + \parens{x_2 + x_3, \frac{1}{2}x_2 + \frac{1}{2}x_3} = u+(v+w)$$
	
	\item Distributive Multiplication over sum of scalars:
	$$\parens{\alpha + \beta}u = \parens{\parens{\alpha + \beta}x_1, \frac{\alpha + \beta}{2}x_1} = \parens{\alpha x_1 + \beta x_1, \frac{\alpha}{2}x_1 + \frac{\beta}{2}x_1} = \alpha u + \beta u$$
	
	\item Distributive Multiplication over sum of vectors:
	$$\alpha\parens{u+v} = \alpha\parens{x_1 + x_2, \frac{1}{2}x_1 + \frac{1}{2}x_2} = \parens{\alpha\parens{x_1 + x_2}, \alpha\parens{\frac{1}{2}x_1 + \frac{1}{2}x_2}}$$
	$$\Rightarrow  = \parens{\alpha x_1 + \alpha x_2, \frac{\alpha}{2}x_1 + \frac{\alpha}{2}x_2} = \alpha u + \alpha v$$
	
	\item Associative Scalar Multiplication:
	$$\parens{\alpha\beta}u = \parens{\parens{\alpha\beta}x_1, \frac{\alpha\beta}{2}x_1} = \parens{\alpha\parens{\beta x_1}, \alpha\parens{\frac{\beta}{2}x_1}} = \alpha\parens{\beta x_1, \frac{\beta}{2}x_1} = \alpha\parens{\beta u}$$
	
	\item Multiplicative Identity, let $1_V = 1 \in \R$:
	$$1_V u = \parens{1_V x_1, \frac{1_V}{2}x_1} = \parens{x_1, \frac{1}{2}x_1} = u$$
	
	\item Additive Identity, let $z = 0 \in \R, \theta = \parens{z, \frac{1}{2}z} = \parens{0,0} \in V$:
	$$u + \theta = \parens{x_1 + 0, \frac{1}{2}x_2 + 0} = \parens{x_1, \frac{1}{2}x_1} = u$$
	
	\item Additive Inverse, let $x_4 = - x_1 \in \R, u' = \parens{x_4, \frac{1}{2}x_4} \in V$:
	$$u + u' = \parens{x_1 + x_4, \frac{1}{2}x_1 + \frac{1}{2}x_4} = \parens{x_1 - x_1, \frac{1}{2}x_1 - \frac{1}{2}x_1} = \parens{0,0} = \theta$$
	
	\end{itemize}
	So $V$ is a vector space over $\R$.
	\end{proof}
	
	\item Let $V = \bracks{\parens{x, y} \mid x,y \in \R,\ x\geq 0, y\geq 0}$. Is $V$ a vector space over $\R$?
	\begin{proof}
	We have that $V$ is not a vector space over $\R$, because $V$ does not the an additive inverse. Let $u = \parens{x_1,y_1}\in V,\ x_1, y_1 \geq 0 \in R$, then let $u' = \parens{x_2, y_2}, x_2, y_2 \in \R: u + u' = \theta,\ \theta = \parens{0,0}$. So:
	$$u + u' = \parens{x_1 + x_2, y_1 + y_2} = \parens{0,0} \Rightarrow \begin{array}{c}
	x_1 + x_2 = 0\\
	y_1 + y_2 = 0
	\end{array} \Rightarrow 
	\begin{array}{c}
	x_2 = -x_1 \\
	y_2 = -y_1 
	\end{array}
	$$
	so clearly $u' = \parens{-x_1, -y_1} \notin V$; and $V$ is not a vector space over $\R$.
	\end{proof}
	
	\item 
	\be[(a)]
		\item Prove that
		$$\forall u,v \in \R^n,\ \gnorm{u-v}^2 = \gnorm{u}^2 + \gnorm{v}^2 - 2\abracks{u, v}$$
		\begin{proof}
		$\R^n$ is a Euclidean Space, so we have
		\begin{align*}
		\gnorm{u-v}^2 &= \abracks{u-v,u-v} \\
		&= \abracks{u, u-v} - \abracks{v,u-v} \\
		&= \abracks{u,u} - \abracks{u,v} - \parens{\abracks{v,u} - \abracks{v,v}} \\
		&= \abracks{u,u} - \abracks{u,v} - \abracks{v,u} + \abracks{v,v} \\
		&= \abracks{u,u} + \abracks{v,v} - \abracks{u,v} - \abracks{u,v} \\
		&= \abracks{u,u} + \abracks{v,v} - 2\abracks{u,v} \\
		\gnorm{u-v}^2 &= \gnorm{u}^2 + \gnorm{v}^2 - 2\abracks{u,v} \\
		\end{align*}
		\end{proof}
		
		\item Prove that
		$$\forall u,v \in \R^n,\ \abracks{u,v} = \frac{\gnorm{u+v}^2 - \gnorm{u-v}^2}{4}$$
		(this last identity is called the \textbf{polarization identity}.)
		\begin{proof}
		We have
		\begin{align*}
		\abracks{u,v} &= \frac{4}{4}\abracks{u,v} \\
		&= \frac{\abracks{u,v} + \abracks{u,v} + \abracks{u,v} + \abracks{u,v}}{4} \\
		&= \frac{\abracks{u,v} + \abracks{v,u} + \abracks{u,v} + \abracks{v,u}}{4} \\
		&= \frac{\abracks{u,u} + \abracks{u,v} + \abracks{v,u} + \abracks{v,v} - \abracks{u,u} + \abracks{u,v} + \abracks{v,u} - \abracks{v,v}}{4} \\
		&= \frac{\abracks{u,u+v} + \abracks{v,u+v} - \abracks{u, v-u} + \abracks{v, u-v}}{4} \\
		&= \frac{\abracks{u,u+v} + \abracks{v,u+v} - \abracks{u, v-u} - \abracks{v, v-u}}{4} \\
		&= \frac{\abracks{u,u+v} + \abracks{v,u+v} - \abracks{v, v-u} - \abracks{u, v-u}}{4} \\
		&= \frac{\abracks{u+v,u+v} - \abracks{v-u,v-u}}{4} \\
		&= \frac{\gnorm{u+v}^2 - \gnorm{v-u}^2}{4} \\
		\abracks{u,v} &= \frac{\gnorm{u+v}^2 - \gnorm{u-v}^2}{4}
		\end{align*}
		\end{proof}
	\ee
\ee
\noindent\makebox[\linewidth]{\rule{\paperwidth}{0.4pt}}
	
\end{document}
