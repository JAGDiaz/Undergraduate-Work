\documentclass{article}
\usepackage{amsmath}
\usepackage{amssymb}
\usepackage{bm}
\usepackage{amsthm}
\usepackage{enumerate}
\usepackage{graphicx}
\usepackage{psfrag}
\usepackage{color}
\usepackage{url}
\usepackage{listings}
\usepackage{xcolor}
\usepackage{tikz}
\usetikzlibrary{positioning}
\tikzset{main node/.style={circle,fill=gray!20,draw,minimum size=.5cm,inner sep=0pt},}

\definecolor{codegreen}{rgb}{0,0.5,0}
\definecolor{codewhite}{rgb}{1,1,1}
\definecolor{codegray}{rgb}{0.5,0.5,0.5}
\definecolor{codepurple}{rgb}{0.58,0,0.82}
\definecolor{codeblack}{rgb}{0,0,0}
\definecolor{codeorange}{rgb}{0.8,0.4,0}

\lstdefinestyle{mystyle}{
    backgroundcolor=\color{codewhite},   
    commentstyle=\color{codegray},
    keywordstyle=\color{codegreen},
    numberstyle=\color{codegray},
    stringstyle=\color{codeorange},
    basicstyle=\ttfamily ,
    breakatwhitespace=false,         
    breaklines=true,                 
    captionpos=b,                    
    keepspaces=true,                 
    numbers=left,                    
    numbersep=5pt,                  
    showspaces=false,                
    showstringspaces=false,
    showtabs=false,                  
    tabsize=4
}
\lstset{style=mystyle}


\setlength{\hoffset}{-1in}
\addtolength{\textwidth}{1.5in}
\setlength{\voffset}{-1in}
\addtolength{\textheight}{1.5in}
\newcommand{\be}{\begin{enumerate}}
\newcommand{\ee}{\end{enumerate}}
\newcommand{\BigO}[1]{\ensuremath\mathcal{O}\left(#1\right)}
\newcommand{\il}[1]{\lstinline!#1!}
\newcommand{\gnorm}[1]{\left|\left|#1\right|\right|}
\newcommand{\abs}[1]{\left|#1\right|}
\newcommand{\parens}[1]{\left(#1\right)}
\newcommand{\bracks}[1]{\left\{#1\right\}}
\newcommand{\vep}{\varepsilon}
\newcommand{\ceiling}[1]{\left\lceil#1\right\rceil}


\begin{document}
	\begin{center}
		\textbf{Spring 2020, Math 530:\ Homework 6} \\
		\textbf{Due:\ Wednesday, April 8th, 2020} \\
		\textbf{Joseph Diaz: 819947915}
	\end{center}
\noindent\makebox[\linewidth]{\rule{\paperwidth}{0.4pt}}
Note: For the first two exercises, you can reuse the results obtained in the previous homework assignment.
\be[(E1)]
	\item Let 
	$$\forall x \in (0,1),\ f_n(x) = \frac{1}{1+nx}$$
	Study the uniform convergence of this sequence of functions.
	\begin{proof}[Solution]
	From the previous homework, we found that $f_n$ converges point-wise to $0$. $f_n$, however, does not converge uniformly: 
	$$\exists \vep >0, \exists N \in \mathbb{N}: \forall n \geq N, \exists x \in (0,1),\ \abs{\frac{1}{1+nx} - 0} \geq \vep$$
	\begin{align*}
	\Rightarrow \abs{\frac{1}{1+nx}} = &\frac{1}{1+nx} \geq \vep\\
	\end{align*}
	Let $x = \frac{1}{2}$, then:
	\begin{align*}
	\frac{1}{1+\frac{n}{2}} &\geq \vep\\
	\frac{1}{\vep} \geq 1& + \frac{n}{2}\\
	\frac{1}{\vep} - 1 &\geq \frac{n}{2}\\
	\frac{1-\vep}{\vep} &\geq \frac{n}{2}\\
	\frac{2-2\vep}{\vep} &\geq n\\
	\end{align*}
	So let $N = \ceiling{\frac{2-2\vep}{\vep}}$, then:
	$$\exists \vep >0, N = \ceiling{\frac{2-2\vep}{\vep}}: \forall n \geq N,\exists x \in (0,1),\ \abs{\frac{1}{1+nx} - 0} \geq \vep$$
	So, $f_n(x)$ does not converge uniformly.
	\end{proof}
	
	\item Let 
	$$\forall x \in [0,1],\ f_n(x) = \frac{x}{1+nx}$$
	Study the uniform convergence of this sequence of functions.
	\begin{proof}
	Again, from the previous homework, we found that $$\lim_{n \to \infty} f_n(x)= 0$$
	So:
	$$\abs{f_n(x) - f(x)} = \abs{\frac{x}{1+nx}-0}= \abs{\frac{x}{1+nx}} = \frac{x}{1+nx} = f_n(x)$$
	Then
	$$f'_n(x) = \frac{1}{(1+nx)^2}$$
	We see that $\forall x \in [0,1],\ f'_n(x) > 0$, so the greatest value that $f_n(x)$ can attain is at the right endpoint, $x=1$. So let $B_n = \frac{1}{n+1}$ and 
	$$\forall n \in \mathbb{N},\ \frac{x}{1+nx} < B_n = \frac{1}{n+1}$$
	also $$\lim_{n\to\infty} B_n = 0$$
	So by Proposition 3.1.1,
	$$\forall x \in [0,1],\ \forall n \in \mathbb{N},\ \abs{\frac{x}{1+nx}} < B_n$$
	and $\bracks{f_n}$ converges uniformly on $[0,1]$.
	\end{proof}	
	
	\item Assume that the two sequences of functions $\bracks{f_n: D \rightarrow \mathbb{R}}$ and $\bracks{g_n: D \rightarrow \mathbb{R}}$ converge uniformly to the function $f : D \rightarrow \mathbb{R}$ and $g : D \rightarrow \mathbb{R}$, respectively.\\ Prove that, $\forall\ \alpha, \beta \in \mathbb{R},\ \bracks{\alpha f_n + \beta g_n : D \rightarrow \mathbb{R}}$ converges uniformly to $\alpha f + \beta g : D \rightarrow \mathbb{R}$.
	\begin{proof}
	It is given that $\bracks{f_n}$ and $\bracks{g_n}$ converge uniformly to $f$ and $g$, respectively, so we have:
	\begin{align*}
	\forall \vep_1 > 0, \exists N_1 \in \mathbb{N}: \forall n \geq N_1, \forall x \in D,\ \abs{f_n - f} < \vep_1 \\
	\forall \vep_2 > 0, \exists N_2 \in \mathbb{N}: \forall n \geq N_2, \forall x \in D,\ \abs{g_n - g} < \vep_2
	\end{align*}
	Then, $\forall \alpha, \beta \in \mathbb{R}$:
	\begin{align*}
	\abs{(\alpha f_n + \beta g_n) - (\alpha f + \beta g)} &= \abs{\alpha f_n - \alpha f + \beta g_n - \beta g}\\
	&= \abs{\alpha (f_n - f) + \beta (g_n - g)}\\
	&\leq \abs{\alpha(f_n-f)} + \abs{\beta(g_n-g)}\\
	&= \abs{\alpha}\abs{f_n-f} + \abs{\beta}\abs{g_n-g}\\
	&< \abs{\alpha}\vep_1 + \abs{\beta}\vep_2
	\end{align*}
	Then
	$$\vep = \abs{\alpha}\vep_1 + \abs{\beta}\vep_2, N = \max\bracks{N_1,N_2} : \forall n \geq N, \forall x \in D,\ \abs{(\alpha f_n + \beta g_n) - (\alpha f + \beta g)} < \vep$$
	Which implies that $\bracks{\alpha f_n + \beta g_n}$ converges to $\alpha f + \beta g$.
	\end{proof}	
	
	\item Assume that $\forall n \in \mathbb{N},\ f_n : \mathbb{R} \rightarrow \mathbb{R}$ is bounded and that the sequence $\bracks{f_n}$ converges uniformly to $f : \mathbb{R} \rightarrow \mathbb{R}$.\\
	Prove that $f$ is also bounded.
	\begin{proof}
	$\bracks{f_n}$ converges uniformly to $f$, so we have:
	$$\forall \vep > 0, \exists N \in \mathbb{N}: \forall n \geq N, \forall x \in \mathbb{R},\ \abs{f_n - f} < \vep$$
	and as $f_n$ is bounded, $\exists m, M \in \mathbb{R}: m \leq f_n \leq M$. Taken together, this gives us:
	\begin{align*}
	\abs{f_n - f} &< \vep\\
	\Rightarrow -\vep < f_n &- f < \vep\\
	\Rightarrow -\vep < f &- f_n < \vep\\
	\Rightarrow f_n - \vep < &f < f_n + \vep\\
	\Rightarrow m - \vep \leq f_n - \vep < &f < f_n + \vep \leq M + \vep\\
	\Rightarrow m - \vep < &f < M + \vep
	\end{align*}
	which implies that $f$ is also bounded.
	\end{proof}
	
	\item Let $\forall n \in \mathbb{N},\ \forall x \in (-1,1),\ f_n(x) = \sqrt{x^2 + \frac{1}{n}}$
	\be[1)] 
		\item Show that the point-wise convergence is $f(x) = \left|x\right|$.
		\begin{proof}[Solution]
		To show the point-wise convergence of $\bracks{f_n}$, we'll consider different values of $x$:
		\be 
			\item[$x = 0$:] $$\lim_{n\to\infty}f_n(0) = \lim_{n\to\infty}\sqrt{0^2 + \frac{1}{n}} = \lim_{n\to\infty}\sqrt{\frac{1}{n}} = 0$$
			\item[$0 < \abs{x} < 1$:] $$\lim_{n\to\infty}\sqrt{\abs{x}^2 + \frac{1}{n}} = \sqrt{\lim_{n\to\infty}\parens{\abs{x}^2 + \frac{1}{n}}} = \sqrt{x^2 + \lim_{n\to\infty}\parens{\frac{1}{n}}} = \sqrt{x^2} = \abs{x}$$
		\ee
		So $$f(x) = \left\{\begin{matrix}
		0, & x = 0\\
		\abs{x}, & x \in (-1,0)\cup(0,1)
		\end{matrix}\right. = \abs{x},\forall x \in (-1,1)$$
		\end{proof}
		
		\item Prove that $\bracks{f_n}$ converges uniformly to $f$ on $(-1,1)$.
		\begin{proof}
		We'll start with:
		\begin{align*}
		\abs{f_n(x) - f(x)} &= \abs{\sqrt{x^2 + \frac{1}{n}} - \abs{x}}\\
		\text{(By sub-additivity) }&\leq \abs{\sqrt{x^2} + \sqrt{\frac{1}{n}} - \abs{x}}\\
		&= \abs{\abs{x} + \sqrt{\frac{1}{n}} - \abs{x}}\\
		&= \abs{\sqrt{\frac{1}{n}}} = \sqrt{\frac{1}{n}}
		\end{align*}
		Let $\sqrt{\frac{1}{n}} < \vep$, then
		$$\sqrt{\frac{1}{n}} < \vep \implies \frac{1}{n} < \vep^2 \implies n > \frac{1}{\vep^2}$$
		Now, letting $N = \ceiling{\frac{1}{\vep^2}}$:
		$$\forall \vep > 0, N = \ceiling{\frac{1}{\vep^2}}: \forall n \geq N, \forall x \in (-1,1),\ \abs{\sqrt{x^2 + \frac{1}{n}} - \abs{x}} < \vep$$
		\end{proof}
		
		\item Check that $\forall n \in \mathbb{N}, \forall x \in (-1,1),\ f_n(x)$ is differentiable.
		\begin{proof}[Solution]
		Differentiating $f_n(x)$ gives:
		$$f'_n(x) = \frac{x}{\sqrt{x^2+\frac{1}{n}}}$$
		Which is defined $\forall n \in \mathbb{N}$, and $\forall x \in (-1,1)$.
		\end{proof}
		
		\item Noticing that $f(x)$ is not differentiable at $x=0$, do we get a contradiction with the theorem of differentiability of uniformly converging sequences?
		\begin{proof}[Solution]
		While $f_n(x)$ is differentiable on $(-1,1)$, $f'_n(x)$ does not converge uniformly to $f'(x)$ due to the singularity at $x = 0$, and so it does not satisfy the requirements for \textit{Theorem 3.1.6}. It would, however, if the interval of interest were constrained to $(-1,0)\cup(0,1)$.
		\end{proof}
	\ee
	
\ee 
\noindent\makebox[\linewidth]{\rule{\paperwidth}{0.4pt}}
	
\end{document}
