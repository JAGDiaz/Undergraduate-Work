\documentclass{article}
\usepackage{amsmath}
\usepackage{amssymb}
\usepackage{bm}
\usepackage{amsthm}
\usepackage{enumerate}
\usepackage{graphicx}
\usepackage{psfrag}
\usepackage{color}
\usepackage{url}
\usepackage{listings}
\usepackage{xcolor}
\usepackage{tikz}
\usetikzlibrary{positioning}
\tikzset{main node/.style={circle,fill=gray!20,draw,minimum size=.5cm,inner sep=0pt},}

\definecolor{codegreen}{rgb}{0,0.5,0}
\definecolor{codewhite}{rgb}{1,1,1}
\definecolor{codegray}{rgb}{0.5,0.5,0.5}
\definecolor{codepurple}{rgb}{0.58,0,0.82}
\definecolor{codeblack}{rgb}{0,0,0}
\definecolor{codeorange}{rgb}{0.8,0.4,0}

\lstdefinestyle{mystyle}{
    backgroundcolor=\color{codewhite},   
    commentstyle=\color{codegray},
    keywordstyle=\color{codegreen},
    numberstyle=\color{codegray},
    stringstyle=\color{codeorange},
    basicstyle=\ttfamily ,
    breakatwhitespace=false,         
    breaklines=true,                 
    captionpos=b,                    
    keepspaces=true,                 
    numbers=left,                    
    numbersep=5pt,                  
    showspaces=false,                
    showstringspaces=false,
    showtabs=false,                  
    tabsize=4
}
\lstset{style=mystyle}


\setlength{\hoffset}{-1in}
\addtolength{\textwidth}{1.5in}
\setlength{\voffset}{-1in}
\addtolength{\textheight}{1.5in}
\newcommand{\be}{\begin{enumerate}}
\newcommand{\ee}{\end{enumerate}}
\newcommand{\BigO}[1]{\ensuremath\mathcal{O}\left(#1\right)}
\newcommand{\il}[1]{\lstinline!#1!}
\newcommand{\gnorm}[1]{\left|\left|#1\right|\right|}
\newcommand{\abs}[1]{\left|#1\right|}
\newcommand{\parens}[1]{\left(#1\right)}
\newcommand{\bracks}[1]{\left\{#1\right\}}
\newcommand{\sqbracks}[1]{\left[#1\right]}
\newcommand{\vep}{\varepsilon}
\newcommand{\ceiling}[1]{\left\lceil#1\right\rceil}
\newcommand{\R}{\mathbb{R}}
\newcommand{\N}{\mathbb{N}}

\begin{document}
	\begin{center}
		\textbf{Spring 2020, Math 530:\ Homework 7} \\
		\textbf{Due:\ Wednesday, April 15th, 2020} \\
		\textbf{Joseph Diaz: 819947915}
	\end{center}
\noindent\makebox[\linewidth]{\rule{\paperwidth}{0.4pt}}
\be[(E1)] 
	\item Let $\forall n \in \N,\ \forall x \in [0,1],\ f_n(x) = nxe^{-nx^2}$
	\be[1)]
		\item Show that the pointwise convergence is $f(x) = 0$.
		\begin{proof}
		We'll show this by considering different values of $x$:
		\be 
			\item[$x=0$:]
			$f_n(0) = 0$
			$$\Rightarrow \lim_{n\to\infty} 0 = 0$$
			
			\item[$0<x<1$:]
			$f_n(x) = nxe^{-nx^2}$
			\begin{align*}
			\Rightarrow\lim_{n\to\infty}f_n(x) &= \lim_{n\to\infty}nxe^{-nx^2}\\
			&= x\lim_{n\to\infty}\frac{n}{e^{nx^2}}\\
			&= x\cdot0\\
			&= 0
			\end{align*}
			
			\item[$x=1$:]
			$f_n(1) = ne^{-n}$
			$$\lim_{n\to\infty}f_n(1) = \lim_{n\to\infty} ne^{-n} = \lim_{n\to\infty}\frac{n}{e^n} = 0$$
		\ee
		So $\forall x \in [0,1], \bracks{f_n}$ converges pointwise to $f(x) = 0$.
		\end{proof}
		
		\item Show that 
		$$\lim_{n\to\infty}\int_0^1 f_n(x)\ dx \neq \int_0^1 f(x)\ dx$$
		\begin{proof}
		We'll prove this directly, first:
		\begin{align*}
		\lim_{n\to\infty}\int_0^1 f_n(x)\ dx &= \lim_{n\to\infty}\int_0^1 nxe^{-nx^2}\ dx\\
		&= -\frac{1}{2}\lim_{n\to\infty}\int_0^1 \parens{-2nxe^{-nx^2}}\ dx\\
		&= -\frac{1}{2}\lim_{n\to\infty}\sqbracks{e^{-nx^2}}_0^1\\
		&= -\frac{1}{2}\lim_{n\to\infty}\sqbracks{e^{-n} - 1}\\
		&= -\frac{1}{2}(0 - 1)\\
		&= \frac{1}{2}
		\end{align*}
		And second:
		\begin{align*}
		\int_0^1 f(x)\ dx &= \int_0^1 0\ dx\\
		&= \sqbracks{C}_0^1 \quad \parens{C \in \R}\\
		&= C - C\\
		&= 0
		\end{align*}
		Clearly, $0 \neq \frac{1}{2}$, so:
		$$\lim_{n\to\infty}\int_0^1 f_n(x)\ dx \neq \int_0^1 f(x)\ dx$$
		\end{proof}
		\item Conclude about the uniform convergence of $\bracks{f_n}$.
		\begin{proof}[Conclusion]
		It is a necessary condition for uniform convergence that
		$$\lim_{n\to\infty}\int_a^b f_n(x)\ dx = \int_a^b f(x)\ dx = \int_a^b\lim_{n\to\infty}f_n(x)\ dx$$
		But we just found that this equality is violated in this case; so we may conclude that $f_n(x)$ does not converge uniformly to $f(x) = 0$.
		\end{proof}
	\ee
	
	\item Give the open interval of convergence and the radius of convergence of the series
	$$\sum_{k=1}^\infty \frac{x^k}{k\cdot5^k}$$
	\begin{proof}[Solution]
	Using the Ratio Test:
	$$a_{k} = \frac{x^k}{k\cdot5^k}, \quad a_{k+1} = \frac{x^{k+1}}{(k+1)\cdot5^{k+1}}$$
	\begin{align*}
	\lim_{k\to\infty}\abs{\frac{a_{k+1}}{a_k}} &= \lim_{k\to\infty}\abs{\frac{x^{k+1}}{(k+1)\cdot5^{k+1}}\cdot\frac{k\cdot5^k}{x^k}}\\
	&= \lim_{k\to\infty}\abs{\frac{x^{k}x}{(k+1)\cdot5^{k}\cdot5}\cdot\frac{k\cdot5^k}{x^k}}\\
	&= \frac{\abs{x}}{5}\lim_{k\to\infty}\abs{\frac{k}{k+1}}\\
	&= \frac{\abs{x}}{5}
	\end{align*}
	Now to find where the series converges:
	$$\frac{\abs{x}}{5}< 1 \implies \abs{x} < 5$$
	So the open interval of convergence is $x \in (-5, 5)$, and the radius of convergence is $R = 5$.
	\end{proof}
	
	\item Give the open interval of convergence and the radius of convergence of the series
	$$\sum_{k=1}^\infty k!x^k$$
	\begin{proof}[Solution]
	Using the Ratio Test:
	$$a_{k} = k!x^k, \quad a_{k+1} = (k+1)!x^{k+1}$$
	\begin{align*}
	\lim_{k\to\infty} \abs{\frac{a_{k+1}}{a_k}} &= \lim_{k\to\infty}\abs{\frac{(k+1)!x^{k+1}}{k!x^k}}\\
	&= \lim_{k\to\infty}\abs{\frac{(k+1)k!x^kx}{k!x^k}}\\
	&= \abs{x}\lim_{k\to\infty}\parens{k+1}\\
	&\to \infty
	\end{align*}
	This result implies that the open interval of convergence is $\emptyset$ and the radius of convergence is $R = 0$.
	\end{proof}
	
	\item Show that $\forall x \in (-1,1),$ 
	$$\frac{1}{1+x} = \sum_{k=0}^\infty (-1)^k x^k$$
	\begin{proof}
	To start the proof itself, we will show that the series on the right converges $\forall x \in (-1,1)$.\\
	Using the Ratio Test:
	$$a_k = (-1)^kx^k, \quad a_{k+1} = (-1)^{k+1}x^{k+1}$$
	\begin{align*}
	\lim_{k\to\infty}\abs{\frac{a_{k+1}}{a_k}} &= \lim_{k\to\infty} \abs{\frac{(-1)^{k+1}x^{k+1}}{(-1)^kx^k}}\\
	&= \lim_{k\to\infty} \abs{\frac{(-1)^{k}x^{k}(-1)x}{(-1)^kx^k}}\\
	&= \abs{x}
	\end{align*}
	So the open interval of convergence is $x \in (-1, 1)$ and the radius of convergence $R = 1$, as desired. Now that we know the series converges, with some algebra:
	\begin{align*}
	\sum_{k=0}^\infty (-1)^k x^k = 1 - x + x^2 - x^3 + x^4 - \cdots\\
	\sum_{k=0}^\infty (-1)^k x^k = 1 - x\parens{1 - x + x^2 - x^3 +\cdots}\\
	\sum_{k=0}^\infty (-1)^k x^k = 1 - x\sum_{k=0}^\infty (-1)^k x^k\\
	\sum_{k=0}^\infty (-1)^k x^k + x\sum_{k=0}^\infty (-1)^k x^k = 1 \\
	\parens{1+x}\sum_{k=0}^\infty (-1)^k x^k = 1\\
	\sum_{k=0}^\infty (-1)^k x^k = \frac{1}{1+x}
	\end{align*}
	As desired.
	\end{proof}
	
	\item Show that $\forall x \in (-1,1),$
	$$\frac{1}{\parens{1+x}^2} = \sum_{k=0}^\infty (-1)^{k+1} kx^{k-1}$$
	(\emph{Hint: use previous exercise.})
	\begin{proof}
	From (E4), we know that 
	$$\sum_{k=0}^\infty (-1)^k x^k = \frac{1}{1+x}$$ 
	and, by Theorem 3.3.4, the series is differentiable. So:
	\begin{align*}
	\sum_{k=0}^\infty (-1)^k x^k &= \frac{1}{1+x}\\
	\frac{d}{dx}\parens{\sum_{k=0}^\infty (-1)^k x^k} &= \frac{d}{dx}\parens{\frac{1}{1+x}}\\
	\sum_{k=1}^\infty (-1)^k kx^{k-1} &= -\frac{1}{\parens{1+x}^2}\\
(\text{Adding 0th summand})\ 	\sum_{k=0}^\infty (-1)^k kx^{k-1} &= -\frac{1}{\parens{1+x}^2}\\
	-\sum_{k=0}^\infty (-1)^k kx^{k-1} &= \frac{1}{\parens{1+x}^2}\\
	\sum_{k=0}^\infty (-1)^{k+1} kx^{k-1} &= \frac{1}{\parens{1+x}^2}\\
	\end{align*}
	As desired.
	\end{proof}
	
	\item Find the $3^{rd}$ order Taylor polynomial approximation of $$f(x) = \int_0^x \frac{1}{1+t^2}\ dt$$
	around $x_0 = 0$.
	\begin{proof}[Solution]
	We want to find 
	\begin{align*}
	f(x) \approx \frac{1}{0!}f(0)x^0 + \frac{1}{1!}f'(0)x^1 + \frac{1}{2!}f''(0)x^2 + \frac{1}{3!}f'''(0)x^3
	\end{align*}
	So now we find the derivatives of $f(x) = \int_0^x \frac{1}{1+t^2}\ dt$:
	\begin{align*}
	f'(x) &= \frac{1}{1+x^2}\quad (\text{By FTC I})\\
	f''(x) &= -\frac{2x}{\parens{1+x^2}^2}\\
	f'''(x) &= \frac{6x^2-2}{\parens{1+x^2}^3}
	\end{align*}
	And evaluating these at $x_0 = 0$ gives $f(0) = 0,\ f'(0) = 1,\ f''(0) = 0,\ f'''(0) = -2$. So the $3^{rd}$ order Taylor polynomial approximation of $f(x)$ is given by:
	$$f(x) \approx x + \frac{1}{3}x^3$$
	
	\end{proof}
	
	\item Find the $3^{rd}$ order Taylor polynomial approximation of $f(x) = \sin(x)$ around $x_0 = 0$.
	\begin{proof}[Solution]
	We want to find 
	\begin{align*}
	f(x) \approx \frac{1}{0!}f(0)x^0 + \frac{1}{1!}f'(0)x^1 + \frac{1}{2!}f''(0)x^2 + \frac{1}{3!}f'''(0)x^3
	\end{align*}
	So now we find the derivatives of $f(x) = \sin(x)$:
	\begin{align*}
	f'(x) &= \cos(x)\\
	f''(x) &= -\sin(x)\\
	f'''(x) &= -\cos(x)
	\end{align*}
	And evaluating these at $x_0 = 0$ gives $f(0) = 0,\ f'(0) = 1,\ f''(0) = 0,\ f'''(0) = -1$. So the $3^{rd}$ order Taylor polynomial approximation of $f(x)$ is given by:
	$$f(x) \approx x - \frac{1}{6}x^3$$
	
	\end{proof}
	
	\item Let $f : \R \to \R$ be an infinitely differentiable function and assume that this function has the following properties:
	$$\forall x \in \R,\ f''(x) + f(x) = e^{-x}\quad \text{and}\quad f(0) = 0,\ f'(0) = 2$$
	Find the $4^{th}$ order Taylor polynomial approximation of $f$ around $x_0 = 0$.
	\begin{proof}[Solution]
	We want to find 
	$$f(x) \approx \frac{1}{0!}f(0)x^0 + \frac{1}{1!}f'(0)x^1 + \frac{1}{2!}f''(0)x^2 + \frac{1}{3!}f'''(0)x^3 + \frac{1}{4!}f^{(4)}(0)x^4$$
	We already have $f(0)$ and $f'(0)$, so to find the rest of the derivatives evaluated at 0 we'll use our other property
	$$f''(x) + f(x) = e^{-x} \implies f''(x) = e^{-x}-f(x)$$
	which allows use to get expressions for the higher derivatives
	\begin{align*}
	f''(x) = e^{-x} - f(x) &\implies f''(0) = e^0 - f(0) = 1\\
	f'''(x) = -e^{-x} - f'(x) &\implies f'''(0) = -e^0 - f'(0) = -3\\
	f^{(4)}(x) = e^{-x} - f''(x) &\implies f^{(4)}(0) = e^0 - f''(0) = 0 
	\end{align*}
	So the $4^{th}$ order Taylor polynomial approximation of $f(x)$ is:
	$$f(x) \approx 2x + \frac{1}{2}x^2 -\frac{1}{2}x^3 $$
	\end{proof}
\ee
\noindent\makebox[\linewidth]{\rule{\paperwidth}{0.4pt}}
	
\end{document}
