\documentclass[12pt,a4paper]{article}
\usepackage[latin1]{inputenc}
\usepackage{amsmath}
\usepackage{amsfonts}
\usepackage{amssymb}
\usepackage{amsthm}
\usepackage{graphicx}
\usepackage{mdframed}
\usepackage{xcolor}
\usepackage{amsmath}
\usepackage[left=0.8in, right=0.8in, top=1.00in, bottom=1.00in]{geometry}
\usepackage{mathtools}
\def\multichoose#1#2{\ensuremath{\left(\kern-.3em\left(\genfrac{}{}{0pt}{}{#1}{#2}\right)\kern-.3em\right)}}
\author{Jackson Autry}

%% New Commands
\newcommand{\C}{\mathbb{C}}
\newcommand{\B}{\mathcal{B}}
\newcommand{\h}{\mathcal{H}}
\renewcommand{\S}{\mathcal{S}}
\newcommand{\Z}{\mathbb{Z}}
\newcommand{\N}{\mathbb{N}}
\newcommand{\R}{\mathbb{R}}
\newcommand{\LL}{\mathcal{L}}
\newcommand{\RR}{\mathcal{R}}
\newcommand{\D}{\mathbb{D}}
\newcommand{\F}{\mathcal{F}}
\newcommand{\cl}{\operatorname{cl}}
\newcommand{\ran}{\operatorname{ran}}
\newcommand{\norm}[1]{\| #1 \|}
\newcommand{\inner}[1]{\langle #1 \rangle}
\renewcommand{\vec}[1]{{\bf #1}}

%%%
%%% Theorem Styles
%%%
% \numberwithin{equation}{section}
\theoremstyle{plain}
\newtheorem{Theorem}[equation]{Theorem}
\newtheorem{Lemma}[equation]{Lemma}
\newtheorem{Proposition}[equation]{Proposition}
\newtheorem{Corollary}[equation]{Corollary}
\theoremstyle{remark}
\newtheorem{Remark}[equation]{Remark}
\theoremstyle{definition}
\newtheorem{Definition}[equation]{Definition}
\newtheorem{Example}[equation]{Example}
\newcounter{question}
\newtheorem{Question}[question]{Question}

\begin{document}
	\newmdenv[linecolor=gray!40,leftmargin=0,%
	innerleftmargin=4pt,
	rightmargin=0pt,
	innerbottommargin=3pt,
	skipbelow=4pt,
	backgroundcolor=gray!40,%
	innertopmargin=3pt,]{ques}

	
\begin{center}
	\textbf{MATH 630A} %############ Insert Course Title! #############
	
		Homework 5	%############### Put Homework # ###################
		
		Jackson Autry
\end{center}

\section*{Exercise 1}

	Let the space $(\mathcal{C}([0,1]),\Vert . \Vert_{L^1})$ where
	$$\forall f \in \mathcal{C}([0,1]),\Vert f \Vert_{L^1} = \int_0^1 \vert f(x) \vert dx$$

\subsection*{1.1. Show that $(\mathcal{C}([0,1]),\Vert . \Vert_{L^1})$ is not a Banach space.}

We look at the sequence $\{f_n\} = \{x^n\}$. First note that $\forall n \in \N, x^N \in \mathcal{C}([0,1])$, and that as $n\to \infty$, $x^n \to \begin{cases}
0 & 0\le x \le 1 \\ 1 & x = 1
\end{cases}$ which is not in $\mathcal{C}([0,1])$. We now show that $\{f_n\}$ is a Cauchy sequence, which shows the space is not complete and hence not a Banach space.\\
We want to show $\forall \epsilon > 0, \exists N \in \N, \forall n,k \in \N, n\ge N, \Vert f_{n+k} - f_{n} \Vert_{L^1} < \epsilon$.\\
\begin{flalign*}
	\Vert f_{n+k} - f_n \Vert_{L^1} = \int_0^1 \vert f_{n+k}(x) - f_n(x) \vert dx = \int_0^1 \vert x^{n+k} - x^n \vert dx &&
\end{flalign*}
Then, $\forall x \in [0,1], x^{n+k} \le x^{n} \Leftrightarrow x^{n+k} - x^n \le 0$, so
\begin{flalign*}
	\Vert f_{n+k} - f_n \Vert_{L^1} &= \int_0^1 \vert x^n - x^{n+k} \vert dx = \int_0^1 x^n - x^{n+k} dx = \left[\frac{x^{n+1}}{n+1} - \frac{x^{n+k+1}}{n+k+1}\right]_0^1 &&\\
	&= \frac{1}{n+1} - \frac{1}{n+k+1} < \frac{1}{n+1} < \frac{1}{n} 
\end{flalign*}
Now, $n \ge N \Leftrightarrow \frac{1}{n} \le \frac{1}{N}$. We want $\frac{1}{N} < \epsilon \Leftrightarrow \frac{1}{\epsilon} < N$. So we have\\
$\forall \epsilon>0, \exists N \in \N, N > \frac{1}{\epsilon}, \forall n,k \in \N, n \ge N, \Vert f_{n+k} - f_{n} \Vert_{L^1} < \frac{1}{N} < \epsilon$\\
Hence $\{f_n\}$ is a Cauchy sequence but does not converge to a function in $\mathcal{C}([0,1])$, hence $(\mathcal{C}([0,1]), \Vert . \Vert_{L^1})$ is not a Banach space.

\subsection*{1.2. Show that if $\forall \{f_n\} \in \mathcal{C}([0,1]), \exists f \in \mathcal{C}([0,1]), f_n \to f$ with respect to $\Vert . \Vert_{\infty}$ then $f_n \to f$ with respect to $\Vert . \Vert_{L^1}$.}

Suppose that $\forall \{f_n\} \in \mathcal{C}([0,1]), \exists f \in \mathcal{C}([0,1]), f_n \to f$ with respect to $\Vert . \Vert_{\infty}$, then we have\\
$\forall \epsilon > 0, \exists N \in N, \forall n \in \N, n \ge N, \Vert f_n - f\Vert_{\infty} < \epsilon$, or $\max\limits_{x \in [0,1]} \vert f_n(x) - f(x) \vert < \epsilon$.\\
Now observe that $\forall n \in \N, \forall x \in [0,1], \vert f_n(x) - f(x) \vert \le \max\limits_{x \in [0,1]} \vert f_n(x) - f(x) \vert$. So we have\\
\begin{flalign*}
	\Vert f_n - f \Vert_{L^1} &= \int_0^1 \vert f_n(x) - f(x) \vert dx \le \int_0^1 \max\limits_{x \in [0,1]} \vert f_n(x) - f(x) \vert dx = \max\limits_{x \in [0,1]} \int_0^1 dx &&\\
	&= \max\limits_{x \in [0,1]} \vert f_n(x) - f(x) \vert = \Vert f_n - f \Vert_{\infty}
\end{flalign*}
Thus we have $\forall \epsilon > 0,  \exists N \in \N, \forall n \in \N, n \ge N, \Vert f_n - f \Vert_{L^1} \le \Vert f_n - f \Vert_{\infty} < \epsilon$,\\
i.e. $f_n \to f$ with respect to $\Vert . \Vert_{L^1}$.
\section*{Exercise 2}
Let denote 
$$\mathcal{C}^1([a,b]) = \{f:[a,b]\to \R/f, f' \text{ are continuous }\}$$
the space of continuously differentiable functions on $[a,b]$. Define
$$\forall f \in \mathcal{C}^1([a,b]), \Vert f \Vert_{\mathcal{C}^1} = \Vert f \Vert_{\infty} + \Vert f' \Vert_{\infty}.$$

\subsection*{2.1. Show that $\Vert . \Vert_{\mathcal{C}^1}$ defines a norm.}

$\forall f,g \in \mathcal{C}^1([a,b]), \forall \alpha \in \R$,\\

\noindent Since $\Vert. \Vert_{\infty}$ is a norm, we have $\Vert f \Vert_{\mathcal{C}^1} = \Vert f \Vert_{\infty} + \Vert f' \Vert_{\infty} \ge 0 + 0 = 0$, and\\
$\Vert f \Vert_{\mathcal{C}^1} = 0 \Leftrightarrow \Vert f \Vert_{\infty} + \Vert f' \Vert_{\infty} = 0 \Leftrightarrow \Vert f \Vert_{\infty} = 0$ and $\Vert f' \Vert_{\infty} = 0 \Leftrightarrow f = 0$ and $f' = 0 \Leftrightarrow f=0$.\\
\begin{flalign*}
\Vert \alpha f \Vert_{\mathcal{C}^1} &= \Vert \alpha f \Vert_{\infty} + \Vert (\alpha f)' \Vert_{\infty} = \Vert \alpha f \Vert_{\infty} + \Vert \alpha f' \Vert_{\infty} = \vert \alpha \vert \Vert f \Vert_{\infty} + \vert \alpha \vert \Vert f' \Vert_{\infty} = \vert \alpha \vert \left(\Vert f \Vert_{\infty} + \Vert f' \Vert_{\infty}\right) &&\\
&= \vert \alpha \vert \Vert f \Vert_{\mathcal{C}^1}
\end{flalign*}

\begin{flalign*}
	\Vert f + g \Vert_{\mathcal{C}^1} &= \Vert f + g \Vert_{\infty} + \Vert (f + g)' \Vert_{\infty} = \Vert f + g \Vert_{\infty} + \Vert f' + g' \Vert_{\infty} \le \Vert f \Vert_{\infty} + \Vert g \Vert_{\infty} + \Vert f' \Vert_{\infty} + \Vert g' \Vert_{\infty} &&\\
	&= \Vert f \Vert_{\mathcal{C}^1} + \Vert g \Vert_{\mathcal{C}^1} 
\end{flalign*}
Hence $\Vert . \Vert_{\mathcal{C}^1}$ defines a norm on $\mathcal{C}^1([a,b])$.

\subsection*{2.2. Show that $(\mathcal{C}^1([a,b]),\Vert . \Vert_{\mathcal{C}^1})$ is a Banach space.}

Let $\{f_n\}$ be an arbitrary Cauchy sequence in $(\mathcal{C}^1([a,b]),\Vert . \Vert_{\mathcal{C}^1})$. Then we have \\
$\forall \epsilon > 0, \exists N \in \N, \forall n,m \in \N, n\ge N, m\ge N, \Vert f_m - f_n \Vert_{\mathcal{C}^1} < \epsilon$\\
$\Leftrightarrow \forall \epsilon > 0, \exists N \in \N, \forall n,m \in \N, n\ge N, m\ge N, \Vert f_m - f_n \Vert_{\infty} + \Vert f_m' - f_n' \Vert_{\infty} < \epsilon$\\
$\Leftrightarrow \forall \epsilon > 0, \exists N \in \N, \forall n,m \in \N, n\ge N, m\ge N, \left(\Vert f_m - f_n \Vert_{\infty} < \epsilon \land \Vert f_m' - f_n' \Vert_{\infty} < \epsilon\right)$\\
i.e. $\{f_n\}$ and $\{f'_n\}$ are Cauchy sequences in $(\mathcal{C}^1([a,b]),\Vert . \Vert_{\infty})$. By Theorem 3.6.1, $(\mathcal{C}^1([a,b]), \Vert . \Vert_{\infty})$ is complete, so $f_n \to f \in (\mathcal{C}^1([a,b]), \Vert . \Vert_{\infty})$ and $f_n' \to g \in (\mathcal{C}^1([a,b]), \Vert . \Vert_{\infty})$. It remains to show that $g = f'$.\\
Since $\{f_n\}$ is a sequence of continuous functions and converges to $f$ with respect to $\Vert . \Vert_{\infty}$, then it converges to $f$ uniformly, i.e. if $f_n \to f$, then $f_n' \to f'$, i.e. $g = f'$.\\
So we have that $\Vert f_n - f \Vert_{\infty}<\epsilon$ and $\Vert f_n' - f' \Vert_{\infty} < \epsilon$, so $\Vert f_n - f \Vert_{\mathcal{C}^1} < 2\epsilon = \epsilon'$, i.e. $\{f_n\}$ converges to a function in $(\mathcal{C}^1([a,b]), \Vert . \Vert_{\infty})$.\\
Hence $(\mathcal{C}^1([a,b]), \Vert . \Vert_{\infty})$ is a Banach space.

\pagebreak
\section*{Exercise 3}
Let $w:[0,1]\to \R$ be a nonnegative continuous function. We define the \emph{weighted supremum} norm by 
$$\forall f \in \mathcal{C}([0,1]), \Vert f \Vert_w = \sup\limits_{x \in [0,1]} w(x)\vert f(x) \vert$$

\subsection*{3.1. If $\forall x \in (0,1), w(x) > 0$, show that $\Vert.\Vert_w$ is a norm on $\mathcal{C}([0,1])$.}

$\forall f,g \in \mathcal{C}([0,1]), \forall \alpha \in \R$, we have\\

\noindent since $\forall x \in [0,1] w(x) \ge 0$, we have $\Vert f \Vert_w = \sup\limits_{x \in [0,1]} w(x) \vert f(x) \vert \ge 0$, and\\
if $f = 0$, then $\Vert f \Vert_w = \sup\limits_{x \in [0,1]} w(x) \vert 0 \vert = \sup\limits_{x \in [0,1]} 0 = 0$. If $\Vert f \Vert_w = 0$, then $\sup\limits_{x \in [0,1]} w(x) \vert f \vert = 0$, so for each $x \in (0,1)$, either $w(x) = 0$ or $f(x) = 0$.\\
Since $\forall x \in (0,1), w(x) > 0$, then $\forall x \in (0,1), f(x) = 0$. Since $f(x)$ is continuous, $f(0) = \lim\limits_{x \to 0^+} f(x) = 0$ and $f(1) = \lim\limits_{x \to 1^-} f(x) = 0$. Hence $f(x) = 0$.\\
So $\Vert f \Vert_w = 0 \Leftrightarrow f = 0$.\\
\begin{flalign*}
	\Vert \alpha f \Vert_w &= \sup\limits_{x \in [0,1]} w(x) \vert \alpha f(x) \vert = \sup\limits_{x \in [0,1]} w(x) \vert a \vert \vert f(x) \vert = \vert \alpha \vert \sup\limits_{x \in [0,1]} w(x) \vert f(x) \vert = \vert \alpha \vert \Vert f \Vert_w &&
\end{flalign*}
\begin{flalign*}
	\Vert f+g \Vert_w &= \sup\limits_{x \in [0,1]} w(x) \vert (f+g)(x) \vert = \sup\limits_{x \in [0,1]} w(x) \vert f(x) + g(x) \vert \le \sup\limits_{x \in [0,1]} w(x) \left(\vert f(x) \vert + \vert g(x) \vert\right) &&\\
	&\le \sup\limits_{x \in [0,1]} w(x) \vert f(x) \vert + \sup\limits_{x \in [0,1]} w(x) \vert g(x) \vert = \Vert f \Vert_w + \Vert g \Vert_w
\end{flalign*}

\subsection*{3.2. If $\forall x \in [0,1], w(x) > 0$, show that $\Vert . \Vert_w$ is equivalent to $\Vert . \Vert_{\infty}$.}

We show $\exists c,c' > 0, \forall f \in \mathcal{C}([0,1]), \left(\Vert f \Vert_{\infty} \le c \Vert f \Vert_w \land \Vert f \Vert_w \le c' \Vert f \Vert_{\infty}\right)$.\\
\begin{flalign*}
	\Vert f \Vert_{\infty} &= \sup\limits_{x \in [0,1]} \vert f(x) \vert = \sup\limits_{x \in [0,1]} \frac{w(x)}{w(x)} \vert f(x) \vert \le \left(\sup\limits_{x \in [0,1]} \frac{1}{w(x)}\right)\sup\limits_{x \in [0,1]} w(x) \vert f(x) \vert &&\\
	&= \left(\sup\limits_{x \in [0,1]} \frac{1}{w(x)} \right) \Vert f \Vert_w 
\end{flalign*}
Note that since $\forall x \in [0,1], w(x) > 0$, it is valid to divide by $w(x)$.\\
Also $\forall x \in [0,1], \frac{1}{w(x)} > 0$, so $\sup\limits_{x \in [0,1]} \frac{1}{w(x)} > 0$.
\begin{flalign*}
	\Vert f \Vert_w = \sup\limits_{x \in [0,1]}w(x) \vert f(x) \vert \le \left(\sup\limits_{x \in [0,1]} w(x)\right) \sup\limits_{x \in [0,1]} \vert f(x) \vert = \left(\sup\limits_{x \in [0,1]} w(x)\right) \Vert f \Vert_{\infty}
\end{flalign*}
Note here that since $\forall x \in [0,1], w(x) > 0$ then $\sup\limits_{x \in [0,1]} w(x) > 0$.\\

Hence $\Vert . \Vert_w$ and $\Vert . \Vert_{\infty}$ are equivalent norms.

	
\end{document}
