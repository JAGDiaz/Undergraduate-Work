\documentclass[12pt,a4paper]{article}
\usepackage[latin1]{inputenc}
\usepackage{amsmath}
\usepackage{amsfonts}
\usepackage{amssymb}
\usepackage{amsthm}
\usepackage{graphicx}
\usepackage{mdframed}
\usepackage{xcolor}
\usepackage{amsmath}
\usepackage[left=0.8in, right=0.8in, top=1.00in, bottom=1.00in]{geometry}
\usepackage{mathtools}
\def\multichoose#1#2{\ensuremath{\left(\kern-.3em\left(\genfrac{}{}{0pt}{}{#1}{#2}\right)\kern-.3em\right)}}
\author{Jackson Autry}

%% New Commands
\newcommand{\C}{\mathbb{C}}
\newcommand{\B}{\mathcal{B}}
\newcommand{\h}{\mathcal{H}}
\renewcommand{\S}{\mathcal{S}}
\newcommand{\Z}{\mathbb{Z}}
\newcommand{\N}{\mathbb{N}}
\newcommand{\R}{\mathbb{R}}
\newcommand{\LL}{\mathcal{L}}
\newcommand{\RR}{\mathcal{R}}
\newcommand{\D}{\mathbb{D}}
\newcommand{\F}{\mathcal{F}}
\newcommand{\cl}{\operatorname{cl}}
\newcommand{\ran}{\operatorname{ran}}
\newcommand{\norm}[1]{\| #1 \|}
\newcommand{\inner}[1]{\langle #1 \rangle}
\renewcommand{\vec}[1]{{\bf #1}}

%%%
%%% Theorem Styles
%%%
% \numberwithin{equation}{section}
\theoremstyle{plain}
\newtheorem{Theorem}[equation]{Theorem}
\newtheorem{Lemma}[equation]{Lemma}
\newtheorem{Proposition}[equation]{Proposition}
\newtheorem{Corollary}[equation]{Corollary}
\theoremstyle{remark}
\newtheorem{Remark}[equation]{Remark}
\theoremstyle{definition}
\newtheorem{Definition}[equation]{Definition}
\newtheorem{Example}[equation]{Example}
\newcounter{question}
\newtheorem{Question}[question]{Question}

\begin{document}
	\newmdenv[linecolor=gray!40,leftmargin=0,%
	innerleftmargin=4pt,
	rightmargin=0pt,
	innerbottommargin=3pt,
	skipbelow=4pt,
	backgroundcolor=gray!40,%
	innertopmargin=3pt,]{ques}

	
\begin{center}
	\textbf{MATH 630A} %############ Insert Course Title! #############
	
		Homework 6	%############### Put Homework # ###################
		
		Jackson Autry
\end{center}

\section*{Exercise 1}

	Let $\delta : \mathcal{C}([0,1]) \to \R$ the operator which evaluates a function $f \in \mathcal{C}([0,1])$ at the origin, i.e. $\delta f = f(0).$

\subsection*{1.1. Show that if $\mathcal{C}([0,1])$ is equipped with the max-norm, $\Vert f \Vert_{\infty} = \sup\limits_{x \in [0,1]}\vert f(x) \vert$, then $\delta$ is a bounded operator and compute its norm.}

$$\vert \delta f \vert = \vert f(0) \vert \le \sup\limits_{x \in [0,1]} \vert f(x) \vert = \Vert f \Vert_{\infty},$$
i.e. $\delta$ is bounded.\\
Then 
$$\Vert \delta \Vert = \underset{f \ne 0}{\sup\limits_{f \in \mathcal{C}([0,1])}} \frac{\vert \delta f \vert}{\Vert f \Vert_{\infty}} \le \underset{f \ne 0}{\sup\limits_{f \in \mathcal{C}([0,1])}}\frac{\Vert f \Vert_{\infty}}{\Vert f \Vert_{\infty}} = 1$$
So $\Vert \delta \Vert \le 1$. We now show $\Vert \delta \Vert \ge 1$.\\
Let $f(x) = 1$. Then $f(x) \in \mathcal{C}([0,1])$, and $\vert \delta f \vert = \vert 1 \vert =1$, and $\Vert f \Vert_{\infty} = \sup\limits_{x \in [0,1]} \vert 1 \vert = 1$. Thus $\Vert \delta \Vert \ge 1$.\\
Hence $\Vert \delta \Vert = 1$.

\subsection*{1.2. Show that if $\mathcal{C}([0,1])$ is equipped with the one-norm, $\Vert f\Vert_1 = \int_0^1 \vert f(x) \vert dx$, then $\delta$ is unbounded.}

We show $\ker \delta$ is not closed and so $\delta$ is not bounded.\\
We study the sequence $\{f_n\}$ defined by $\forall n \in \N, f_n = x^{1/n}$.\\
Then $\forall n \in \N, f_n \in \mathcal{C}([0,1])$, and $\lim\limits_{n\to \infty} f_n = \lim\limits_{n\to \infty}x^{1/n} = 1 \in \mathcal{C}([0,1])$.\\
But $\forall n \in \N, \delta (x^{1/n}) = 0$, and $\delta 1 = 1$. Hence $f_n \in \ker \delta,$ but $\lim\limits_{n\to \infty}f_n = 1 \not\in \ker\delta$, i.e. $\ker \delta$ is not closed.\\
Hence $\delta$ is unbounded.

\pagebreak
\section*{Exercise 2}
Let $T \in \mathcal{B}(\mathcal{C}([0,1]))$ defined by
$$\forall f \in \mathcal{C}([0,1]), \forall x\in [0,1], (Tf)(x) = xf(x)$$
and consider that $\mathcal{B}(\mathcal{C}([0,1]))$ is equipped with the max-norm. Find $\Vert T\Vert$.

$$\Vert T \Vert = \underset{\Vert f \Vert_{\infty} = 1}{\sup\limits_{f \in \mathcal{C}([0,1])}} \Vert Tf \Vert_{\infty}$$
If $\Vert f \Vert_{\infty} = 1$, then $\sup\limits_{x \in [0,1]} \vert f(x) \vert = 1$. So\\
$\Vert Tf \Vert_{\infty} = \sup\limits_{x \in [0,1]} \vert x f(x) \vert \le \sup\limits_{x \in [0,1]} \vert x \vert \sup\limits_{x \in [0,1]}\vert f(x) \vert = \sup\limits_{x \in [0,1]} \vert x \vert = 1$.\\
Thus $\Vert T \Vert \le 1$. We now show that $\Vert T \Vert \ge 1$.\\
Let $f(x) = 1$. Then $f\in\mathcal{C}([0,1])$ and $\Vert f \Vert_{\infty} = 1$.\\
Then $\Vert Tf \Vert_{\infty} = \sup\limits_{x \in [0,1]} \vert x (1) \vert = \sup\limits_{x \in [0,1]} \vert x \vert = 1$\\
Thus $\Vert T \Vert \ge 1$ and we have that $\Vert T \Vert = 1$.

\section*{3}
Let a linear operator $A:\R^n \to \R^m$, $A$ can be written as a matrix, let write this matrix as $A = (A_1 A_2 \ldots A_n)$ where $\forall i = 1,2,\ldots,n$, $A_i$ represents the $i-th$ column of $A$.

\subsection*{3.1. Find $\Vert A \Vert$ when $\R^n$ and $\R^m$ are both equipped with $\Vert . \Vert_{\ell^1}$}
\begin{flalign*}
	\Vert A \Vert & = \underset{\Vert x \Vert_{\ell^1} = 1}{\sup\limits_{x \in \R^n}} \Vert Ax \Vert_{\ell^1} = \underset{\Vert x \Vert_{\ell^1} = 1}{\sup\limits_{x \in \R^n}} \left\Vert \sum_{i = 1}^{n} A_i x_i \right\Vert_{\ell^1} \le \underset{\Vert x \Vert_{\ell^1} = 1}{\sup\limits_{x \in \R^n}} \sum_{i = 1}^n \Vert A_i x_i \Vert_{\ell^1} = \underset{\Vert x \Vert_{\ell^1} = 1}{\sup\limits_{x \in \R^n}} \sum_{i = 1}^n \vert x_i \vert \Vert A_i \Vert_{\ell^1} &&\\
	&\le \underset{\Vert x \Vert_{\ell^1} = 1}{\sup\limits_{x \in \R^n}} \sum_{i = 1}^n \vert x_i \vert \max\limits_{j = 1,2,\ldots n} \Vert A_j \Vert_{\ell^1} = \underset{\Vert x \Vert_{\ell^1} = 1}{\sup\limits_{x \in \R^n}} \max\limits_{j = 1,2,\ldots n} \Vert A_j \Vert_{\ell^1} \sum_{i = 1} \vert x_i \vert \\
	&= \max\limits_{j = 1,2,\ldots n} \Vert A_j \Vert_{\ell^1} \underset{\Vert x \Vert_{\ell^1} = 1}{\sup\limits_{x \in \R^n}}  \sum_{i = 1} \vert x_i \vert = \max\limits_{j = 1,2,\ldots n} \Vert A_j \Vert_{\ell^1} \underset{\Vert x \Vert_{\ell^1} = 1}{\sup\limits_{x \in \R^n}} \Vert x \Vert_{\ell^1}  = \max\limits_{j = 1,2,\ldots n} \Vert A_j \Vert_{\ell^1}
\end{flalign*}

Thus $\Vert A \Vert \le \max\limits_{j = 1,2,\ldots n} \Vert A_j \Vert_{\ell^1}$. We now show $\Vert A \Vert \ge \max\limits_{j = 1,2,\ldots n} \Vert A_j \Vert_{\ell^1}$.\\
Let $j \in \{1,2,\ldots,n\}$ such that $\Vert A_j \Vert_{\ell^1} = \max\limits_{i = 1,2,\ldots,n} \Vert A_i \Vert_{\ell^1}$ and $e_j$ be the vector whose $j-th$ entry is 1 and the rest $0$. Clearly $\Vert e_j \Vert_{\ell^1} = 1$, and we have $Ae_j = A_j$, so 
$$\Vert Ae_j \Vert_{\ell^1} = \Vert A_j \Vert_{\ell^1} = \max\limits_{i = 1,2,\ldots,n} \Vert A_i \Vert_{\ell^1}$$
Thus $\Vert A \Vert \ge \max\limits_{j = 1,2,\ldots n} \Vert A_j \Vert_{\ell^1}$, and so we have\\
$\Vert A \Vert = \max\limits_{i = 1,2,\ldots,n} \Vert A_i \Vert_{\ell^1}$.

\pagebreak
\subsection*{3.2. Find $\Vert A \Vert$ when $\R^n$ and $\R^m$ are both equipped with $\Vert . \Vert_{\ell^{\infty}}$}

Denote $A_{ij}$ the element of $A$ in the $i-th$ column and the $j-th$ row.\\
If $\Vert x \Vert_{\ell^{\infty}} = 1,$ then $\forall i = 1,2,\ldots, n, \vert x_i \vert \le 1$. Then we have
\begin{flalign*}
	\Vert A \Vert & = \underset{\Vert x \Vert_{\ell^{\infty}} =1}{\sup\limits_{x\in \R^n}} \left\Vert \sum_{i =1}^n A_ix_i \right\Vert = \underset{\Vert x \Vert_{\ell^{\infty}} =1}{\sup\limits_{x\in \R^n}} \left(\max\limits_{j = 1,2,\ldots m} \left\vert \sum_{i = 1}^n A_{ij}x_i\right\vert\right) \le \underset{\Vert x \Vert_{\ell^{\infty}} =1}{\sup\limits_{x\in \R^n}} \left(\max\limits_{j = 1,2,\ldots m}  \sum_{i = 1}^n \vert A_{ij}\vert \vert x_i\vert\right) &&\\
	&\le \underset{\Vert x \Vert_{\ell^{\infty}} =1}{\sup\limits_{x\in \R^n}} \left(\max\limits_{j = 1,2,\ldots m} \sum_{i = 1}^n \vert A_{ij}\vert \right) = \max\limits_{j = 1,2,\ldots m} \sum_{i = 1}^n \vert A_{ij} \vert
\end{flalign*}
Thus $\Vert A \Vert \le \max\limits_{j = 1,2,\ldots m} \sum\limits_{i = 1}^n \vert A_{ij} \vert$. We now show that $\Vert A \Vert \ge \max\limits_{j = 1,2,\ldots m} \sum\limits_{i = 1}^n \vert A_{ij} \vert$.\\
Let $A_{\ast k} = (A_{1k}, A_{2k}, \cdots, A_{nk})$ be the row of $A$ such that $\sum\limits_{i=1}^n \vert A_{ik} \vert = \max\limits_{j = 1,2,\ldots m} \sum\limits_{i=1}^n \vert A_{ij} \vert$ and $x = (x_1,x_2,\ldots, x_n) \in \R^n$ defined by $\forall i= 1,2,\ldots, n, x_i = \begin{cases}
1 & A_{ik} \ge 0 \\ -1 & A_{ik} < 0
\end{cases}$. Then $\Vert x \Vert_{\ell^{\infty}} = 1$.\\
Furthermore, $\forall i = 1,2,\ldots, n, A_{ik} x_i = \vert A_{ik} \vert$, and since \\
$\forall j = 1,2,\ldots m, \left\vert \sum\limits_{i=1}^n A_{ij}x_i \right\vert \le \sum\limits_{i = 1}^n \vert A_{ij}x_i \vert \le \sum\limits_{i=1}^n \vert A_{ik} \vert$, we have\\
\begin{flalign*}
\Vert Ax \Vert_{\ell^{\infty}} = \left\Vert \sum_{i =1}^n A_ix_i \right\Vert_{\ell^{\infty}} = \max_{j=1,2,\ldots,m} \left\vert \sum_{i = 1}^n A_{ij} x_i \right\vert = \sum_{i =1}^n \vert A_{ik} \vert = \max_{j=1,2,\ldots,m} \sum_{i =1}^n \vert A_{ij}\vert &&
\end{flalign*}

Thus $\Vert A \Vert \ge \max\limits_{j = 1,2,\ldots m} \sum\limits_{i = 1}^n \vert A_{ij} \vert$ and we have\\
$\Vert A \Vert = \max\limits_{j = 1,2,\ldots m} \sum\limits_{i = 1}^n \vert A_{ij} \vert$.

	
\end{document}
