\documentclass[12pt,a4paper]{article}
\usepackage[latin1]{inputenc}
\usepackage{amsmath}
\usepackage{amsfonts}
\usepackage{amssymb}
\usepackage{graphicx}
\usepackage{mdframed}
\usepackage{xcolor}
\usepackage{amsmath}
\usepackage[parfill]{parskip}
\setlength{\parskip}{0 mm}
\usepackage[left=0.50in, right=0.50in, top=1.00in, bottom=1.00in]{geometry}
\author{Jackson Autry}
\begin{document}
	\newmdenv[linecolor=gray!40,leftmargin=0,%
	innerleftmargin=4pt,
	rightmargin=40pt,
	innerbottommargin=0pt,
	skipbelow=1pt,
	backgroundcolor=gray!40,%
	innertopmargin=4pt,]{ques}

	
\begin{center}
	\textbf{MATH 630A} %############ Insert Course Title! #############
	
		Homework Assignment 1		%############### Put Homework # ###################
	
\end{center}
1. 
\begin{ques}
	Use the $\epsilon$-definition of limits to prove that
	\[ 
	\lim\limits_{x \rightarrow 2}\frac{x^2-3x+2}{x-2} = 1.
	\]
\end{ques}
	
	\[
	\left| \frac{x^2-3x+2}{x-2} - 1 \right| = \left| \frac{x^2-3x+2}{x-2} - \frac{x-2}{x-2} \right| = \left| \frac{x^2 - 3x + 2 - x + 2}{x - 2} \right| = \left| \frac{x^2 - 4x + 4}{x-2} \right| 
	\]
	\[
	= \left| \frac{(x - 2)^2}{x-2} \right| = \left| {x - 2} \right|
	\]
	Setting $\delta = \epsilon$, we have 
	\[
	\left| \frac{x^2-3x+2}{x-2} - 1 \right| = \left| x-2 \right| < \delta = \epsilon
	\]
	hence, $\forall \epsilon > 0, \exists \delta > 0, \forall x \in \mathbb{R}, |x - 2| < \delta \longrightarrow \left| \frac{x^2 - 3x + 2}{x - 2} - 1 \right| < \epsilon$ 
	
2.
\begin{ques}
	Let the sequence $\{x_n\}_{n \in \mathbb{N}}$ defined by $x_n = \frac{3n}{n+1}.$
\end{ques}

2.1
\begin{ques}
	Show that $\{x_n\}$ is a Cauchy sequence.
\end{ques}

\begin{equation*}
\begin{split}
\left| \frac{3n}{n+1} - \frac{3m}{m+1} \right| & = \left| \frac{3n(m+1) - 3m(n+1)}{(n+1)(m+1)} \right| = \left| \frac{3nm + 3n - 3nm - 3m}{(n+1)(m+1)} \right| = \left| \frac{3n - 3m}{nm + n + m + 1} \right| \\
 & \leq \left| \frac{3n}{nm + n + m + 1} \right| + \left| \frac{3m}{nm + n + m + 1} \right| < \left| \frac{3n}{nm + n} \right| + \left| \frac{3m}{nm + m} \right| \\
 & = \left| \frac{3}{m + 1} \right| + \left| \frac{3}{n + 1} \right| \leq \frac{3}{N + 1} + \frac{3}{N + 1} = \frac{6}{N + 1}
\end{split}
\end{equation*}

By setting $N > \frac{6}{\epsilon} - 1$, we have

\[
\left| \frac{3n}{n+1} - \frac{3m}{m+1} \right| < \frac{6}{N+1} < \epsilon
\]
hence $\forall \epsilon, \exists N \in \mathbb{N}, \forall n,m \in \mathbb{N}, n \leq N, m \leq N, \left| \frac{3n}{n+1} - \frac{3m}{m+1} \right| < \epsilon$, i.e. $\{x_n\}$ is Cauchy.

2.2
\begin{ques}
	Does this sequence converge?
\end{ques}

Since this sequence is Cauchy, $\forall n \in \mathbb{N}, x_n \in \mathbb{R}$, and the real numbers are complete, this sequence converges.

\pagebreak
3
\begin{ques}
	Let the sequence of functions $\{f_n(x)\}$ where $\forall x \in (0,1), \forall n \in \mathbb{N}, f_n(x) = \sqrt{x^4 + \frac{1}{n^2}}$
\end{ques}
3.1
\begin{ques}
	Find the pointwise convergence of this sequence on $(0,1)$.
\end{ques}

Let $\forall x \in (0,1)$, $f_n(x) = \sqrt{x^4 + \frac{1}{n^2}}$, so

\[
\lim\limits_{n \rightarrow \infty} f_n(x) = \lim\limits_{n \rightarrow \infty}\sqrt{x^4 + \frac{1}{n^2}} \overset{*}{=} \sqrt{x^4 + \lim\limits_{n \rightarrow \infty} \frac{1}{n^2}} = \sqrt{x^4} = x^2
\]
* Note that we can move the limit inside the function since $\sqrt{x}$ is continuous on $(0,1)$.

So $f_n(x)$ converges pointwise to $x^2$.

3.2
\begin{ques}
	Does this sequence converge uniformly on $(0,1)$?
\end{ques}

\begin{align*}
\left| \sqrt{x^4 + \frac{1}{n^2}} - x^2 \right| & = \left| \left(\sqrt{x^4 + \frac{1}{n^2}} - x^2\right) \frac{\sqrt{x^4 + \frac{1}{n^2}} + x^2}{\sqrt{x^4 + \frac{1}{n^2}} + x^2} \right| = \left| \frac{x^4 + \frac{1}{n^2}-x^4}{\sqrt{x^4 + \frac{1}{n^2}} + x^2} \right| = \left| \frac{1}{n^2(\sqrt{x^4 + \frac{1}{n^2}} + x^2)}\right|\\
& = \left| \frac{1}{n(\sqrt{n^2x^4 + 1} + nx^2)}\right| \overset{*}{<} \frac{1}{n}
\end{align*}

* Note that since $\sqrt{n^2x^4 + 1} + nx^2 > 1$, we have this inequality.

By setting $B_n = \frac{1}{n}$ we can see that $\lim\limits_{n \rightarrow \infty} B_n = 0$, and so,

\[\forall x \in (0,1), \forall n \in \mathbb{N}, \left| \sqrt{x^4 + \frac{1}{n^2}} - x^2 \right| \leq B_n
\]

i.e. $\{f_n(x)\}$ converges uniformly to $x^2$.

\end{document}
