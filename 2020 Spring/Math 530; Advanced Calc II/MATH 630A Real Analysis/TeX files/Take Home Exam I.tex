\documentclass[12pt,a4paper]{article}
\usepackage[latin1]{inputenc}
\usepackage{amsmath}
\usepackage{amsfonts}
\usepackage{amssymb}
\usepackage{amsthm}
\usepackage{graphicx}
\usepackage{mdframed}
\usepackage{xcolor}
\usepackage{amsmath}
\usepackage[left=0.8in, right=0.8in, top=1.00in, bottom=1.00in]{geometry}
\usepackage{mathtools}
\def\multichoose#1#2{\ensuremath{\left(\kern-.3em\left(\genfrac{}{}{0pt}{}{#1}{#2}\right)\kern-.3em\right)}}
\author{Jackson Autry}

%% New Commands
\newcommand{\C}{\mathbb{C}}
\newcommand{\B}{\mathcal{B}}
\newcommand{\h}{\mathcal{H}}
\renewcommand{\S}{\mathcal{S}}
\newcommand{\Z}{\mathbb{Z}}
\newcommand{\N}{\mathbb{N}}
\newcommand{\R}{\mathbb{R}}
\newcommand{\K}{\mathbb{K}}
\newcommand{\LL}{\mathcal{L}}
\newcommand{\RR}{\mathcal{R}}
\newcommand{\D}{\mathbb{D}}
\newcommand{\F}{\mathcal{F}}
\newcommand{\cl}{\operatorname{cl}}
\newcommand{\ran}{\operatorname{ran}}
\newcommand{\norm}[1]{\| #1 \|}
\newcommand{\inner}[1]{\langle #1 \rangle}
\renewcommand{\vec}[1]{{\bf #1}}

%%%
%%% Theorem Styles
%%%
% \numberwithin{equation}{section}
\theoremstyle{plain}
\newtheorem{Theorem}[equation]{Theorem}
\newtheorem{Lemma}[equation]{Lemma}
\newtheorem{Proposition}[equation]{Proposition}
\newtheorem{Corollary}[equation]{Corollary}
\theoremstyle{remark}
\newtheorem{Remark}[equation]{Remark}
\theoremstyle{definition}
\newtheorem{Definition}[equation]{Definition}
\newtheorem{Example}[equation]{Example}
\newcounter{question}
\newtheorem{Question}[question]{Question}

\begin{document}
	\newmdenv[linecolor=gray!40,leftmargin=0,%
	innerleftmargin=4pt,
	rightmargin=0pt,
	innerbottommargin=3pt,
	skipbelow=4pt,
	backgroundcolor=gray!40,%
	innertopmargin=3pt,]{ques}

	
\begin{center}
	\textbf{MATH 630A} %############ Insert Course Title! #############
	
		Take Home Exam 1	%############### Put Homework # ###################
		
		Jackson Autry
\end{center}

\section*{Problem 1 (Minkowski's inequality)}
We start by proving the first two easy properties:\\

\noindent Note that in both cases of $\K, \forall u \in \K, \vert u \vert \geq 0,$ and $\vert u \vert = 0 \Leftrightarrow u = o$. Then we have\\
$\forall x = (x_1,x_2,\ldots,x_n) \in \K^n, \forall k = 1,2,\ldots,n, \vert x_k \vert \ge 0 \Rightarrow \vert x_k \vert \ge 0$, and so\\
$\sum\limits_{k=1}^n \vert x_k \vert^p \ge 0 \Rightarrow \left( \sum\limits_{k=1}^n \vert x_k \vert^p \right)^{1/p} \ge 0$.\\
Furthermore, $\left( \sum\limits_{k=1}^n \vert x_k \vert^p \right)^{1/p} = 0$ if and only if each $\vert x_k \vert = 0 \Leftrightarrow x_k = 0$, i.e. $x = 0$. Hence we have\\
$\forall x \in \K^n, \Vert x \Vert_p \ge 0$ and $\Vert x \Vert_p = 0 \Leftrightarrow x = 0$.\\

\noindent Now, $\forall \alpha,y \in \K, \vert \alpha y \vert = \vert \alpha \vert \vert y \vert$, so $\forall \alpha \in \K$ we have
\begin{flalign*}
\Vert \alpha x \Vert_p &= \left( \sum\limits_{k=1}^n \vert \alpha x_k \vert^p \right)^{1/p} = \left( \sum\limits_{k=1}^n \vert\alpha\vert^p \vert x_k \vert^p \right)^{1/p} = \left( \vert\alpha\vert^p\sum\limits_{k=1}^n \vert x_k \vert^p \right)^{1/p} = \vert\alpha\vert\left( \sum\limits_{k=1}^n \vert x_k \vert^p \right)^{1/p} &&\\
&= \vert\alpha\vert\Vert x \Vert_p
\end{flalign*}

\noindent Now for the triangle inequality:\\
\noindent We start in an unusual way, looking at $\forall x,y \in \K^n, \frac{\Vert x + y \Vert_p}{\Vert x \Vert_p + \Vert y \Vert_p}$ and prove this is at most 1.\\
Let $x,y\in \K^n$. Then
\begin{flalign*}
	\frac{\Vert x+y \Vert_p}{\Vert x \Vert_p + \Vert y \Vert_p} &= \left(\frac{1}{(\Vert x \Vert_p + \Vert y \Vert_p)^p} \sum_{k=1}^n (\vert x_k + y_k \vert ^p \right)^{1/p} = \left(\sum_{k=1}^n \left(\frac{\vert x_k + y_k \vert}{\Vert x \Vert_p + \Vert y \Vert_p}\right)^p\right)^{1/p} &&\\
	&\le \left(\sum_{k=1}^n \left(\frac{\vert x_k \vert + \vert y_k \vert}{\Vert x \Vert_p + \Vert y \Vert_p}\right)^p\right)^{1/p} = \left(\sum_{k=1}^n \left(\frac{\vert x_k \vert}{\Vert x \Vert_p + \Vert y \Vert_p} + \frac{\vert y_k \vert}{\Vert x \Vert_p + \Vert y \Vert_p}\right)^p\right)^{1/p} \\
	& = \left(\sum_{k=1}^n \left(\frac{\Vert x \Vert_p}{\Vert x \Vert_p + \Vert y \Vert_p}\frac{\vert x_k \vert}{\Vert x \Vert_p} + \frac{\Vert y \Vert_p}{\Vert x \Vert_p + \Vert y \Vert_p}\frac{\vert y_k \vert}{\Vert y \Vert_p}\right)^p\right)^{1/p}
\end{flalign*}
Now each term is of the form $(\alpha u + (1-\alpha)v)^p$,\\
with $\alpha = \frac{\Vert x \Vert_p}{\Vert x \Vert_p + \Vert y \Vert_p}, 1 - \alpha = 1 - \frac{\Vert x \Vert_p}{\Vert x \Vert_p + \Vert y \Vert_p} = \frac{\Vert y \Vert_p}{\Vert x \Vert_p + \Vert y \Vert_p}, u = \frac{\vert x_k \vert}{\Vert x \Vert_p}, v = \frac{\vert y_k\vert}{\Vert y \Vert_p}$.\\
So using the convexity hint, we have
\begin{flalign*}
	\frac{\Vert x+y \Vert_p}{\Vert x \Vert_p + \Vert y \Vert_p} & \le \left(\sum_{k=1}^n \left(\frac{\Vert x \Vert_p}{\Vert x \Vert_p + \Vert y \Vert_p}\frac{\vert x_k \vert}{\Vert x \Vert_p} + \frac{\Vert y \Vert_p}{\Vert x \Vert_p + \Vert y \Vert_p}\frac{\vert y_k \vert}{\Vert y \Vert_p}\right)^p\right)^{1/p} &&\\
	& \le \left(\sum_{k=1}^n \frac{\Vert x \Vert_p}{\Vert x \Vert_p + \Vert y \Vert_p}\left(\frac{\vert x_k \vert}{\Vert x \Vert_p}\right)^p + \frac{\Vert y \Vert_p}{\Vert x \Vert_p + \Vert y \Vert_p}\left(\frac{\vert y_k \vert}{\Vert y \Vert_p}\right)^p \right)^{1/p}\\
	& = \left( \frac{\Vert x \Vert_p}{\Vert x \Vert_p + \Vert y \Vert_p}\frac{1}{\Vert x \Vert_p^p} \sum_{k=1}^n \vert x_k \vert^p + \frac{\Vert y \Vert_p}{\Vert x \Vert_p + \Vert y \Vert_p}\frac{1}{\Vert y \Vert_p^p} \sum_{k=1}^n \vert y_k \vert^p \right)^{1/p} \\
	& = \left(\frac{\Vert x \Vert_p}{\Vert x \Vert_p + \Vert y \Vert_p}\frac{1}{\Vert x \Vert_p^p}\Vert x \Vert_p^p + \frac{\Vert y \Vert_p}{\Vert x \Vert_p + \Vert y \Vert_p}\frac{1}{\Vert y \Vert_p^p}\Vert y \Vert_p^p \right)^{1/p}\\
	& = \left(\frac{\Vert x \Vert_p}{\Vert x \Vert_p + \Vert y \Vert_p} + \frac{\Vert y \Vert_p}{\Vert x \Vert_p + \Vert y \Vert_p} \right)^{1/p} = \left(1\right)^{1/p} = 1
\end{flalign*}
Hence we have

$$\frac{\Vert x+y \Vert_p}{\Vert x \Vert_p + \Vert y \Vert_p} \le 1 \Longleftrightarrow \Vert x+y \Vert_p \le \Vert x \Vert_p + \Vert y \Vert_p $$
And we have the triangle inequality. Hence $\Vert . \Vert_p$ defines a norm.

\section*{Problem 2 (H\"{o}lder's inequality)}
\subsection*{2.1}
We indeed study the function $\forall a \in \R_+, f(a) = \frac{a^p}{p} + \frac{b^q}{q} - ab,$\\
where $b \in \R_+, p,q \in \R, 1 < p < \infty, \frac{1}{p} + \frac{1}{q} = 1$.\\
We first note that $f(0) = \frac{0^p}{p} + \frac{b^q}{q} - (0)b = \frac{b^q}{q} \ge 0$\\
So $f(a)$ is nonnegative at $a = 0$. Now let's look at $f'(a)$:\\
$f'(a) = \frac{d}{da} \frac{a^p}{p} + \frac{b^q}{q} - ab = \frac{pa^{p-1}}{p} -b = a^{p-1}-b$.\\
Looking at critical points, i.e. where $f' = 0$, we see there is only one, at $a = b^{\frac{1}{p-1}}$.\\
Plugging this back into $f$, we have\\
$f (b^{\frac{1}{p-1}}) = \frac{b^{\frac{1}{p-1}p}}{p} + \frac{b^q}{q} - b^{\frac{1}{p-1}}b = \frac{b^q}{p} + \frac{b^q}{q} -b^{q} = b^q\left(\frac{1}{p} + \frac{1}{q}\right) - b^q = b^q (1) - b^q = 0$.\\
So not only is $a = b^{\frac{1}{p-1}}$ a critical point, but it is also a root. Hence, $\forall a \in [0,b^{\frac{1}{p-1}}], f(a) \ge 0$ (since there only one critical point, $f$ is decreasing until $a = b^{\frac{1}{p-1}}$). Now we only need to evaluate $f$ at one point greater than $a = b^{\frac{1}{p-1}}$ to find the sign of $f(a)$ for $a> b^{\frac{1}{p-1}}$ (since there is only one critical point). We evaluate at $a = (pb)^{\frac{1}{p-1}} > b^{\frac{1}{p-1}}$, since $p > 1$.\\
$ f((pb)^{\frac{1}{p-1}}) = \frac{(pb)^{\frac{1}{p-1}p}}{p} = \frac{b^q}{q} - (pb)^{\frac{1}{p-1}}b = \frac{p^qb^q}{p} + \frac{b^q}{q} - p^{\frac{1}{p-1}}b^{\frac{1}{p-1}}b = p^\frac{1}{p-1}b^q + \frac{b^q}{q} - p^{\frac{1}{p-1}}b^q = \frac{b^q}{q} \ge 0$.\\
Note that when we evaluated all points ($0,b^{\frac{1}{p-1}}, (pb)^{\frac{1}{p-1}}$), there are different if $b \neq 0$. The case where $b=0$ is simple, $f(a) = \frac{a^p}{p} \ge 0$ for all $a \in \R_+$.\\
Hence we have $\forall a,b\in \R_+, \forall p,q \in \R, 1<p<\infty, \frac{1}{p} + \frac{1}{q} = 1, \frac{a^p}{p} + \frac{b^q}{q} - ab \ge 0 \Longleftrightarrow \frac{a^p}{p} + \frac{b^q}{q} \ge ab$.

\subsection*{2.2}
Before choosing $a,b$ we motivate the choice by starting similarly to proving Minkowski's inequality, and seeing where it would be nice to use H\"{o}lder's inequality:\\
We look at $\frac{1}{\Vert x \Vert_p\Vert y\Vert_q}\sum\limits_{k=1}^n \vert x_k \vert\vert y_k \vert$ and aim to prove this is at most 1.\\
\begin{flalign*}
	\frac{1}{\Vert x \Vert_p\Vert y\Vert_q}\sum_{k=1}^n \vert x_k \vert\vert y_k \vert = \sum_{k=1}^n \frac{\vert x_k \vert}{\Vert x \Vert_p}\frac{\vert y_k \vert}{\Vert y \Vert_p}&&
\end{flalign*}
Now we see it would be nice to have this as a sum,\\
so we choose $a = \frac{\vert x_k \vert}{\Vert x \Vert_p} \in \R_+$ and $b = \frac{\vert y_k \vert}{\Vert y \Vert_p} \in \R_+$. So we have:
\begin{flalign*}
	\frac{1}{\Vert x \Vert_p\Vert y\Vert_q}\sum_{k=1}^n \vert x_k \vert\vert y_k \vert &= \sum_{k=1}^n \frac{\vert x_k \vert}{\Vert x \Vert_p}\frac{\vert y_k \vert}{\Vert y \Vert_p} \le \sum_{k=1}^n \left(\frac{\left(\vert x_k \vert / \Vert x \Vert_p\right)^p}{p} + \frac{\left(\vert y_k \vert / \Vert y \Vert_q\right)^q}{q}\right) = &&\\
	&= \sum_{k=1}^n \left(\frac{\vert x_k \vert^p}{p\Vert x \Vert_p^p} + \frac{\vert y_k \vert^q}{q \Vert y \Vert_q^q}\right) = \frac{1}{p\Vert x \Vert_p^p} \sum_{k=1}^n \vert x_k \vert^p + \frac{1}{q\Vert y \Vert_q^q}\sum_{k=1}^n \vert y_k \vert^q\\
	& = \frac{\Vert x \Vert_p^p}{p\Vert x \Vert_p^p} + \frac{\Vert y \Vert_q^q}{q \Vert y \Vert_q^q} = \frac{1}{p} + \frac{1}{q} = 1
\end{flalign*}
So we have
\begin{flalign*}
	\frac{1}{\Vert x \Vert_p\Vert y\Vert_q}\sum_{k=1}^n \vert x_k \vert\vert y_k \vert \le 1 \Longleftrightarrow \sum_{k=1}^n \vert x_k \vert\vert y_k \vert \le \Vert x \Vert_p\Vert y\Vert_q&&
\end{flalign*}

\subsection*{2.3}
When $p = q = 2$, this is  Cauchy's Inequality, or the Cauchy-Schwartz Inequality.

\pagebreak
\section*{Problem 3 (Lipschitz Spaces)}
\subsection*{3.1}
We use the usual function addition and scalar multiplication, i.e.\\
$\forall x \in X, \forall f,g \in Lip(X), \forall \alpha \in \K, (\alpha f + g)(x) = \alpha f(x) + g(x)$.\\
Then $\forall f,g \in Lip(X)$, we have:\\
Since $f,g \in Lip(X), \exists c_1,c_2 > 0, \forall x,y \in X, \vert f(x) - f(y) \le c_1d(x,y)$ and $\vert g(x) - g(y) \le c_2d(x,y)$. Then $(f+g):X \to \R$ and\\
$\vert (f+g)(x) - (f+g)(y)\vert = \vert f(x) + g(x) - f(y) - g(y) \vert \\
\le \vert f(x) - f(y) \vert + \vert g(x) - g(y) \vert < c_1 d(x,y) + c_2 d(x,y) = (c_1 + c_2) d(x,y)$.\\
So $\exists c = c_1 + c_2 > 0, \forall x,y \in X, \vert(f+g)(x) - (f+g)(y)\vert < cd(x,y)$, i.e.
$\forall f,g \in Lip(X), f+g \in Lip(X)$.\\
Also, the following properties follow from the definition of function addition and scalar multiplication:\\
$\forall f,g,h \in Lip(X), \forall \alpha, \beta \in \K,$,
\begin{itemize}
	\item 
	$f + g = g + f$
	
	\item 
	$(f + g) + h = f + (g + h)$
	
	\item 
	$(\alpha + \beta)f = \alpha f + \beta f$
	
	\item
	$\alpha(f + g) = \alpha f + \alpha g$
	
	\item 
	$(\alpha\beta)f = \alpha(\beta f)$
\end{itemize}

Then $f(x) = 1 \in Lip(X)$, since setting $c = 0$, we have\\
$\forall x,y \in X, \vert f(x) - f(y) \vert = \vert 1 - 1 \vert = 0 \le 0d(x,y)$, and $\forall g \in Lip(X), 1g = g$.\\
So we have a multiplicative identity.

We also have that $f(x) = 0 \in Lip(X)$, since again setting $c = 0$, we have\\
$\forall x,y \in X, \vert f(x) - f(y) \vert = \vert 0 - 0 \vert = 0 \le 0d(x,y)$, and $\forall g \in Lip(X), 0 + g = g$.\\
So we have an additive identity.\\

It only remains to prove that for any function in $Lip(X)$, there exists a unique additive inverse.\\
Let $f \in Lip(X)$. then $\exists c >0, \forall x,y \in X, \vert f(x) - f(y) \vert \le cd(x,y)\\
\Leftrightarrow \forall x,y \in X, \vert -f(x) - (-f(y)) \vert \le cd(x,y)$, i.e. $-f \in Lip(X)$,\\
and $f + (-f) = 0$. Now we prove the uniqueness:\\
Suppose $\exists g \in Lip(X), f+g=0$. Then\\
$f+g+(-f) = 0 + (-f) \Longleftrightarrow f-f+g = -f \Longleftrightarrow g = -f$. Hence $-f$ is unique.\\

Therefore $Lip(X)$ indeed defines a vector space.

\pagebreak
\subsection*{3.2}
\subsubsection*{3.2.1}
$\forall x,y \in X, \forall f \in Lip(X), \Vert f \Vert_{Lip,a} = \vert f(a) \vert + Lip(f) = \vert f(a) \vert \sup\limits_{x\neq y} \frac{\vert f(x) - f(y) \vert}{d(x,y)}$.\\
Since $\forall a \in X, \forall f \in Lip(X), \vert f(a) \vert \ge 0, \vert f(x) - f(y) \vert \ge 0,$ and $d(x,y) \ge 0$, then $\vert f(a) \vert \sup\limits_{x\neq y} \frac{\vert f(x) - f(y) \vert}{d(x,y)} \ge 0$.\\
If $\vert f(a) \vert \sup\limits_{x\neq y} \frac{\vert f(x) - f(y) \vert}{d(x,y)} = 0$, then $f(a) = 0$ and $\sup\limits_{x\neq y} \frac{\vert f(x) - f(y) \vert}{d(x,y)} = 0 \Leftrightarrow f(x) = 0, \forall x \in X$, i.e. $f = 0$.\\
Hence $\forall f \in Lip(X), \Vert f \Vert_{Lip,a} \ge 0$ and $\Vert f \Vert_{Lip,a} = 0 \Leftrightarrow f = 0$.\\

\noindent Now, $\forall \alpha \in \K, \Vert \alpha f \Vert_{Lip,a} = \vert \alpha f(a) \vert + Lip(\alpha f) = \vert \alpha f(a) \vert + \sup\limits_{x\neq y}\frac{\vert \alpha f(x) - \alpha f(y)\vert}{d(x,y)} \\
= \vert \alpha \vert \vert f(a) \vert + \sup\limits_{x\neq y} \frac{\vert \alpha \vert \vert f(x) - f(y) \vert}{d(x,y)} = \vert \alpha \vert \vert f(a) \vert + \vert \alpha \vert \sup\limits_{x\neq y}\frac{\vert f(x) - f(y)\vert}{d(x,y)} = \vert \alpha \vert \left(\vert f(a) \vert + \sup\limits_{x\neq y}\frac{\vert f(x) - f(y)\vert}{d(x,y)}\right)\\
= \vert \alpha \vert \Vert f \Vert_{Lip,a}$ \\

\noindent Finally, for the triangle inequality, we have\\
$\Vert f+g \Vert_{Lip,a} = \vert (f+g)(a) \vert + Lip(f+g) = \vert f(a) + g(a) \vert + \sup\limits_{x\neq y} \frac{\vert(f+g)(x) - (f+g)(y) \vert}{d(x,y)} \\
= \vert f(a) + g(a) \vert + \sup\limits_{x\neq y}\frac{ \vert f(x) - f(y) + g(x) - g(y) \vert}{d(x,y)} \le \vert f(a) \vert + \vert g(a) \vert + \sup\limits_{x\neq y} \frac{ \vert f(x) - f(y) \vert + \vert g(x) - g(y) \vert}{d(x,y)} \\
\le \vert f(a)\vert + \vert g(a) \vert + \sup\limits_{x\neq y} \frac{\vert f(x) - f(y) \vert}{d(x,y)} + \sup\limits_{x\neq y} \frac{\vert g(x) - g(y) \vert}{d(x,y)} = \vert f(a) \vert + Lip(f) + \vert g(a) \vert + Lip(g) \\
= \Vert f \Vert_{Lip,a} + \Vert g \Vert_{Lip,a}$\\

\noindent Hence $(Lip(X),\Vert . \Vert_{Lip,a})$ does indeed define a normed vector space.

\subsubsection*{3.2.2}
By definition of $Lip(f)$, we have
$$\frac{ \vert f(b) - f(a) \vert}{d(b,a)} \le Lip(f) \Leftrightarrow \vert f(b) - f(a) \vert \le d(b,a)Lip(f)$$
Then
\begin{flalign*}
	\Vert f \Vert_{Lip,b} &= \vert f(b) \vert + Lip(f) = \vert f(b) - f(a) + f(a) \vert + Lip(f) \le \vert f(b) - f(a) \vert + \vert f(a) + Lip(f) &&\\
	&\le \vert f(a) \vert + d(b,a)Lip(f) + Lip(f) = \vert f(a) \vert + (1 + d(b,a))Lip(f)
\end{flalign*}
Since $\forall a,b \in X, d(b,a) \ge 0, 1+ d(a,b) \ge 1 \Leftrightarrow \vert f(a)\vert \le (1+d(b,a)) \vert f(a) \vert$, and we have
\begin{flalign*}
	\Vert f \Vert_{Lip,b} &\le (1 + d(b,a))\vert f(a) \vert + (1 + d(b,a))Lip(f) = (1 + d(b,a))(\vert f(a) \vert + Lip(f)) &&\\
	&= (1 + d(b,a))\Vert f \Vert_{Lip,a}
\end{flalign*}
Hence $\exists c>0, c = 1 + d(b,a), \forall f \in Lip(X), \Vert f \Vert_{Lip,b} \le c\Vert f \Vert_{Lip,a}$.\\
Using the exact same steps but switching $a$ and $b$, we have\\
$\exists c>0, c = 1 + d(a,b), \forall f \in Lip(X), \Vert f \Vert_{Lip,a} \le c\Vert f \Vert_{Lip,b}$.\\
Hence $\Vert.\Vert_{Lip,a}$ and $\Vert.\Vert_{Lip,b}$ are equivalent norms.

\pagebreak
\subsection*{3.3}
\subsubsection*{3.3.1}
Since $\forall x \in X, f_n(x) \in \R$, then for fixed $x, \{f_n(x)\}$ is a sequence of real numbers, furthermore, it is a Cauchy Sequence of real numbers. Since the reals is complete, for any $x \in X$, the sequence $\{f_n(x)\}$ converges to a real number. Then define $f(x) = \lim\limits_{n\to \infty} f_n(x)$. Hence $\{f_n\}$ converges pointwise to the function $f$.

\subsubsection*{3.3.2}
\begin{flalign*}
	\lim\limits_{n\to \infty} \Vert f_n - f\Vert_{Lip,a} &= \lim\limits_{n\to \infty}\left( \vert f_n(a) - f(a) \vert + Lip(f_n + f)\right) &&\\
	&= \lim\limits_{n\to \infty}\left( \vert f_n(a) - f(a) \vert + \sup\limits_{x\neq y} \frac{\vert(f_n + f)(x) - (f_n+f)(y)\vert}{d(x,y)}\right) \\
	&= \lim\limits_{n\to \infty}\left( \vert f_n(a) - f(a) \vert + \sup\limits_{x\neq y} \frac{\vert f_n(x) + f(x) - f_n(y)-f(y)\vert}{d(x,y)}\right)\\
	& \le  \lim\limits_{n\to \infty}\left( \vert f_n(a) - f(a) \vert + \sup\limits_{x\neq y} \frac{\vert f_n(x) - f_n(y) \vert  + \vert f(x) - f(y)\vert}{d(x,y)}\right)\\
	& \le \lim\limits_{n\to \infty}\left( \vert f_n(a) - f(a) \vert + \sup\limits_{x\neq y} \frac{\vert f_n(x) - f_n(y) \vert}{d(x,y)} + \sup\limits_{x\neq y}\frac{\vert f(x) - f(y)\vert}{d(x,y)}\right)
\end{flalign*}
Since $\forall x \in X, \lim\limits_{n\to\infty} f_n(x) = f(x)$, we have $\lim\limits_{n\to\infty} \vert f_n(a) - f(a) \vert = 0,$ and\\
$\lim\limits_{n\to\infty} \sup\limits_{x\neq y} \frac{\vert f_n(x) - f_n(y)\vert}{d(x,y)} = \sup\limits_{x\neq y} \frac{\vert f(x) - f(y)\vert}{d(x,y)}$. Hence we have 
\begin{flalign*}
	\lim\limits_{n\to\infty} \Vert f_n - f \Vert_{Lip,a} \le 0 \Rightarrow \lim\limits_{n\to\infty} \Vert f_n - f \Vert_{Lip,a} = 0&&
\end{flalign*}

\subsubsection*{3.3.3}
Since $\forall n \in \N, f_n \in Lip(x)$, then $\exists C > 0, \forall n \in N, \forall x,y \in X, \vert f_n(x) - f_n(y) \vert \le Cd(x,y)\\
 \Leftrightarrow \lim\limits_{n\to\infty}\vert f_n(x) - f_n(y)\vert \le \lim\limits_{n\to\infty}Cd(x,y) \Leftrightarrow \vert f(x) - f(y) \vert \le Cd(x,y)$, i.e. $f \in Lip(X)$.\\
 Now, since any arbitrary Cauchy sequence $\{f_n\}$ in $Lip(X)$ converges to a function $f \in Lip(X)$, $(Lip(X),\Vert.\Vert_{Lip,a})$ is complete.


\end{document}
